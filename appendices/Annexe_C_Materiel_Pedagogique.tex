%!TEX root = ../main.tex
\chapter{Matériel Pédagogique et Dispositif Technique}
\label{annexe:materiel}

Cette annexe présente l'ensemble du matériel pédagogique utilisé dans les études empiriques, ainsi que les spécifications techniques des dispositifs déployés. L'organisation suit la structure des études pour faciliter la consultation.

%=============================================================================
\section{Contenus Historiques -- Étude 1}
\label{annexe:contenus-e1}
%=============================================================================

\subsection{Le Premier Empire (Classes de 4\textsuperscript{ème})}

Le contenu pédagogique pour les classes de 4\textsuperscript{ème} portait sur le Premier Empire français (1804-1815). Les thématiques abordées comprenaient:

\begin{itemize}
    \item L'ascension de Napoléon Bonaparte: du général au consul, du consul à l'empereur
    \item Les grandes campagnes militaires: Austerlitz, Iéna, la campagne de Russie
    \item Les réformes administratives et juridiques: le Code civil, la réorganisation administrative
    \item La vie quotidienne sous l'Empire: conscription, économie continentale, propagande
    \item La chute de l'Empire: les Cent-Jours, Waterloo, l'exil à Sainte-Hélène
\end{itemize}

\subsection{La Rome Antique (Classes de 6\textsuperscript{ème})}

Pour les classes de 6\textsuperscript{ème}, le contenu portait sur la Rome antique à l'époque de Jules César:

\begin{itemize}
    \item La République romaine: institutions, vie politique, citoyenneté
    \item La conquête de la Gaule: les campagnes de César, Vercingétorix, Alésia
    \item La vie quotidienne à Rome: alimentation, habitat, loisirs (thermes, jeux)
    \item L'armée romaine: organisation, équipement, tactiques
    \item Les relations entre Romains et Gaulois: romanisation, échanges culturels
\end{itemize}

\subsection{La Seconde Guerre Mondiale et la Résistance (Classes de Terminale)}

Pour les classes de Terminale, le contenu portait sur la Résistance française:

\begin{itemize}
    \item L'Occupation: vie quotidienne, rationnement, collaboration
    \item Les formes de résistance: réseaux, maquis, presse clandestine
    \item Figures de la Résistance: Jean Moulin, de Gaulle, résistants anonymes
    \item La Libération: débarquement, insurrections, épuration
    \item Mémoire et commémoration: travail de mémoire, lieux de mémoire
\end{itemize}

%=============================================================================
\section{Représentation Visuelle des Agents -- Étude 1}
\label{annexe:agents-visuels-e1}
%=============================================================================

\subsection{Outils de Génération}

Les représentations visuelles des agents ont été générées à l'aide de Midjourney, un service de génération d'images par intelligence artificielle. L'objectif était de produire des portraits plausibles mais stylisés, évitant le photoréalisme pour prévenir l'effet de vallée dérangeante (\textit{uncanny valley}).

\begin{table}[htbp]
\centering
\caption{Outils utilisés pour la génération des représentations visuelles (Étude 1)}
\label{tab:outils-visuels-e1}
\begin{tabular}{lll}
\toprule
\textbf{Personnage} & \textbf{Outil} & \textbf{Version} \\
\midrule
Napoléon Bonaparte & Midjourney & V5 \\
Jules César & Midjourney & V5 \\
Charles de Gaulle & Midjourney & V5 \\
Louis (Résistant) & Midjourney & V6 (Remix Mode) \\
Agent Neutre & Midjourney & V5 \\
\bottomrule
\end{tabular}
\end{table}

\subsection{Spécifications des Personnages}

\paragraph{Napoléon Bonaparte (Expérimentation 1.1).} L'agent représentait Napoléon dans un portrait stylisé avec l'uniforme impérial caractéristique. L'expression faciale combinait autorité et une certaine distance formelle, conformément au style de communication adopté (formel, autoritaire, narration à la première personne).

\paragraph{Agent Neutre -- Condition Contrôle (Expérimentation 1.1).} L'agent neutre présentait un portrait sobre sans attributs historiques spécifiques. L'expression était neutre et professionnelle, cohérente avec le style de communication adopté (exposé factuel à la troisième personne, ton neutre).

\paragraph{Jules César (Expérimentation 1.2).} L'agent figurait César dans un portrait stylisé avec la couronne de laurier traditionnelle. Les traits du visage s'inspiraient des bustes antiques tout en étant rendus accessibles. Le style de communication était plus chaleureux et accessible que pour Napoléon, incluant des touches de vulnérabilité et d'humour.

\paragraph{Charles de Gaulle (Expérimentation 1.3).} L'agent présentait de Gaulle dans un portrait stylisé basé sur les photographies historiques. La voix a été recréée par clonage vocal (ElevenLabs) à partir d'enregistrements de discours historiques, permettant une reconnaissance immédiate du personnage.

\paragraph{Louis le Résistant (Expérimentation 1.3).} Ce personnage fictif a été créé à l'aide du Mode Remix de Midjourney V6, en utilisant le portrait de de Gaulle comme base pour générer un jeune combattant de la résistance (environ 20 ans). Le style de communication était informel et accessible, conçu comme une figure de pair pour les lycéens.

%=============================================================================
\section{Équipement Technique -- Étude 1}
\label{annexe:equipement-e1}
%=============================================================================

L'Étude 1 s'est déroulée dans des salles de classe standard équipées du matériel suivant:

\begin{itemize}
    \item Ordinateur portable Dell Precision 7780 (Intel Core i7-1185G7, NVIDIA RTX 4090, 16 Go RAM)
    \item Haut-parleur de conférence Jabra Speak 750 (capture et diffusion audio)
    \item Connexion au système d'affichage de la classe (vidéoprojecteur)
\end{itemize}

Cette configuration permettait à toute la classe de voir et d'entendre simultanément l'agent virtuel pendant les sessions interactives. Pour les conditions non interactives, des vidéos pré-enregistrées ont été diffusées sur le même système.

%=============================================================================
\section{Matériel Vidéo -- Condition Non-Interactive (Étude 1)}
\label{annexe:video-e1}
%=============================================================================

Pour les conditions non interactives des Expérimentations 1.1 et 1.2, des vidéos pré-enregistrées ont été produites. Ces vidéos présentaient le même contenu que les interactions en temps réel, mais sans possibilité d'échange. Les réponses des agents virtuels aux questions prédéfinies fournies par les enseignants ont été enregistrées puis éditées en séquences vidéo cohérentes.

\begin{table}[htbp]
\centering
\caption{Spécifications des vidéos pré-enregistrées (Étude 1)}
\label{tab:video-specs-e1}
\begin{tabular}{ll}
\toprule
\textbf{Paramètre} & \textbf{Valeur} \\
\midrule
Format & MP4 (H.264) \\
Résolution & 1920 × 1080 (Full HD) \\
Framerate & 30 fps \\
Durée moyenne & 10 minutes \\
Audio & AAC, 48 kHz, stéréo \\
\bottomrule
\end{tabular}
\end{table}

\clearpage
%=============================================================================
\section{Contenus Historiques -- Étude 2}
\label{annexe:contenus-e2}
%=============================================================================

L'Étude 2 se concentrait sur les pratiques funéraires de l'Égypte ancienne, avec un focus particulier sur la momification:

\begin{itemize}
    \item \textbf{Les croyances égyptiennes}: le ka et le ba, le voyage dans l'au-delà, le jugement d'Osiris
    \item \textbf{Le processus de momification}: les étapes techniques (éviscération, déshydratation au natron, bandelettage), la durée (environ 70 jours), les différentes qualités selon le statut social
    \item \textbf{Les acteurs}: les prêtres embaumeurs, leur formation, les rituels accompagnant chaque étape
    \item \textbf{Le mobilier funéraire}: sarcophages, vases canopes, amulettes, Livre des Morts
    \item \textbf{Les lieux d'inhumation}: mastabas, pyramides, hypogées de la Vallée des Rois
    \item \textbf{Les rituels}: l'Ouverture de la Bouche, les offrandes, le culte des morts
\end{itemize}

%=============================================================================
\section{Représentation Visuelle des Agents -- Étude 2}
\label{annexe:agents-visuels-e2}
%=============================================================================

\subsection{Condition Agent Humanoïde}

L'agent humanoïde a été généré à l'aide de HeyGen\footnote{HeyGen: https://www.heygen.com}, une plateforme de génération vidéo basée sur l'intelligence artificielle. Il apparaissait comme un professionnel masculin d'âge moyen s'adressant directement à la caméra, avec des mouvements naturels de la tête et des expressions faciales synchronisées avec la parole. L'environnement de bureau était neutre et la tenue formelle.

\subsection{Condition Agent Abstrait}

L'agent abstrait consistait en une visualisation audio-réactive non anthropomorphique, implémentée à l'aide de l'API HTML5 Canvas et de l'API Web Audio. L'interface comprenait:

\begin{itemize}
    \item Un cercle blanc pulsant central dont la taille variait avec l'amplitude audio
    \item 48 barres radiales réagissant à des bandes de fréquences spécifiques du spectre vocal
    \item Un fond neutre sans caractéristiques anthropomorphiques
\end{itemize}

\subsection{Synthèse Vocale}

Les deux conditions utilisaient un contenu vocal synthétique strictement identique, généré par ElevenLabs\footnote{ElevenLabs: https://elevenlabs.io}. La voix a été sélectionnée pour sa clarté, son ton formel et ses caractéristiques neutres. Le contenu audio pour la condition abstraite a été extrait des vidéos de la condition humanoïde, garantissant un débit de parole, une intonation et une prosodie identiques entre les conditions.

\begin{table}[htbp]
\centering
\caption{Caractéristiques comparées des agents par condition expérimentale (Étude 2)}
\label{tab:agents-conditions-e2}
\small
\begin{tabularx}{\textwidth}{>{\raggedright\arraybackslash}p{3cm}XX}
\toprule
\textbf{Caractéristique} & \textbf{Condition Humanoïde} & \textbf{Condition Abstraite} \\
\midrule
Apparence & Personnage humain réaliste (HeyGen) & Visualisation audio-réactive \\
Voix & Synthèse vocale ElevenLabs & Synthèse vocale ElevenLabs (identique) \\
Animation & Mouvements de tête, expressions faciales & Pulsations et barres réactives \\
Contenu verbal & Identique & Identique \\
\bottomrule
\end{tabularx}
\end{table}

%=============================================================================
\section{Équipement Technique -- Étude 2}
\label{annexe:equipement-e2}
%=============================================================================

L'Étude 2 utilisait un dispositif de projection collective:

\begin{itemize}
    \item Ordinateur portable de l'expérimentateur
    \item Vidéoprojecteur haute définition
    \item Système audio amplifié (enceintes de classe)
    \item Écran de projection
    \item Microphone mobile pour les porte-paroles des groupes
    \item Protocoles papier-crayon pour les élèves (livrets individuels)
\end{itemize}

%=============================================================================
\section{Architecture Logicielle Merlin -- Étude 2}
\label{annexe:merlin-e2}
%=============================================================================

Le logiciel Merlin, développé spécifiquement pour l'Étude 2, orchestrait l'ensemble des composants techniques de l'agent conversationnel.

\subsection{Spécifications Techniques}

\begin{table}[htbp]
\centering
\caption{Spécifications techniques de Merlin (Étude 2)}
\label{tab:merlin-specs-e2}
\begin{tabular}{ll}
\toprule
\textbf{Composant} & \textbf{Spécification} \\
\midrule
Framework & Electron v28.0.0 \\
Langage & JavaScript/TypeScript \\
Interface & HTML5/CSS3 \\
Rendu & Chromium embedded \\
Backend & Node.js \\
\bottomrule
\end{tabular}
\end{table}

\subsection{Architecture Tri-Composants}

Merlin reposait sur une architecture modulaire à trois composants principaux:

\paragraph{Interface Opérateur.} Permettait au chercheur de sélectionner les conditions expérimentales, de déclencher les réponses de l'agent et de contrôler les paramètres en temps réel (volume audio, vitesse de transition, mode plein écran). L'interface incluait un journal d'activité avec horodatage et des contrôles de préchargement des médias.

\paragraph{Interface Participant.} Affichait l'agent virtuel sur un écran visible par toute la classe. Cette fenêtre était configurée avec une résolution et des paramètres d'affichage cohérents à travers toutes les sessions expérimentales.

\paragraph{Gestion des Médias.} Pour la condition agent humanoïde, des séquences vidéo ont été créées pour différents états d'interaction (animation d'attente, message de bienvenue, réponses aux questions, message de clôture). Le système implémentait un double buffering pour des transitions fluides entre les séquences vidéo. Pour la condition agent abstrait, la visualisation audio-réactive était synchronisée avec les réponses audio par une analyse FFT (Fast Fourier Transform) réalisée sur le signal vocal.

\subsection{Génération du Contenu}

Les réponses de l'agent aux six questions prédéfinies ont été générées à l'aide de GPT-5\footnote{GPT-5, développé par OpenAI: https://openai.com} puis validées par un enseignant d'histoire-géographie pour leur exactitude historique et leur adéquation pédagogique à l'âge des participants. Ces réponses étaient identiques dans les deux conditions expérimentales.