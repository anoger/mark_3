%!TEX root = ../main.tex
\chapter{Considérations Éthiques et Consentement}
\label{annexe:ethique}

Cette annexe détaille les procédures éthiques mises en œuvre pour les études empiriques de cette thèse, incluant les approbations institutionnelles, les protocoles de recrutement et de consentement, ainsi que les mesures de protection des données.

%=============================================================================
\section{Approbations des Comités d'Éthique}
\label{annexe:approbations}
%=============================================================================

Les études de cette thèse ont fait l'objet d'évaluations par des comités d'éthique de la recherche conformément à la réglementation en vigueur pour les recherches impliquant des participants humains mineurs.

\subsection{Étude 1 -- Comité d'Éthique Inserm}

L'Étude 1, portant sur l'influence de l'incarnation et de l'interactivité des agents conversationnels sur l'intérêt des élèves, a été soumise au Comité d'Éthique de l'Inserm (Institut national de la santé et de la recherche médicale).

\begin{table}[htbp]
\centering
\caption{Approbation éthique -- Étude 1}
\label{tab:ethique-e1}
\begin{tabular}{ll}
\toprule
\textbf{Élément} & \textbf{Information} \\
\midrule
Comité & Comité d'Éthique Inserm (CEI) \\
Numéro d'avis & \textbf{CEEI/IRB n\textdegree{}23-1061} \\
Titre de l'étude & Influence de l'incarnation et de l'interactivité des agents \\
& conversationnels historiques sur l'intérêt des élèves \\
Responsable scientifique & Dr. Sylvain FLEURY \\
Avis & \textbf{Favorable} \\
\bottomrule
\end{tabular}
\end{table}

\subsection{Étude 2 -- Comité d'Éthique Arts et Métiers}

L'Étude 2, examinant l'influence du design d'agents virtuels alimentés par l'IA sur l'illusion de compréhension, a reçu un avis favorable du Comité d'Éthique de la Recherche (CER) Arts et Métiers.

\begin{table}[htbp]
\centering
\caption{Approbation éthique -- Étude 2}
\label{tab:ethique-e2}
\begin{tabular}{ll}
\toprule
\textbf{Élément} & \textbf{Information} \\
\midrule
Comité & Comité d'Éthique de la Recherche Arts et Métiers (CER ENSAM) \\
Numéro d'avis & \textbf{CER\_AM\_2025\_007} \\
Date de la réunion & 05 juin 2025 \\
Titre de l'étude & Étude de l'influence du design d'agents virtuels alimentés \\
& par l'IA sur l'illusion de compréhension chez les collégiens \\
Responsable scientifique & Dr. Sylvain FLEURY \\
Demandeur & Antoine OGER \\
Laboratoire & LAMPA (Laboratoire Angevin de Mécanique, Procédés et Innovation) \\
Avis & \textbf{Favorable} \\
\bottomrule
\end{tabular}
\end{table}

Le CER ENSAM a rappelé que l'investigateur s'engage à respecter le protocole déposé et à suivre les recommandations du comité.

\subsection{Étude 3 -- Procédure en Cours}

L'Étude 3, prévue pour février 2026, est actuellement en cours d'instruction auprès du Comité d'Éthique de la Recherche Arts et Métiers. Cette étude examinera les effets combinés de l'incarnation et de la prosodie dans un design expérimental 2×2.

%=============================================================================
\section{Recrutement des Participants}
\label{annexe:recrutement}
%=============================================================================

\subsection{Population Cible}

Les études ont ciblé des élèves du secondaire dans des établissements de la région Pays de la Loire:

\begin{table}[htbp]
\centering
\caption{Caractéristiques des populations par étude}
\label{tab:populations}
\small
\begin{tabularx}{\textwidth}{>{\raggedright\arraybackslash}p{3.5cm}XXX}
\toprule
\textbf{Caractéristique} & \textbf{Étude 1} & \textbf{Étude 2} & \textbf{Étude 3 (prévu)} \\
\midrule
Niveau scolaire & 6\textsuperscript{ème}, 3\textsuperscript{ème} & 5\textsuperscript{ème}, 4\textsuperscript{ème} & 2\textsuperscript{nde} \\
Tranche d'âge & 11--15 ans & 12--14 ans & 15--16 ans \\
Effectif total & 113 élèves & 119 élèves & $\approx$140 élèves \\
Établissement(s) & Collèges partenaires & Collège Saint Aubin La Salle (Verrières-en-Anjou) & À confirmer \\
\bottomrule
\end{tabularx}
\end{table}

\subsection{Procédure de Recrutement}

Le recrutement suivait une procédure standardisée:

\begin{enumerate}
    \item \textbf{Contact institutionnel}: Prise de contact avec la direction de l'établissement et présentation du projet de recherche
    \item \textbf{Accord de l'établissement}: Obtention de l'autorisation du chef d'établissement
    \item \textbf{Coordination pédagogique}: Collaboration avec les enseignants d'histoire-géographie pour l'intégration dans le programme
    \item \textbf{Information des familles}: Envoi d'une lettre d'information aux parents/tuteurs légaux
    \item \textbf{Recueil des consentements}: Collecte des formulaires de consentement parental signés
    \item \textbf{Assentiment des élèves}: Vérification de l'accord oral des élèves le jour de l'expérimentation
\end{enumerate}

%=============================================================================
\section{Consentement Éclairé}
\label{annexe:consentement}
%=============================================================================

\subsection{Principes Appliqués}

Conformément aux principes éthiques de la recherche impliquant des mineurs, un double niveau de consentement a été mis en œuvre:

\paragraph{Consentement parental.} Les parents ou tuteurs légaux ont reçu une information écrite détaillant les objectifs de l'étude, les modalités de participation, les données collectées, leur traitement et leurs droits. Leur signature était requise préalablement à toute participation.

\paragraph{Assentiment de l'élève.} Au début de chaque session, les élèves ont été informés oralement de la nature de l'étude. Leur participation était volontaire et ils pouvaient se retirer à tout moment sans conséquence.

\subsection{Information Fournie aux Participants}

Les documents d'information comprenaient:

\begin{itemize}
    \item Description générale de l'étude et de ses objectifs pédagogiques
    \item Nature des activités proposées (interaction avec un agent virtuel, questionnaires)
    \item Durée estimée de la participation
    \item Types de données collectées (réponses aux questionnaires, aucun enregistrement audio/vidéo des élèves)
    \item Garanties d'anonymat et de confidentialité
    \item Droits des participants (accès, rectification, retrait)
    \item Coordonnées du responsable scientifique pour toute question
\end{itemize}

\subsection{Critères d'Inclusion et d'Exclusion}

\paragraph{Critères d'inclusion:}
\begin{itemize}
    \item Élève inscrit dans une classe participante
    \item Consentement parental écrit obtenu
    \item Assentiment oral de l'élève
\end{itemize}

\paragraph{Critères d'exclusion:}
\begin{itemize}
    \item Refus parental ou de l'élève
    \item Absence le jour de l'expérimentation
    \item Retrait en cours de session (données non utilisées)
\end{itemize}

%=============================================================================
\section{Protection des Données}
\label{annexe:protection-donnees}
%=============================================================================

\subsection{Anonymisation}

L'anonymat des participants a été garanti par:

\begin{itemize}
    \item Attribution d'un code participant unique (non nominatif)
    \item Aucune collecte d'informations directement identifiantes
    \item Séparation des données de consentement (nominatives) et des données de recherche (anonymes)
    \item Stockage sécurisé avec accès restreint à l'équipe de recherche
\end{itemize}

\subsection{Données Collectées}

\begin{table}[htbp]
\centering
\caption{Nature des données collectées par étude}
\label{tab:donnees-collectees}
\begin{threeparttable}
\begin{tabular}{lccc}
\toprule
\textbf{Type de donnée} & \textbf{Étude 1} & \textbf{Étude 2} & \textbf{Étude 3} \\
\midrule
Âge & $\checkmark$ & $\checkmark$ & $\checkmark$ \\
Sexe & $\checkmark$ & $\checkmark$ & $\checkmark$ \\
Niveau scolaire & $\checkmark$ & $\checkmark$ & $\checkmark$ \\
Réponses aux questionnaires & $\checkmark$ & $\checkmark$ & $\checkmark$ \\
Productions écrites & -- & $\checkmark$ & -- \\
Scores aux tests de connaissances & -- & $\checkmark$ & $\checkmark$ \\
Enregistrements audio/vidéo & -- & -- & -- \\
\bottomrule
\end{tabular}
\begin{tablenotes}
\small
\item \textit{Note.} Aucun enregistrement audio ou vidéo des élèves n'a été réalisé dans aucune des études.
\end{tablenotes}
\end{threeparttable}
\end{table}

\subsection{Stockage et Conservation}

Les données ont été stockées conformément aux recommandations du RGPD:

\begin{itemize}
    \item Stockage sur serveur sécurisé de l'institution (Arts et Métiers)
    \item Accès restreint aux membres de l'équipe de recherche
    \item Durée de conservation limitée aux besoins de l'analyse et de la publication
    \item Destruction programmée à l'issue du projet de recherche
\end{itemize}

%=============================================================================
\section{Considérations Spécifiques aux Mineurs}
\label{annexe:considerations-mineurs}
%=============================================================================

La recherche impliquant des mineurs requiert une attention particulière à plusieurs dimensions:

\paragraph{Bénéfice pédagogique.} Les activités proposées s'inscrivaient dans le cadre des programmes scolaires d'histoire-géographie, assurant un bénéfice éducatif direct pour les participants.

\paragraph{Absence de risque.} Les interactions avec les agents virtuels ne présentaient aucun risque identifié pour le bien-être physique ou psychologique des élèves. Le contenu était adapté à l'âge et au niveau scolaire.

\paragraph{Encadrement.} Les sessions se sont déroulées en présence de l'enseignant habituel de la classe et de l'expérimentateur, dans l'environnement familier de l'établissement scolaire.

\paragraph{Droit de retrait.} Les élèves pouvaient cesser leur participation à tout moment, sans justification nécessaire. En cas de non-participation, une activité alternative était proposée par l'enseignant.

%=============================================================================
\section{Débriefing}
\label{annexe:debriefing}
%=============================================================================

\subsection{Débriefing -- Étude 1}

À l'issue de chaque session de l'Étude 1, un débriefing collectif immédiat était réalisé:

\begin{itemize}
    \item Explication de l'objectif de l'étude (comparer différentes façons de présenter l'histoire)
    \item Clarification sur la nature de l'agent virtuel (programme informatique, pas une vraie personne)
    \item Temps de questions-réponses avec les élèves
    \item Remerciements pour la participation
\end{itemize}

\subsection{Débriefing -- Étude 2}

L'Étude 2 a mis en place une procédure de débriefing plus élaborée compte tenu de la thématique de l'illusion de compréhension:

\paragraph{Débriefing immédiat (5 minutes).} En fin de session, explication générale sur l'objectif de l'étude et remerciements.

\paragraph{Séance de débriefing approfondi (30--40 minutes).} Une semaine après l'expérimentation, une séance dédiée a été organisée avec chaque classe participante:

\begin{itemize}
    \item Présentation des objectifs réels de l'étude (mesure de l'illusion de compréhension)
    \item Explication du concept d'illusion de compréhension avec des exemples accessibles
    \item Discussion sur les différences entre <<croire comprendre>> et <<comprendre réellement>>
    \item Réflexion collective sur les stratégies d'apprentissage efficaces
    \item Échanges sur la place de l'IA dans l'éducation
    \item Temps de questions libres
\end{itemize}

Cette séance visait à transformer l'expérience de recherche en opportunité pédagogique de métacognition.

%=============================================================================
\section{Communication avec les Parties Prenantes}
\label{annexe:communication}
%=============================================================================

\subsection{Information des Établissements}

Les établissements scolaires partenaires ont reçu:

\begin{itemize}
    \item Présentation détaillée du projet de recherche
    \item Protocole expérimental complet
    \item Calendrier prévisionnel des interventions
    \item Coordonnées de l'équipe de recherche
\end{itemize}

À l'issue des études, une restitution des résultats généraux (sans données individuelles) a été proposée aux établissements.

\subsection{Information des Familles}

Au-delà du formulaire de consentement, les familles ont été informées:

\begin{itemize}
    \item Par courrier individuel préalable à l'étude
    \item Via le carnet de liaison ou l'ENT de l'établissement
    \item Par la possibilité de contacter directement l'équipe de recherche
\end{itemize}

\subsection{Retour aux Participants}

Les élèves ont bénéficié d'un retour sur leur participation à travers:

\begin{itemize}
    \item Le débriefing post-expérimentation
    \item La séance de réflexion métacognitive (Étude 2)
    \item La possibilité de poser des questions sur la recherche
\end{itemize}