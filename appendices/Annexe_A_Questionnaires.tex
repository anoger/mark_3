%!TEX root = ../main.tex
\chapter{Questionnaires et Échelles de Mesure}
\label{annexe:questionnaires}

Cette annexe présente l'intégralité des instruments de mesure utilisés dans les études empiriques de cette thèse. Pour chaque échelle, nous fournissons la version originale anglaise ainsi que l'adaptation française employée lors des passations. Les instruments sont organisés par étude afin de faciliter la consultation.

%=============================================================================
\section{Intrinsic Motivation Inventory (IMI) -- Étude 1}
\label{annexe:imi}
%=============================================================================

L'Intrinsic Motivation Inventory \citep{deci1994} a été utilisé dans l'Étude 1 pour mesurer l'intérêt des élèves. Les items évaluaient l'intérêt selon trois dimensions : l'activité elle-même, le personnage historique et le thème de la leçon. Pour chaque item, les participants répondaient en cochant une case sur une échelle en 7 points allant de 1 (<<~pas du tout d'accord~>>) à 7 (<<~tout à fait d'accord~>>).

\subsection{Items Relatifs à l'Activité}

\paragraph{Avant l'interaction :}
\begin{itemize}
    \item J'ai vraiment envie de faire cette activité.
    \item Je pense que cette activité est très intéressante.
    \item Je pense que cette activité est ennuyeuse. (I)
\end{itemize}

\paragraph{Après l'interaction :}
\begin{itemize}
    \item J'ai vraiment apprécié faire cette activité.
    \item J'ai trouvé cette activité très intéressante.
    \item J'ai trouvé cette activité ennuyeuse. (I)
\end{itemize}

\subsection{Items Relatifs au Personnage Historique}

\paragraph{Avant et après l'interaction :}
\begin{itemize}
    \item J'ai vraiment envie d'en savoir plus sur Napoléon.
    \item Je pense que découvrir des choses sur Napoléon sera très intéressant.
    \item Je pense que découvrir des choses sur Napoléon est ennuyeux. (I)
\end{itemize}

\subsection{Items Relatifs au Thème de la Leçon}

\paragraph{Avant et après l'interaction :}
\begin{itemize}
    \item J'ai vraiment envie d'en savoir plus sur l'Égypte ancienne.
    \item Je pense que découvrir des choses sur l'Égypte antique est très intéressant.
    \item Je pense que découvrir des choses sur l'Égypte antique est ennuyeux. (I)
\end{itemize}

\vspace{0.5cm}
\noindent\textit{Note.} (I) = item inversé. Le nom du personnage historique et le thème de la leçon étaient adaptés selon la sous-expérience (Napoléon/Empire, César/Rome, de Gaulle/Résistance).

\section{Procédure d'Administration -- Étude 1}
\label{annexe:procedures-admin-e1}

Les questionnaires de l'Étude 1 ont été administrés sous forme de livrets papier distribués aux élèves en début et en fin de session. Les items des trois dimensions (intérêt pour l'activité, intérêt pour le personnage historique, intérêt pour le contenu de la leçon) ont été randomisés pour minimiser les effets d'ordre et les biais de réponse.

%=============================================================================
\section{Protocole IOED -- Étude 2}
\label{annexe:ioed}
%=============================================================================

Le protocole IOED \citep{rozenblit2002} a été adapté pour l'Étude 2 afin de mesurer l'illusion de compréhension chez les collégiens. Cette procédure en quatre phases combine des auto-évaluations subjectives et des évaluations objectives de la compréhension.

\subsection{Procédure de Mesure Standardisée}

\paragraph{Phase 1 : Auto-évaluation initiale (avant interaction).}
\textit{Instructions}: <<Avant d'interagir avec l'agent, évalue ta compréhension des concepts suivants sur une échelle de 1 à 7.>>
Échelle: 1 = <<Je ne comprends pas du tout ce concept>> à 7 = <<Je comprends parfaitement ce concept>>

\paragraph{Phase 2 : Production d'explications (avant interaction).}
\textit{Instructions}: <<Explique brièvement, avec tes propres mots, chacun des concepts suivants. Sois aussi précis que possible dans l'espace prévu.>>
Format: Espace limité à 10 lignes par concept pour standardiser la longueur des réponses.

\paragraph{Phase 3 : Auto-évaluation post-interaction.}
\textit{Instructions}: <<Après avoir interagi avec l'agent, évalue à nouveau ta compréhension des mêmes concepts sur une échelle de 1 à 7.>>
Échelle: Identique à la Phase 1 (1--7).

\paragraph{Phase 4 : Évaluation objective (après interaction).}
\textit{Instructions}: <<Réponds aux questions suivantes concernant les concepts étudiés.>>
Format du mini-test: Pour chaque concept, 2 questions à choix multiples (3 options chacune) et 1 question courte nécessitant une réponse rédigée de 2--3 phrases.

\subsection{Calcul des Scores}

\begin{enumerate}
    \item \textbf{Degré d'illusion initial}: Différence entre l'auto-évaluation initiale (Phase 1) et la qualité des explications produites (Phase 2). La qualité des explications écrites est évaluée par deux chercheurs indépendants sur une échelle standardisée de 1 à 7, utilisant une grille critériée préétablie tenant compte de la précision factuelle, de la complexité et de la cohérence.
    
    \item \textbf{Ajustement métacognitif}: Différence entre l'auto-évaluation post-interaction (Phase 3) et l'auto-évaluation initiale (Phase 1). Un score positif indique une augmentation de la confiance après exposition aux explications de l'agent.
    
    \item \textbf{Persistance de l'illusion}: Différence entre l'auto-évaluation post-interaction (Phase 3) et la performance objective au mini-test (Phase 4). Le score au mini-test est converti sur une échelle de 1 à 7 pour permettre la comparaison.
\end{enumerate}

%=============================================================================
\section{Godspeed Questionnaire -- Étude 2}
\label{annexe:godspeed}
%=============================================================================

La sous-échelle Anthropomorphisme du Godspeed Questionnaire Series \citep{bartneck2009} évalue la manière dont les utilisateurs perçoivent le caractère humain d'entités artificielles. Cette échelle sémantique différentielle en 5 points a été utilisée dans l'Étude 2.

\begin{table}[htbp]
\centering
\caption{Items du Godspeed Questionnaire (sous-échelle Anthropomorphisme) -- Étude 2}
\label{tab:godspeed-items}
\small
\begin{tabular}{lccccccl}
\toprule
\textbf{Pôle négatif} & & \textbf{1} & \textbf{2} & \textbf{3} & \textbf{4} & \textbf{5} & \textbf{Pôle positif} \\
\midrule
\textit{Fake} / Faux & & $\square$ & $\square$ & $\square$ & $\square$ & $\square$ & \textit{Natural} / Naturel \\[0.3cm]
\textit{Machinelike} / Machine & & $\square$ & $\square$ & $\square$ & $\square$ & $\square$ & \textit{Humanlike} / Humain \\[0.3cm]
\textit{Unconscious} / Inconscient & & $\square$ & $\square$ & $\square$ & $\square$ & $\square$ & \textit{Conscious} / Conscient \\[0.3cm]
\textit{Artificial} / Artificiel & & $\square$ & $\square$ & $\square$ & $\square$ & $\square$ & \textit{Lifelike} / Vivant \\[0.3cm]
\textit{Rigid} / Rigide & & $\square$ & $\square$ & $\square$ & $\square$ & $\square$ & \textit{Elegant} / Élégant \\
\bottomrule
\end{tabular}

\vspace{0.2cm}
\footnotesize\textit{Note.} Instructions: <<En pensant à l'agent virtuel avec lequel tu viens d'interagir, indique à quel point tu le décrirais en cochant une case pour chaque ligne.>>
\end{table}

%=============================================================================
\section{Multi-Dimensional Measure of Trust (MDMT) -- Étude 2}
\label{annexe:mdmt}
%=============================================================================

Le Multi-Dimensional Measure of Trust \citep{malle2021} évalue la confiance accordée à l'agent virtuel en distinguant la confiance liée à la \textit{performance} de celle liée à la \textit{moralité}. L'adaptation française présentée ci-dessous a été utilisée dans l'Étude 2 (échelle de Likert en 7 points: 1 = Pas du tout ; 7 = Tout à fait).

\begin{table}[htbp]
\centering
\caption{Items MDMT -- Dimension Performance (Étude 2)}
\label{tab:mdmt-performance}
\small
\begin{tabular}{p{2.5cm}p{8cm}p{2.5cm}}
\toprule
\textbf{Item Original} & \textbf{Adaptation Française} & \textbf{Sous-dimension} \\
\midrule
Reliable & Cet agent virtuel est fiable. & Fiabilité \\[0.15cm]
Predictable & Cet agent virtuel est prévisible dans ses réponses. & Fiabilité \\[0.15cm]
Can count on & Je peux compter sur cet agent virtuel. & Fiabilité \\[0.15cm]
Consistent & Cet agent virtuel est cohérent dans ses réponses. & Fiabilité \\[0.15cm]
\midrule
Capable & Cet agent virtuel est capable de bien répondre. & Capacité \\[0.15cm]
Skilled & Cet agent virtuel est compétent. & Capacité \\[0.15cm]
Competent & Cet agent virtuel est compétent. & Capacité \\[0.15cm]
Meticulous & Cet agent virtuel est précis dans ses informations. & Capacité \\
\bottomrule
\end{tabular}
\end{table}

\begin{table}[htbp]
\centering
\caption{Items MDMT -- Dimension Moralité (Étude 2)}
\label{tab:mdmt-moralite}
\small
\begin{tabular}{p{2.5cm}p{8cm}p{2.5cm}}
\toprule
\textbf{Item Original} & \textbf{Adaptation Française} & \textbf{Sous-dimension} \\
\midrule
Sincere & Cet agent virtuel est sincère. & Sincérité \\[0.15cm]
Genuine & Cet agent virtuel est authentique. & Sincérité \\[0.15cm]
Candid & Cet agent virtuel est franc et direct. & Sincérité \\[0.15cm]
Authentic & Cet agent virtuel dit la vérité. & Sincérité \\[0.15cm]
\midrule
Ethical & Cet agent virtuel est respectable. & Éthique \\[0.15cm]
Respectable & Cet agent virtuel a de bonnes valeurs. & Éthique \\[0.15cm]
Principled & Cet agent virtuel ne ment pas. & Éthique \\[0.15cm]
Has integrity & Cet agent virtuel a de l'intégrité. & Éthique \\
\bottomrule
\end{tabular}

\vspace{0.2cm}
\footnotesize\textit{Note.} Dans le protocole imprimable utilisé avec les élèves, une version simplifiée à 13 items a été administrée, omettant les items \textit{Genuine}, \textit{Authentic} et \textit{Has integrity} pour éviter la redondance perçue par les collégiens.
\end{table}

%=============================================================================
\section{Échelle de Persuasion Perçue -- Étude 2}
\label{annexe:persuasion}
%=============================================================================

L'échelle de Persuasion Perçue, adaptée de \citet{thomas2019}, mesure la perception de la capacité de l'agent virtuel à influencer les attitudes ou les comportements de l'utilisateur. Elle distingue trois facteurs: Efficacité, Qualité et Capacité (échelle de Likert en 7 points: 1 = Pas du tout d'accord ; 7 = Tout à fait d'accord).

\begin{table}[htbp]
\centering
\caption{Items de l'Échelle de Persuasion Perçue -- Étude 2}
\label{tab:persuasion-items}
\small
\begin{tabular}{p{4.5cm}p{6.5cm}p{2cm}}
\toprule
\textbf{Item Original} & \textbf{Adaptation Française} & \textbf{Facteur} \\
\midrule
\multicolumn{3}{l}{\textit{Facteur 1 : Efficacité}} \\
\midrule
This message will cause changes in my behavior. & Cet agent virtuel peut influencer mes actions et mon comportement. & Efficacité \\[0.2cm]
This message causes me to make some changes in my behavior. & Cet agent virtuel m'encourage à changer mon comportement. & Efficacité \\[0.2cm]
After viewing this message, I will make changes in my attitude. & Après avoir discuté avec cet agent virtuel, mon attitude a changé. & Efficacité \\
\midrule
\multicolumn{3}{l}{\textit{Facteur 2 : Qualité}} \\
\midrule
This message is accurate. & Les informations données par ce guide virtuel sont fiables. & Qualité \\[0.2cm]
This message is trustworthy. & Je crois que les informations données par ce guide virtuel sont vraies. & Qualité \\[0.2cm]
I believe this message is true. & Ce guide virtuel peut changer ma façon de voir les choses. & Qualité \\
\midrule
\multicolumn{3}{l}{\textit{Facteur 3 : Capacité}} \\
\midrule
This message has the potential to change user behavior. & Cet agent virtuel a le potentiel d'influencer mon comportement. & Capacité \\[0.2cm]
This message has the potential to influence user behavior. & Cet agent virtuel peut me faire agir différemment. & Capacité \\[0.2cm]
This message has the potential to inspire users. & Cet agent virtuel m'inspire et me motive. & Capacité \\
\bottomrule
\end{tabular}
\end{table}

%=============================================================================
\section{Test de Connaissances -- Étude 2}
\label{annexe:test-connaissances}
%=============================================================================

Le test de connaissances post-interaction de l'Étude 2, co-conçu avec l'enseignant d'histoire-géographie, évaluait la compréhension du processus de momification égyptienne. Il comprenait 3 questions à choix multiples et 2 questions ouvertes.

\subsection{Questions à Choix Multiples}

\begin{enumerate}
    \item \textbf{Combien de temps durait environ le processus complet de momification ?}
    \begin{enumerate}[label=\alph*)]
        \item Quelques jours (3--5 jours)
        \item Environ un mois (30 jours)
        \item Environ 70 jours (plus de 2 mois) \textbf{[Réponse correcte]}
    \end{enumerate}
    
    \item \textbf{Quel était l'élément principal utilisé pour déshydrater le corps ?}
    \begin{enumerate}[label=\alph*)]
        \item Du sable chaud du désert
        \item Du natron (un sel minéral) \textbf{[Réponse correcte]}
        \item Des huiles parfumées
    \end{enumerate}
    
    \item \textbf{Comment les embaumeurs retiraient-ils le cerveau ?}
    \begin{enumerate}[label=\alph*)]
        \item En ouvrant la tête par le haut
        \item En passant par les narines avec des outils spéciaux \textbf{[Réponse correcte]}
        \item Ils ne retiraient jamais le cerveau
    \end{enumerate}
\end{enumerate}

\subsection{Questions Ouvertes}

\begin{enumerate}
    \setcounter{enumi}{3}
    \item \textbf{Explique brièvement pourquoi tous les Égyptiens n'étaient pas momifiés de la même façon.}
    
    \textit{Éléments de réponse attendus}: Références aux différences de statut social, au coût du processus, aux différentes qualités de momification disponibles selon les moyens financiers.
    
    \item \textbf{Explique pourquoi les Égyptiens momifiaient les morts (2--3 phrases).}
    
    \textit{Éléments de réponse attendus}: Croyances en la vie après la mort, préservation du corps pour le voyage dans l'au-delà, nécessité d'un corps intact pour que l'âme (ka et ba) puisse y retourner.
\end{enumerate}

\subsection{Barème de Notation}

\begin{table}[htbp]
\centering
\caption{Barème de notation du test de connaissances -- Étude 2}
\label{tab:bareme-test}
\begin{tabular}{lcc}
\toprule
\textbf{Type de question} & \textbf{Points par item} & \textbf{Total} \\
\midrule
Questions à choix multiples (3 items) & 1 point & 3 points \\[0.15cm]
Questions ouvertes (2 items) & 0--3 points & 6 points \\
\midrule
\textbf{Score total} & & \textbf{9 points} \\
\bottomrule
\end{tabular}

\vspace{0.2cm}
\footnotesize\textit{Note.} Pour les questions ouvertes, l'évaluation tenait compte de la précision factuelle (0--1 pt), de la complétude (0--1 pt) et de la cohérence de l'explication (0--1 pt).
\end{table}

\section{Procédure d'Administration -- Étude 2}
\label{annexe:procedures-admin-e2}

L'Étude 2 a utilisé un protocole papier-crayon structuré en six parties:

\begin{enumerate}
    \item \textbf{Partie 1}: Auto-évaluation initiale des connaissances (IOED Phase 1)
    \item \textbf{Partie 2}: Production d'explications écrites (IOED Phase 2)
    \item \textbf{Partie 3}: Auto-évaluation pré-interaction (IOED Phase 3 partielle)
    \item \textbf{Partie 4}: Auto-évaluation post-interaction (IOED Phase 3 finale)
    \item \textbf{Partie 5}: Évaluation de l'agent (Godspeed, MDMT, Persuasion)
    \item \textbf{Partie 6}: Test de connaissances objectif (IOED Phase 4)
\end{enumerate}

Chaque participant disposait d'un code anonyme pour garantir la confidentialité tout en permettant l'appariement des données.