\begin{otherlanguage}{english}
\chapter*{Abstract}
\addcontentsline{toc}{chapter}{Abstract}

\vspace{0.5cm}

Secondary school history education faces a persistent challenge: students acknowledge the importance of the discipline while finding its traditional classroom format disengaging. The emergence of large language models opens unprecedented possibilities---conversing with representations of historical figures---that could transform this relationship. However, this promise carries a risk: the very fluency of these agents may induce unwarranted confidence in one's own understanding.

This thesis investigates the design and impact of AI-powered embodied conversational agents in this educational context. It adopts a dual focus, examining both student engagement and metacognitive risks through two empirical studies conducted in ecological settings.

Study~1, comprising three experiments with 339~students across sixth, eighth, and twelfth grades, examines the influence of interactivity and agent characteristics on interest. Results reveal that oral interaction with a historical agent significantly increases interest compared to passive video presentation, regardless of grade level. The effect of agent representation proves more nuanced, modulated by communicative style rather than thematic alignment.

Study~2, conducted with 119~seventh-grade students, examines the influence of agent visual design (humanoid vs. abstract) on the illusion of understanding. Contrary to initial hypotheses, both conditions produce comparable metacognitive profiles: students overestimate their understanding regardless of agent appearance.

Methodologically, the thesis introduces MemorIA, a modular technical architecture integrating speech recognition, language generation, voice synthesis, and facial animation. The findings suggest cautious optimism: interactivity serves as a robust engagement lever, yet the challenge remains to design systems that engage without deactivating students' critical awareness.

\vspace{0.8cm}

\textbf{Keywords:} Conversational agents, Generative artificial intelligence, History education, Situational interest, Illusion of understanding, Human-Computer Interaction

\end{otherlanguage}
\cleardoublepage
