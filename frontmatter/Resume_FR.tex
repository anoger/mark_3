\chapter*{Résumé}
\addcontentsline{toc}{chapter}{Résumé}

\vspace{0.5cm}

L'enseignement de l'histoire au secondaire se heurte à un paradoxe : les élèves reconnaissent l'importance de cette discipline tout en la percevant comme peu engageante dans sa forme scolaire traditionnelle. L'émergence des grands modèles de langage ouvre une possibilité inédite --- converser avec des représentations de figures historiques --- qui pourrait transformer ce rapport. Cette promesse se double cependant d'un risque : la fluidité même de ces agents pourrait induire une confiance excessive dans sa propre compréhension.

Cette thèse examine la conception et l'impact des agents conversationnels incarnés alimentés par l'intelligence artificielle générative dans ce contexte éducatif. Elle adopte une double focale, articulant l'étude de l'engagement des élèves avec celle des risques métacognitifs, à travers deux études empiriques conduites en contexte écologique.

L'Étude~1, composée de trois expérimentations auprès de 339~élèves de sixième, quatrième et terminale, examine l'influence de l'interactivité et des caractéristiques de l'agent sur l'intérêt. Les résultats révèlent que l'interactivité orale avec un agent historique augmente significativement l'intérêt par rapport à une présentation vidéo passive, et ce quel que soit le niveau scolaire. L'effet de la représentation apparaît plus nuancé, modulé par le style communicationnel de l'agent plutôt que par son alignement thématique formel.

L'Étude~2, conduite auprès de 119~collégiens de cinquième, examine l'influence du design visuel de l'agent (humanoïde vs. abstrait) sur l'illusion de compréhension. Contrairement aux hypothèses initiales, les deux conditions produisent des profils métacognitifs comparables : les élèves surestiment leur compréhension indépendamment de l'apparence de l'agent.

Sur le plan méthodologique, la thèse présente MemorIA, une architecture technique modulaire intégrant reconnaissance vocale, génération de langage, synthèse vocale et animation faciale. Les résultats invitent à un optimisme mesuré : l'interactivité constitue un levier d'engagement robuste, mais l'enjeu demeure de concevoir des dispositifs qui engagent sans désactiver la vigilance critique des élèves.

\vspace{0.8cm}

\textbf{Mots-clés :} Agents conversationnels, Intelligence artificielle générative, Enseignement de l'histoire, Intérêt situationnel, Illusion de compréhension, Interaction Homme-Machine

\cleardoublepage
