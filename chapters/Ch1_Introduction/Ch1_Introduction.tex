\chapter{Introduction Générale}
\label{ch:introduction}

\sommairechapitre

\section{Contexte : L'Enseignement de l'Histoire face aux Nouveaux Médias}
\label{sec:intro_contexte}

L'enseignement de l'histoire au secondaire se heurte à une difficulté récurrente. Les élèves reconnaissent la pertinence de cette discipline : interrogés sur son utilité, ils évoquent spontanément la compréhension du présent, l'évitement des erreurs passées, la formation citoyenne. Cette reconnaissance coexiste pourtant avec un désengagement face au cours magistral, aux exercices de mémorisation, aux évaluations écrites répétitives. Le paradoxe traverse les enquêtes internationales : les élèves jugent l'histoire ennuyeuse dans sa forme scolaire habituelle, mais importante pour décrypter l'actualité \citep{kristinHenkaline2023}.

Les enquêtes comparatives entre disciplines éclairent cette tension. L'histoire se classe en position intermédiaire dans les préférences des élèves, devancée par les disciplines requérant une activité concrète : éducation physique, technologie, arts plastiques \citep{haydn2010}. Ces matières impliquent une manipulation, une production tangible, un engagement corporel ou créatif que le cours d'histoire peine à offrir. Les variations observées entre établissements suggèrent néanmoins que le problème n'est pas inhérent à la discipline. Les écarts d'appréciation atteignent plusieurs dizaines de points de pourcentage selon les contextes, soulignant l'effet déterminant des choix pédagogiques.

La comparaison avec les disciplines scientifiques et techniques révèle une asymétrie supplémentaire. Les STIM bénéficient d'une perception d'utilité professionnelle que l'histoire ne peut revendiquer avec la même évidence \citep{grever2011}. Les débouchés des filières scientifiques apparaissent clairement aux élèves et à leurs familles. L'histoire semble cantonnée à l'enseignement, à l'archéologie, aux métiers du patrimoine --- des secteurs perçus comme moins attractifs économiquement. Cette hiérarchie implicite influence les choix d'orientation, malgré la transférabilité des compétences historiques : analyse critique des sources, argumentation nuancée, synthèse documentaire.

Les travaux en didactique de l'histoire convergent sur un point : les méthodes interactives modifient le rapport des élèves à la discipline. Jeux de rôle, débats structurés, simulations de procès, analyse de sources primaires génèrent un engagement que le cours transmissif ne suscite pas \citep{vanStraaten2016}. Les approches expérientielles --- vidéos immersives, inclusion de figures historiques diverses --- activent l'empathie historique et établissent des connexions explicites entre passé et présent \citep{kristinHenkaline2023}. La manière d'enseigner détermine l'engagement davantage que le contenu.

L'émergence des technologies d'intelligence artificielle générative s'inscrit dans ce contexte. Les grands modèles de langage ouvrent une possibilité inédite : converser avec des représentations de figures historiques. La comparaison entre dialogues interactifs et lecture traditionnelle indique des effets positifs sur la motivation, bien que la contribution spécifique de l'interactivité reste à clarifier \citep{pataranutaporn2023}. Les méta-analyses sur les agents virtuels alimentés par l'IA confirment ces résultats, avec des effets plus marqués pour les agents combinant apparence humanoïde et interactions structurées \citep{dai2024}.

Les technologies de synthèse vocale et d'animation faciale complètent ce dispositif. Elles incarnent visuellement et vocalement le personnage. Le \og témoin virtuel \fg{} du passé devient une réalité technique. Un élève peut poser des questions à Jules César et obtenir des réponses immédiates, formulées à la première personne. Cette possibilité suscite un enthousiasme compréhensible au regard des difficultés documentées de l'enseignement traditionnel.

L'enthousiasme appelle toutefois à la prudence. La littérature sur les technologies éducatives en histoire souffre de limitations méthodologiques récurrentes : échantillons restreints, durées d'intervention courtes, effet de nouveauté non contrôlé, prédominance des mesures auto-rapportées \citep{veletsianos2014}. Les études en contexte écologique --- en classe réelle, avec des contraintes institutionnelles authentiques --- demeurent rares. La fluidité même qui rend ces agents attrayants pourrait constituer leur talon d'Achille pédagogique. Quand la technologie se fait oublier, l'agent cesse d'être perçu comme un outil. L'élève risque alors de lui accorder une confiance que l'interaction ne justifie pas.


\section{Problématique et Positionnement de la Recherche}
\label{sec:intro_problematique}

La convergence des grands modèles de langage et des médias synthétiques permet de créer des agents pédagogiques d'un réalisme saisissant. Ces agents combinent génération de discours en temps réel, incarnation visuelle et interaction orale naturelle. Cette configuration répond aux facteurs identifiés comme efficaces dans l'enseignement de l'histoire : interactivité, personnification, dimension narrative.

Les premières études sur ces dispositifs confirment l'intérêt des collégiens pour les agents incarnant des personnages historiques \citep{kim2025}. Des plateformes de simulation permettent d'explorer des environnements passés via des cartes interactives et de dialoguer avec des personnages générés par l'IA. Les enseignants peuvent créer des scénarios alignés sur leurs objectifs pédagogiques \citep{park2025}. Le paradigme CASA (\textit{Computers Are Social Actors}) éclaire ce phénomène : les individus appliquent spontanément aux technologies interactives les règles sociales qui régissent les relations humaines, créant un sentiment de présence qui transforme l'apprentissage en expérience relationnelle \citep{nass1994}.

Cette promesse se double cependant d'un risque que la recherche commence à peine à explorer. L'intérêt déclaré ne garantit pas l'apprentissage effectif. Un élève captivé par l'échange conversationnel peut surestimer ce qu'il en retire. La clarté apparente des explications fournies par l'agent, optimisées pour la lisibilité par les algorithmes sous-jacents, pourrait induire une confiance excessive dans sa propre compréhension. Ce phénomène est documenté en psychologie cognitive sous le terme d'illusion de compréhension \citep{rozenblit2002}. Il désigne la tendance à surestimer la profondeur de ses propres connaissances jusqu'à ce qu'une tâche d'explication révèle les lacunes.

Le mécanisme sous-jacent est l'heuristique de fluidité : une information facile à traiter est perçue comme familière, et cette facilité est attribuée par erreur à sa propre maîtrise du sujet plutôt qu'aux qualités de la présentation \citep{reber1999}. L'effet de fluidité de l'instructeur (\textit{instructor fluency}) illustre ce biais : les comportements non verbaux d'un enseignant dynamique et fluide biaisent les jugements d'apprentissage des élèves, qui confondent la qualité de la prestation avec la qualité de leur propre apprentissage \citep{toftness2018}. Les agents génératifs maximisent cette fluidité par leur capacité à produire un discours instantané, structuré et linguistiquement impeccable.

La présente recherche se positionne à l'intersection de ces enjeux. Elle relève du champ de l'Interaction Homme-Machine, mais s'ancre résolument dans le terrain éducatif. Son originalité tient à trois caractéristiques. Elle mobilise d'abord des protocoles expérimentaux contrôlés en contexte écologique --- des classes réelles, pendant les heures de cours, avec des contraintes institutionnelles authentiques. Elle examine ensuite non seulement les effets sur l'intérêt, mais aussi les risques métacognitifs associés à l'interaction avec des agents génératifs. Elle couvre enfin un spectre développemental étendu, de la sixième à la terminale, permettant d'observer comment l'âge module les effets observés.

Cette double focale --- engagement et vigilance critique --- répond à une lacune de la littérature. Les travaux existants privilégient les mesures d'engagement, de satisfaction, de motivation \citep{wu2024ai}. La question métacognitive demeure sous-explorée, alors même que les spécificités des LLM (génération fluide, hallucinations plausibles) la rendent particulièrement saillante. La recherche interroge donc cet équilibre : comment concevoir des agents qui engagent sans désactiver la conscience critique de l'outil ?

Trois axes structurent cette investigation. Le premier concerne la modalité d'interaction elle-même. Le cadre ICAP distingue les modes passif et interactif et prédit un avantage cognitif du second \citep{chi2014}. La théorie de l'agence sociale postule que les indices sociaux émis par un agent activent des réponses sociales favorisant un traitement plus approfondi \citep{moreno2001, mayer2012}. Ces cadres n'ont cependant pas été testés avec des agents génératifs déployés en classe. C'est cette lacune qui motive notre première question de recherche :

\textbf{QR1 : Dans quelle mesure l'interactivité orale directe avec un agent historique alimenté par l'IA influence-t-elle l'intérêt des élèves par rapport à une présentation vidéo passive ?}

Le deuxième axe porte sur les caractéristiques de l'agent et leur interaction avec l'âge des apprenants. La théorie du développement de l'intérêt distingue le déclenchement (suscité par la nouveauté) du maintien (entretenu par la pertinence personnelle) \citep{hidi2006, renninger2015-yq}. L'alignement thématique entre l'agent et le contenu pédagogique pourrait renforcer cette pertinence \citep{schmidt2019}. Or aucune étude ne teste l'incarnation d'un personnage historique comme variable d'alignement, ni ne mesure comment le stade développemental module cette réponse. Cette lacune motive notre deuxième question :

\textbf{QR2 : Comment l'alignement thématique de l'agent (personnage historique vs. neutre ; pair vs. autorité) et son style de présentation (formel vs. accessible) modulent-ils cet intérêt en fonction du stade développemental des élèves ?}

Le troisième axe déplace l'analyse vers le versant métacognitif. L'heuristique de fluidité et le paradigme CASA prédisent qu'une incarnation humanoïde renforce la crédibilité perçue et réduit la vigilance critique. Le protocole IOED (\textit{Illusion of Explanatory Depth}) permet de mesurer l'écart entre confiance subjective et performance objective \citep{rozenblit2002}. L'effet d'une incarnation visuelle --- humanoïde ou abstraite --- sur cette illusion n'a pas été testé en contexte d'interaction avec un agent génératif. Ce constat motive notre troisième question :

\textbf{QR3 : L'apparence de l'agent (humanoïde vs. abstrait) influence-t-elle la propension des élèves à l'illusion de compréhension, leur confiance et la crédibilité perçue des informations délivrées ?}


\section{Périmètre de la recherche}
\label{sec:intro_perimetre}

Il convient de circonscrire le périmètre de cette recherche afin d'en expliciter les limites et la portée. Nos investigations portent sur l'intérêt des élèves pour le dispositif pédagogique et le contenu disciplinaire, mesuré par des échelles auto-rapportées adaptées de l'IMI (\textit{Intrinsic Motivation Inventory}) \citep{deci1994}. L'apprentissage factuel --- la rétention de connaissances historiques --- ne fait pas partie des variables dépendantes de la série d'études 1. Ce choix est délibéré : il permet d'isoler l'effet motivationnel du dispositif avant de mesurer ses conséquences cognitives. L'étude 2, en revanche, intègre des mesures de connaissances et de confiance pour évaluer l'illusion de compréhension.

Par ailleurs, nos recherches se focalisent sur des agents incarnés, dotés d'une voix synthétique et d'une animation faciale. Les chatbots textuels, les assistants vocaux sans représentation visuelle et les environnements de réalité virtuelle immersive ne relèvent pas du périmètre de cette thèse. Ce choix reflète la spécificité de notre dispositif (la plateforme MemorIA) et la question de recherche centrale : l'effet de l'incarnation visuelle sur les processus motivationnels et métacognitifs.

L'ensemble des participants recrutés pour nos expérimentations est constitué d'élèves du secondaire, de la sixième à la terminale. Ce spectre développemental couvre l'adolescence dans sa globalité --- de 11 à 18 ans --- et correspond à la tranche d'âge où les effets des agents pédagogiques sont les plus documentés (10-14 ans selon les méta-analyses). Ce choix permet de tester le rôle modérateur de l'âge sur les variables d'intérêt et de vigilance critique. Les élèves du primaire et les populations adultes ne sont pas inclus. Cette restriction, bien qu'elle limite la généralisation des résultats, permet de maintenir une cohérence entre le public cible et le contenu disciplinaire : les programmes d'histoire au secondaire offrent un cadre curriculaire stable pour l'intégration du dispositif.


\section{Contributions de la Thèse}
\label{sec:intro_contributions}

Cette recherche apporte quatre types de contributions.

\textbf{Contribution empirique.} Elle fournit des données quantitatives issues de quatre études impliquant 458 élèves répartis sur quatre niveaux scolaires (sixième, cinquième, quatrième, terminale). Ces études ont été conduites en contexte écologique, intégrées au programme scolaire, avec des mesures pré/post et des designs expérimentaux contrôlés. Ce corpus constitue l'une des premières investigations systématiques des agents conversationnels historiques alimentés par LLM en classe réelle dans le contexte français. Les résultats éclairent l'effet de l'interactivité sur l'intérêt à différents stades développementaux, ainsi que l'influence du design de l'agent sur les processus métacognitifs.

\textbf{Contribution théorique.} Elle interroge les mécanismes psychologiques à l'œuvre dans l'interaction élève-agent à l'ère des modèles génératifs. Les résultats invitent à reconsidérer le paradigme CASA : la fluidité conversationnelle des LLM semble reconfigurer la hiérarchie des indices sociaux, la qualité du discours pouvant éclipser l'effet de l'incarnation visuelle. La thèse contribue également à la compréhension de l'illusion de compréhension dans les contextes d'apprentissage médiatisés par l'IA \citep{paik2013}. Elle articule ce biais métacognitif avec les spécificités techniques des agents génératifs (fluidité maximisée, absence de marqueurs d'incertitude).

\textbf{Contribution méthodologique.} Elle présente MemorIA, une architecture technique modulaire permettant le déploiement d'agents conversationnels historiques en milieu scolaire. Cette plateforme intègre reconnaissance vocale, génération de langage, synthèse vocale et animation faciale dans un pipeline temps réel. L'authenticité historique et le contrôle comportemental des agents constituent des préoccupations centrales pour ce type de système : l'exactitude factuelle doit guider le développement pour éviter une représentation simplifiée du passé \citep{daCosta2025}. La recherche propose également une adaptation du protocole IOED aux populations scolaires et aux contextes d'interaction avec agents génératifs, incluant des phases de production d'explications écrites et des mesures de confiance avant et après interaction.

\textbf{Contribution pratique.} Elle formule des recommandations pour la conception d'agents pédagogiques qui engagent les élèves sans désactiver leur vigilance critique. Les résultats suggèrent que l'interactivité constitue un levier d'engagement robuste, indépendant du niveau scolaire. L'effet de la représentation de l'agent apparaît plus nuancé, modulé par le développement cognitif et les attentes des élèves. Ces observations alimentent une réflexion sur l'équilibre entre engagement et conscience de l'outil, nécessaire à une intégration pédagogique responsable de ces technologies.


\section{Structure du Manuscrit}
\label{sec:intro_structure}

Le manuscrit s'organise en six chapitres.

Le \textbf{chapitre 2} établit le cadre théorique de la recherche. Il articule les théories de la motivation et de l'intérêt avec les spécificités de l'enseignement de l'histoire, puis examine les mécanismes de la présence sociale et de l'incarnation qui sous-tendent l'interaction avec un agent. L'analyse des capacités et des risques des IA génératives conduit à identifier la tension structurelle entre intérêt et vigilance critique, dont découlent les questions de recherche.

Le \textbf{chapitre 3} présente la plateforme MemorIA, conçue pour opérationnaliser ces questions en contexte scolaire réel. Il décrit les choix d'architecture (reconnaissance vocale, génération de langage, synthèse vocale, animation faciale) et les contraintes de déploiement en classe. La validation lors d'une phase pilote permet d'identifier les ajustements nécessaires et de préparer le protocole de l'évaluation expérimentale.

Le \textbf{chapitre 4} rapporte la série d'études 1, composée de trois expérimentations menées auprès d'élèves de 6\textsuperscript{e}, 4\textsuperscript{e} et terminale. Ces expérimentations testent l'effet de l'interactivité orale et de l'alignement thématique de l'agent sur l'intérêt des élèves (QR1 et QR2). La comparaison inter-études permet d'observer comment l'âge module les effets de l'interactivité et du personnage.

Le \textbf{chapitre 5} rapporte l'étude 2, qui déplace l'analyse vers le versant métacognitif. Elle examine l'influence du design visuel de l'agent (humanoïde vs. abstrait) sur l'illusion de compréhension chez des élèves de 5\textsuperscript{e} (QR3). Le protocole IOED adapté permet de mesurer l'écart entre confiance subjective et compréhension réelle, éclairant le revers potentiel de l'intérêt documenté au chapitre précédent.

Le \textbf{chapitre 6} conclut le manuscrit en articulant les résultats des deux études. Il confronte les bénéfices motivationnels de l'interactivité aux risques métacognitifs de la fluence, formule des recommandations pour la conception d'agents pédagogiques et ouvre des perspectives de recherche.
