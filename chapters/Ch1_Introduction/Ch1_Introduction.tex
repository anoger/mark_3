\chapter{Introduction Générale}
\label{ch:introduction}

\section{Contexte : L'Enseignement de l'Histoire face aux Nouveaux Médias}
\label{sec:intro_contexte}

L'enseignement de l'histoire au secondaire se heurte à une difficulté récurrente. Les élèves reconnaissent largement la pertinence de cette discipline pour comprendre le monde contemporain \citep{haydn2010pupil}. Interrogés sur l'utilité de l'histoire, ils évoquent spontanément la compréhension du présent, l'évitement des erreurs passées, la formation citoyenne. Cette reconnaissance coexiste pourtant avec une critique des modalités d'enseignement traditionnelles. Le cours magistral, les exercices de mémorisation, les évaluations écrites répétitives génèrent un désengagement que les enseignants constatent quotidiennement.

Les enquêtes comparatives entre disciplines éclairent cette tension. L'histoire se classe en position intermédiaire dans les préférences des élèves, devancée par les disciplines requérant une activité concrète : éducation physique, technologie, arts plastiques \citep{haydn2010pupil}. Ces matières partagent une caractéristique commune : elles impliquent une manipulation, une production tangible, un engagement corporel ou créatif. L'histoire, dans sa forme scolaire traditionnelle, peine à offrir cette dimension active. Les variations observées entre établissements suggèrent néanmoins que le problème n'est pas inhérent à la discipline. Les écarts d'appréciation atteignent trente-cinq points de pourcentage selon les contextes, soulignant l'effet déterminant des choix pédagogiques et de la posture enseignante \citep{haydn2010pupil}.

La comparaison avec les disciplines scientifiques et techniques révèle une asymétrie supplémentaire. Les STIM bénéficient d'une perception d'utilité professionnelle que l'histoire ne peut revendiquer avec la même évidence \citep{grever2011high}. Les débouchés des filières scientifiques apparaissent clairement aux élèves et à leurs familles. L'histoire semble cantonnée à l'enseignement, à l'archéologie, aux métiers du patrimoine --- des secteurs perçus comme moins attractifs économiquement. Cette hiérarchie implicite influence les choix d'orientation, malgré la transférabilité des compétences historiques : analyse critique des sources, argumentation nuancée, synthèse documentaire.

Les travaux en didactique de l'histoire convergent sur un point : les méthodes interactives transforment le rapport des élèves à la discipline. Les jeux de rôle, les débats structurés, les simulations historiques, l'analyse de sources primaires génèrent un engagement que le cours transmissif ne peut susciter \citep{vanstraaten2015making}. Ces approches activent l'empathie historique --- cette capacité à se projeter dans les perspectives du passé --- et établissent des connexions explicites entre les temporalités. La manière d'enseigner détermine l'engagement davantage que le contenu enseigné.

L'émergence des technologies d'intelligence artificielle générative s'inscrit dans ce contexte. Les grands modèles de langage produisent un discours fluide et contextuellement adapté. Ils ouvrent une possibilité inédite : converser avec des représentations de figures historiques. Les travaux empiriques sur les agents virtuels alimentés par l'IA rapportent des effets positifs sur l'apprentissage \citep{dai2024effects}. Une méta-analyse portant sur vingt-deux études établit un effet global favorable (g = 0.43). Les agents combinant apparence humanoïde et interactions structurées produisent les effets les plus marqués \citep{dai2024effects}. Ces données suggèrent que l'interactivité conversationnelle avec un agent incarné pourrait répondre au déficit d'engagement identifié dans l'enseignement de l'histoire.

Les technologies de synthèse vocale et d'animation faciale complètent ce dispositif. Elles incarnent visuellement et vocalement le personnage. Le \og témoin virtuel \fg{} du passé devient une réalité technique. Un élève peut poser des questions à Jules César et obtenir des réponses immédiates, formulées à la première personne. Cette possibilité suscite un enthousiasme compréhensible au regard des difficultés documentées de l'enseignement traditionnel.

L'enthousiasme appelle toutefois à la prudence. La littérature sur les technologies éducatives en histoire souffre de limitations méthodologiques récurrentes : échantillons restreints, durées d'intervention courtes, effet de nouveauté non contrôlé, prédominance des mesures auto-rapportées \citep{veletsianos2013pedagogical}. Les études en contexte écologique --- en classe réelle, avec des contraintes institutionnelles authentiques --- demeurent rares. La fluidité même qui rend ces agents attrayants pourrait constituer leur talon d'Achille pédagogique. Quand la technologie se fait oublier, l'agent cesse d'être perçu comme un outil. L'élève risque alors de lui accorder une confiance que l'interaction ne justifie pas.


\section{Problématique et Positionnement de la Recherche}
\label{sec:intro_problematique}

La convergence des grands modèles de langage et des médias synthétiques permet de créer des agents pédagogiques d'un réalisme saisissant. Ces agents combinent trois caractéristiques : une capacité à générer un discours adapté en temps réel, une incarnation visuelle potentiellement réaliste, et une interaction orale naturelle. Cette configuration technique répond aux facteurs identifiés comme efficaces dans l'enseignement de l'histoire : l'interactivité, la personnification, la dimension narrative et émotionnelle \citep{vanstraaten2015making}.

Les premières études sur ces dispositifs rapportent des effets positifs sur l'engagement et la motivation. Les collégiens montrent une réceptivité marquée aux agents conversationnels \citep{kim2025histochat}. L'interaction sociale pseudo-humaine active les mécanismes d'engagement social et de réciprocité. Les agents créent un sentiment de présence qui transforme l'apprentissage passif en expérience relationnelle. Le paradigme CASA (\textit{Computers Are Social Actors}) éclaire ce phénomène : les individus appliquent spontanément aux technologies interactives les règles sociales qui régissent les relations humaines \citep{nass1994computers}.

Cette promesse se double cependant d'un risque que la recherche commence à peine à explorer. L'intérêt déclaré ne garantit pas l'apprentissage effectif. Un élève captivé par l'échange conversationnel peut surestimer ce qu'il en retire. La clarté apparente des explications fournies par l'agent, optimisées pour la lisibilité par les algorithmes sous-jacents, pourrait induire une confiance excessive dans sa propre compréhension. Ce phénomène est documenté en psychologie cognitive sous le terme d'illusion de compréhension \citep{rozenblit2002misunderstood}. Il désigne la tendance à surestimer la profondeur de ses propres connaissances jusqu'à ce qu'une tâche d'explication révèle les lacunes.

Le mécanisme sous-jacent est l'heuristique de fluidité : une information facile à traiter est perçue comme familière, et cette facilité est attribuée par erreur à sa propre maîtrise du sujet plutôt qu'aux qualités de la présentation \citep{reber1999effects}. L'effet de fluidité de l'instructeur (\textit{instructor fluency}) illustre ce biais : les comportements non verbaux d'un enseignant dynamique et fluide biaisent les jugements d'apprentissage des élèves, qui confondent la qualité de la prestation avec la qualité de leur propre apprentissage \citep{toftness2018instructor}. Les agents génératifs maximisent cette fluidité par leur capacité à produire un discours instantané, structuré et linguistiquement impeccable.

La présente recherche se positionne à l'intersection de ces enjeux. Elle relève du champ de l'Interaction Homme-Machine, mais s'ancre résolument dans le terrain éducatif. Son originalité tient à trois caractéristiques. Elle mobilise d'abord des protocoles expérimentaux contrôlés en contexte écologique --- des classes réelles, pendant les heures de cours, avec des contraintes institutionnelles authentiques. Elle examine ensuite non seulement les effets sur l'intérêt, mais aussi les risques métacognitifs associés à l'interaction avec des agents génératifs. Elle couvre enfin un spectre développemental étendu, de la sixième à la terminale, permettant d'observer comment l'âge module les effets observés.

Cette double focale --- engagement et vigilance critique --- répond à une lacune de la littérature. Les travaux existants privilégient les mesures d'engagement, de satisfaction, de motivation \citep{wu2024ai}. La question métacognitive demeure sous-explorée, alors même que les spécificités des LLM (génération fluide, hallucinations plausibles) la rendent particulièrement saillante. La recherche interroge donc cet équilibre : comment concevoir des agents qui engagent sans désactiver la conscience critique de l'outil ?


\section{Questions de Recherche}
\label{sec:intro_questions}

Cette recherche s'articule autour de trois questions complémentaires, organisées selon une progression dialectique : de l'exploration des bénéfices potentiels vers l'examen des risques associés.

\textbf{QR1 : Dans quelle mesure l'interactivité orale directe avec un agent historique alimenté par l'IA influence-t-elle l'intérêt des élèves par rapport à une présentation vidéo passive ?}

Cette première question teste l'hypothèse issue de la littérature : l'interactivité constitue-t-elle un levier d'engagement supérieur aux formats passifs, même lorsque le contenu informationnel est équivalent ? Elle opérationnalise la distinction entre les modes \og passif \fg{} et \og interactif \fg{} du cadre ICAP \citep{chi2014icap} en contexte d'agent conversationnel. La théorie de l'agence sociale prédit que les indices sociaux émis par un agent virtuel activent des réponses sociales chez l'apprenant, favorisant un traitement cognitif plus approfondi \citep{moreno2001case, mayer2012role}.

\textbf{QR2 : Comment l'alignement thématique de l'agent (personnage historique vs. neutre ; pair vs. autorité) et son style de présentation (formel vs. accessible) modulent-ils cet intérêt en fonction du stade développemental des élèves ?}

Cette question explore les conditions aux limites de l'effet d'interactivité. Elle examine si les caractéristiques de l'agent --- son identité, son style communicationnel --- interagissent avec l'âge des apprenants pour produire des effets différenciés. L'alignement thématique entre l'agent et le contenu pédagogique pourrait renforcer l'expérience d'apprentissage en la rendant plus attractive et stimulante \citep{schmidt2019effects}. La théorie du développement de l'intérêt distingue l'intérêt situationnel déclenché (suscité par des éléments nouveaux ou surprenants) de l'intérêt situationnel maintenu (entretenu par la pertinence personnelle et l'engagement significatif) \citep{hidi2006four, renninger2015psychology}. L'interaction avec un personnage historique pourrait activer ces deux mécanismes différemment selon l'âge.

\textbf{QR3 : L'apparence de l'agent (humanoïde vs. abstrait) influence-t-elle la propension des élèves à l'illusion de compréhension, leur confiance et la crédibilité perçue des informations délivrées ?}

Cette troisième question déplace l'analyse du versant motivationnel vers le versant métacognitif. Elle interroge le revers potentiel de l'engagement : la fluidité de l'interaction conduit-elle les élèves à surestimer leur compréhension ? Le design visuel de l'agent amplifie-t-il ce risque ? Le protocole IOED (\textit{Illusion of Explanatory Depth}) permet de mesurer l'écart entre la confiance subjective et la performance objective \citep{rozenblit2002misunderstood}. L'hypothèse sous-jacente est que l'incarnation humanoïde, en renforçant la présence sociale et l'autorité perçue, pourrait amplifier l'illusion de compréhension par rapport à une représentation abstraite.


\section{Contributions de la Thèse}
\label{sec:intro_contributions}

Cette recherche apporte quatre types de contributions.

\textbf{Contribution empirique.} Elle fournit des données quantitatives issues de quatre études impliquant 458 élèves répartis sur quatre niveaux scolaires (sixième, cinquième, quatrième, terminale). Ces études ont été conduites en contexte écologique, intégrées au programme scolaire, avec des mesures pré/post et des designs expérimentaux contrôlés. Ce corpus constitue l'une des premières investigations systématiques des agents conversationnels historiques alimentés par LLM en classe réelle dans le contexte français. Les résultats éclairent l'effet de l'interactivité sur l'intérêt à différents stades développementaux, ainsi que l'influence du design de l'agent sur les processus métacognitifs.

\textbf{Contribution théorique.} Elle interroge les mécanismes psychologiques à l'œuvre dans l'interaction élève-agent à l'ère des modèles génératifs. Les résultats invitent à reconsidérer le paradigme CASA : la fluidité conversationnelle des LLM semble reconfigurer la hiérarchie des indices sociaux, la qualité du discours pouvant éclipser l'effet de l'incarnation visuelle. La thèse contribue également à la compréhension de l'illusion de compréhension dans les contextes d'apprentissage médiatisés par l'IA \citep{paik2013illusion}. Elle articule ce biais métacognitif avec les spécificités techniques des agents génératifs (fluidité maximisée, absence de marqueurs d'incertitude).

\textbf{Contribution méthodologique.} Elle présente MemorIA, une architecture technique modulaire permettant le déploiement d'agents conversationnels historiques en milieu scolaire. Cette plateforme intègre reconnaissance vocale, génération de langage, synthèse vocale et animation faciale dans un pipeline temps réel. Elle propose également une adaptation du protocole IOED aux populations scolaires et aux contextes d'interaction avec agents génératifs. Cette adaptation inclut des phases de production d'explications écrites et des mesures de confiance avant et après interaction.

\textbf{Contribution pratique.} Elle formule des recommandations pour la conception d'agents pédagogiques qui engagent les élèves sans désactiver leur vigilance critique. Les résultats suggèrent que l'interactivité constitue un levier d'engagement robuste, indépendant du niveau scolaire. L'effet de la représentation de l'agent apparaît plus nuancé, modulé par le développement cognitif et les attentes des élèves. Ces observations alimentent une réflexion sur l'équilibre entre engagement et conscience de l'outil, nécessaire à une intégration pédagogique responsable de ces technologies.


\section{Structure du Manuscrit}
\label{sec:intro_structure}

Le manuscrit s'organise en six chapitres.

Le \textbf{chapitre 2} établit le cadre théorique de la recherche. Il articule les théories de la motivation et de l'intérêt (théorie de l'autodétermination, modèle du développement de l'intérêt), les spécificités de l'enseignement de l'histoire (comparaison avec les STIM, perception des élèves, pratiques pédagogiques), le paradigme CASA et la théorie de l'agence sociale, les mécanismes de l'illusion de compréhension, pour aboutir à la formulation des hypothèses.

Le \textbf{chapitre 3} présente la plateforme expérimentale MemorIA : son architecture technique, ses modules (reconnaissance vocale, génération de langage, synthèse vocale, animation faciale), et sa validation lors d'une étude pilote. Il expose les contraintes de conception liées au déploiement en milieu scolaire et les choix techniques effectués.

Le \textbf{chapitre 4} expose l'Étude 1, une série de trois expérimentations examinant l'impact de l'interactivité et de la représentation de l'agent sur l'intérêt des élèves, à trois niveaux scolaires distincts (sixième, quatrième, terminale). Chaque expérimentation est présentée avec sa méthodologie, ses résultats et sa discussion spécifique. Une synthèse croisée conclut le chapitre.

Le \textbf{chapitre 5} présente l'Étude 2, qui examine l'influence du design visuel de l'agent (humanoïde vs. abstrait) sur l'illusion de compréhension chez des élèves de cinquième. Le protocole IOED adapté est détaillé, ainsi que les mesures complémentaires de crédibilité, confiance et anthropomorphisme perçu.

Le \textbf{chapitre 6} conclut le manuscrit en résumant les contributions et en ouvrant des perspectives de recherche future, notamment sur les études longitudinales, les modalités alternatives d'interaction, et les questions éthiques soulevées par l'utilisation d'agents IA personnifiant des figures historiques.
