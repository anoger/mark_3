% Section Résultats à ajouter après les Hypothèses de l'Exp 1.2

\subsection{Résultats}
\label{subsec:exp12-resultats}

\subsubsection{Analyse de l'Intérêt des Élèves de 6\textsuperscript{ème}}

Les effets de l'interactivité et de l'alignement du personnage sur les trois dimensions d'intérêt sont illustrés dans les Figures~\ref{fig:caesar_post_act}, \ref{fig:caesar_post_les}, et \ref{fig:caesar_post_char}. Les ANCOVA n'ont révélé aucun effet d'interaction significatif entre l'interactivité et l'alignement du personnage pour Post\_Act (F(1, 108) = 2,505, p = ,116, $\eta^2$ = ,011), Post\_Les (F(1, 108) = 3,027, p = ,085, $\eta^2$ = ,011), ou Post\_Char (F(1, 108) = 2,043, p = ,156, $\eta^2$ = ,008).

Pour l'intérêt pour l'activité d'apprentissage (Post\_Act ; voir Figure~\ref{fig:caesar_post_act}), un effet principal de l'interactivité a été observé (F(1, 108) = 76,58, p < ,001, $\eta^2$ = ,330), avec des participants dans les conditions interactives rapportant un intérêt plus élevé (M = 6,449, ET = 0,841 pour le personnage historique ; M = 5,949, ET = 1,531 pour le personnage neutre). Un effet principal de l'alignement du personnage a également été trouvé (F(1, 108) = 11,425, p = ,001, $\eta^2$ = ,049). Les élèves dans les conditions vidéo ont montré un intérêt plus faible (M = 4,704, ET = 1,760 pour le personnage historique ; M = 3,161, ET = 1,599 pour le personnage neutre).

\begin{figure}[htbp]
    \centering
    \includegraphics[width=0.8\textwidth]{images/ch4/caesar_post_act.png}
    \caption{Intérêt des élèves de 6\textsuperscript{ème} pour l'activité d'apprentissage (Post\_Act) en fonction de l'interactivité et de l'alignement du personnage}
    \label{fig:caesar_post_act}
\end{figure}

Pour l'intérêt pour le contenu de la leçon (Post\_Les, voir Figure~\ref{fig:caesar_post_les}), un effet principal de l'interactivité a été trouvé (F(1, 108) = 52,118, p < ,001, $\eta^2$ = ,187). Les conditions interactives ont produit un intérêt plus élevé (M = 5,610, ET = 1,330 pour le personnage historique ; M = 5,294, ET = 1,419 pour le personnage neutre). Un effet principal de l'alignement du personnage a également été observé (F(1, 108) = 15,603, p < ,001, $\eta^2$ = ,056). Les conditions vidéo ont montré un intérêt plus faible (M = 4,828, ET = 1,265 pour le personnage historique ; M = 3,559, ET = 1,284 pour le personnage neutre).

\begin{figure}[htbp]
    \centering
    \includegraphics[width=0.8\textwidth]{images/ch4/caesar_post_les.png}
    \caption{Intérêt des élèves de 6\textsuperscript{ème} pour le contenu de la leçon (Post\_Les) en fonction de l'interactivité et de l'alignement du personnage}
    \label{fig:caesar_post_les}
\end{figure}

Concernant l'intérêt pour le personnage historique (Post\_Char, voir Figure~\ref{fig:caesar_post_char}), un effet principal de l'interactivité a été trouvé (F(1, 108) = 47,261, p < ,001, $\eta^2$ = ,174). Les conditions interactives ont montré un intérêt plus élevé (M = 5,712, ET = 1,178 pour le personnage historique ; M = 5,462, ET = 1,387 pour le personnage neutre). Un effet principal de l'alignement du personnage a également été observé (F(1, 108) = 10,048, p = ,002, $\eta^2$ = ,037). Les conditions vidéo ont montré un intérêt plus faible (M = 4,543, ET = 1,820 pour le personnage historique ; M = 3,419, ET = 1,481 pour le personnage neutre).

\begin{figure}[htbp]
    \centering
    \includegraphics[width=0.8\textwidth]{images/ch4/caesar_post_char.png}
    \caption{Intérêt des élèves de 6\textsuperscript{ème} pour le personnage historique (Post\_Char) en fonction de l'interactivité et de l'alignement du personnage}
    \label{fig:caesar_post_char}
\end{figure}
