% Section Résultats à ajouter après les Hypothèses de l'Exp 1.3

\subsection{Résultats}
\label{subsec:exp13-resultats}

\subsubsection{Comparabilité des Groupes}

Les analyses préliminaires ont montré une différence significative dans l'intérêt pré-intervention pour l'activité (Pre\_Act : p = ,037) entre les groupes, tandis que les autres mesures pré-intervention n'ont montré aucune différence significative (Pre\_Les : p = ,299, Pre\_Char : p = ,428). Cette différence initiale a été prise en compte dans les analyses subséquentes à travers l'utilisation des mesures pré-intervention comme covariables.

\subsubsection{Analyse de l'Intérêt}

Les ANCOVA n'ont révélé aucun effet significatif du type de personnage sur l'intérêt pour l'activité (Post\_Act ; voir Figure~\ref{fig:degaulle_louis_post_act} ; F(1, 110) = 0,057, p = ,812, $\eta^2$ = ,0003). L'intérêt pré-intervention (Pre\_Act) est apparu comme un prédicteur significatif (F(1, 110) = 78,359, p < ,001, $\eta^2$ = ,408). Les élèves ont montré des niveaux d'intérêt similaires à travers les conditions (M = 5,117, ET = 1,419 pour le personnage pair ; M = 5,519, ET = 1,393 pour le personnage historique).

\begin{figure}[htbp]
    \centering
    \includegraphics[width=0.8\textwidth]{images/ch4/degaulle_louis_post_act.png}
    \caption{Intérêt des lycéens de Terminale pour l'activité (Post\_Act) en fonction du type de personnage. Les barres d'erreur représentent les écarts-types}
    \label{fig:degaulle_louis_post_act}
\end{figure}

Pour le contenu de la leçon (Post\_Les ; voir Figure~\ref{fig:degaulle_louis_post_les}), l'analyse n'a montré aucun effet significatif du type de personnage (F(1, 110) = 1,330, p = ,251, $\eta^2$ = ,005), avec l'intérêt pré-intervention (Pre\_Les) montrant un effet significatif (F(1, 110) = 154,760, p < ,001, $\eta^2$ = ,584). Les niveaux d'intérêt moyens étaient similaires entre les conditions (M = 5,407, ET = 1,332 pour le personnage pair ; M = 5,435, ET = 1,368 pour le personnage historique).

\begin{figure}[htbp]
    \centering
    \includegraphics[width=0.8\textwidth]{images/ch4/degaulle_louis_post_les.png}
    \caption{Intérêt des lycéens de Terminale pour le contenu de la leçon (Post\_Les) en fonction du type de personnage. Les barres d'erreur représentent les écarts-types}
    \label{fig:degaulle_louis_post_les}
\end{figure}

Pour l'intérêt pour le personnage (Post\_Char ; voir Figure~\ref{fig:degaulle_louis_post_char}), aucun effet significatif du type de personnage n'a été observé (F(1, 110) = 0,555, p = ,458, $\eta^2$ = ,002), tandis que l'intérêt pré-intervention était un prédicteur significatif (F(1, 110) = 126,043, p < ,001, $\eta^2$ = ,532). Les niveaux d'intérêt sont restés comparables entre les conditions (M = 4,969, ET = 1,509 pour le personnage pair ; M = 4,774, ET = 1,382 pour le personnage historique).

\begin{figure}[htbp]
    \centering
    \includegraphics[width=0.8\textwidth]{images/ch4/degaulle_louis_post_char.png}
    \caption{Intérêt des lycéens de Terminale pour le personnage (Post\_Char) en fonction du type de personnage. Les barres d'erreur représentent les écarts-types}
    \label{fig:degaulle_louis_post_char}
\end{figure}
