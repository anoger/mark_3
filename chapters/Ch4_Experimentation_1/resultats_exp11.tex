% Section Résultats à ajouter après les Hypothèses de l'Exp 1.1

\subsection{Résultats}
\label{subsec:exp11-resultats}

Nous avons conduit des analyses de covariance bidirectionnelles (ANCOVA) pour chaque variable dépendante (intérêt pour l'activité d'apprentissage [Post\_Act], intérêt pour le contenu de la leçon [Post\_Les], et intérêt pour le personnage historique [Post\_Char]), avec l'interactivité et l'alignement du personnage comme variables indépendantes et les mesures pré-intervention correspondantes comme covariables. Cette approche nous a permis de contrôler les différences préexistantes entre les participants tout en testant nos hypothèses sur les effets de l'interactivité et de l'alignement du personnage sur l'intérêt des élèves. Des analyses préliminaires ont été conduites pour vérifier que les hypothèses de l'ANCOVA étaient satisfaites, et des corrections ont été appliquées lorsque nécessaire pour ajuster les violations d'homogénéité de variance. Les données qualitatives issues des discussions de groupe post-intervention ont complété nos analyses quantitatives, fournissant une compréhension plus complète des mécanismes psychologiques sous-jacents aux effets observés.

Avant d'analyser les effets de nos variables indépendantes, nous avons vérifié la comparabilité initiale des groupes expérimentaux. Des tests de Mann-Whitney ont comparé les scores d'intérêt initiaux entre les conditions d'interactivité (Vidéo vs. Interactive) et d'alignement du personnage (Neutre vs. Historique). Aucune différence significative n'a été trouvée entre les groupes Vidéo et Interactif, ni entre les groupes Neutre et Historique, démontrant une bonne comparabilité initiale. Ces résultats supportent la validité de nos analyses subséquentes, car ils démontrent que les effets observés ne peuvent être attribués à des différences préexistantes entre les groupes.

\subsubsection{Analyses Principales}

Aucun effet d'interaction significatif entre l'interactivité et l'alignement du personnage n'a été observé pour aucune des mesures d'intérêt (Post\_Act : F(1, 108) = 1,399, p = ,240, $\eta^2$ = ,010 ; Post\_Les : F(1, 108) = 0,457, p = ,501, $\eta^2$ = ,002 ; Post\_Char : F(1, 108) = 1,336, p = ,250, $\eta^2$ = ,007). 

Un effet principal de l'interactivité a été observé sur l'intérêt pour l'activité d'apprentissage (Post\_Act ; F(1, 108) = 23,547, p < ,001, $\eta^2$ = ,166), tandis que l'alignement du personnage n'a montré aucun effet significatif (F(1, 108) = 2,269, p = ,135, $\eta^2$ = ,016). Les élèves dans les conditions interactives ont rapporté un intérêt plus élevé (M = 5,699, ET = 1,133 pour le personnage historique ; M = 5,844, ET = 1,176 pour le personnage neutre) comparé à ceux dans les conditions vidéo (M = 4,226, ET = 1,513 pour le personnage historique ; M = 4,960, ET = 1,338 pour le personnage neutre).

\begin{figure}[htbp]
    \centering
    \includegraphics[width=0.8\textwidth]{images/ch4/napoleon_post_act.png}
    \caption{Intérêt des élèves de 4\textsuperscript{ème} pour l'activité d'apprentissage (Post\_Act) en fonction de l'interactivité et de l'alignement du personnage. Les barres d'erreur représentent les écarts-types}
    \label{fig:napoleon_post_act}
\end{figure}

Pour l'intérêt pour le contenu de la leçon (Post\_Les), un effet principal de l'interactivité a été trouvé (F(1, 108) = 11,231, p = ,001, $\eta^2$ = ,052), sans effet significatif de l'alignement du personnage (F(1, 108) = 1,377, p = ,243, $\eta^2$ = ,006). Les scores d'intérêt étaient plus élevés dans les conditions interactives (M = 5,455, ET = 1,218 pour le personnage neutre ; M = 4,789, ET = 1,198 pour le personnage historique) comparé aux conditions vidéo (M = 4,500, ET = 1,375 pour le personnage historique ; M = 5,160, ET = 1,265 pour le personnage neutre).

De manière similaire, pour l'intérêt pour le personnage historique (Post\_Char), l'analyse a révélé un effet principal de l'interactivité (F(1, 108) = 5,901, p = ,017, $\eta^2$ = ,030), sans effet significatif de l'alignement du personnage (F(1, 108) = 0,337, p = ,563, $\eta^2$ = ,002). Les conditions interactives ont montré un intérêt plus élevé (M = 5,655, ET = 1,153 pour le personnage neutre ; M = 5,033, ET = 1,198 pour le personnage historique) comparé aux conditions vidéo (M = 4,904, ET = 1,323 et M = 4,906, ET = 1,346 respectivement).

\begin{figure}[htbp]
    \centering
    \includegraphics[width=0.8\textwidth]{images/ch4/napoleon_post_char.png}
    \caption{Intérêt des élèves de 4\textsuperscript{ème} pour le personnage historique (Post\_Char) en fonction de l'interactivité et de l'alignement du personnage. Les barres d'erreur représentent les écarts-types}
    \label{fig:napoleon_post_char}
\end{figure}
