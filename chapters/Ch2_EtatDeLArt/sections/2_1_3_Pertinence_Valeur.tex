% ============================================================================
% SOUS-SECTION 2.1.3 - La Théorie de la Pertinence et de la Valeur
% ============================================================================

\subsection{La Théorie de la Pertinence et de la Valeur}

La décision de s'engager dans une tâche d'apprentissage repose sur deux évaluations distinctes mais interdépendantes : l'attente de succès et la valeur attribuée à la tâche \citep{EcclesWigfield2002}. La théorie Expectancy-Value postule qu'un individu choisit de persister dans une activité exigeante lorsqu'il estime à la fois pouvoir y réussir et que cette activité présente une valeur suffisante pour justifier l'investissement requis. Ces deux composantes interagissent de manière multiplicative : une forte attente de succès ne compense pas l'absence de valeur perçue, et inversement, une tâche hautement valorisée mais perçue comme inaccessible ne suscitera pas d'engagement durable. Les croyances relatives à soi-même, incluant le sentiment de compétence et l'auto-efficacité, déterminent la composante expectation, tandis que les croyances relatives à la tâche déterminent la composante valeur \citep{EcclesWigfield2002}. Ce cadre théorique permet de prédire les choix de cours, la persistence dans une filière et les orientations professionnelles, en particulier dans les domaines scientifiques où les abandons précoces constituent un phénomène documenté.

La valeur subjective attribuée à une tâche se décompose en quatre dimensions distinctes \citep{WigfieldEccles1992}. La valeur intrinsèque correspond au plaisir inhérent à l'exécution de la tâche, indépendamment de ses conséquences. La valeur d'accomplissement renvoie à l'importance de la tâche pour l'identité de l'individu : réussir dans ce domaine confirme une facette valorisée du soi. La valeur d'utilité désigne la perception de l'utilité de la tâche pour atteindre des objectifs futurs pertinents pour la vie de l'individu \citep{Harackiewicz2014}. Le coût, enfin, représente les aspects négatifs associés à l'engagement dans la tâche : le temps requis, l'effort cognitif, l'anxiété de performance ou le renoncement à d'autres activités valorisées. Ces quatre dimensions ne sont pas mutuellement exclusives : un même apprenant peut simultanément apprécier une activité pour elle-même, y voir un enjeu identitaire, la considérer comme utile pour ses projets et en percevoir le coût. La configuration relative de ces dimensions varie selon les individus et les contextes, ce qui explique en partie la diversité des patterns d'engagement observés face à un même contenu d'apprentissage.

Parmi ces quatre dimensions, la valeur d'utilité présente une caractéristique particulière : elle repose sur la perception de connexions entre la tâche immédiate et des activités, objectifs ou contextes futurs, ce qui la rend particulièrement susceptible d'intervention externe \citep{Harackiewicz2014}. Un enseignant ou un parent peut expliciter ces connexions, aider l'apprenant à percevoir la pertinence d'un contenu pour ses projets personnels ou professionnels. Cette malléabilité distingue la valeur d'utilité des autres formes de valeur, plus directement liées aux caractéristiques intrinsèques de la tâche ou à l'histoire personnelle de l'individu. Les interventions visant à promouvoir la perception de valeur d'utilité ont démontré des effets positifs sur l'intérêt et la persistence, en particulier lorsque les apprenants génèrent eux-mêmes les connexions plutôt que de les recevoir passivement \citep{HullemanHarackiewicz2009}. Cette auto-génération des liens entre le contenu et la vie personnelle s'avère particulièrement efficace pour les apprenants initialement moins confiants dans leurs capacités, tandis que la simple présentation d'informations sur l'utilité bénéficie davantage aux apprenants déjà intéressés \citep{Harackiewicz2014}.

La pertinence personnelle constitue le mécanisme psychologique sous-jacent à l'efficacité de ces interventions sur la valeur d'utilité \citep{AlbrechtKarabenick2017}. Rendre un contenu pertinent implique d'établir des connexions significatives entre ce contenu et les expériences, les objectifs ou l'identité de l'apprenant \citep{Priniski2018}. Cette conception rejoint la notion de "psychologisation" du curriculum proposée par Dewey : transformer le contenu disciplinaire en l'ancrant dans l'expérience immédiate et les préoccupations actuelles de l'apprenant, de sorte qu'un obstacle intellectuel présent crée le besoin d'acquérir la connaissance en question \citep{AlbrechtKarabenick2017}. La distinction entre pertinence personnelle, affectivement orientée et liée aux intérêts et à l'identité, et pertinence impersonnelle, cognitivement orientée et liée aux applications pratiques, suggère que les voies vers l'engagement peuvent varier selon les individus \citep{AlbrechtKarabenick2017}. Pour l'enseignement de l'histoire, discipline souvent perçue comme déconnectée du quotidien des élèves, ces considérations soulèvent la question des moyens par lesquels le contenu historique peut être rendu personnellement significatif.

% ============================================================================
% RÉFÉRENCES UTILISÉES DANS CETTE SOUS-SECTION :
% - Eccles & Wigfield (2002) : Expectancy-Value Theory, interaction des composantes
% - Wigfield & Eccles (1992) : quatre types de valeur subjective
% - Harackiewicz et al. (2014) : utility value, interventions, auto-génération
% - Hulleman & Harackiewicz (2009) : self-generated utility value
% - Albrecht & Karabenick (2017) : pertinence personnelle, Dewey, psychologisation
% - Priniski et al. (2018) : connexions significatives, identité
% ============================================================================
