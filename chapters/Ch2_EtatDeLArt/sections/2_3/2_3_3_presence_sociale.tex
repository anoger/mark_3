% ============================================================================
% Sous-section 2.3.3 : De l'Artefact au Partenaire — Présence Sociale
% ============================================================================
% Sources : IJCCI_extraction (§2.5), illusion_extraction, Protocole CER
% Calibrage : ~600 mots
% Type : Révision fusionnant 2_3_2_CASA.tex et 2_3_3_agence_sociale.tex
% ============================================================================

\subsection{De l'Artefact au Partenaire~: Mécanismes de la Présence Sociale}
\label{subsec:presence_sociale}

La section précédente a établi les contraintes cognitives qui encadrent l'efficacité des agents. Mais l'impact d'un agent ne se réduit pas à sa fonction de canal d'information~: il dépend aussi de sa capacité à être perçu comme un partenaire social. Cette perception transforme l'interaction technique en expérience relationnelle.

Le paradigme CASA (\textit{Computers Are Social Actors}) constitue le cadre explicatif central de ce phénomène~\citep{nass2000machines}. Les travaux fondateurs de Nass et ses collaborateurs ont démontré que les utilisateurs appliquent inconsciemment aux ordinateurs les règles sociales qu'ils utiliseraient avec des humains~: politesse, réciprocité, attribution de personnalité. Ce traitement social tend à s'opérer même lorsque l'utilisateur \textit{sait} qu'il interagit avec une machine --- un processus qualifié d'\textit{ethopoeia}.

Des indices sociaux minimaux suffisent à déclencher ces scripts relationnels. Une voix, qu'elle soit synthétique ou humaine, active des réponses sociales. Les utilisateurs évaluent différemment un ordinateur selon le genre de sa voix, reproduisant les stéréotypes sociaux~: une voix masculine est perçue comme plus compétente sur les sujets techniques. Ces attributions automatiques expliquent pourquoi même des agents rudimentaires peuvent générer un engagement significatif.

La présence sociale désigne le sentiment d'être \og avec\fg{} une autre intelligence dans un environnement médiatisé~\citep{biocca2003toward}. Ce sentiment ne requiert pas un interlocuteur humain~; il peut émerger de l'interaction avec un agent virtuel pourvu que certaines conditions soient réunies. L'utilisateur doit percevoir l'agent non comme un outil passif, mais comme une entité dotée d'une forme d'intentionnalité et de réactivité.

Cette présence sociale constitue la condition préalable à l'activation des mécanismes d'apprentissage social. Sans elle, l'interaction tend à rester instrumentale~: l'utilisateur traite l'information sans s'engager pleinement dans la relation. Avec elle, l'agent devient un partenaire dont on respecte implicitement le \og contrat de communication\fg{}.

La théorie de l'agence sociale (\textit{Social Agency Theory}) explicite le lien entre présence sociale et apprentissage~\citep{moreno2001case, mayer2012embodiment}. Lorsque l'apprenant perçoit l'agent comme un partenaire social, un contrat implicite s'établit~: l'apprenant s'engage à \og honorer\fg{} cette relation par un effort cognitif accru. Cet effort supplémentaire se traduit par un traitement plus profond du contenu.

Les méta-analyses confirment ce mécanisme~: les agents anthropomorphisés produisent des effets supérieurs aux agents non anthropomorphisés, mais ces effets concernent davantage les mesures affectives et motivationnelles que les mesures cognitives pures~\citep{schroeder2025designing, dai2022meta}.

