% ============================================================================
% Sous-section 2.3.1 : Cartographie des Agents Pédagogiques Virtuels
% ============================================================================
% Sources : Rapport Analyse Research Paper Mapper (17 méta-analyses)
% Calibrage : ~800 mots
% Type : NOUVEAU
% ============================================================================

\subsection{Cartographie des Agents Pédagogiques~: 25 Ans d'Évolution}
\label{subsec:cartographie_agents}

L'histoire des agents pédagogiques virtuels peut être structurée en trois générations distinctes, chacune caractérisée par des paradigmes technologiques et des objectifs de recherche spécifiques~\citep{johnson2016face, alfaro2020new}.

La \textbf{première génération (1997--2011)} correspond à l'ère des pionniers. Les premiers agents pédagogiques animés émergent à la fin des années 1990 avec des systèmes comme Steve (\textit{Soar Training Expert for Virtual Environments})~\citep{johnson1998steve}, agent 3D guidant les marins dans l'apprentissage de procédures techniques, ou Herman the Bug~\citep{lester1997lifelike}, agent 2D anthropomorphisé sous forme d'insecte qui a démontré l'\textit{effet de persona}~: sa seule présence améliorait la motivation des apprenants, indépendamment de la qualité du feedback~\citep{heidig2011pedagogical}. AutoTutor~\citep{graesser2004autotutor} a introduit le dialogue socratique automatisé, posant les bases des agents conversationnels éducatifs. Ces systèmes pionniers reposaient sur des scripts prédéfinis et des arbres de décision, ce qui limitait leur flexibilité conversationnelle.

La \textbf{deuxième génération (2012--2019)} se caractérise par la consolidation empirique. Les méta-analyses se multiplient pour évaluer l'efficacité réelle des agents selon différentes variables de design~: effets de l'affect, impact des gestes, comparaisons entre apparences 2D et 3D, voix humaines et synthétiques, niveaux d'anthropomorphisme~\citep{guo2015affect, davis2018impact}. Un constat émerge de ces synthèses~: sur 15 études avec groupe contrôle, 9 ne montrent aucune différence d'apprentissage significative, appelant à une approche plus rigoureuse~\citep{heidig2011pedagogical}. Cette génération établit les bases empiriques du domaine, mais les agents restent contraints par leurs scripts préprogrammés.

La \textbf{troisième génération (2020--présent)} correspond à l'ère générative. L'émergence des grands modèles de langage transforme radicalement le paysage~: les agents peuvent désormais générer des réponses contextuellement appropriées sans scripts prédéfinis. Les chatbots IA en éducation produisent des effets positifs sur les résultats d'apprentissage, mais cette flexibilité conversationnelle accrue s'accompagne de nouveaux défis liés à l'imprévisibilité des réponses~\citep{wu2024chatbots, schroeder2025designing}. Cette génération, à laquelle appartient notre agent historique, hérite des acquis des générations précédentes tout en introduisant des risques inédits --- notamment celui des hallucinations analysé en section~\ref{sec:IA_generatives}.

Au-delà de la chronologie, les agents se différencient selon plusieurs dimensions de design. Le tableau~\ref{tab:typologie_agents} propose une typologie fondée sur les caractéristiques empiriquement étudiées dans la littérature, synthétisée à partir de 17 méta-analyses et revues systématiques.

\begin{table}[htbp]
\centering
\caption{Typologie des agents pédagogiques selon leurs caractéristiques de design. Synthèse établie à partir des méta-analyses de \citet{castroalonso2021effectiveness}, \citet{dai2024effects} et \citet{martha2018design}.}
\label{tab:typologie_agents}
\small
\begin{tabular}{p{2.8cm}p{4cm}p{4cm}p{3cm}}
\toprule
\textbf{Dimension} & \textbf{Modalités} & \textbf{Exemples} & \textbf{Effet sur apprentissage} \\
\midrule
\textit{Représentation visuelle} &
Absent / Statique / Animé 2D / Animé 3D &
Texte seul / Image fixe / Cartoon / Avatar 3D &
Animé > Statique~; 2D $\geq$ 3D \\
\addlinespace
\textit{Niveau d'anthropomorphisme} &
Non-humain / Stylisé / Réaliste / Hyperréaliste &
Robot / Cartoon / Avatar / Deepfake &
Stylisé souvent optimal \\
\addlinespace
\textit{Modalité vocale} &
Texte / Voix synthétique / Voix humaine &
Chat / TTS / Enregistrement &
Humaine > Synthétique \\
\addlinespace
\textit{Rôle pédagogique} &
Tuteur / Compagnon / Teachable agent &
Expert / Assistant / Élève virtuel &
Dépend du contexte \\
\addlinespace
\textit{Moteur conversationnel} &
Script / Règles / NLP-ML / LLM &
Arbre décision / AIML / Seq2Seq / GPT &
LLM~: flexibilité $\uparrow$, contrôle $\downarrow$ \\
\bottomrule
\end{tabular}
\end{table}

L'application des agents pédagogiques à l'enseignement de l'Histoire reste peu explorée et les résultats disponibles sont préoccupants~: l'effet mesuré pour l'histoire est négatif, contre des effets positifs en biologie et en informatique~\citep{castroalonso2021effectiveness}. Cette disparité disciplinaire s'explique par un défi épistémologique spécifique~: contrairement aux domaines STIM où l'agent peut s'appuyer sur des connaissances vérifiables, l'agent historique doit naviguer entre faits documentés, interprétations historiographiques et zones d'incertitude. Cette spécificité justifie une attention particulière aux risques métacognitifs que nous analysons dans cette thèse.

La figure~\ref{fig:evolution_agents} illustre cette évolution à travers deux exemples représentatifs des deux premières générations.

\begin{figure}[htbp]
\centering
% Placeholder pour figure composite
\fbox{\parbox{0.9\textwidth}{\centering\vspace{2cm}
\textit{Figure à insérer~: Deux agents représentatifs}\\[0.5em]
(a) Steve~\citep{johnson1998steve} --- Génération 1\\
(b) Agent expressif de \citet{torre2019} --- Génération 2
\vspace{2cm}}}
\caption{Évolution des agents pédagogiques à travers les deux premières générations. (a)~Steve, agent 3D pionnier pour l'entraînement naval~\citep{johnson1998steve}. (b)~Agent expressif utilisé dans les études sur l'alignement comportemental et contextuel~\citep{torre2019}.}
\label{fig:evolution_agents}
\end{figure}

