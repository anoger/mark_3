\subsection{Architecture cognitive et apprentissage multimédia}
\label{subsec:architecture_cognitive}

Tout agent pédagogique opère dans les limites de la mémoire de travail. La théorie de la charge cognitive identifie cette contrainte : la mémoire de travail ne traite qu'un nombre restreint d'éléments simultanément \citep{sweller1988}. Trois types de charge se partagent cette capacité. La charge intrinsèque dépend de la complexité du contenu et du niveau de l'apprenant. La charge extrinsèque résulte d'un design inadapté : informations redondantes, navigation confuse, éléments visuels non pertinents. La charge pertinente (\textit{germane load}) correspond à l'effort consacré à la construction de schémas mentaux \citep{sweller2011}. Ces trois charges ne sont pas indépendantes : réduire la charge extrinsèque libère des ressources pour la charge pertinente. L'enjeu du design d'un agent pédagogique se formule dans ces termes : minimiser la charge extrinsèque pour maximiser les ressources disponibles à l'apprentissage.

La théorie cognitive de l'apprentissage multimédia (CTML) prolonge ce cadre en précisant comment l'information est traitée quand elle combine texte, image et son \citep{mayer2020}. Trois hypothèses structurent le modèle. L'information entre par deux canaux distincts (visuel et auditif). Chaque canal a une capacité limitée. L'apprentissage requiert un traitement actif : sélection, organisation et intégration de l'information avec les connaissances antérieures. De ces hypothèses découlent des principes de design applicables aux agents pédagogiques. Le principe de modalité établit qu'une narration orale accompagnant une image produit un meilleur apprentissage qu'un texte écrit accompagnant la même image. La narration répartit la charge entre les deux canaux au lieu de surcharger le seul canal visuel. Le principe de redondance montre, à l'inverse, que présenter simultanément un texte écrit et une narration identique tend à dégrader la performance. Le canal visuel traite alors à la fois le texte et l'image, ce qui génère une charge extrinsèque. Ces principes reposent cependant sur des conditions expérimentales précises. Ils supposent un contenu technique accompagné de schémas, un apprenant novice et une présentation linéaire. Leur transposition aux interactions conversationnelles, où l'apprenant contrôle le rythme et le contenu des échanges, n'est pas directement validée. Le principe de modalité fournit néanmoins la justification technique des agents oraux : un agent qui parle libère le canal visuel pour les contenus graphiques que l'apprenant doit analyser --- cartes, schémas, documents historiques \citep{mayer2014-ju}.

Ces principes traitent l'apprenant comme un système de traitement de l'information. L'optimisation des canaux cognitifs explique \textit{comment} l'information est encodée, mais pas \textit{pourquoi} un apprenant consentirait à fournir l'effort nécessaire. La CTML, dans sa formulation initiale, ne distingue pas un agent pédagogique d'un diaporama narré. Les deux exploitent le canal auditif de la même manière. L'extension du modèle vers les indices sociaux (\textit{social cues hypothesis}) comble cette lacune \citep{mayer2014-ju}. Certains signaux --- voix humaine, visage expressif, regard dirigé --- déclenchent un sentiment de présence sociale. Ce sentiment transforme le traitement d'information en échange interpersonnel. L'artefact numérique active alors des réponses ordinairement réservées aux interactions humaines. Le mécanisme psychologique qui sous-tend cette activation fait l'objet de la section suivante.

