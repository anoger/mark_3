\subsection{Architecture cognitive et apprentissage multimédia}
\label{subsec:architecture_cognitive}

Tout agent pédagogique, quelle que soit sa sophistication, opère dans les limites de la mémoire de travail de l'apprenant. La théorie de la charge cognitive identifie cette contrainte fondamentale : la mémoire de travail ne peut traiter qu'un nombre restreint d'éléments simultanément, et toute surcharge dégrade l'apprentissage \citep{sweller1988}. Trois types de charge se partagent cette capacité limitée. La charge intrinsèque dépend de la complexité du contenu et du niveau de l'apprenant --- elle est irréductible. La charge extrinsèque résulte d'un design inadapté : informations redondantes, navigation confuse, éléments visuels non pertinents. La charge pertinente (\textit{germane load}) correspond à l'effort cognitif consacré à la construction de schémas mentaux, c'est-à-dire à l'apprentissage proprement dit \citep{sweller2011}. L'enjeu du design d'un agent pédagogique se formule dans ces termes : minimiser la charge extrinsèque pour libérer des ressources au profit de la charge pertinente.

La théorie cognitive de l'apprentissage multimédia (CTML) prolonge ce cadre en précisant comment l'information est traitée quand elle combine texte, image et son \citep{mayer2021}. Trois hypothèses structurent le modèle : l'information entre par deux canaux distincts (visuel et auditif), chaque canal a une capacité limitée, et l'apprentissage requiert un traitement actif --- sélection, organisation et intégration de l'information avec les connaissances antérieures. De ces hypothèses découlent des principes de design directement applicables aux agents pédagogiques. Le principe de modalité établit qu'une narration orale accompagnant une image produit un meilleur apprentissage qu'un texte écrit accompagnant la même image, parce que la narration répartit la charge entre les deux canaux au lieu de surcharger le canal visuel seul. Le principe de redondance montre, à l'inverse, que présenter simultanément un texte écrit et une narration identique dégrade la performance : le canal visuel traite à la fois le texte et l'image, ce qui génère une charge extrinsèque. Le principe de modalité fournit la justification technique des agents oraux : un agent qui parle libère le canal visuel pour le contenu graphique --- cartes, schémas, documents historiques --- que l'apprenant doit analyser \citep{mayer2014-ju}.

Ces principes traitent l'apprenant comme un système de traitement de l'information. L'optimisation des canaux cognitifs explique \textit{comment} l'information est encodée, mais pas \textit{pourquoi} un apprenant consentirait à fournir l'effort nécessaire à son traitement. La CTML, dans sa formulation initiale, ne distingue pas un agent pédagogique d'un diaporama narré : les deux exploitent le canal auditif de la même manière. L'extension du modèle vers les indices sociaux (\textit{social cues hypothesis}) comble cette lacune. L'hypothèse est la suivante : certains signaux --- une voix humaine plutôt que synthétique, un visage expressif, un regard dirigé --- déclenchent chez l'apprenant un sentiment de présence sociale qui transforme le traitement d'information en échange interpersonnel \citep{mayer2014-ju}. Ce passage du cognitif au social suppose un mécanisme psychologique distinct, par lequel un artefact numérique peut activer des réponses ordinairement réservées aux interactions humaines.

