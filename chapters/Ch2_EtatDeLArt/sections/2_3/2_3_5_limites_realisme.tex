% ============================================================================
% Sous-section 2.3.5 : Limites du Réalisme dans les Agents Pédagogiques
% ============================================================================
% Sources : Rapport Analyse Research Paper Mapper
% Calibrage : ~500 mots
% Type : Révision de 2_3_4_uncanny_valley.tex
% ============================================================================

\subsection{Limites du Réalisme~: Le Paradoxe de l'Hyperréalisme}
\label{subsec:limites_realisme}

Les résultats empiriques présentés dans les sections précédentes convergent vers un constat contre-intuitif~: le réalisme accru des agents ne garantit pas une efficacité pédagogique supérieure. Cette section examine les mécanismes qui expliquent ce paradoxe et ses implications pour le design.

Les comparaisons directes entre agents réalistes et agents stylisés révèlent une absence de différence significative sur les mesures d'apprentissage~\citep{shiban2015appearance}. Les agents stylisés génèrent parfois un engagement supérieur, possiblement parce qu'ils évitent les attentes implicites associées à l'apparence humaine. Lorsqu'un agent ressemble étroitement à un humain, l'apprenant s'attend à des comportements pleinement humains~; les déviations --- latence de réponse, expressions faciales limitées, erreurs de compréhension --- peuvent générer une dissonance qui compromet l'interaction.

L'écart entre agents 2D et agents 3D illustre ce phénomène~\citep{castroalonso2021effectiveness}. Cette différence ne s'explique pas uniquement par la charge cognitive du réalisme visuel~: elle reflète également une inadéquation entre les capacités comportementales des agents et les attentes générées par leur apparence. Un agent 3D photoréaliste qui ne maintient pas un contact visuel approprié ou dont les expressions faciales restent rigides crée une impression d'étrangeté absente chez un agent 2D dont les limitations sont explicitement acceptées.

Ce décalage constitue un défi persistant du domaine~\citep{johnson2016face}. Les progrès technologiques permettent désormais des rendus visuels sophistiqués, mais les comportements interactifs --- synchronisation multimodale, gestion des tours de parole, réactivité émotionnelle --- n'ont pas progressé au même rythme. Cette asymétrie crée une vallée de l'étrange (\textit{uncanny valley}) où l'agent est suffisamment réaliste pour activer les attentes sociales, mais insuffisamment capable pour les satisfaire.

Les agents hyperréalistes peuvent également réduire l'engagement par un mécanisme psychologique distinct~\citep{dai2022systematic}~: face à un agent perçu comme ``intelligent'' et ``compétent'', certains apprenants développent une anxiété de performance qui inhibe leur participation. Les agents stylisés, perçus comme moins menaçants, favoriseraient une interaction plus détendue et un engagement plus authentique.

Ces résultats ont des implications directes pour le design des agents historiques. Un personnage historique rendu de manière hyperréaliste risque de générer des attentes impossibles à satisfaire~: l'apprenant s'attend à une interaction ``comme avec un vrai humain'', alors que l'agent reste limité par ses capacités techniques et la fiabilité de ses connaissances. Une représentation stylisée, explicitement non réaliste, pourrait paradoxalement favoriser une interaction plus productive en établissant d'emblée les limites de l'échange.

Cette analyse conduit à reformuler la question du design~: plutôt que de maximiser le réalisme, il s'agit d'optimiser la congruence entre l'apparence de l'agent, ses capacités comportementales et les attentes de l'apprenant. Ces principes, établis pour les agents traditionnels à scripts prédéfinis, doivent être réexaminés à la lumière de la rupture technologique introduite par les IA génératives, dont les capacités conversationnelles modifient substantiellement les modalités d'interaction.

