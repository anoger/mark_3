% ============================================================================
% Section 2.3 : Les Agents Pédagogiques Virtuels (V3)
% ============================================================================
% Corrections appliquées :
% - 2_3_1 : Aération citations, consolidation effet persona
% - 2_3_2 : Ajout transition vers ICAP
% - 2_3_4 : Regroupement citations davis2018impact
% ============================================================================

\section{Les Agents Pédagogiques Virtuels~: Typologie, Signaux Sociaux et Efficacité}
\label{sec:agents_pedagogiques}

% ----------------------------------------------------------------------------
% Introduction de section (~200 mots)
% ----------------------------------------------------------------------------
Depuis le tuteur SCHOLAR de Carbonell en 1970, les agents pédagogiques virtuels ont connu trois générations de développement. La première (2000--2011) a exploré les possibilités des agents animés avec des systèmes comme Steve et Herman the Bug. La deuxième (2012--2019) a consolidé les bases empiriques à travers de nombreuses méta-analyses. La troisième (2020--présent) intègre les capacités des grands modèles de langage, transformant radicalement les possibilités d'interaction.

Cette section dresse un état de l'art systématique du domaine. Nous proposons d'abord une cartographie des agents pédagogiques à travers leur évolution historique et leurs caractéristiques de design (\S\ref{subsec:cartographie_agents}). Nous analysons ensuite les fondements cognitifs qui contraignent leur efficacité (\S\ref{subsec:fondements_cognitifs}), puis les mécanismes par lesquels ils génèrent une présence sociale (\S\ref{subsec:presence_sociale}). Une taxonomie des signaux sociaux et de leur efficacité empirique est ensuite présentée (\S\ref{subsec:taxonomie_signaux}). Nous examinons enfin les limites du réalisme (\S\ref{subsec:limites_realisme}).

% ----------------------------------------------------------------------------
% Sous-sections : v3 pour 2_3_1, 2_3_2, 2_3_4 ; originaux pour 2_3_3, 2_3_5
% ----------------------------------------------------------------------------
\input{chapters/Ch2_EtatDeLArt/sections/2_3/2_3_1_cartographie_agents_v3.tex}
% ============================================================================
% Sous-section 2.3.2 : Fondements Cognitifs de l'Apprentissage avec Agents (V3)
% ============================================================================
% Corrections appliquées :
% - Lacune 19 : Ajout transition explicite vers ICAP (cf. section 2.1.4)
% ============================================================================

\subsection{Fondements Cognitifs~: Contraintes et Principes de Design}
\label{subsec:fondements_cognitifs}

L'efficacité des agents pédagogiques est contrainte par l'architecture cognitive humaine. Deux cadres théoriques informent directement leur conception~: la Théorie de la Charge Cognitive et la Théorie Cognitive de l'Apprentissage Multimédia. Ces cadres complètent la hiérarchie d'engagement du cadre ICAP présenté en section~\ref{subsec:ICAP}~: si ICAP décrit \textit{ce que fait} l'apprenant, les théories qui suivent expliquent \textit{pourquoi} certaines activités sont plus efficaces que d'autres.

La mémoire de travail constitue le goulot d'étranglement de l'apprentissage~\citep{sweller2011cognitive}. Sa capacité limitée impose de distinguer trois types de charge~: la charge \textit{intrinsèque} (complexité du contenu), la charge \textit{extrinsèque} (inefficacités de présentation), et la charge \textit{pertinente} (effort d'apprentissage productif). Pour les agents pédagogiques, cette distinction a une implication directe~: les éléments de design non fonctionnels peuvent constituer une charge extrinsèque susceptible de compromettre l'apprentissage.

Un agent visuellement complexe --- animations élaborées, environnement 3D détaillé, expressions faciales sophistiquées --- peut paradoxalement nuire à l'apprentissage s'il détourne des ressources cognitives du contenu éducatif. Les méta-analyses confirment ce risque~: les agents 2D surpassent les agents 3D sur les mesures d'apprentissage~\citep{castroalonso2021effectiveness}. Ce résultat contre-intuitif s'explique par le coût cognitif du réalisme~: le traitement d'un environnement 3D immersif consomme des ressources au détriment du contenu.

La Théorie Cognitive de l'Apprentissage Multimédia~\citep{mayer2014cambridge} postule l'existence de deux canaux de traitement distincts~: visuel-pictural et auditif-verbal. Le \textit{principe de modalité} qui en découle stipule que l'apprentissage est favorisé lorsque les explications verbales sont présentées sous forme audio plutôt que textuelle. En déléguant l'information verbale au canal auditif, le concepteur libère le canal visuel pour les éléments graphiques pertinents.

Ce principe fournit la justification théorique des agents vocaux~: un agent qui \textit{parle} plutôt qu'il n'affiche du texte permet une répartition optimale de la charge entre les canaux. Les voix humaines s'avèrent plus efficaces que les voix synthétiques, bien que cet écart se réduise avec les progrès technologiques. Le \textit{principe de personnalisation} complète cette recommandation~: un style conversationnel surpasse un style formel.

Les gestes de l'agent constituent un cas particulier~: ils peuvent soit faciliter l'apprentissage en guidant l'attention, soit le compromettre en ajoutant une charge extrinsèque. Les données empiriques révèlent des effets modérés des gestes sur le transfert proche et la rétention~\citep{davis2018impact}. Ces effets sont conditionnés par la \textit{congruence sémantique}~: les gestes doivent être alignés avec le contenu verbal. Un geste déictique pointant vers un élément pertinent du graphique facilite l'intégration~; un geste générique non relié au contenu constitue une distraction.

Ces résultats convergent vers un paradoxe apparent~: les agents les plus efficaces ne sont pas les plus réalistes, mais les plus parcimonieux. Un agent doit fournir suffisamment d'indices pour activer l'engagement social (voir section suivante), sans surcharger le système cognitif par des éléments non fonctionnels. Cette recommandation de parcimonie entre en tension avec l'évolution technologique~: les capacités croissantes de rendu réaliste ne garantissent pas une efficacité pédagogique accrue. La question n'est pas \og que peut-on techniquement réaliser?\fg{} mais \og qu'est-ce qui sert effectivement l'apprentissage?\fg{}.

% ============================================================================
% Sous-section 2.3.3 : De l'Artefact au Partenaire — Présence Sociale
% ============================================================================
% Sources : IJCCI_extraction (§2.5), illusion_extraction, Protocole CER
% Calibrage : ~600 mots
% Type : Révision fusionnant 2_3_2_CASA.tex et 2_3_3_agence_sociale.tex
% ============================================================================

\subsection{De l'Artefact au Partenaire~: Mécanismes de la Présence Sociale}
\label{subsec:presence_sociale}

La section précédente a établi les contraintes cognitives qui encadrent l'efficacité des agents. Mais l'impact d'un agent ne se réduit pas à sa fonction de canal d'information~: il dépend aussi de sa capacité à être perçu comme un partenaire social. Cette perception transforme l'interaction technique en expérience relationnelle.

Le paradigme CASA (\textit{Computers Are Social Actors}) constitue le cadre explicatif central de ce phénomène~\citep{nass2000machines}. Les travaux fondateurs de Nass et ses collaborateurs ont démontré que les utilisateurs appliquent inconsciemment aux ordinateurs les règles sociales qu'ils utiliseraient avec des humains~: politesse, réciprocité, attribution de personnalité. Ce traitement social tend à s'opérer même lorsque l'utilisateur \textit{sait} qu'il interagit avec une machine --- un processus qualifié d'\textit{ethopoeia}.

Des indices sociaux minimaux suffisent à déclencher ces scripts relationnels. Une voix, qu'elle soit synthétique ou humaine, active des réponses sociales. Les utilisateurs évaluent différemment un ordinateur selon le genre de sa voix, reproduisant les stéréotypes sociaux~: une voix masculine est perçue comme plus compétente sur les sujets techniques. Ces attributions automatiques expliquent pourquoi même des agents rudimentaires peuvent générer un engagement significatif.

La présence sociale désigne le sentiment d'être \og avec\fg{} une autre intelligence dans un environnement médiatisé~\citep{biocca2003toward}. Ce sentiment ne requiert pas un interlocuteur humain~; il peut émerger de l'interaction avec un agent virtuel pourvu que certaines conditions soient réunies. L'utilisateur doit percevoir l'agent non comme un outil passif, mais comme une entité dotée d'une forme d'intentionnalité et de réactivité.

Cette présence sociale constitue la condition préalable à l'activation des mécanismes d'apprentissage social. Sans elle, l'interaction tend à rester instrumentale~: l'utilisateur traite l'information sans s'engager pleinement dans la relation. Avec elle, l'agent devient un partenaire dont on respecte implicitement le \og contrat de communication\fg{}.

La théorie de l'agence sociale (\textit{Social Agency Theory}) explicite le lien entre présence sociale et apprentissage~\citep{moreno2001case, mayer2012embodiment}. Lorsque l'apprenant perçoit l'agent comme un partenaire social, un contrat implicite s'établit~: l'apprenant s'engage à \og honorer\fg{} cette relation par un effort cognitif accru. Cet effort supplémentaire se traduit par un traitement plus profond du contenu.

Les méta-analyses confirment ce mécanisme~: les agents anthropomorphisés produisent des effets supérieurs aux agents non anthropomorphisés, mais ces effets concernent davantage les mesures affectives et motivationnelles que les mesures cognitives pures~\citep{schroeder2025designing, dai2022meta}.


% ============================================================================
% Sous-section 2.3.4 : Taxonomie et Efficacité des Signaux Sociaux (V3)
% ============================================================================
% Corrections appliquées :
% - Lacune 21 : Regroupement des citations davis2018impact (3x -> 1x)
% ============================================================================

\subsection{Taxonomie et Efficacité des Signaux Sociaux}
\label{subsec:taxonomie_signaux}

Les agents pédagogiques mobilisent une variété de signaux sociaux pour établir la présence et faciliter l'apprentissage. Cette section propose une taxonomie systématique de ces signaux, fondée sur les résultats empiriques des méta-analyses du domaine. Le tableau~\ref{tab:signaux_sociaux} synthétise les principaux effets documentés.

\begin{table}[htbp]
\centering
\caption{Efficacité des signaux sociaux dans les agents pédagogiques. Synthèse des résultats issus des méta-analyses.}
\label{tab:signaux_sociaux}
\small
\begin{tabular}{p{2.5cm}p{3.5cm}p{3cm}p{4cm}}
\toprule
\textbf{Catégorie} & \textbf{Signal} & \textbf{Effet} & \textbf{Source} \\
\midrule
\multirow{3}{*}{\textit{Représentation}}
& Présence vs absence & Effet positif & \citet{castroalonso2021effectiveness} \\
& 2D vs 3D & 2D > 3D & \citet{castroalonso2021effectiveness} \\
& Animé vs statique & Animé > Statique & \citet{dai2024effects} \\
\addlinespace
\multirow{2}{*}{\textit{Voix}}
& Voix humaine vs synthétique & Humaine > Synthétique & \citet{davis2018impact} \\
& Style conversationnel & Effet positif modéré & \citet{wang2023effects} \\
\addlinespace
\multirow{2}{*}{\textit{Gestes}}
& Gestes déictiques & Effet positif (transfert) & \citet{davis2018impact} \\
& Gestes génériques & Effet modéré (rétention) & \citet{davis2018impact} \\
\addlinespace
\multirow{2}{*}{\textit{Affect}}
& Expression émotionnelle & Effet positif modéré & \citet{wang2023effects} \\
& Affect positif & Corrélation positive & \citet{guo2015affect} \\
\addlinespace
\multirow{2}{*}{\textit{Interaction}}
& Chatbots IA & Effet positif important & \citet{wu2024chatbots} \\
& Feedback adaptatif & Effet positif & \citet{martha2018design} \\
\bottomrule
\end{tabular}
\end{table}

La \textbf{représentation visuelle} constitue le premier niveau de signaux. La présence d'un agent produit un effet global positif, mais ce résultat masque des variations importantes selon le type de représentation~\citep{castroalonso2021effectiveness}. Les agents 2D surpassent les agents 3D, un écart qui s'explique par le coût cognitif du réalisme~: le traitement d'environnements 3D immersifs consomme des ressources au détriment du contenu pédagogique. L'animation constitue néanmoins un atout~: les agents animés surpassent les agents statiques, l'animation permettant de guider l'attention et de maintenir l'engagement~\citep{dai2024effects}.

La \textbf{voix} représente un signal social particulièrement puissant. Les voix humaines conservent un avantage sur les voix synthétiques, bien que cet écart se réduise avec les progrès technologiques~: les voix synthétiques de haute qualité produisent désormais des résultats d'apprentissage équivalents, voire supérieurs, pour le transfert de connaissances~\citep{craig2017schroeder, davis2018impact}. La qualité prosodique s'avère plus déterminante que l'incarnation visuelle pour la perception de naturalité~: un agent réaliste avec une prosodie inadéquate sera jugé moins naturel qu'un agent désincarné avec une prosodie correcte~\citep{ehret2021}. Au-delà de la qualité vocale, le style discursif influence l'apprentissage~: le style conversationnel produit un effet positif par rapport au style formel~\citep{wang2023effects}. Ce résultat s'aligne avec le principe de personnalisation de la théorie cognitive de l'apprentissage multimédia~: un discours adressé directement à l'apprenant (``vous'') favorise l'engagement par rapport à un discours impersonnel.

Les \textbf{gestes} de l'agent constituent une catégorie de signaux aux effets différenciés. Une méta-analyse distingue deux types de gestes~\citep{davis2018impact}~: les gestes déictiques --- pointage vers des éléments pertinents de l'interface --- produisent un effet important sur le transfert en guidant explicitement l'attention visuelle et en facilitant l'intégration des informations verbales et graphiques~; les gestes génériques --- mouvements non liés au contenu --- produisent un effet plus modeste sur la rétention. Cette différence confirme l'importance de la congruence sémantique~: les gestes doivent être alignés avec le contenu pour maximiser leur efficacité.

La combinaison de plusieurs types de communication non verbale n'amplifie pas nécessairement les effets~: un seul type de signal approprié au résultat d'apprentissage visé (gestes \textit{ou} expressions faciales) surpasse la combinaison des deux~\citep{baylor2009design}. Les expressions faciales favorisent l'apprentissage attitudinal, tandis que les gestes facilitent l'apprentissage procédural. Ce résultat contre-intuitif s'explique par la théorie de la charge cognitive~: la multiplication des signaux peut surcharger l'apprenant plutôt que l'assister. L'inconsistance des résultats dans la littérature confirme que l'efficacité des signaux non verbaux dépend davantage de leur appropriation au contexte que de leur quantité~\citep{wang2021examining}.

L'\textbf{affect} de l'agent influence l'apprentissage à travers plusieurs mécanismes. Une corrélation positive existe entre affect de l'agent et apprentissage~\citep{guo2015affect}. L'expression émotionnelle se distingue de la simple présence affective~\citep{wang2023effects}. Les expressions faciales doivent être alignées avec le contexte pour produire leurs effets~\citep{liew2016}~; lorsqu'elles sont coordonnées avec le discours, elles influencent positivement la motivation~\citep{liew2017}. Toutefois, l'expressivité émotionnelle positive peut s'avérer contre-productive~: elle peut sembler inappropriée et nuire à la confiance lors de tâches critiques où la neutralité est attendue~\citep{torre2019}. Les effets affectifs sont également plus marqués sur les mesures motivationnelles que sur les mesures cognitives pures~\citep{tao2022exploring}.

La dimension \textbf{interactive} distingue les agents conversationnels des agents unidirectionnels. Les chatbots IA en éducation produisent des effets positifs importants~\citep{wu2024chatbots}, suggérant que la capacité de dialogue --- recevoir les questions de l'apprenant et y répondre de manière contextuelle --- constitue un amplificateur majeur de l'efficacité pédagogique.

La \textbf{motivation} constitue une variable médiatrice importante. Les effets des chatbots éducatifs varient selon le type de motivation mesurée~: l'auto-efficacité montre des effets importants, tandis que la motivation intrinsèque présente des effets plus modérés~\citep{gladstone2025motivation}. Ces résultats suggèrent que les agents agissent principalement en renforçant la confiance de l'apprenant dans ses capacités plutôt qu'en modifiant son intérêt intrinsèque pour le contenu.

L'\textbf{alignement thématique} entre l'agent et le contenu constitue un facteur modulateur. Les apprenants interagissant avec un agent thématiquement aligné (un astronaute enseignant des concepts spatiaux) rapportent des niveaux supérieurs de présence sociale et d'engagement comparés à ceux interagissant avec un agent neutre~\citep{schmidt2019}. Ce résultat suggère que la cohérence entre l'identité de l'agent et le domaine enseigné peut amplifier les effets des signaux sociaux.

Une analyse transversale révèle un patron cohérent~: les effets des signaux sociaux sont systématiquement plus importants sur les mesures affectives et motivationnelles que sur les mesures cognitives. Sur 15 études avec groupe contrôle, 9 ne montrent aucune différence d'apprentissage significative, alors que les effets sur l'engagement sont généralement positifs~\citep{heidig2011pedagogical}. Une dissociation frappante émerge entre perception et performance~: les apprenants apprécient davantage la leçon, la jugent plus intéressante et ont l'impression de mieux apprendre, mais leurs résultats de compréhension objective diminuent --- un phénomène résumé par l'expression ``\textit{more gaze, but less learning}''~\citep{wilson2018}.

La figure~\ref{fig:signaux_efficacite} visualise ces relations entre signaux sociaux et types d'effets.

\begin{figure}[htbp]
\centering
\fbox{\parbox{0.9\textwidth}{\centering\vspace{2cm}
\textit{Figure à insérer~: Modèle des effets des signaux sociaux}\\[0.5em]
Schéma montrant~:\\
(1) Les catégories de signaux (représentation, voix, gestes, affect, interaction)\\
(2) Les mécanismes médiateurs (présence sociale, engagement, charge cognitive)\\
(3) Les types d'effets (affectif, motivationnel, cognitif)\\
Avec indication des tailles d'effet pour chaque relation
\vspace{2cm}}}
\caption{Modèle intégratif des effets des signaux sociaux des agents pédagogiques. Les signaux influencent l'apprentissage à travers des mécanismes médiateurs, avec des effets différenciés selon le type de mesure.}
\label{fig:signaux_efficacite}
\end{figure}

Ces résultats convergent vers une recommandation de parcimonie stratégique~: multiplier les signaux sociaux n'amplifie pas nécessairement l'efficacité pédagogique. Chaque signal doit être évalué selon son rapport bénéfice (engagement, guidage attentionnel) / coût (charge cognitive, distraction). Cette tension entre richesse des signaux et surcharge potentielle conduit naturellement à examiner les limites du réalisme dans la conception des agents.

% ============================================================================
% Sous-section 2.3.5 : Limites du Réalisme dans les Agents Pédagogiques
% ============================================================================
% Sources : Rapport Analyse Research Paper Mapper
% Calibrage : ~500 mots
% Type : Révision de 2_3_4_uncanny_valley.tex
% ============================================================================

\subsection{Limites du Réalisme~: Le Paradoxe de l'Hyperréalisme}
\label{subsec:limites_realisme}

Les résultats empiriques présentés dans les sections précédentes convergent vers un constat contre-intuitif~: le réalisme accru des agents ne garantit pas une efficacité pédagogique supérieure. Cette section examine les mécanismes qui expliquent ce paradoxe et ses implications pour le design.

Les comparaisons directes entre agents réalistes et agents stylisés révèlent une absence de différence significative sur les mesures d'apprentissage~\citep{shiban2015appearance}. Les agents stylisés génèrent parfois un engagement supérieur, possiblement parce qu'ils évitent les attentes implicites associées à l'apparence humaine. Lorsqu'un agent ressemble étroitement à un humain, l'apprenant s'attend à des comportements pleinement humains~; les déviations --- latence de réponse, expressions faciales limitées, erreurs de compréhension --- peuvent générer une dissonance qui compromet l'interaction.

L'écart entre agents 2D et agents 3D illustre ce phénomène~\citep{castroalonso2021effectiveness}. Cette différence ne s'explique pas uniquement par la charge cognitive du réalisme visuel~: elle reflète également une inadéquation entre les capacités comportementales des agents et les attentes générées par leur apparence. Un agent 3D photoréaliste qui ne maintient pas un contact visuel approprié ou dont les expressions faciales restent rigides crée une impression d'étrangeté absente chez un agent 2D dont les limitations sont explicitement acceptées.

Ce décalage constitue un défi persistant du domaine~\citep{johnson2016face}. Les progrès technologiques permettent désormais des rendus visuels sophistiqués, mais les comportements interactifs --- synchronisation multimodale, gestion des tours de parole, réactivité émotionnelle --- n'ont pas progressé au même rythme. Cette asymétrie crée une vallée de l'étrange (\textit{uncanny valley}) où l'agent est suffisamment réaliste pour activer les attentes sociales, mais insuffisamment capable pour les satisfaire.

Les agents hyperréalistes peuvent également réduire l'engagement par un mécanisme psychologique distinct~\citep{dai2022systematic}~: face à un agent perçu comme ``intelligent'' et ``compétent'', certains apprenants développent une anxiété de performance qui inhibe leur participation. Les agents stylisés, perçus comme moins menaçants, favoriseraient une interaction plus détendue et un engagement plus authentique.

Ces résultats ont des implications directes pour le design des agents historiques. Un personnage historique rendu de manière hyperréaliste risque de générer des attentes impossibles à satisfaire~: l'apprenant s'attend à une interaction ``comme avec un vrai humain'', alors que l'agent reste limité par ses capacités techniques et la fiabilité de ses connaissances. Une représentation stylisée, explicitement non réaliste, pourrait paradoxalement favoriser une interaction plus productive en établissant d'emblée les limites de l'échange.

Cette analyse conduit à reformuler la question du design~: plutôt que de maximiser le réalisme, il s'agit d'optimiser la congruence entre l'apparence de l'agent, ses capacités comportementales et les attentes de l'apprenant. Ces principes, établis pour les agents traditionnels à scripts prédéfinis, doivent être réexaminés à la lumière de la rupture technologique introduite par les IA génératives, dont les capacités conversationnelles modifient substantiellement les modalités d'interaction.


