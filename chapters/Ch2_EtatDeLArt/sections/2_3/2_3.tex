% ============================================================================
% Section 2.3 : Les Agents Pédagogiques Virtuels
% ============================================================================
% Objectif : État de l'art des agents pédagogiques : typologie, signaux sociaux,
%            efficacité et limites
% Sources : Rapport Analyse Research Paper Mapper (17 méta-analyses),
%           IJCCI_extraction, illusion_extraction, Vault.xlsx
% Calibrage : ~4 200 mots
% ============================================================================

\section{Les Agents Pédagogiques Virtuels~: Typologie, Signaux Sociaux et Efficacité}
\label{sec:agents_pedagogiques}

% ----------------------------------------------------------------------------
% Introduction de section (~200 mots)
% ----------------------------------------------------------------------------
Depuis le tuteur SCHOLAR de Carbonell en 1970, les agents pédagogiques virtuels ont connu trois générations de développement. La première (2000--2011) a exploré les possibilités des agents animés avec des systèmes comme Steve et Herman the Bug. La deuxième (2012--2019) a consolidé les bases empiriques à travers de nombreuses méta-analyses. La troisième (2020--présent) intègre les capacités des grands modèles de langage, transformant radicalement les possibilités d'interaction.

Cette section dresse un état de l'art systématique du domaine. Nous proposons d'abord une cartographie des agents pédagogiques à travers leur évolution historique et leurs caractéristiques de design (\S\ref{subsec:cartographie_agents}). Nous analysons ensuite les fondements cognitifs qui contraignent leur efficacité (\S\ref{subsec:fondements_cognitifs}), puis les mécanismes par lesquels ils génèrent une présence sociale (\S\ref{subsec:presence_sociale}). Une taxonomie des signaux sociaux et de leur efficacité empirique est ensuite présentée (\S\ref{subsec:taxonomie_signaux}). Nous examinons enfin les limites du réalisme (\S\ref{subsec:limites_realisme}).

% ----------------------------------------------------------------------------
% Sous-sections
% ----------------------------------------------------------------------------
% ============================================================================
% Sous-section 2.3.1 : Cartographie des Agents Pédagogiques Virtuels
% ============================================================================
% Sources : Rapport Analyse Research Paper Mapper (17 méta-analyses)
% Calibrage : ~800 mots
% Type : NOUVEAU
% ============================================================================

\subsection{Cartographie des Agents Pédagogiques~: 25 Ans d'Évolution}
\label{subsec:cartographie_agents}

L'histoire des agents pédagogiques virtuels peut être structurée en trois générations distinctes, chacune caractérisée par des paradigmes technologiques et des objectifs de recherche spécifiques~\citep{johnson2016face, alfaro2020new}.

La \textbf{première génération (1997--2011)} correspond à l'ère des pionniers. Les premiers agents pédagogiques animés émergent à la fin des années 1990 avec des systèmes comme Steve (\textit{Soar Training Expert for Virtual Environments})~\citep{johnson1998steve}, agent 3D guidant les marins dans l'apprentissage de procédures techniques, ou Herman the Bug~\citep{lester1997lifelike}, agent 2D anthropomorphisé sous forme d'insecte qui a démontré l'\textit{effet de persona}~: sa seule présence améliorait la motivation des apprenants, indépendamment de la qualité du feedback~\citep{heidig2011pedagogical}. AutoTutor~\citep{graesser2004autotutor} a introduit le dialogue socratique automatisé, posant les bases des agents conversationnels éducatifs. Ces systèmes pionniers reposaient sur des scripts prédéfinis et des arbres de décision, ce qui limitait leur flexibilité conversationnelle.

La \textbf{deuxième génération (2012--2019)} se caractérise par la consolidation empirique. Les méta-analyses se multiplient pour évaluer l'efficacité réelle des agents selon différentes variables de design~: effets de l'affect, impact des gestes, comparaisons entre apparences 2D et 3D, voix humaines et synthétiques, niveaux d'anthropomorphisme~\citep{guo2015affect, davis2018impact}. Un constat émerge de ces synthèses~: sur 15 études avec groupe contrôle, 9 ne montrent aucune différence d'apprentissage significative, appelant à une approche plus rigoureuse~\citep{heidig2011pedagogical}. Cette génération établit les bases empiriques du domaine, mais les agents restent contraints par leurs scripts préprogrammés.

La \textbf{troisième génération (2020--présent)} correspond à l'ère générative. L'émergence des grands modèles de langage transforme radicalement le paysage~: les agents peuvent désormais générer des réponses contextuellement appropriées sans scripts prédéfinis. Les chatbots IA en éducation produisent des effets positifs sur les résultats d'apprentissage, mais cette flexibilité conversationnelle accrue s'accompagne de nouveaux défis liés à l'imprévisibilité des réponses~\citep{wu2024chatbots, schroeder2025designing}. Cette génération, à laquelle appartient notre agent historique, hérite des acquis des générations précédentes tout en introduisant des risques inédits --- notamment celui des hallucinations analysé en section~\ref{sec:IA_generatives}.

Au-delà de la chronologie, les agents se différencient selon plusieurs dimensions de design. Le tableau~\ref{tab:typologie_agents} propose une typologie fondée sur les caractéristiques empiriquement étudiées dans la littérature, synthétisée à partir de 17 méta-analyses et revues systématiques.

\begin{table}[htbp]
\centering
\caption{Typologie des agents pédagogiques selon leurs caractéristiques de design. Synthèse établie à partir des méta-analyses de \citet{castroalonso2021effectiveness}, \citet{dai2024effects} et \citet{martha2018design}.}
\label{tab:typologie_agents}
\small
\begin{tabular}{p{2.8cm}p{4cm}p{4cm}p{3cm}}
\toprule
\textbf{Dimension} & \textbf{Modalités} & \textbf{Exemples} & \textbf{Effet sur apprentissage} \\
\midrule
\textit{Représentation visuelle} &
Absent / Statique / Animé 2D / Animé 3D &
Texte seul / Image fixe / Cartoon / Avatar 3D &
Animé > Statique~; 2D $\geq$ 3D \\
\addlinespace
\textit{Niveau d'anthropomorphisme} &
Non-humain / Stylisé / Réaliste / Hyperréaliste &
Robot / Cartoon / Avatar / Deepfake &
Stylisé souvent optimal \\
\addlinespace
\textit{Modalité vocale} &
Texte / Voix synthétique / Voix humaine &
Chat / TTS / Enregistrement &
Humaine > Synthétique \\
\addlinespace
\textit{Rôle pédagogique} &
Tuteur / Compagnon / Teachable agent &
Expert / Assistant / Élève virtuel &
Dépend du contexte \\
\addlinespace
\textit{Moteur conversationnel} &
Script / Règles / NLP-ML / LLM &
Arbre décision / AIML / Seq2Seq / GPT &
LLM~: flexibilité $\uparrow$, contrôle $\downarrow$ \\
\bottomrule
\end{tabular}
\end{table}

L'application des agents pédagogiques à l'enseignement de l'Histoire reste peu explorée et les résultats disponibles sont préoccupants~: l'effet mesuré pour l'histoire est négatif, contre des effets positifs en biologie et en informatique~\citep{castroalonso2021effectiveness}. Cette disparité disciplinaire s'explique par un défi épistémologique spécifique~: contrairement aux domaines STIM où l'agent peut s'appuyer sur des connaissances vérifiables, l'agent historique doit naviguer entre faits documentés, interprétations historiographiques et zones d'incertitude. Cette spécificité justifie une attention particulière aux risques métacognitifs que nous analysons dans cette thèse.

La figure~\ref{fig:evolution_agents} illustre cette évolution à travers deux exemples représentatifs des deux premières générations.

\begin{figure}[htbp]
\centering
% Placeholder pour figure composite
\fbox{\parbox{0.9\textwidth}{\centering\vspace{2cm}
\textit{Figure à insérer~: Deux agents représentatifs}\\[0.5em]
(a) Steve~\citep{johnson1998steve} --- Génération 1\\
(b) Agent expressif de \citet{torre2019} --- Génération 2
\vspace{2cm}}}
\caption{Évolution des agents pédagogiques à travers les deux premières générations. (a)~Steve, agent 3D pionnier pour l'entraînement naval~\citep{johnson1998steve}. (b)~Agent expressif utilisé dans les études sur l'alignement comportemental et contextuel~\citep{torre2019}.}
\label{fig:evolution_agents}
\end{figure}


\input{chapters/Ch2_EtatDeLArt/sections/2_3/2_3_2_fondements_cognitifs.tex}
% ============================================================================
% Sous-section 2.3.3 : De l'Artefact au Partenaire — Présence Sociale
% ============================================================================
% Sources : IJCCI_extraction (§2.5), illusion_extraction, Protocole CER
% Calibrage : ~600 mots
% Type : Révision fusionnant 2_3_2_CASA.tex et 2_3_3_agence_sociale.tex
% ============================================================================

\subsection{De l'Artefact au Partenaire~: Mécanismes de la Présence Sociale}
\label{subsec:presence_sociale}

La section précédente a établi les contraintes cognitives qui encadrent l'efficacité des agents. Mais l'impact d'un agent ne se réduit pas à sa fonction de canal d'information~: il dépend aussi de sa capacité à être perçu comme un partenaire social. Cette perception transforme l'interaction technique en expérience relationnelle.

Le paradigme CASA (\textit{Computers Are Social Actors}) constitue le cadre explicatif central de ce phénomène~\citep{nass2000machines}. Les travaux fondateurs de Nass et ses collaborateurs ont démontré que les utilisateurs appliquent inconsciemment aux ordinateurs les règles sociales qu'ils utiliseraient avec des humains~: politesse, réciprocité, attribution de personnalité. Ce traitement social tend à s'opérer même lorsque l'utilisateur \textit{sait} qu'il interagit avec une machine --- un processus qualifié d'\textit{ethopoeia}.

Des indices sociaux minimaux suffisent à déclencher ces scripts relationnels. Une voix, qu'elle soit synthétique ou humaine, active des réponses sociales. Les utilisateurs évaluent différemment un ordinateur selon le genre de sa voix, reproduisant les stéréotypes sociaux~: une voix masculine est perçue comme plus compétente sur les sujets techniques. Ces attributions automatiques expliquent pourquoi même des agents rudimentaires peuvent générer un engagement significatif.

La présence sociale désigne le sentiment d'être \og avec\fg{} une autre intelligence dans un environnement médiatisé~\citep{biocca2003toward}. Ce sentiment ne requiert pas un interlocuteur humain~; il peut émerger de l'interaction avec un agent virtuel pourvu que certaines conditions soient réunies. L'utilisateur doit percevoir l'agent non comme un outil passif, mais comme une entité dotée d'une forme d'intentionnalité et de réactivité.

Cette présence sociale constitue la condition préalable à l'activation des mécanismes d'apprentissage social. Sans elle, l'interaction tend à rester instrumentale~: l'utilisateur traite l'information sans s'engager pleinement dans la relation. Avec elle, l'agent devient un partenaire dont on respecte implicitement le \og contrat de communication\fg{}.

La théorie de l'agence sociale (\textit{Social Agency Theory}) explicite le lien entre présence sociale et apprentissage~\citep{moreno2001case, mayer2012embodiment}. Lorsque l'apprenant perçoit l'agent comme un partenaire social, un contrat implicite s'établit~: l'apprenant s'engage à \og honorer\fg{} cette relation par un effort cognitif accru. Cet effort supplémentaire se traduit par un traitement plus profond du contenu.

Les méta-analyses confirment ce mécanisme~: les agents anthropomorphisés produisent des effets supérieurs aux agents non anthropomorphisés, mais ces effets concernent davantage les mesures affectives et motivationnelles que les mesures cognitives pures~\citep{schroeder2025designing, dai2022meta}.


\input{chapters/Ch2_EtatDeLArt/sections/2_3/2_3_4_taxonomie_signaux.tex}
% ============================================================================
% Sous-section 2.3.5 : Limites du Réalisme dans les Agents Pédagogiques
% ============================================================================
% Sources : Rapport Analyse Research Paper Mapper
% Calibrage : ~500 mots
% Type : Révision de 2_3_4_uncanny_valley.tex
% ============================================================================

\subsection{Limites du Réalisme~: Le Paradoxe de l'Hyperréalisme}
\label{subsec:limites_realisme}

Les résultats empiriques présentés dans les sections précédentes convergent vers un constat contre-intuitif~: le réalisme accru des agents ne garantit pas une efficacité pédagogique supérieure. Cette section examine les mécanismes qui expliquent ce paradoxe et ses implications pour le design.

Les comparaisons directes entre agents réalistes et agents stylisés révèlent une absence de différence significative sur les mesures d'apprentissage~\citep{shiban2015appearance}. Les agents stylisés génèrent parfois un engagement supérieur, possiblement parce qu'ils évitent les attentes implicites associées à l'apparence humaine. Lorsqu'un agent ressemble étroitement à un humain, l'apprenant s'attend à des comportements pleinement humains~; les déviations --- latence de réponse, expressions faciales limitées, erreurs de compréhension --- peuvent générer une dissonance qui compromet l'interaction.

L'écart entre agents 2D et agents 3D illustre ce phénomène~\citep{castroalonso2021effectiveness}. Cette différence ne s'explique pas uniquement par la charge cognitive du réalisme visuel~: elle reflète également une inadéquation entre les capacités comportementales des agents et les attentes générées par leur apparence. Un agent 3D photoréaliste qui ne maintient pas un contact visuel approprié ou dont les expressions faciales restent rigides crée une impression d'étrangeté absente chez un agent 2D dont les limitations sont explicitement acceptées.

Ce décalage constitue un défi persistant du domaine~\citep{johnson2016face}. Les progrès technologiques permettent désormais des rendus visuels sophistiqués, mais les comportements interactifs --- synchronisation multimodale, gestion des tours de parole, réactivité émotionnelle --- n'ont pas progressé au même rythme. Cette asymétrie crée une vallée de l'étrange (\textit{uncanny valley}) où l'agent est suffisamment réaliste pour activer les attentes sociales, mais insuffisamment capable pour les satisfaire.

Les agents hyperréalistes peuvent également réduire l'engagement par un mécanisme psychologique distinct~\citep{dai2022systematic}~: face à un agent perçu comme ``intelligent'' et ``compétent'', certains apprenants développent une anxiété de performance qui inhibe leur participation. Les agents stylisés, perçus comme moins menaçants, favoriseraient une interaction plus détendue et un engagement plus authentique.

Ces résultats ont des implications directes pour le design des agents historiques. Un personnage historique rendu de manière hyperréaliste risque de générer des attentes impossibles à satisfaire~: l'apprenant s'attend à une interaction ``comme avec un vrai humain'', alors que l'agent reste limité par ses capacités techniques et la fiabilité de ses connaissances. Une représentation stylisée, explicitement non réaliste, pourrait paradoxalement favoriser une interaction plus productive en établissant d'emblée les limites de l'échange.

Cette analyse conduit à reformuler la question du design~: plutôt que de maximiser le réalisme, il s'agit d'optimiser la congruence entre l'apparence de l'agent, ses capacités comportementales et les attentes de l'apprenant. Ces principes, établis pour les agents traditionnels à scripts prédéfinis, doivent être réexaminés à la lumière de la rupture technologique introduite par les IA génératives, dont les capacités conversationnelles modifient substantiellement les modalités d'interaction.


