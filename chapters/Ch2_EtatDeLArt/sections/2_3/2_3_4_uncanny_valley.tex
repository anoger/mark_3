\subsection{Les ruptures de présence : la vallée de l'étrange}
\label{subsec:uncanny_valley}

La relation entre réalisme de l'agent et réponse affective de l'utilisateur n'est pas linéaire. La vallée de l'étrange (\textit{uncanny valley}) désigne une zone où l'affinité, après avoir crû avec le degré de ressemblance humaine, chute brutalement avant de remonter lorsque l'agent devient indiscernable d'un humain \citep{mori2012}. Cette vallée n'est pas un artefact de laboratoire : elle se manifeste dans les jugements de confiance et d'attractivité portés sur des visages numériques dont le réalisme varie systématiquement \citep{mcdonnell2012}. Le mécanisme proposé repose sur un conflit perceptif : un agent presque humain active les attentes propres aux interactions humaines, mais les violations subtiles --- un mouvement oculaire trop lent, une micro-expression absente, un sourire asymétrique --- produisent un malaise que les agents stylisés ne déclenchent pas, précisément parce qu'ils n'activent pas ces attentes.

La théorie des violations d'attentes (\textit{Expectancy Violation Theory}) fournit le cadre qui explique ce mécanisme au-delà de la seule apparence visuelle \citep{burgoon2015}. Toute interaction repose sur des attentes implicites concernant le comportement de l'interlocuteur. Une violation positive --- un agent qui répond mieux que prévu --- renforce l'engagement. Une violation négative --- un comportement incohérent avec les indices émis par l'agent --- produit un inconfort qui détourne les ressources cognitives du contenu vers l'évaluation de la source. Un agent dont le réalisme comportemental ne correspond pas aux attentes créées par son apparence tend à susciter davantage d'inconfort qu'un agent aux capacités modestes qui n'active pas ces attentes \citep{groom2009, haresamudram2024}. L'alignement multimodal --- la cohérence entre apparence visuelle, qualité vocale et comportement gestuel --- prédit la réponse affective de l'utilisateur mieux que le réalisme de chaque modalité prise isolément \citep{alimardani2024}. Ces attentes dépendent en outre du contexte : un agent pédagogique n'est pas évalué selon les mêmes critères qu'un agent de divertissement, et le seuil de tolérance aux incohérences varie avec le rôle attribué à l'agent \citep{torre2019}.

La leçon convergente de ces travaux est que le réalisme n'est pas un objectif de design en soi. Un agent trop réaliste pour son niveau de comportement peut produire une distraction cognitive susceptible de réduire l'apprentissage \citep{parmar2022}. Les agents au style cartoon et les agents photoréalistes produisent des effets d'apprentissage comparables, mais les premiers génèrent moins de distraction et réduisent le risque de tomber dans la vallée de l'étrange \citep{li2024}. Le principe de design qui en découle privilégie la cohérence sur le réalisme : viser un niveau de ressemblance humaine que le comportement de l'agent --- verbal, vocal et gestuel --- peut soutenir sans rupture. Les IA génératives modifient cette équation. La fluence du langage naturel et la capacité d'adaptation contextuelle augmentent la cohérence comportementale, ce qui permet de soutenir des niveaux de réalisme plus élevés sans déclencher de violations d'attentes. Cette avancée ouvre de nouvelles possibilités, mais introduit simultanément de nouveaux risques que les sections suivantes examineront.

