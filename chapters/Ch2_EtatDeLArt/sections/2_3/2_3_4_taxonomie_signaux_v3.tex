% ============================================================================
% Sous-section 2.3.4 : Taxonomie et Efficacité des Signaux Sociaux (V3)
% ============================================================================
% Corrections appliquées :
% - Lacune 21 : Regroupement des citations davis2018impact (3x -> 1x)
% ============================================================================

\subsection{Taxonomie et Efficacité des Signaux Sociaux}
\label{subsec:taxonomie_signaux}

Les agents pédagogiques mobilisent une variété de signaux sociaux pour établir la présence et faciliter l'apprentissage. Cette section propose une taxonomie systématique de ces signaux, fondée sur les résultats empiriques des méta-analyses du domaine. Le tableau~\ref{tab:signaux_sociaux} synthétise les principaux effets documentés.

\begin{table}[htbp]
\centering
\caption{Efficacité des signaux sociaux dans les agents pédagogiques. Synthèse des résultats issus des méta-analyses.}
\label{tab:signaux_sociaux}
\small
\begin{tabular}{p{2.5cm}p{3.5cm}p{3cm}p{4cm}}
\toprule
\textbf{Catégorie} & \textbf{Signal} & \textbf{Effet} & \textbf{Source} \\
\midrule
\multirow{3}{*}{\textit{Représentation}}
& Présence vs absence & Effet positif & \citet{castroalonso2021effectiveness} \\
& 2D vs 3D & 2D > 3D & \citet{castroalonso2021effectiveness} \\
& Animé vs statique & Animé > Statique & \citet{dai2024effects} \\
\addlinespace
\multirow{2}{*}{\textit{Voix}}
& Voix humaine vs synthétique & Humaine > Synthétique & \citet{davis2018impact} \\
& Style conversationnel & Effet positif modéré & \citet{wang2023effects} \\
\addlinespace
\multirow{2}{*}{\textit{Gestes}}
& Gestes déictiques & Effet positif (transfert) & \citet{davis2018impact} \\
& Gestes génériques & Effet modéré (rétention) & \citet{davis2018impact} \\
\addlinespace
\multirow{2}{*}{\textit{Affect}}
& Expression émotionnelle & Effet positif modéré & \citet{wang2023effects} \\
& Affect positif & Corrélation positive & \citet{guo2015affect} \\
\addlinespace
\multirow{2}{*}{\textit{Interaction}}
& Chatbots IA & Effet positif important & \citet{wu2024chatbots} \\
& Feedback adaptatif & Effet positif & \citet{martha2018design} \\
\bottomrule
\end{tabular}
\end{table}

La \textbf{représentation visuelle} constitue le premier niveau de signaux. La présence d'un agent produit un effet global positif, mais ce résultat masque des variations importantes selon le type de représentation~\citep{castroalonso2021effectiveness}. Les agents 2D surpassent les agents 3D, un écart qui s'explique par le coût cognitif du réalisme~: le traitement d'environnements 3D immersifs consomme des ressources au détriment du contenu pédagogique. L'animation constitue néanmoins un atout~: les agents animés surpassent les agents statiques, l'animation permettant de guider l'attention et de maintenir l'engagement~\citep{dai2024effects}.

La \textbf{voix} représente un signal social particulièrement puissant. Les voix humaines conservent un avantage sur les voix synthétiques, bien que cet écart se réduise avec les progrès technologiques~: les voix synthétiques de haute qualité produisent désormais des résultats d'apprentissage équivalents, voire supérieurs, pour le transfert de connaissances~\citep{craig2017schroeder, davis2018impact}. La qualité prosodique s'avère plus déterminante que l'incarnation visuelle pour la perception de naturalité~: un agent réaliste avec une prosodie inadéquate sera jugé moins naturel qu'un agent désincarné avec une prosodie correcte~\citep{ehret2021}. Au-delà de la qualité vocale, le style discursif influence l'apprentissage~: le style conversationnel produit un effet positif par rapport au style formel~\citep{wang2023effects}. Ce résultat s'aligne avec le principe de personnalisation de la théorie cognitive de l'apprentissage multimédia~: un discours adressé directement à l'apprenant (``vous'') favorise l'engagement par rapport à un discours impersonnel.

Les \textbf{gestes} de l'agent constituent une catégorie de signaux aux effets différenciés. Une méta-analyse distingue deux types de gestes~\citep{davis2018impact}~: les gestes déictiques --- pointage vers des éléments pertinents de l'interface --- produisent un effet important sur le transfert en guidant explicitement l'attention visuelle et en facilitant l'intégration des informations verbales et graphiques~; les gestes génériques --- mouvements non liés au contenu --- produisent un effet plus modeste sur la rétention. Cette différence confirme l'importance de la congruence sémantique~: les gestes doivent être alignés avec le contenu pour maximiser leur efficacité.

La combinaison de plusieurs types de communication non verbale n'amplifie pas nécessairement les effets~: un seul type de signal approprié au résultat d'apprentissage visé (gestes \textit{ou} expressions faciales) surpasse la combinaison des deux~\citep{baylor2009design}. Les expressions faciales favorisent l'apprentissage attitudinal, tandis que les gestes facilitent l'apprentissage procédural. Ce résultat contre-intuitif s'explique par la théorie de la charge cognitive~: la multiplication des signaux peut surcharger l'apprenant plutôt que l'assister. L'inconsistance des résultats dans la littérature confirme que l'efficacité des signaux non verbaux dépend davantage de leur appropriation au contexte que de leur quantité~\citep{wang2021examining}.

L'\textbf{affect} de l'agent influence l'apprentissage à travers plusieurs mécanismes. Une corrélation positive existe entre affect de l'agent et apprentissage~\citep{guo2015affect}. L'expression émotionnelle se distingue de la simple présence affective~\citep{wang2023effects}. Les expressions faciales doivent être alignées avec le contexte pour produire leurs effets~\citep{liew2016}~; lorsqu'elles sont coordonnées avec le discours, elles influencent positivement la motivation~\citep{liew2017}. Toutefois, l'expressivité émotionnelle positive peut s'avérer contre-productive~: elle peut sembler inappropriée et nuire à la confiance lors de tâches critiques où la neutralité est attendue~\citep{torre2019}. Les effets affectifs sont également plus marqués sur les mesures motivationnelles que sur les mesures cognitives pures~\citep{tao2022exploring}.

La dimension \textbf{interactive} distingue les agents conversationnels des agents unidirectionnels. Les chatbots IA en éducation produisent des effets positifs importants~\citep{wu2024chatbots}, suggérant que la capacité de dialogue --- recevoir les questions de l'apprenant et y répondre de manière contextuelle --- constitue un amplificateur majeur de l'efficacité pédagogique.

La \textbf{motivation} constitue une variable médiatrice importante. Les effets des chatbots éducatifs varient selon le type de motivation mesurée~: l'auto-efficacité montre des effets importants, tandis que la motivation intrinsèque présente des effets plus modérés~\citep{gladstone2025motivation}. Ces résultats suggèrent que les agents agissent principalement en renforçant la confiance de l'apprenant dans ses capacités plutôt qu'en modifiant son intérêt intrinsèque pour le contenu.

L'\textbf{alignement thématique} entre l'agent et le contenu constitue un facteur modulateur. Les apprenants interagissant avec un agent thématiquement aligné (un astronaute enseignant des concepts spatiaux) rapportent des niveaux supérieurs de présence sociale et d'engagement comparés à ceux interagissant avec un agent neutre~\citep{schmidt2019}. Ce résultat suggère que la cohérence entre l'identité de l'agent et le domaine enseigné peut amplifier les effets des signaux sociaux.

Une analyse transversale révèle un patron cohérent~: les effets des signaux sociaux sont systématiquement plus importants sur les mesures affectives et motivationnelles que sur les mesures cognitives. Sur 15 études avec groupe contrôle, 9 ne montrent aucune différence d'apprentissage significative, alors que les effets sur l'engagement sont généralement positifs~\citep{heidig2011pedagogical}. Une dissociation frappante émerge entre perception et performance~: les apprenants apprécient davantage la leçon, la jugent plus intéressante et ont l'impression de mieux apprendre, mais leurs résultats de compréhension objective diminuent --- un phénomène résumé par l'expression ``\textit{more gaze, but less learning}''~\citep{wilson2018}.

La figure~\ref{fig:signaux_efficacite} visualise ces relations entre signaux sociaux et types d'effets.

\begin{figure}[htbp]
\centering
\fbox{\parbox{0.9\textwidth}{\centering\vspace{2cm}
\textit{Figure à insérer~: Modèle des effets des signaux sociaux}\\[0.5em]
Schéma montrant~:\\
(1) Les catégories de signaux (représentation, voix, gestes, affect, interaction)\\
(2) Les mécanismes médiateurs (présence sociale, engagement, charge cognitive)\\
(3) Les types d'effets (affectif, motivationnel, cognitif)\\
Avec indication des tailles d'effet pour chaque relation
\vspace{2cm}}}
\caption{Modèle intégratif des effets des signaux sociaux des agents pédagogiques. Les signaux influencent l'apprentissage à travers des mécanismes médiateurs, avec des effets différenciés selon le type de mesure.}
\label{fig:signaux_efficacite}
\end{figure}

Ces résultats convergent vers une recommandation de parcimonie stratégique~: multiplier les signaux sociaux n'amplifie pas nécessairement l'efficacité pédagogique. Chaque signal doit être évalué selon son rapport bénéfice (engagement, guidage attentionnel) / coût (charge cognitive, distraction). Cette tension entre richesse des signaux et surcharge potentielle conduit naturellement à examiner les limites du réalisme dans la conception des agents.
