\subsection{La présence sociale et le paradigme CASA}
\label{subsec:presence_sociale_CASA}

La présence sociale désigne le sentiment subjectif d'être en présence d'une autre entité intelligente \citep{biocca2003}. Ce construit comporte trois dimensions. La co-présence désigne la perception qu'un autre être partage le même espace. L'implication psychologique renvoie à l'attention portée à cet être. L'engagement comportemental correspond à la tendance à interagir avec lui. La présence sociale n'est pas une propriété de la technologie. Un même agent peut susciter un fort sentiment de présence chez un apprenant et aucun chez un autre, selon le contexte, les attentes et la qualité des indices sociaux. Ce point est décisif pour le design pédagogique : la question n'est pas de savoir si un agent \textit{est} social, mais s'il \textit{est perçu} comme tel.

Le paradigme CASA (\textit{Computers Are Social Actors}) explique comment cette perception émerge. Les travaux fondateurs montrent que les utilisateurs appliquent spontanément aux ordinateurs des règles réservées aux interactions humaines : politesse, réciprocité, attribution de traits de personnalité \citep{nass1994}. Un ordinateur qui se présente comme un ``partenaire d'équipe'' reçoit des évaluations plus favorables qu'un ordinateur neutre --- alors même que les participants déclarent savoir qu'il s'agit d'une machine \citep{nass2000}. Le mécanisme en jeu n'est pas une croyance consciente dans l'humanité de la machine. Il relève de ce que Nass nomme \textit{mindlessness} : l'activation automatique de scripts sociaux en réponse à des indices minimaux. La présence d'une voix, d'un nom ou d'un tour de parole suffit à déclencher ces réponses.

Les conséquences pour l'apprentissage découlent directement de ce mécanisme. Si l'apprenant traite l'agent comme un partenaire social, il s'engage dans un contrat conversationnel implicite : effort accru pour comprendre le message, traitement plus profond de l'information, volonté de réciprocité dans l'échange. La théorie de l'agence sociale formalise cette chaîne causale : les indices sociaux activent la présence sociale, qui génère un partenariat perçu, lequel augmente l'effort cognitif et améliore l'apprentissage \citep{mayer2012}. La perception d'un interlocuteur transforme un exercice de traitement d'information en échange social --- un déplacement qui peut faciliter la transition vers le mode interactif du cadre ICAP (cf.~\ref{subsec:icap}), où l'apprenant co-construit le savoir plutôt que de le recevoir.

L'intensité de la présence sociale varie avec le degré d'anthropomorphisme de l'agent. La tendance humaine à attribuer des caractéristiques humaines aux entités non humaines suit un gradient : plus un agent présente d'indices humains, plus il est perçu comme doté d'intentions et de compétences \citep{epley2007}. Ce gradient s'observe empiriquement dans un continuum qui va du texte seul à un agent doté d'une voix, d'un visage et d'un corps \citep{gong2008}. L'attribution ne se limite pas à la compétence : les apprenants projettent sur l'agent des intentions, des émotions et une personnalité, ce qui produit un investissement émotionnel dans le processus d'apprentissage \citep{kim2006}. Chaque ajout d'indice social augmente la crédibilité perçue et la confiance accordée à l'agent \citep{li2023}. La congruence entre l'apparence de l'agent et le contenu enseigné constitue un facteur supplémentaire : un agent thématiquement aligné avec le domaine --- par exemple un astronaute enseignant l'astronomie --- génère une présence sociale et un engagement supérieurs à ceux d'un agent générique \citep{schmidt2019-et}. Cette hypothèse de congruence se heurte cependant à des résultats contradictoires. Une étude manipulant la tenue vestimentaire d'un agent (appropriée vs formelle) et le décor de la vidéo (atelier vs salon) n'observe aucun effet principal de ces facteurs sur l'apprentissage \citep{decker2023-ss}. Une interaction inattendue émerge : la combinaison tenue appropriée et décor inapproprié produit les meilleurs résultats, avec une taille d'effet faible ($\eta^2_p = 0{,}02$). La tenue appropriée augmente la crédibilité perçue sans affecter l'expertise perçue. Ces données suggèrent que l'effet de congruence thématique ne se généralise pas systématiquement aux aspects vestimentaires et contextuels : le type d'alignement --- thématique, vestimentaire ou environnemental --- pourrait moduler différemment la perception de l'agent et son efficacité pédagogique.

