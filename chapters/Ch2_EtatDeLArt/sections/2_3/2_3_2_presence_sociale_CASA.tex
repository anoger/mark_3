\subsection{La présence sociale et le paradigme CASA}
\label{subsec:presence_sociale_CASA}

La présence sociale désigne le sentiment subjectif d'être en présence d'une autre entité intelligente \citep{biocca2003}. Ce construit comporte trois dimensions : la co-présence (perception d'un espace partagé), l'implication psychologique (attention portée à l'entité) et l'engagement comportemental (tendance à interagir avec elle). Ces dimensions ne sont pas indépendantes. La co-présence constitue un préalable : sans perception d'un agent présent, aucune implication ne se développe. L'implication conditionne à son tour l'engagement. La présence sociale n'est pas une propriété de la technologie. Un même agent peut susciter un fort sentiment de présence chez un apprenant et aucun chez un autre, selon le contexte, les attentes et la qualité des indices sociaux. La question pour le design pédagogique n'est pas de savoir si un agent \textit{est} social, mais s'il \textit{est perçu} comme tel.

Le paradigme CASA (\textit{Computers Are Social Actors}) explique comment cette perception émerge. Les utilisateurs appliquent spontanément aux ordinateurs des règles réservées aux interactions humaines : politesse, réciprocité, attribution de traits de personnalité \citep{nass1994}. Un ordinateur présenté comme un \og partenaire d'équipe \fg{} reçoit des évaluations plus favorables qu'un ordinateur neutre. Les participants déclarent pourtant savoir qu'il s'agit d'une machine \citep{nass2000}. Le mécanisme n'est pas une croyance consciente. Nass le nomme \textit{mindlessness} : l'activation automatique de scripts sociaux en réponse à des indices minimaux. Une voix, un nom ou un tour de parole peut suffire à déclencher ces réponses. Le caractère automatique du processus est important : il opère indépendamment de ce que l'utilisateur croit sur la nature de son interlocuteur.

L'intensité de cette activation varie avec le degré d'anthropomorphisme de l'agent. La tendance à attribuer des caractéristiques humaines aux entités non humaines suit un gradient \citep{epley2007}. Un agent qui présente davantage d'indices humains tend à être perçu comme doté d'intentions et de compétences, dans les limites décrites en~\ref{subsec:uncanny_valley}. Ce gradient s'observe dans un continuum allant du texte seul à un agent doté d'une voix, d'un visage et d'un corps \citep{gong2008}. L'attribution ne se limite pas à la compétence. Les apprenants projettent sur l'agent des intentions, des émotions et une personnalité \citep{kim2006}. Chaque ajout d'indice social augmente la crédibilité perçue et la confiance accordée \citep{li2023}.

Ces mécanismes --- activation automatique de scripts sociaux et attribution graduée de compétences --- ont des conséquences directes sur l'apprentissage. Si l'apprenant traite l'agent comme un partenaire social, il s'engage dans un contrat conversationnel implicite. Ce contrat peut se manifester par un effort accru de compréhension, un traitement plus profond de l'information et une volonté de réciprocité. La perception d'un interlocuteur transforme un exercice de traitement d'information en échange social. Ce déplacement peut faciliter la transition vers le mode interactif du cadre ICAP (cf.~\ref{subsec:icap}), où l'apprenant co-construit le savoir plutôt que de le recevoir passivement. La théorie de l'agence sociale formalise cette chaîne causale et en détaille les conditions. Elle fait l'objet de la section suivante.

