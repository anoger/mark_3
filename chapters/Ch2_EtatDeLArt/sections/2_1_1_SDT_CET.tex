% ============================================================================
% SOUS-SECTION 2.1.1 - Dynamiques Motivationnelles : SDT et CET
% ============================================================================

\subsection{Dynamiques Motivationnelles : SDT et CET}

La compréhension des processus d'apprentissage repose sur l'identification des mécanismes psychologiques qui orientent le comportement des apprenants. La Théorie de l'Autodétermination (SDT) propose un cadre explicatif articulé autour de trois besoins psychologiques fondamentaux dont la satisfaction conditionne le développement d'une motivation de qualité \citep{DeciRyan2000}. Le besoin d'autonomie correspond à la nécessité de se percevoir comme l'agent causal de ses propres actions, plutôt que contrôlé par des forces externes. Le besoin de compétence renvoie au sentiment d'efficacité dans ses interactions avec l'environnement et à la capacité d'atteindre les résultats souhaités. Le besoin d'affiliation, enfin, traduit la recherche de connexions sociales significatives et le sentiment d'appartenance à un groupe \citep{deci1985}. Ces trois besoins ne fonctionnent pas de manière isolée : leur satisfaction conjointe dans un contexte donné détermine le degré d'internalisation des comportements et la qualité de l'engagement dans une activité. Les environnements qui offrent des choix significatifs, des défis adaptés au niveau de l'apprenant, des retours constructifs et des interactions sociales soutenantes tendent à favoriser le développement d'une motivation autodéterminée \citep{DeciRyan2000}.

La Théorie de l'Évaluation Cognitive (CET), intégrée à la SDT, précise les mécanismes par lesquels les événements externes modulent la motivation intrinsèque \citep{DeciRyan1985}. Selon ce cadre théorique, l'effet d'un événement sur la motivation dépend de son interprétation par l'individu selon trois dimensions fonctionnelles. Les aspects informationnels fournissent un retour sur la compétence de l'individu : un feedback positif tend à renforcer le sentiment d'efficacité et, par conséquent, la motivation intrinsèque. Les aspects contrôlants exercent une pression sur le comportement et orientent l'individu vers des résultats spécifiques, ce qui déplace le locus de causalité perçu vers l'extérieur et tend à diminuer la motivation intrinsèque. Les aspects amotivants, enfin, signalent une incompétence et sapent à la fois le sentiment d'efficacité et l'envie de poursuivre l'activité. Un même événement, tel qu'une récompense ou un feedback de l'enseignant, peut être interprété différemment selon le contexte et la manière dont il est présenté. Une récompense perçue comme une reconnaissance de compétence aura un effet différent d'une récompense perçue comme un moyen de contrôle comportemental \citep{DeciRyan1985}.

L'application de ces cadres théoriques aux contextes éducatifs révèle une tendance préoccupante : la motivation intrinsèque tend à décliner tout au long de la scolarité, avec une inflexion particulièrement marquée lors de la transition vers l'enseignement secondaire \citep{GnambsHanfstingl2016}. Cette période se caractérise non seulement par une augmentation de la pression liée à la performance, mais également par des changements biologiques et une élévation des niveaux d'anxiété chez les élèves \citep{GnambsHanfstingl2016}. Les structures scolaires traditionnelles, en limitant les opportunités d'exploration et en privilégiant des formats d'enseignement transmissifs, peuvent restreindre l'expression de la curiosité naturelle des élèves \citep{Engel2009, Engel2011}. Ce déclin motivationnel ne constitue pas une fatalité développementale mais reflète en partie l'inadéquation entre les besoins psychologiques des apprenants et les caractéristiques de leur environnement éducatif. Les disciplines perçues comme éloignées des préoccupations quotidiennes des élèves, telles que l'histoire, peuvent se trouver particulièrement affectées par ce phénomène.

La motivation, comprise comme la direction et l'intensité de l'effort vers un but, se distingue conceptuellement de l'intérêt, qui désigne un état psychologique caractérisé par une attention focalisée, un affect positif et une volonté de réengagement avec un contenu spécifique \citep{Bergin1999}. Alors que la motivation peut être orientée vers des objectifs extrinsèques sans rapport avec le contenu lui-même, l'intérêt implique une relation particulière avec un domaine ou une activité spécifique. Une activité est considérée comme intrinsèquement motivante lorsqu'elle est poursuivie en l'absence de récompense externe apparente \citep{DeciPorac1978}. L'intérêt, en revanche, inclut des émotions positives envers l'objet et une activité autodirigée, non instrumentale \citep{Bergin1999}. Cette distinction possède des implications pratiques : un élève peut être motivé à réussir un examen d'histoire pour des raisons extrinsèques sans pour autant développer un intérêt pour la discipline elle-même.

% ============================================================================
% RÉFÉRENCES UTILISÉES DANS CETTE SOUS-SECTION :
% - Deci & Ryan (1985) : CET, mécanismes de la motivation intrinsèque
% - Deci & Ryan (2000) : SDT, trois besoins psychologiques fondamentaux
% - Gnambs & Hanfstingl (2016) : déclin motivation intrinsèque, transition secondaire
% - Engel (2009, 2011) : structures scolaires et curiosité
% - Bergin (1999) : distinction intérêt/motivation
% - Deci & Porac (1978) : définition motivation intrinsèque
% ============================================================================
