% ============================================================================
% Section 2.4 : La Rupture Technologique des IA Génératives en Éducation
% ============================================================================
% Objectif : Contextualiser l'étude dans l'ère post-2023 et les nouvelles
%            affordances des LLM
% Calibrage : ~100 mots (introduction)
% ============================================================================

\section{La Rupture Technologique des IA Génératives en Éducation}
\label{sec:IA_generatives}

Les cadres théoriques présentés jusqu'ici ont été élaborés dans un contexte technologique où les agents pédagogiques fonctionnaient sur la base de scripts prédéfinis. L'émergence des Grands Modèles de Langage (LLM) et des technologies de synthèse multimédia a transformé ce paysage, créant de nouvelles possibilités mais aussi de nouveaux risques.

Cette section examine le saut qualitatif représenté par les agents génératifs (\S\ref{subsec:agents_generatifs}), leurs capacités de personnalisation en temps réel (\S\ref{subsec:personnalisation_temps_reel}), et le défi épistémique posé par leur propension aux hallucinations (\S\ref{subsec:hallucinations}).

% Inclusion des sous-sections
% ============================================================================
% Sous-section 2.4.1 : Des Agents Scriptés aux Agents Génératifs
% ============================================================================
% Sources : IJCCI_extraction (§2.4), illusion_extraction (§2.2.3)
% Calibrage : ~500 mots
% Type : C (Rédaction originale)
% ============================================================================

\subsection{Des Agents Scriptés aux Agents Génératifs~: le Saut Qualitatif}
\label{subsec:agents_generatifs}

Les agents pédagogiques ont évolué au cours des dernières décennies, passant de simples outils de diffusion d'information à des partenaires d'apprentissage plus sophistiqués~\citep{johnson2016face}. L'émergence des Grands Modèles de Langage (\textit{Large Language Models}, LLM) a particulièrement accéléré cette évolution.

Les agents pédagogiques traditionnels fonctionnaient sur la base de scripts prédéfinis et d'arbres de décision. Leur comportement était entièrement déterminé par les anticipations de leurs concepteurs~: chaque question possible devait être prévue, chaque réponse pré-rédigée. Cette architecture présentait des avantages --- prévisibilité, contrôle du contenu, absence d'erreurs factuelles --- mais aussi des limites fondamentales. La rigidité constituait le principal écueil~: l'agent ne pouvait répondre qu'aux questions anticipées, dans les formulations anticipées. Toute déviation se heurtait à des réponses génériques. Cette rigidité contrastait avec la fluidité du dialogue humain et limitait le sentiment de présence sociale.

Les LLM ont transformé cette équation. Ces modèles peuvent générer des réponses personnalisées, s'adapter aux besoins des élèves en temps réel, et produire du contenu éducatif contextuellement pertinent~\citep{kasneci2023chatgpt, labadze2023role}. Ces avancées s'inscrivent dans la continuité des travaux sur les environnements d'apprentissage multimodaux interactifs~\citep{moreno2007interactive}. La fluidité conversationnelle des LLM constitue leur caractéristique la plus distinctive~: ils génèrent un discours structuré, linguistiquement cohérent, et adapté au contexte de l'échange. Cette fluidité peut activer les mécanismes d'agence sociale décrits en \S\ref{subsec:agence_sociale}~: l'apprenant perçoit l'agent comme un interlocuteur plutôt qu'un système automatique.

Cette évolution s'accompagne d'avancées parallèles en synthèse multimédia. Les réseaux antagonistes génératifs (GAN) permettent de créer des représentations visuelles hyperréalistes, brouillant la frontière entre réel et artificiel~\citep{whittaker2020deepfakes}. L'animation faciale par apprentissage profond, le clonage vocal, et la génération vidéo atteignent des niveaux de réalisme inédits. Ces technologies peuvent atténuer l'effet de vallée de l'étrange (cf. \S\ref{subsec:uncanny_valley}), produisant des agents synthétiques perçus comme attractifs~\citep{xu2025recorded}. Certains travaux indiquent que ces agents peuvent atteindre des niveaux de performance et de perception comparables à ceux d'instructeurs humains~\citep{leiker2023generative, lim2024potential}.

C'est précisément cette convergence --- moteurs conversationnels fluides et interfaces visuelles réalistes --- qui multiplie les enjeux. D'un côté, des \og moteurs\fg{} capables de produire un discours éloquent~; de l'autre, des \og interfaces\fg{} visuelles pour les incarner. Cette combinaison ouvre des possibilités pédagogiques inédites, mais comporte également des risques que les sections suivantes examineront.

% ============================================================================
% Sous-section 2.4.2 : La Personnalisation en Temps Réel
% ============================================================================
% Sources : IJCCI_extraction (§2.4), Leong et al., Pataranutaporn et al.
% Calibrage : ~450 mots
% Type : C (Rédaction originale)
% ============================================================================

\subsection{La Personnalisation en Temps Réel}
\label{subsec:personnalisation_temps_reel}

L'une des capacités distinctives des agents alimentés par LLM réside dans leur aptitude à personnaliser le contenu d'apprentissage en temps réel, sans nécessiter de programmation préalable pour chaque profil d'apprenant. Cette capacité répond aux défis de mise en œuvre identifiés en \S\ref{subsec:personnalisation}.

L'apprentissage adaptatif traditionnel reposait sur des algorithmes prédéfinis~: en fonction des réponses de l'élève à des questions diagnostiques, le système orientait vers des parcours prédéterminés~\citep{walkington2018personalization}. Les LLM permettent une forme d'adaptation plus fine~: l'agent peut moduler son vocabulaire, ses exemples, son niveau de détail en fonction du flux de la conversation elle-même.

L'étude de la personnalisation dans l'apprentissage du vocabulaire illustre ce potentiel~: un système développant des exemples et des récits adaptés aux intérêts individuels conduit à une augmentation de la motivation intrinsèque et à un sentiment renforcé de compétence et d'autonomie~\citep{leong2024putting}. Ces résultats démontrent la faisabilité d'une personnalisation automatique à grande échelle. L'exploration des interactions conversationnelles avec des figures historiques à travers des interfaces textuelles indique une amélioration de la motivation et des résultats d'apprentissage comparés à la lecture traditionnelle~\citep{pataranutaporn2023living}. L'évaluation de l'interactivité textuelle en éducation financière confirme que permettre aux étudiants de dialoguer avec l'instructeur virtuel conduit à une motivation et un engagement accrus comparés à l'instruction vidéo passive~\citep{prasongpongchai2024influence}.

Ce qui distingue la personnalisation par LLM est son caractère émergent. L'adaptation n'est pas programmée explicitement~: elle émerge de la capacité du modèle à générer des réponses contextuellement appropriées. L'élève pose une question selon ses propres termes, l'agent répond en s'adaptant. Si l'élève manifeste une incompréhension, l'agent peut reformuler spontanément. Cette fluidité adaptative présente un avantage pédagogique~: chaque interaction devient unique, calibrée sur les besoins du moment.

Elle présente également un risque~: l'adaptation peut masquer l'absence de compréhension réelle. L'agent qui reformule efficacement peut donner l'impression à l'élève qu'il a compris, alors que c'est l'agent qui a simplifié son discours au point de ne plus transmettre le concept dans sa complexité. Ce mécanisme constitue l'une des sources potentielles de l'illusion de compréhension.

L'examen des agents conversationnels conçus pour favoriser la curiosité chez les enfants du primaire révèle des résultats prometteurs~: un agent encourageant le questionnement divergent conduit à une amélioration de la qualité des questions et à des activités exploratoires soutenues~\citep{abdelghani2024exploring}. Ces résultats suggèrent que la personnalisation peut être mise au service de l'engagement cognitif authentique plutôt que de la simple facilitation.

% ============================================================================
% Sous-section 2.4.3 : Fiabilité et Hallucinations — Le Défi Épistémique
% ============================================================================
% Sources : illusion_extraction (§2.2.2), Zhang et al. (2025)
% Calibrage : ~450 mots
% Type : C (Rédaction originale)
% ============================================================================

\subsection{Fiabilité et Hallucinations~: le Défi Épistémique}
\label{subsec:hallucinations}

La fluidité conversationnelle des LLM masque une faille intrinsèque~: leur propension à générer des informations factuellement incorrectes présentées avec une confiance apparente. Ce phénomène, qualifié d'\og hallucination\fg{}, représente un défi épistémique majeur pour les applications éducatives.

Les LLM sont des modèles probabilistes qui prédisent le mot suivant le plus probable étant donné le contexte. Cette nature stochastique implique qu'ils ne \og connaissent\fg{} pas les faits au sens humain~: ils génèrent des séquences statistiquement plausibles~\citep{zhang2025sirens}. Une taxonomie des hallucinations distingue les \textit{hallucinations factuelles} (assertions fausses sur le monde), les \textit{hallucinations de fidélité} (déformations de l'information source), et les \textit{hallucinations d'entrée} (fabrication d'éléments non présents dans la requête). Dans un contexte éducatif, chaque type présente des risques spécifiques~: date erronée, citation inventée, personnage historique attribué à la mauvaise période.

L'enseignement de l'Histoire présente une vulnérabilité particulière à ces hallucinations. La discipline repose sur des faits précis --- dates, lieux, protagonistes --- dont l'exactitude est vérifiable. Or, les LLM excellent dans la production de récits plausibles et cohérents~; ils peuvent générer une narration parfaitement fluide qui contient néanmoins des erreurs factuelles. Cette tension est d'autant plus problématique que la fluidité du discours constitue un signal de crédibilité (cf. \S\ref{subsec:fluency_heuristic})~: un récit bien construit est intuitivement perçu comme plus vrai qu'un récit hésitant~\citep{reber1999effects}. Le LLM, en produisant un discours maximalement fluide, maximise cette heuristique --- indépendamment de la véracité de ce qu'il affirme.

Les avancées technologiques soulèvent plusieurs défis connexes. L'engagement des élèves peut fluctuer en raison d'effets de nouveauté~\citep{fryer2019bots}, tandis que les questions de fiabilité, de biais, de confidentialité et d'intégrité académique nécessitent une attention particulière~\citep{labadze2023role, dempere2023impact, berson2024childrens}. Au-delà de l'établissement de lignes directrices pour l'intégration de l'IA en éducation, des opportunités de recherche existent pour examiner comment l'interaction verbale avec des agents pourrait compléter les pratiques pédagogiques actuelles. La modalité orale, en tirant parti des dynamiques naturelles de la classe, pourrait créer des patterns d'engagement différents comparés aux interactions textuelles individuelles~\citep{moreno2007interactive}.

Ces considérations informent directement notre programme expérimental. L'Étude~2, en exposant les participants à un agent capable de produire des informations historiques incorrectes mais présentées avec fluidité, teste précisément ce risque~: la fluidité de l'agent conduit-elle les élèves à accepter des informations fausses et à surestimer leur propre compréhension?

