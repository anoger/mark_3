% ============================================================================
% Sous-section 2.4.2 : La Personnalisation de l'Apprentissage par IA
% ============================================================================
% Sources :
%   - IJCCI_extraction : §2.4 (Leong et al., 2024; Pataranutaporn et al., 2023)
%   - Rapport Technologies Éducatives : Adaptive Learning
%   - Vault.xlsx : références personnalisation IA
% Calibrage : ~500-600 mots
% Type : C (Rédaction originale)
% ============================================================================

\subsection{La Personnalisation de l'Apprentissage par IA}
\label{subsec:personnalisation_ia}

% TODO: Rédiger à partir des sources identifiées
%
% Structure attendue :
% 1. Le concept d'apprentissage adaptatif (Adaptive Learning)
%    - Ajustement du contenu au profil de l'apprenant
%    - Adaptation du rythme et de la difficulté
%    - Historique : systèmes à règles vs. systèmes génératifs
%
% 2. Adaptation sémantique et stylistique en temps réel
%    - Vocabulaire ajusté au niveau de l'élève
%    - Exemples personnalisés selon les intérêts (Leong et al., 2024)
%    - Ton conversationnel modulable
%
% 3. Personnalisation de la représentation (Pataranutaporn et al., 2023)
%    - Choix du personnage incarné
%    - Exploitation de la familiarité (figures admirées)
%    - Transfert de crédibilité depuis relations existantes
%
% 4. Bénéfices documentés
%    - Augmentation de la motivation intrinsèque
%    - Sentiment accru de compétence et d'autonomie
%    - Amélioration de l'engagement (mais pas toujours des outcomes)
%
% Extrait IJCCI_extraction :
% "In vocabulary learning, Leong et al. (2024) studied content personalization
% using generative AI. Their system developed customized examples and narratives
% based on individual interests, leading to increased intrinsic motivation and
% enhanced feelings of competence and autonomy among participants."
%
% Extrait illusion_extraction :
% "The use of visual representations based on public personalities admired by
% the learner has proven to be a lever for improving motivation and perceived
% credibility (Pataranutaporn et al., 2022)."
%
% Citations clés :
%   \cite{leong2024generative} — Personnalisation vocabulaire
%   \cite{pataranutaporn2023living} — Personnages historiques IA
