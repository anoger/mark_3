% ============================================================================
% Sous-section 2.4.2 : La Personnalisation en Temps Réel
% ============================================================================
% Sources : IJCCI_extraction (§2.4), Leong et al., Pataranutaporn et al.
% Calibrage : ~450 mots
% Type : C (Rédaction originale)
% ============================================================================

\subsection{La Personnalisation en Temps Réel}
\label{subsec:personnalisation_temps_reel}

L'une des capacités distinctives des agents alimentés par LLM réside dans leur aptitude à personnaliser le contenu d'apprentissage en temps réel, sans nécessiter de programmation préalable pour chaque profil d'apprenant. Cette capacité répond aux défis de mise en œuvre identifiés en \S\ref{subsec:personnalisation}.

L'apprentissage adaptatif traditionnel reposait sur des algorithmes prédéfinis~: en fonction des réponses de l'élève à des questions diagnostiques, le système orientait vers des parcours prédéterminés~\citep{walkington2018personalization}. Les LLM permettent une forme d'adaptation plus fine~: l'agent peut moduler son vocabulaire, ses exemples, son niveau de détail en fonction du flux de la conversation elle-même.

L'étude de la personnalisation dans l'apprentissage du vocabulaire illustre ce potentiel~: un système développant des exemples et des récits adaptés aux intérêts individuels conduit à une augmentation de la motivation intrinsèque et à un sentiment renforcé de compétence et d'autonomie~\citep{leong2024putting}. Ces résultats démontrent la faisabilité d'une personnalisation automatique à grande échelle. L'exploration des interactions conversationnelles avec des figures historiques à travers des interfaces textuelles indique une amélioration de la motivation et des résultats d'apprentissage comparés à la lecture traditionnelle~\citep{pataranutaporn2023living}. L'évaluation de l'interactivité textuelle en éducation financière confirme que permettre aux étudiants de dialoguer avec l'instructeur virtuel conduit à une motivation et un engagement accrus comparés à l'instruction vidéo passive~\citep{prasongpongchai2024influence}.

Ce qui distingue la personnalisation par LLM est son caractère émergent. L'adaptation n'est pas programmée explicitement~: elle émerge de la capacité du modèle à générer des réponses contextuellement appropriées. L'élève pose une question selon ses propres termes, l'agent répond en s'adaptant. Si l'élève manifeste une incompréhension, l'agent peut reformuler spontanément. Cette fluidité adaptative présente un avantage pédagogique~: chaque interaction devient unique, calibrée sur les besoins du moment.

Elle présente également un risque~: l'adaptation peut masquer l'absence de compréhension réelle. L'agent qui reformule efficacement peut donner l'impression à l'élève qu'il a compris, alors que c'est l'agent qui a simplifié son discours au point de ne plus transmettre le concept dans sa complexité. Ce mécanisme constitue l'une des sources potentielles de l'illusion de compréhension.

L'examen des agents conversationnels conçus pour favoriser la curiosité chez les enfants du primaire révèle des résultats prometteurs~: un agent encourageant le questionnement divergent conduit à une amélioration de la qualité des questions et à des activités exploratoires soutenues~\citep{abdelghani2024exploring}. Ces résultats suggèrent que la personnalisation peut être mise au service de l'engagement cognitif authentique plutôt que de la simple facilitation.
