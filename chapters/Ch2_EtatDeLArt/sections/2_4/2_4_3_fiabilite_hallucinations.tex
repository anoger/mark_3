\subsection{Fiabilité et hallucinations : le défi épistémique}
\label{subsec:fiabilite_hallucinations}

Les LLM génèrent du texte en prédisant le token le plus probable étant donné le contexte. Cette génération est intrinsèquement stochastique : le même prompt peut produire des réponses différentes. Le modèle ne dispose pas d'une représentation du monde qu'il consulterait pour vérifier ses énoncés. Il produit des séquences statistiquement plausibles, sans mécanisme interne de validation factuelle. Cette architecture explique le phénomène des hallucinations : la génération d'informations factuellement incorrectes présentées avec une apparente certitude \citep{zhang2025}.

Ces hallucinations peuvent concerner des faits (dates erronées, événements inventés), la fidélité au contexte d'entrée ou la validité logique du raisonnement \citep{zhang2025}. Quelle que soit leur forme, elles partagent une caractéristique commune : le modèle les produit avec la même fluence et la même assurance apparente que ses énoncés corrects. Aucun signal textuel ne distingue le vrai du faux.

Cette absence de marqueurs d'incertitude pose un problème spécifique en contexte pédagogique. L'apprenant ne dispose pas d'indices textuels pour distinguer les énoncés fiables des énoncés erronés. La fluence du discours --- sa fluidité, sa cohérence apparente, son absence d'hésitation --- agit comme un signal de compétence. Le mécanisme de la fluence de traitement (\textit{processing fluency}) éclaire ce biais : une information facile à traiter cognitivement est perçue comme plus vraie, plus familière et plus fiable \citep{reber1999}. Les LLM maximisent cette fluence par construction : ils sont entraînés à produire un discours naturel et bien structuré.

La tension entre fluence et exactitude prend une forme particulière en histoire. Le discours historique mobilise des récits, des personnages et des enchaînements causaux qui se prêtent naturellement à la narration. Un LLM peut générer un récit historique parfaitement cohérent sur le plan narratif tout en contenant des erreurs factuelles. La plausibilité narrative masque l'inexactitude factuelle. L'apprenant, confronté à un récit fluide et engageant, peut accepter des informations erronées sans les questionner, précisément parce qu'elles s'insèrent dans une trame cohérente.

Une étude de faisabilité sur la reconstruction de Joseph Lister révèle la nature du problème \citep{dacosta2025}. L'agent, construit avec un système de récupération augmentée, produit des réponses fidèles à la voix historique du personnage. Les évaluations détectent cependant des lacunes dans le cadrage temporel et des embellissements occasionnels. Le modèle situe incorrectement certains événements ou enrichit des détails au-delà de ce que les sources permettent d'affirmer.

Ce cas illustre une difficulté structurelle. Les distorsions ne relèvent pas d'erreurs grossières que l'apprenant pourrait détecter. Elles s'insèrent dans un discours par ailleurs cohérent et engageant. Seul un expert du domaine peut les identifier --- précisément le type de connaissance que l'apprenant novice ne possède pas. La récupération augmentée réduit le risque d'hallucinations flagrantes, mais ne garantit pas l'exactitude des nuances, des interprétations et du cadrage temporel. En histoire, ces nuances constituent souvent l'essentiel de la compréhension disciplinaire.

Les données empiriques confirment ce risque et précisent le mécanisme. Les explications générées par IA sont perçues comme plus claires, plus engageantes et plus crédibles que des explications statiques, même lorsque leur contenu est identique \citep{anderl2024, huschens2023}. Cette crédibilité accrue s'explique par le même mécanisme de fluence de traitement : la présentation conversationnelle, fluide et bien structurée, active des heuristiques sociales qui réduisent la vigilance critique. Les utilisateurs surestiment la précision des modèles sur la base de la confiance apparente de leurs réponses \citep{steyvers2025}. Cet écart de calibration ne se réduit que lorsque le modèle signale explicitement son niveau d'incertitude, une fonctionnalité rarement implémentée dans les interfaces grand public.

Le problème se complexifie avec l'incarnation visuelle de l'agent. Un visage humain, des expressions naturelles et une voix convaincante ajoutent des indices de crédibilité au discours. Ces signaux activent la chaîne causale de l'agence sociale (cf.~\ref{subsec:incarnation_agence_sociale}) : l'apprenant perçoit un partenaire et accorde une attention accrue à son discours. La question devient : cette attention accrue favorise-t-elle un traitement critique ou, au contraire, une acceptation confiante ? Les travaux sur la fluence de l'instructeur suggèrent la seconde hypothèse. Un enseignant au discours fluide et aux comportements non verbaux engageants produit des jugements d'apprentissage élevés sans amélioration correspondante de la performance \citep{toftness2018}. L'apprenant confond la qualité de la présentation avec la qualité de sa propre compréhension.

Cette dynamique peut induire une illusion de compréhension. L'apprenant, exposé à un discours fluide et incarné, peut surestimer sa maîtrise du contenu. Les études récentes documentent ce phénomène avec les LLM : l'assistance de l'IA améliore la performance immédiate sur une tâche tout en dégradant la capacité de l'utilisateur à évaluer cette performance \citep{fernandes2026}. La calibration métacognitive habituelle disparaît. L'utilisateur ne sait plus ce qu'il sait vraiment, parce que l'aisance de l'interaction lui donne l'impression de comprendre.

Le risque est amplifié chez les apprenants novices. Ces derniers manquent des connaissances nécessaires pour détecter les erreurs du modèle et pour calibrer leur propre confiance \citep{kruger1999}. Les jeunes apprenants présentent une vulnérabilité supplémentaire : leur tendance à l'anthropomorphisme les conduit à attribuer plus facilement des intentions et des compétences aux agents \citep{kidd2023}. Un agent hyperréaliste au discours fluide peut être perçu comme une autorité fiable, alors même qu'il produit des hallucinations. Le paradoxe est que les apprenants qui auraient le plus besoin d'un accompagnement sont aussi les moins équipés pour en évaluer la fiabilité.

Le contexte disciplinaire module ce risque. En histoire, les erreurs peuvent concerner des faits vérifiables (dates, noms, lieux) mais aussi des interprétations, des causalités et des nuances que l'apprenant novice ne peut pas évaluer. Les misconceptions historiques courantes --- le présentisme, la téléologie, la simplification des causalités --- constituent des pièges particuliers. Une explication incorrecte mais intuitive peut paraître plus convaincante qu'une explication correcte mais contre-intuitive \citep{kulgemeyer2023}. L'alignement avec ces conceptions naïves renforce l'illusion de compréhension : l'explication semble juste parce qu'elle confirme ce que l'apprenant croyait déjà.

Ces constats posent la question de l'illusion de compréhension dans sa dimension métacognitive. La section suivante examine ce phénomène, ses mécanismes et ses manifestations spécifiques dans les interactions avec des agents pédagogiques génératifs.

