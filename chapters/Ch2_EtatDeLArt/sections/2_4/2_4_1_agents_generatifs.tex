% ============================================================================
% Sous-section 2.4.1 : Des Agents Scriptés aux Agents Génératifs
% ============================================================================
% Sources : IJCCI_extraction (§2.4), illusion_extraction (§2.2.3)
% Calibrage : ~500 mots
% Type : C (Rédaction originale)
% ============================================================================

\subsection{Des Agents Scriptés aux Agents Génératifs~: le Saut Qualitatif}
\label{subsec:agents_generatifs}

Les agents pédagogiques ont évolué au cours des dernières décennies, passant de simples outils de diffusion d'information à des partenaires d'apprentissage plus sophistiqués~\citep{johnson2016face}. L'émergence des Grands Modèles de Langage (\textit{Large Language Models}, LLM) a particulièrement accéléré cette évolution.

Les agents pédagogiques traditionnels fonctionnaient sur la base de scripts prédéfinis et d'arbres de décision. Leur comportement était entièrement déterminé par les anticipations de leurs concepteurs~: chaque question possible devait être prévue, chaque réponse pré-rédigée. Cette architecture présentait des avantages --- prévisibilité, contrôle du contenu, absence d'erreurs factuelles --- mais aussi des limites fondamentales. La rigidité constituait le principal écueil~: l'agent ne pouvait répondre qu'aux questions anticipées, dans les formulations anticipées. Toute déviation se heurtait à des réponses génériques. Cette rigidité contrastait avec la fluidité du dialogue humain et limitait le sentiment de présence sociale.

Les LLM ont transformé cette équation. Ces modèles peuvent générer des réponses personnalisées, s'adapter aux besoins des élèves en temps réel, et produire du contenu éducatif contextuellement pertinent~\citep{kasneci2023chatgpt, labadze2023role}. Ces avancées s'inscrivent dans la continuité des travaux sur les environnements d'apprentissage multimodaux interactifs~\citep{moreno2007interactive}. La fluidité conversationnelle des LLM constitue leur caractéristique la plus distinctive~: ils génèrent un discours structuré, linguistiquement cohérent, et adapté au contexte de l'échange. Cette fluidité peut activer les mécanismes d'agence sociale décrits en \S\ref{subsec:agence_sociale}~: l'apprenant perçoit l'agent comme un interlocuteur plutôt qu'un système automatique.

Cette évolution s'accompagne d'avancées parallèles en synthèse multimédia. Les réseaux antagonistes génératifs (GAN) permettent de créer des représentations visuelles hyperréalistes, brouillant la frontière entre réel et artificiel~\citep{whittaker2020deepfakes}. L'animation faciale par apprentissage profond, le clonage vocal, et la génération vidéo atteignent des niveaux de réalisme inédits. Ces technologies peuvent atténuer l'effet de vallée de l'étrange (cf. \S\ref{subsec:uncanny_valley}), produisant des agents synthétiques perçus comme attractifs~\citep{xu2025recorded}. Certains travaux indiquent que ces agents peuvent atteindre des niveaux de performance et de perception comparables à ceux d'instructeurs humains~\citep{leiker2023generative, lim2024potential}.

C'est précisément cette convergence --- moteurs conversationnels fluides et interfaces visuelles réalistes --- qui multiplie les enjeux. D'un côté, des \og moteurs\fg{} capables de produire un discours éloquent~; de l'autre, des \og interfaces\fg{} visuelles pour les incarner. Cette combinaison ouvre des possibilités pédagogiques inédites, mais comporte également des risques que les sections suivantes examineront.
