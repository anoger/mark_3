\subsection{L'heuristique de fluidité}
\label{subsec:heuristique_fluidite}

La fluence de traitement (\textit{processing fluency}) désigne la facilité subjective avec laquelle une information est perçue et traitée cognitivement. Cette facilité n'est pas un indicateur neutre. Elle fonctionne comme une heuristique : le cerveau interprète la facilité de traitement comme un signal de familiarité, de vérité et de fiabilité \citep{reber1999}. Une phrase typographiquement lisible est jugée plus vraie qu'une phrase difficilement déchiffrable. Un nom facile à prononcer inspire plus de confiance qu'un nom complexe. Le mécanisme est automatique et préconscient : l'individu ne décide pas de faire confiance à l'information facile. Il le fait par défaut, sauf effort délibéré de correction.

L'erreur d'attribution qui en résulte est spécifique. L'apprenant tend à confondre la source de la facilité. L'information est facile à traiter non pas parce qu'il la maîtrise, mais parce qu'elle est bien présentée. Il attribue pourtant cette facilité à sa propre compréhension. Le résultat est une dissociation entre confiance subjective, qui augmente, et apprentissage objectif, qui stagne ou décline. Ce mécanisme peut contribuer à expliquer pourquoi les présentations les plus appréciées ne produisent pas toujours les meilleurs résultats d'apprentissage.

L'effet de fluence de l'instructeur (\textit{instructor fluency effect}) illustre cette dissociation dans le contexte pédagogique. Les comportements non verbaux de l'enseignant --- dynamisme, contact visuel, aisance verbale, absence d'hésitation --- peuvent biaiser les jugements d'apprentissage des élèves \citep{toftness2018}. Face à un enseignant au discours fluide, les apprenants rapportent une confiance élevée dans leur apprentissage. Leur performance aux tests ne reflète pas nécessairement cette confiance. L'apprenant tend à confondre la qualité de la prestation avec la qualité de sa propre compréhension. La fluence de l'instructeur agit comme un \textit{proxy} trompeur : elle signale une compétence de présentation, que l'apprenant interprète comme un signal de sa propre maîtrise.

La composante visuelle de l'instruction amplifie cet effet. L'observation passive d'une démonstration --- un tour de magie, un pas de danse, une manipulation technique --- peut générer une illusion d'acquisition de compétence \citep{kardas2018}. L'observateur confond la fluence visuelle du geste (le fait que l'action paraît simple) avec sa propre capacité à l'exécuter. Plus la démonstration est fluide et bien réalisée, plus l'écart entre la confiance et la compétence réelle se creuse. La facilité perçue du stimulus est interprétée comme une facilité d'exécution.

Ce mécanisme interagit avec la présence visuelle de l'instructeur dans les environnements multimédia. Les données oculométriques suggèrent que la présence d'un visage à l'écran tend à détourner une partie de l'attention du contenu pédagogique (cf.~\ref{subsec:incarnation_agence_sociale}). La satisfaction de l'apprenant augmente, mais sa compréhension objective diminue \citep{wilson2018}. L'heuristique de fluidité fournit une explication de cette dissociation : la facilité perceptuelle générée par la présence visuelle est réattribuée à la compréhension. L'apprenant ne distingue pas entre le plaisir de l'interaction et la qualité de son apprentissage. La présence visuelle fonctionne alors comme un amplificateur de l'illusion de compréhension.

Les agents conversationnels alimentés par des LLM tendent à réunir ces conditions de manière prononcée, du fait de leur mode de génération. Leur discours est instantané, structuré, linguistiquement fluide et exempt d'hésitations. Ils ne cherchent pas leurs mots, ne reformulent pas maladroitement, ne marquent aucune pause de réflexion. Ces caractéristiques, qui résultent de la nature même de la génération de texte, produisent un discours dont la fluence peut excéder celle de nombreux instructeurs humains. Présentées en format conversationnel plutôt que comme du texte statique, les explications générées par IA sont perçues comme plus claires, plus engageantes et plus crédibles, même lorsque leur contenu est identique \citep{anderl2024, huschens2023}. La présentation conversationnelle peut activer des heuristiques sociales susceptibles de réduire la vigilance critique. Les utilisateurs tendent à surestimer la précision des modèles sur la base de la confiance apparente de leurs réponses \citep{steyvers2025}. Cet écart de calibration semble se réduire notamment lorsque le modèle signale explicitement son niveau d'incertitude --- une fonctionnalité rarement implémentée dans les interfaces grand public.

La fluence technique des LLM se transforme ainsi en risque pédagogique. Le discours qui rend l'agent convaincant est aussi celui qui peut rendre ses erreurs plus difficiles à détecter. L'apprenant, exposé à un flux verbal fluide et structuré, active l'heuristique de fluidité et en déduit qu'il comprend. Cette dynamique est amplifiée lorsque l'agent qui produit ce discours présente des caractéristiques anthropomorphes. La section suivante examine comment l'incarnation visuelle de l'agent module la vigilance critique de l'apprenant.

