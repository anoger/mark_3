% ============================================================================
% Section 2.5 : Le Phénomène de l'Illusion de Compréhension (IoU)
% ============================================================================
% Objectif : Définir le risque majeur exploré par la thèse,
%            revers de la médaille de la fluidité
% Sources : illusion_extraction (§2.2.1, §2.2.2), Vault.xlsx
% Calibrage : ~2 000 mots
% ============================================================================

\section{Le Phénomène de l'Illusion de Compréhension}
\label{sec:illusion_comprehension}

La fluidité des agents conversationnels génératifs constitue leur force pédagogique principale --- mais aussi leur risque le plus insidieux. Lorsqu'un agent présente l'information de manière parfaitement articulée et convaincante, l'apprenant peut confondre la facilité de réception avec la qualité de sa propre compréhension. Ce phénomène, que nous désignons par \textit{illusion de compréhension} (ou \textit{Illusion of Understanding}, IoU), représente le risque métacognitif central examiné par cette thèse. Cette section analyse d'abord les fondements cognitifs de ce phénomène à travers l'Illusion de Profondeur Explicative (\S\ref{subsec:metacognition_IOED}), puis examine le mécanisme de l'heuristique de fluidité (\S\ref{subsec:fluency_heuristic}), avant d'explorer comment l'autorité perçue de l'agent peut compromettre la vigilance épistémique (\S\ref{subsec:autorite_vigilance}).

% ----------------------------------------------------------------------------
% Sous-sections
% ----------------------------------------------------------------------------
% ============================================================================
% Sous-section 2.5.1 : Métacognition et Calibration de la Confiance
% ============================================================================
% Sources : illusion_extraction (§2.2.1), illusion_ai_agent_draft
% Calibrage : ~550 mots
% Type : C (Rédaction originale)
% ============================================================================

\subsection{Métacognition et Calibration de la Confiance}
\label{subsec:metacognition_IOED}

La métacognition désigne la cognition sur la cognition~: la capacité d'un individu à surveiller et réguler ses propres processus cognitifs~\citep{flavell1979metacognition}. Dans le contexte de l'apprentissage, cette capacité se manifeste par deux fonctions distinctes~: le \textit{monitoring}, qui consiste à évaluer sa propre compréhension, et le \textit{control}, qui permet d'ajuster ses stratégies en conséquence. L'auto-évaluation de la compréhension constitue un processus intrinsèquement faillible~\citep{glenberg1982automatic}~: les apprenants croient fréquemment avoir compris un contenu alors même qu'ils échouent à détecter des contradictions évidentes.

Ce déficit métacognitif trouve sa formalisation dans le concept d'\textit{Illusion of Explanatory Depth} (IOED). Ce phénomène désigne la tendance des individus à surestimer la profondeur de leur compréhension des systèmes causaux complexes --- jusqu'au moment où ils sont contraints de produire une explication détaillée~\citep{rozenblit2002misunderstood}. Le protocole expérimental classique procède en trois phases~: une auto-évaluation initiale (T1), la production d'une explication écrite, puis une seconde auto-évaluation (T2). La révélation est systématique~: confrontés à l'exercice d'explication, les participants découvrent que leur compréhension était moins profonde qu'ils ne le croyaient, se traduisant par une chute significative entre T1 et T2.

Ce phénomène s'avère pertinent pour l'étude des interactions avec les agents conversationnels. Certaines modalités de présentation peuvent donner l'impression que le matériel est facile à traiter~\citep{paik2013effects}. Cette facilité apparente conduit les apprenants à sous-estimer la difficulté de la tâche et à développer une métacompréhension excessivement optimiste. Le résultat est une dissociation entre la confiance subjective, qui augmente, et l'apprentissage objectif, qui stagne.

Cette illusion s'inscrit dans un déficit métacognitif plus large. Le phénomène décrit par l'effet Dunning-Kruger met en lumière un \og double fardeau\fg{}~: les individus les moins compétents manquent également des compétences métacognitives nécessaires pour reconnaître leur propre incompétence~\citep{kruger1999unskilled}. Dans le contexte d'une interaction avec un agent conversationnel fluide, ce déficit devient particulièrement problématique~: la facilité apparente de l'échange peut renforcer une confiance injustifiée.

La notion de \textit{calibration} désigne l'écart entre la confiance subjective et la performance objective~\citep{koriat2008subjective}. Les études révèlent une tendance systématique à la \textit{surconfiance}~: les apprenants surestiment leur niveau de compréhension. Cette surconfiance constitue un prédicteur robuste de mauvais apprentissage, car elle conduit à un désengagement prématuré de l'effort cognitif.

Dans le contexte des agents alimentés par IA générative, ce problème prend une dimension nouvelle. Un \og paradoxe métacognitif\fg{} émerge~: bien que l'assistance de l'IA puisse améliorer la performance immédiate, elle dégrade la capacité de l'utilisateur à évaluer cette même performance~\citep{fernandes2026metacognitive}. L'agent, en fournissant des réponses instantanées et fluides, prive l'apprenant des \og difficultés désirables\fg{} --- l'effort de construction et de réorganisation des connaissances --- pourtant essentielles à un apprentissage durable~\citep{bjork2013self}.

% ============================================================================
% Sous-section 2.5.2 : L'Heuristique de Fluidité (Fluency Heuristic)
% ============================================================================
% Sources : illusion_extraction (§2.2.1, §2.2.2), illusion_ai_agent_draft
% Calibrage : ~550 mots
% Type : C (Rédaction originale)
% ============================================================================

\subsection{L'Heuristique de Fluidité}
\label{subsec:fluency_heuristic}

Le mécanisme central de l'illusion de compréhension réside dans ce que la littérature désigne par \textit{processing fluency}~: l'information facile à percevoir et à traiter cognitivement est vécue comme familière~\citep{reber1999effects}. Cette facilité de traitement est alors attribuée par erreur à la propre maîtrise du sujet par l'apprenant plutôt qu'aux qualités de la présentation. L'heuristique de fluidité constitue un raccourci cognitif par lequel nous jugeons plus vraie, plus crédible et mieux comprise l'information qui \og passe bien\fg{}.

La simple facilité de perception d'un énoncé --- police lisible, articulation claire, formulation syntaxiquement simple --- augmente sa crédibilité perçue, indépendamment de sa véracité objective. L'individu interprète la facilité cognitive comme un signal de familiarité, et la familiarité comme un indice de vérité. Dans le contexte des agents conversationnels alimentés par LLM, cette heuristique prend une dimension particulière~: ces modèles possèdent des caractéristiques techniques qui maximisent la fluidité de traitement --- discours instantané, parfaitement structuré et linguistiquement fluide~\citep{shanahan2024talking}. Cette double fluidité --- linguistique et auditive lorsque couplée à une synthèse vocale --- crée les conditions idéales pour que l'heuristique opère.

Une variante pertinente est l'\textit{instructor fluency effect}, qui décrit comment les comportements non-verbaux de l'enseignant --- dynamisme, contact visuel, fluidité verbale --- peuvent biaiser les jugements d'apprentissage~\citep{toftness2018instructor}. Les apprenants confondent la qualité de la délivrance pédagogique avec la qualité de leur propre apprentissage, rapportant une confiance élevée sans gains de performance correspondants. La présence visuelle d'un instructeur peut augmenter la satisfaction et l'apprentissage perçu tout en détournant l'attention du contenu~\citep{wilson2018instructor}. L'apprenant se trouve dans une situation paradoxale où son expérience subjective positive masque un apprentissage objectivement dégradé.

Au-delà de la compréhension conceptuelle, l'heuristique de fluidité peut générer une \og illusion d'acquisition de compétence\fg{}~: l'observation passive d'une démonstration peut conduire l'observateur à confondre la fluidité de traitement visuel avec sa propre capacité à exécuter la tâche~\citep{kardas2018illusion}. Par analogie, un apprenant qui observe un agent expliquer un phénomène avec aisance peut confondre la clarté de la présentation avec sa propre maîtrise du sujet.

La convergence de ces mécanismes crée un risque métacognitif majeur. L'interaction conversationnelle tend à augmenter la crédibilité perçue et à réduire la détection des inexactitudes par rapport au texte statique, car elle active des heuristiques sociales qui diminuent la vigilance critique~\citep{anderl2024conversational}. Même lorsque le contenu n'est pas jugé globalement plus crédible, il est souvent perçu comme plus clair et plus engageant --- une qualité qui peut conduire à une acceptation non critique~\citep{huschens2023unambiguous}. Cette configuration favorise une forme de \og paresse métacognitive\fg{}, où les apprenants délèguent les processus cognitifs coûteux et renoncent à l'effort de construction personnelle des connaissances~\citep{fan2023metacognitive}.

% ============================================================================
% Sous-section 2.5.3 : L'Autorité de l'Agent et la Vigilance Épistémique
% ============================================================================
% Sources : illusion_extraction (§2.2.1, §2.2.2), illusion_ai_agent_draft
% Calibrage : ~550 mots
% Type : C (Rédaction originale)
% ============================================================================

\subsection{L'Autorité de l'Agent et la Vigilance Épistémique}
\label{subsec:autorite_vigilance}

Au-delà de l'illusion de compréhension --- qui constitue une erreur d'auto-évaluation ---, les agents conversationnels génèrent un second risque épistémique~: la surconfiance accordée à la source elle-même. Cette distinction conceptuelle mérite une analyse séparée.

La surconfiance épistémique ne désigne pas une erreur d'auto-évaluation, mais une erreur concernant l'information externe~\citep{kulgemeyer2023epistemic}. Ce biais se manifeste lorsque les apprenants accordent une confiance excessive à des informations factuellement incorrectes parce que la source apparaît autoritaire ou que l'explication est intuitivement séduisante. Les explications fondées sur des conceptions erronées, parce qu'elles s'alignent avec l'expérience quotidienne, sont souvent jugées plus convaincantes que des explications scientifiquement correctes mais contre-intuitives. Cette dynamique prend une dimension critique face aux \og hallucinations\fg{} des LLM --- la génération d'informations incorrectes présentées avec assurance~\citep{zhang2025hallucination}. La fluidité discursive de l'agent crée les conditions d'une acceptation non critique d'informations potentiellement fausses.

L'\textit{illusory truth effect} constitue un mécanisme cognitif qui contribue à cette surconfiance~\citep{fazio2015knowledge}. La simple répétition d'un énoncé, même faux, augmente sa crédibilité perçue en accroissant sa familiarité. Dans le contexte d'interactions répétées avec un agent, ce mécanisme peut consolider des croyances erronées initialement introduites par une hallucination. Il convient toutefois de distinguer cet effet de l'illusion de maîtrise mesurée par le protocole IOED~: l'effet de vérité illusoire constitue une variable explicative de la persistance des fausses croyances plutôt qu'une mesure de l'auto-évaluation. Les deux phénomènes peuvent néanmoins se renforcer~: un apprenant qui croit maîtriser un sujet sera moins enclin à questionner les informations reçues.

L'anthropomorphisme --- notre tendance à attribuer des caractéristiques humaines à des entités non-humaines~\citep{epley2007seeing} --- joue un rôle central dans l'attribution d'autorité aux agents. L'apparence visuelle déclenche ce processus, mais les travaux récents suggèrent que la fluidité conversationnelle pourrait constituer un signal social de premier ordre~\citep{nass1994computers}. Cette dynamique s'avère préoccupante pour les jeunes apprenants, dont la tendance naturelle à l'anthropomorphisme est plus prononcée~\citep{kidd2023anthropomorphism}. Les adolescents, dont les modèles mentaux de la technologie sont en développement, sont plus dépendants des indices de surface pour évaluer la crédibilité~\citep{andries2023trust}.

La convergence de ces mécanismes --- fluidité de traitement, autorité perçue, anthropomorphisme --- tend à désactiver la vigilance épistémique de l'apprenant. Les explications trompeuses générées par l'IA peuvent être plus persuasives que les explications honnêtes~\citep{danry2024don}. Le \og halo de confiance\fg{} observé dans les interactions avec les agents --- où les élèves généralisent la compétence perçue à l'ensemble du domaine --- agit comme un \og bouclier de crédibilité\fg{}. L'apprenant, ayant catégorisé l'agent comme une source fiable, cesse d'appliquer les filtres critiques qu'il mobiliserait face à une source dont l'autorité n'est pas établie. Ce phénomène de \og négligence épistémique\fg{} opère de manière inconsciente, échappant à la régulation métacognitive de l'individu.


% Transition vers Section 2.6
L'analyse de l'illusion de compréhension révèle une tension fondamentale~: les mêmes caractéristiques qui rendent les agents conversationnels engageants --- fluidité, réalisme, crédibilité --- peuvent simultanément compromettre la profondeur de l'apprentissage en désactivant les mécanismes de vigilance critique. Cette tension entre engagement et vigilance constitue le fil conducteur de notre problématique de recherche, qu'il convient maintenant de formuler explicitement.
