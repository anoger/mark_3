% ============================================================================
% Sous-section 2.5.3 : L'Autorité de l'Agent et la Vigilance Épistémique
% ============================================================================
% Sources : illusion_extraction (§2.2.1, §2.2.2), illusion_ai_agent_draft
% Calibrage : ~550 mots
% Type : C (Rédaction originale)
% ============================================================================

\subsection{L'Autorité de l'Agent et la Vigilance Épistémique}
\label{subsec:autorite_vigilance}

Au-delà de l'illusion de compréhension --- qui constitue une erreur d'auto-évaluation ---, les agents conversationnels génèrent un second risque épistémique~: la surconfiance accordée à la source elle-même. Cette distinction conceptuelle mérite une analyse séparée.

La surconfiance épistémique ne désigne pas une erreur d'auto-évaluation, mais une erreur concernant l'information externe~\citep{kulgemeyer2023epistemic}. Ce biais se manifeste lorsque les apprenants accordent une confiance excessive à des informations factuellement incorrectes parce que la source apparaît autoritaire ou que l'explication est intuitivement séduisante. Les explications fondées sur des conceptions erronées, parce qu'elles s'alignent avec l'expérience quotidienne, sont souvent jugées plus convaincantes que des explications scientifiquement correctes mais contre-intuitives. Cette dynamique prend une dimension critique face aux \og hallucinations\fg{} des LLM --- la génération d'informations incorrectes présentées avec assurance~\citep{zhang2025hallucination}. La fluidité discursive de l'agent crée les conditions d'une acceptation non critique d'informations potentiellement fausses.

L'\textit{illusory truth effect} constitue un mécanisme cognitif qui contribue à cette surconfiance~\citep{fazio2015knowledge}. La simple répétition d'un énoncé, même faux, augmente sa crédibilité perçue en accroissant sa familiarité. Dans le contexte d'interactions répétées avec un agent, ce mécanisme peut consolider des croyances erronées initialement introduites par une hallucination. Il convient toutefois de distinguer cet effet de l'illusion de maîtrise mesurée par le protocole IOED~: l'effet de vérité illusoire constitue une variable explicative de la persistance des fausses croyances plutôt qu'une mesure de l'auto-évaluation. Les deux phénomènes peuvent néanmoins se renforcer~: un apprenant qui croit maîtriser un sujet sera moins enclin à questionner les informations reçues.

L'anthropomorphisme --- notre tendance à attribuer des caractéristiques humaines à des entités non-humaines~\citep{epley2007seeing} --- joue un rôle central dans l'attribution d'autorité aux agents. L'apparence visuelle déclenche ce processus, mais les travaux récents suggèrent que la fluidité conversationnelle pourrait constituer un signal social de premier ordre~\citep{nass1994computers}. Cette dynamique s'avère préoccupante pour les jeunes apprenants, dont la tendance naturelle à l'anthropomorphisme est plus prononcée~\citep{kidd2023anthropomorphism}. Les adolescents, dont les modèles mentaux de la technologie sont en développement, sont plus dépendants des indices de surface pour évaluer la crédibilité~\citep{andries2023trust}.

La convergence de ces mécanismes --- fluidité de traitement, autorité perçue, anthropomorphisme --- tend à désactiver la vigilance épistémique de l'apprenant. Les explications trompeuses générées par l'IA peuvent être plus persuasives que les explications honnêtes~\citep{danry2024don}. Le \og halo de confiance\fg{} observé dans les interactions avec les agents --- où les élèves généralisent la compétence perçue à l'ensemble du domaine --- agit comme un \og bouclier de crédibilité\fg{}. L'apprenant, ayant catégorisé l'agent comme une source fiable, cesse d'appliquer les filtres critiques qu'il mobiliserait face à une source dont l'autorité n'est pas établie. Ce phénomène de \og négligence épistémique\fg{} opère de manière inconsciente, échappant à la régulation métacognitive de l'individu.
