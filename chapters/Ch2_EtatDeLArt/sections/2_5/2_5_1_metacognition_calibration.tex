\subsection{Métacognition et calibration de la confiance}
\label{subsec:metacognition_calibration}

L'auto-évaluation de sa propre compréhension est un processus actif et imparfait. La métacognition désigne cette capacité à surveiller et réguler ses propres processus cognitifs \citep{flavell1979}. Elle implique deux opérations distinctes : estimer ce que l'on sait (\textit{monitoring}) et ajuster son comportement en conséquence (\textit{control}). L'efficacité de l'apprentissage dépend en partie de la précision de cette estimation. Un apprenant qui surestime sa compréhension tend à réduire ses efforts d'approfondissement. Un apprenant qui la sous-estime peut investir des ressources au-delà de ce que la tâche exige. La calibration métacognitive --- l'écart entre confiance subjective et performance objective --- constitue un indicateur de la qualité de l'autorégulation.

Cette calibration est fréquemment défaillante. Des apprenants échouent à détecter des contradictions explicites dans un texte qu'ils affirment avoir compris \citep{glenberg1982}. Le phénomène ne semble pas relever principalement d'un manque d'attention. Il reflète une difficulté structurelle : évaluer sa compréhension requiert les mêmes compétences que comprendre. Les individus les moins performants manquent aussi des ressources métacognitives nécessaires pour reconnaître leur propre incompétence \citep{kruger1999}. Ce double déficit --- faible performance et faible calibration --- suggère que les apprenants qui auraient le plus besoin de régulation sont aussi ceux qui en disposent le moins.

L'Illusion de Compréhension (\textit{Illusion of Understanding}, IoU) désigne la forme spécifique de ce déficit où la confiance subjective dans sa propre maîtrise d'un sujet excède la connaissance objectivement mesurée. Ce biais se distingue de deux autres phénomènes avec lesquels il est parfois confondu. La surconfiance envers la source (\textit{epistemic overconfidence}) porte sur l'information externe : l'apprenant accorde un crédit excessif à une source incorrecte parce qu'elle paraît compétente ou intuitivement plausible \citep{kulgemeyer2023}. L'effet de vérité illusoire (\textit{illusory truth effect}) décrit un mécanisme de répétition : un énoncé entendu plusieurs fois est perçu comme plus crédible, indépendamment de sa véracité \citep{fazio2015}. Ces deux biais portent sur le jugement de l'information reçue. L'IoU porte sur le jugement de soi : l'apprenant croit avoir compris, alors qu'il n'a pas compris.

L'Illusion de Profondeur Explicative (\textit{Illusion of Explanatory Depth}, IOED) constitue une manifestation documentée de ce biais. Les individus surestiment leur compréhension de systèmes causaux courants --- le fonctionnement d'une fermeture éclair, d'une chasse d'eau, du système politique --- jusqu'à ce qu'on leur demande d'en produire une explication détaillée \citep{rozenblit2002}. La demande d'explication révèle l'écart entre la familiarité perçue et la compréhension réelle. Ce résultat indique que le simple sentiment de savoir ne garantit pas la possession du savoir. L'exposition répétée à un sujet, ou l'aisance avec laquelle on peut en évoquer les termes, peut suffire à générer une confiance qui ne repose pas nécessairement sur une compréhension structurée.

Le contexte de présentation influence directement cette illusion. Le modèle de l'illusion de compréhension en apprentissage multimédia identifie les conditions de déclenchement : certaines modalités de présentation --- animations fluides, vidéos passives, démonstrations dynamiques --- donnent l'impression que le contenu est simple à maîtriser \citep{paik2013}. Cette impression ne reflète pas nécessairement la difficulté réelle du contenu. Elle tend plutôt à refléter la facilité de traitement de la présentation. L'apprenant attribue cette facilité à sa propre compréhension et développe une métacompréhension excessivement optimiste. Le résultat est un désengagement cognitif prématuré : convaincu d'avoir compris, l'apprenant peut réduire l'effort nécessaire à un apprentissage profond.

Le rôle de l'effort perçu dans cette dynamique est documenté. Un média perçu comme \og facile \fg{} réduit la quantité d'effort mental investi par l'apprenant (\textit{Amount of Invested Mental Effort}, AIME) \citep{salomon1984}. La télévision, perçue comme un média sans effort, produit un apprentissage inférieur à celui de textes imprimés perçus comme plus exigeants, même lorsque le contenu est identique. L'effort ne constitue pas nécessairement un obstacle à l'apprentissage. Selon le cadre des \textit{desirable difficulties}, il peut en être une condition. Lorsque la présentation réduit l'effort perçu, elle peut simultanément réduire l'effort réel, avec des conséquences directes sur la rétention et la compréhension.

Un mécanisme central de cette illusion est la fluence de traitement : la facilité avec laquelle une information est perçue et traitée cognitivement. La section suivante examine ce mécanisme et ses manifestations dans les contextes d'apprentissage médiatisé.

