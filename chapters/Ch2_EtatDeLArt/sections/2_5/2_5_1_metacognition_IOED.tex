% ============================================================================
% Sous-section 2.5.1 : Métacognition et Calibration de la Confiance
% ============================================================================
% Sources : illusion_extraction (§2.2.1), illusion_ai_agent_draft
% Calibrage : ~550 mots
% Type : C (Rédaction originale)
% ============================================================================

\subsection{Métacognition et Calibration de la Confiance}
\label{subsec:metacognition_IOED}

La métacognition désigne la cognition sur la cognition~: la capacité d'un individu à surveiller et réguler ses propres processus cognitifs~\citep{flavell1979metacognition}. Dans le contexte de l'apprentissage, cette capacité se manifeste par deux fonctions distinctes~: le \textit{monitoring}, qui consiste à évaluer sa propre compréhension, et le \textit{control}, qui permet d'ajuster ses stratégies en conséquence. L'auto-évaluation de la compréhension constitue un processus intrinsèquement faillible~\citep{glenberg1982automatic}~: les apprenants croient fréquemment avoir compris un contenu alors même qu'ils échouent à détecter des contradictions évidentes.

Ce déficit métacognitif trouve sa formalisation dans le concept d'\textit{Illusion of Explanatory Depth} (IOED). Ce phénomène désigne la tendance des individus à surestimer la profondeur de leur compréhension des systèmes causaux complexes --- jusqu'au moment où ils sont contraints de produire une explication détaillée~\citep{rozenblit2002misunderstood}. Le protocole expérimental classique procède en trois phases~: une auto-évaluation initiale (T1), la production d'une explication écrite, puis une seconde auto-évaluation (T2). La révélation est systématique~: confrontés à l'exercice d'explication, les participants découvrent que leur compréhension était moins profonde qu'ils ne le croyaient, se traduisant par une chute significative entre T1 et T2.

Ce phénomène s'avère pertinent pour l'étude des interactions avec les agents conversationnels. Certaines modalités de présentation peuvent donner l'impression que le matériel est facile à traiter~\citep{paik2013effects}. Cette facilité apparente conduit les apprenants à sous-estimer la difficulté de la tâche et à développer une métacompréhension excessivement optimiste. Le résultat est une dissociation entre la confiance subjective, qui augmente, et l'apprentissage objectif, qui stagne.

Cette illusion s'inscrit dans un déficit métacognitif plus large. Le phénomène décrit par l'effet Dunning-Kruger met en lumière un \og double fardeau\fg{}~: les individus les moins compétents manquent également des compétences métacognitives nécessaires pour reconnaître leur propre incompétence~\citep{kruger1999unskilled}. Dans le contexte d'une interaction avec un agent conversationnel fluide, ce déficit devient particulièrement problématique~: la facilité apparente de l'échange peut renforcer une confiance injustifiée.

La notion de \textit{calibration} désigne l'écart entre la confiance subjective et la performance objective~\citep{koriat2008subjective}. Les études révèlent une tendance systématique à la \textit{surconfiance}~: les apprenants surestiment leur niveau de compréhension. Cette surconfiance constitue un prédicteur robuste de mauvais apprentissage, car elle conduit à un désengagement prématuré de l'effort cognitif.

Dans le contexte des agents alimentés par IA générative, ce problème prend une dimension nouvelle. Un \og paradoxe métacognitif\fg{} émerge~: bien que l'assistance de l'IA puisse améliorer la performance immédiate, elle dégrade la capacité de l'utilisateur à évaluer cette même performance~\citep{fernandes2026metacognitive}. L'agent, en fournissant des réponses instantanées et fluides, prive l'apprenant des \og difficultés désirables\fg{} --- l'effort de construction et de réorganisation des connaissances --- pourtant essentielles à un apprentissage durable~\citep{bjork2013self}.
