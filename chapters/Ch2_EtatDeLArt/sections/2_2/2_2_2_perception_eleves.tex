% ============================================================================
% Sous-section 2.2.2 : Le Paradoxe de la Perception Élève
% ============================================================================
% Sources : État de l'Art Histoire (Harris & Haydn 2006, Haydn & Harris 2010,
%           Henkaline 2023, Grever et al. 2011, Miguel-Revilla 2021,
%           Cairns & Garrard 2024)
% Calibrage : ~600 mots
% Type : C (Rédaction originale)
% ============================================================================

\subsection{Le Paradoxe de la Perception Élève~: Pertinent mais Ennuyeux}
\label{subsec:perception_eleves}

Un paradoxe traverse les enquêtes internationales sur la perception de l'histoire scolaire~: les élèves reconnaissent massivement la pertinence de la discipline tout en critiquant ses modalités d'enseignement. Ce décalage entre valeur reconnue et expérience vécue constitue le nœud du problème pédagogique.

La reconnaissance de cette pertinence atteint des niveaux remarquablement stables à travers les contextes nationaux~: 69,3\% en Grande-Bretagne~\citep{haydn2010pupil}, 46\% parmi les élèves australiens qui abandonnent pourtant la discipline au lycée~\citep{cairns2024learning}. Cette perception croît avec le niveau d'études, passant de 3,57 chez les lycéens à 4,46 chez les étudiants spécialisés~\citep{miguelrevilla2021relevance}. Ces données convergentes invalident l'hypothèse d'un rejet de principe~: ce n'est pas la valeur de l'histoire que les élèves contestent, mais la manière dont on la leur enseigne.

Le phénomène \og ennuyeux mais pertinent\fg{} décrit précisément cette tension~\citep{henkaline2023eighth}~: les élèves de quatrième trouvent l'histoire répétitive et passive dans sa forme scolaire, mais la jugent importante pour comprendre les injustices actuelles et développer une conscience citoyenne. Cette contradiction traverse les contextes nationaux selon des modalités distinctes~: les élèves australiens qui poursuivent l'histoire valorisent sa capacité à éclairer le présent, tandis que ceux qui l'abandonnent privilégient son rôle prospectif~--- éviter de répéter les erreurs du passé~\citep{cairns2024learning}. Dans les deux cas, la valeur est reconnue~; seule diffère son articulation.

Un indice révélateur de ce paradoxe réside dans la difficulté des élèves à \textit{formuler} pourquoi l'histoire est utile. Sur 1500 réponses collectées, 658 s'avèrent tautologiques~: \og l'histoire est utile parce qu'il est utile de connaître l'histoire\fg{}~\citep{haydn2010pupil}. Cette incapacité à expliciter une valeur pourtant perçue traduit un déficit de médiation~: les élèves intuïtionnent une pertinence que l'enseignement ne leur fournit pas les cadres conceptuels pour articuler. Ce déficit explique pourquoi la reconnaissance de la pertinence ne se traduit pas en engagement soutenu.

Les variations observées entre établissements confirment que le problème relève des pratiques plutôt que de la discipline elle-même. Les écarts d'appréciation vont de 51,1\% à 85,9\% selon les écoles~--- un différentiel de près de trente-cinq points qui ne s'explique pas par les caractéristiques socio-économiques des publics~\citep{harris2006pupils}. L'effet enseignant et l'effet département surpassent les déterminants structurels, suggérant que l'histoire \textit{peut} engager les élèves lorsque les conditions pédagogiques sont réunies.

Des variations significatives apparaissent également selon les profils. Les filles manifestent une appréciation affective plus élevée, tandis que les élèves issus de l'immigration rapportent une appréciation comportementale supérieure~\citep{henkaline2023eighth}. Dans les zones urbaines multiculturelles~--- Rotterdam, Londres, Nord-Pas-de-Calais~---, les élèves non-natifs valorisent l'histoire des migrations et l'histoire religieuse, des contenus marginaux dans les curricula nationaux où les garçons natifs privilégient davantage l'histoire nationale~\citep{grever2011high}. Ces résultats invitent à questionner les approches mono-perspectives qui, en ne reflétant qu'un récit dominant, excluent une partie des élèves de la possibilité de s'identifier au contenu enseigné.

Analysé par le prisme des besoins psychologiques fondamentaux (cf.~\S\ref{subsec:SDT_CET}), ce désengagement trouve une explication structurelle. L'enseignement traditionnel de l'histoire satisfait mal le besoin d'autonomie~--- peu de choix offerts~---, fragilise le sentiment de compétence~--- critères d'évaluation opaques, feedback différé~--- et limite l'appartenance sociale~--- travail individuel prédominant. Le paradoxe \og pertinent mais ennuyeux\fg{} traduit ainsi un échec de médiation~: la discipline possède les ressources pour engager, mais ses modalités d'enseignement dominantes neutralisent ce potentiel.
