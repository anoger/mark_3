\subsection{Le Paradoxe Perceptif : « Ennuyeux mais Important »}
\label{subsec:perception_eleves}

Les études empiriques convergent vers un constat transculturel : les élèves reconnaissent massivement la pertinence de l'histoire tout en critiquant ses modalités d'enseignement. En Grande-Bretagne, 69,8\% des élèves trouvent l'histoire appréciable, mais la discipline se classe derrière toutes les matières impliquant une activité concrète \citep{harris2006}. Aux États-Unis, les élèves la jugent répétitive et passive mais importante pour comprendre les injustices actuelles \citep{henkaline2023}. En Australie, les élèves qui abandonnent l'histoire au secondaire supérieur valorisent néanmoins son rôle prospectif --- apprendre du passé pour éclairer l'avenir --- tandis que ceux qui la poursuivent privilégient sa capacité à éclairer le présent \citep{cairns2024}. En Espagne, les lycéens obtiennent les scores de pertinence perçue les plus bas ($M = 3{,}57$), un score qui s'améliore significativement chez les étudiants universitaires ($M = 4{,}46$), avec des tailles d'effet importantes ($\eta^2 = 0{,}13$ à $0{,}25$) \citep{miguelrevilla2021}. Ce paradoxe n'est pas un artefact culturel local : il émerge systématiquement de la conjonction entre la reconnaissance sociale de la discipline et l'inadéquation du format pédagogique dominant.

Le cadre théorique de l'intérêt (cf.~\ref{subsec:architecture_interet}) éclaire ce paradoxe. L'histoire possède un potentiel narratif et dramatique capable de déclencher l'intérêt situationnel (phase~1), mais le format transmissif empêche la transition vers l'intérêt maintenu (phase~2) parce qu'il ne crée pas de connexion personnelle avec le contenu. La théorie de l'attente-valeur (cf.~\ref{subsec:pertinence_valeur}) affine le diagnostic : la valeur d'atteinte --- l'importance de l'histoire pour l'identité sociale et la citoyenneté --- coexiste avec un déficit de valeur d'utilité et de valeur intrinsèque. Sur 1\,500 réponses recueillies auprès d'élèves britanniques, 658 sont tautologiques : les élèves peinent à articuler pourquoi l'histoire est utile au-delà de formulations circulaires \citep{haydn2010}. Le cadre ICAP (cf.~\ref{subsec:icap}) complète l'analyse : le cours magistral, la prise de notes verbatim et la lecture de manuels placent l'élève en mode passif --- le niveau le plus bas d'engagement cognitif --- tandis que les données empiriques montrent que l'efficacité pédagogique en histoire repose sur l'activation de l'élève \citep{audigier2010}.

Les données empiriques identifient avec précision les leviers d'engagement et les facteurs de désaffection. Les méthodes interactives --- jeu de rôle, débats, simulations --- recueillent massivement les évaluations positives : 614 commentaires favorables dans une étude portant sur 1\,740 élèves, dont 295 pour le jeu de rôle et 108 pour les discussions \citep{harris2006}. Les connexions explicites entre passé, présent et futur augmentent la perception de pertinence \citep{cairns2024, vanStraaten2016}. La qualité relationnelle de l'enseignant --- son enthousiasme, sa capacité à respecter les élèves --- pèse davantage que les facteurs socio-économiques pour expliquer les variations d'appréciation entre établissements \citep{harris2006}. La diversification des perspectives historiques engage les élèves d'origines multiculturelles : une enquête menée auprès de 678 lycéens dans trois zones urbaines (Rotterdam, Londres, Nord-Pas-de-Calais) montre que les élèves issus de l'immigration s'intéressent davantage à l'histoire transnationale et religieuse qu'au récit national, et que les tentatives de raviver l'histoire en termes purement nationaux risquent de produire un modèle scolaire que beaucoup considèrent comme non pertinent \citep{grever2011}. À l'inverse, le travail écrit excessif, les tests fréquents et les cours magistraux prolongés génèrent un désengagement durable \citep{harris2006}. La formation des enseignants aux méthodes actives produit des améliorations significatives de la motivation et de l'apprentissage perçu chez 467 élèves espagnols ($r = 0{,}44$) \citep{gomez2021}.

Ce diagnostic converge vers un principe : l'engagement en histoire dépend moins du contenu enseigné que de la modalité d'enseignement. Les facteurs d'engagement identifiés --- interactivité, connexion personnelle, multiperspectivité, feedback --- correspondent aux conditions de satisfaction des besoins psychologiques fondamentaux (autonomie, compétence, affiliation ; cf.~\ref{subsec:sdt_cet}) et aux niveaux supérieurs du cadre ICAP (constructif et interactif ; cf.~\ref{subsec:icap}). Des pratiques pédagogiques innovantes tentent d'activer ces leviers.
