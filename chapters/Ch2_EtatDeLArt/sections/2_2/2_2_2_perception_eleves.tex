\subsection{Le paradoxe perceptif : importance reconnue, engagement faible}
\label{subsec:perception_eleves_histoire}

Le décalage entre exigences cognitives de la discipline et pratiques transmissives semble produire un paradoxe documenté dans plusieurs contextes nationaux : les élèves du secondaire reconnaissent l'importance de l'histoire mais la vivent comme ennuyeuse dans sa forme scolaire \citep{harris2006, henkaline2023}. En termes de valeur subjective (cf.~\ref{subsec:pertinence_valeur}), la valeur d'atteinte --- l'importance accordée à la réussite en histoire pour l'image de soi --- reste élevée, tandis que la valeur intrinsèque --- le plaisir retiré de l'apprentissage --- reste faible. Les données britanniques et australiennes éclairent la nature de ce paradoxe : environ 70\,\% des élèves jugent l'histoire utile, mais la majorité peine à expliquer pourquoi, se limitant à des justifications vagues ou aux débouchés professionnels spécialisés \citep{haydn2010}. Les élèves australiens qui poursuivent l'histoire valorisent sa capacité à éclairer le présent, tandis que ceux qui l'abandonnent n'y voient qu'un éclairage rétrospectif \citep{cairns2024}. La perception de pertinence croît avec l'âge et le niveau de spécialisation, les lycéens obtenant les scores les plus faibles \citep{miguelrevilla2021}. Par ailleurs, les approches centrées sur l'histoire nationale excluent les élèves dont l'expérience familiale ne s'y retrouve pas \citep{grever2011}. Ce paradoxe ne semble pas relever d'un contexte éducatif particulier : sa récurrence dans plusieurs pays suggère une propriété qui pourrait être structurelle.

Le modèle en quatre phases de l'intérêt (cf.~\ref{subsec:architecture_interet}) offre une grille de lecture pour situer la défaillance. L'histoire semble posséder des déclencheurs d'intérêt situationnel --- narration, dilemmes moraux, altérité des acteurs --- susceptibles de capter l'attention (phase~1). Les élèves expriment d'ailleurs une préférence pour les jeux de rôle, les débats et les simulations. Le problème se situe au maintien (phase~2) : le cours magistral prolongé, la prise de notes passive et la restitution factuelle ne fournissent ni l'engagement actif ni la connexion personnelle nécessaires à la transition vers un intérêt soutenu \citep{gomez2021}. Les variations d'appréciation entre établissements --- de 51\,\% à 86\,\% --- suggèrent que la variable déterminante pourrait être moins le contenu que la manière dont il est enseigné. La théorie de l'évaluation cognitive (cf.~\ref{subsec:sdt_cet}) complète ce diagnostic : les évaluations centrées sur la mémorisation factuelle présentent un aspect contrôlant qui sape le sentiment d'autonomie, et l'absence de feedback immédiat empêche le soutien du sentiment de compétence. L'histoire semble posséder un potentiel d'engagement --- narratif, empathique, argumentatif --- que le format pédagogique standard peine à activer.

