\subsection{Le paradoxe perceptif : reconnaissance sociale et désengagement individuel}
\label{subsec:perception_eleves_histoire}

Le décalage entre exigences épistémologiques et pratiques pédagogiques produit un phénomène que les données empiriques documentent de manière convergente dans plusieurs contextes nationaux. Les enquêtes à grande échelle auprès d'élèves du secondaire révèlent un paradoxe : l'histoire est reconnue comme importante mais vécue comme ennuyeuse dans sa forme scolaire \citep{harris2006, henkaline2023}. Ce paradoxe se manifeste par un écart entre deux composantes de la valeur subjective (cf.~\ref{subsec:pertinence_valeur}) : la valeur d'atteinte --- l'importance accordée à la réussite en histoire pour l'image de soi et la reconnaissance sociale --- reste élevée, tandis que la valeur intrinsèque --- le plaisir retiré de l'activité d'apprentissage --- et la valeur d'utilité restent faibles. Une enquête britannique portant sur 1740~élèves de 11 à 14~ans situe l'histoire en cinquième position dans le classement des matières appréciées, derrière les disciplines impliquant une manipulation physique directe (éducation physique, technologie, informatique, arts), avec un taux d'appréciation de 69,8\,\%, et 69,3\,\% des élèves jugent l'histoire utile, mais la majorité produit des justifications tautologiques ou limitées aux débouchés professionnels spécialisés \citep{haydn2010}. Les élèves peinent à articuler la nature de cette utilité au-delà de formules convenues.

Ce paradoxe se retrouve dans des contextes culturels distincts. En Australie, une enquête nationale auprès de 290~élèves du secondaire supérieur confirme que les élèves qui abandonnent l'histoire reconnaissent sa pertinence pour éclairer l'avenir, tandis que ceux qui la poursuivent valorisent sa capacité à éclairer le présent \citep{cairns2024}. En Espagne, une analyse comparative portant sur 403~participants (lycéens, futurs enseignants, étudiants en histoire) montre que la perception de pertinence croît avec l'âge et le niveau de spécialisation, les lycéens obtenant les scores les plus faibles \citep{miguelrevilla2021}. Dans trois zones urbaines multiculturelles (Rotterdam, Londres, Nord-Pas-de-Calais), une enquête auprès de 678~lycéens identifie cinq profils d'intérêt historique liés aux origines culturelles des élèves : les approches centrées exclusivement sur l'histoire nationale excluent une partie des apprenants dont l'expérience familiale ne s'y retrouve pas \citep{grever2011}. La convergence transculturelle de ces résultats indique que le paradoxe perceptif ne relève pas d'un contexte éducatif particulier, mais d'une propriété structurelle de la relation entre la discipline et son enseignement.

Le modèle en quatre phases de l'intérêt (cf.~\ref{subsec:architecture_interet}) permet de localiser la défaillance. L'histoire possède des déclencheurs d'intérêt situationnel intrinsèques : la narration, les dilemmes moraux, les retournements de situation, l'altérité des acteurs historiques constituent autant de stimuli capables de capter l'attention (phase~1). Les élèves interrogés expriment une préférence marquée pour les jeux de rôle, les débats et les simulations --- des formats qui activent ces déclencheurs \citep{harris2006, henkaline2023}. Le problème se situe au niveau du maintien (phase~2) : la pédagogie transmissive dominante --- cours magistral prolongé, prise de notes passive, restitution factuelle lors d'évaluations --- ne fournit ni l'engagement actif ni la connexion personnelle nécessaires à la transition vers un intérêt soutenu \citep{gomez2021}. Le travail écrit excessif, les tests fréquents et la sur-utilisation de manuels concentrent la majorité des critiques négatives des élèves, et les variations d'appréciation entre établissements --- de 51\,\% à 86\,\% dans l'enquête britannique --- confirment que la variable déterminante n'est pas le contenu disciplinaire mais la manière dont il est enseigné. La théorie de l'évaluation cognitive (cf.~\ref{subsec:sdt_cet}) complète ce diagnostic : les modalités d'évaluation centrées sur la mémorisation factuelle --- dates, noms, événements isolés de leur contexte --- présentent un aspect contrôlant qui sape le sentiment d'autonomie, et l'absence de feedback immédiat caractéristique de l'histoire empêche le soutien du sentiment de compétence. L'histoire possède ainsi un potentiel d'engagement latent --- narratif, empathique, argumentatif --- que le format pédagogique standard ne parvient pas à activer.

