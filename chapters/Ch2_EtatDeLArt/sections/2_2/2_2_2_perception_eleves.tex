\subsection{Les Pratiques Actives en Histoire : Au-Delà du Cours Magistral}
\label{subsec:pratiques_actives}

Les recherches en didactique de l'histoire ont identifié des formats pédagogiques qui déplacent l'élève de la réception passive d'un récit vers la construction active de sa compréhension. La discussion collaborative et la résolution de problèmes en groupe augmentent la compréhension des concepts historiques et l'intérêt pour la discipline par rapport à l'apprentissage individuel : les échanges entre pairs obligent chaque participant à expliciter son raisonnement, à confronter ses interprétations et à intégrer des perspectives divergentes \citep{delfavero2007}. Le récit interactif numérique, dans lequel les apprenants rencontrent des points de décision stratégiques au fil d'une narration historique, produit un effet convergent : les choix narratifs stimulent la discussion de groupe et favorisent une compréhension en profondeur des enjeux historiques, parce qu'ils contraignent l'apprenant à évaluer les conséquences de décisions prises dans un contexte passé \citep{petousi2022}.

L'immersion dans des environnements virtuels renforce ces effets en ajoutant une dimension spatiale et sensorielle à l'expérience historique. La reconstruction tridimensionnelle de cités antiques améliore la performance et l'engagement des élèves, qui consacrent davantage de temps à l'exploration du contenu historique que les groupes exposés au même contenu par le texte ou la vidéo \citep{ijaz2017}. Les dispositifs muséaux interactifs, qui combinent objets tangibles et récits à forte charge émotionnelle, favorisent l'empathie historique et la pensée critique chez les adolescents : la manipulation physique d'artefacts liés à des récits personnels du passé crée une connexion affective que le texte seul ne parvient pas à produire \citep{roussou2024}. Ces résultats convergent vers un principe commun : l'apprentissage de l'histoire gagne en profondeur lorsque l'élève passe de la réception d'un récit clos à la construction active de sa propre compréhension du passé --- un déplacement qui correspond au passage du mode passif vers les modes constructif et interactif du cadre ICAP (cf.~\ref{subsec:icap}). Trois formats illustrent ce principe avec une force particulière : la simulation, l'incarnation par le jeu de rôle et la multimodalité du jeu historique.
