% ============================================================================
% Sous-section 2.2.4 : Technologies Éducatives en Histoire — Bilan Empirique
% ============================================================================
% Sources : IJCCI_extraction, État de l'Art Histoire
% Calibrage : ~650 mots
% Type : C (Rédaction originale)
% ============================================================================

\subsection{Technologies Éducatives en Histoire~: Bilan Empirique}
\label{subsec:technologies_histoire}

L'intégration des technologies numériques dans l'enseignement de l'Histoire a fait l'objet d'expérimentations variées, dont les résultats permettent d'identifier les approches prometteuses et celles qui se sont révélées décevantes.

\subsubsection*{Approches ayant produit des résultats probants}

Plusieurs dispositifs technologiques ont démontré une efficacité mesurable sur l'engagement ou l'apprentissage en contexte historique.

La narration numérique interactive constitue l'approche la mieux documentée. L'intégration de points de décision stratégiques dans des récits historiques stimule des discussions de groupe significatives et une compréhension approfondie~\citep{petousi2022interactive}. Dans une étude sur l'Athènes antique, les élèves confrontés à des choix narratifs ont développé une réflexion historique plus élaborée que ceux exposés à un récit linéaire. L'efficacité repose sur l'activation du mode \textit{Constructif} du cadre ICAP~: les apprenants génèrent des hypothèses sur les conséquences de leurs choix plutôt que de recevoir passivement l'information.

Les environnements virtuels immersifs produisent des gains d'apprentissage et d'engagement lorsqu'ils permettent une exploration autonome. Une reconstruction 3D de l'ancienne Uruk a conduit à une amélioration des performances aux tests et à un temps d'exploration accru comparé aux méthodes traditionnelles~\citep{ijaz2017immersion}. L'immersion spatiale semble activer des processus de mémorisation épisodique qui renforcent la rétention des informations contextuelles.

Les installations muséales combinant interactions tangibles et récits émotionnels favorisent le développement de l'empathie historique~\citep{roussou2024emotions}. L'exploration de perspectives émotionnelles multiples sur des personnages historiques permet aux adolescents de dépasser la connaissance factuelle pour développer une compréhension des motivations et des contextes. Cette approche répond au déficit de connexion personnelle identifié dans la perception de l'Histoire par les élèves.

Les discussions de groupe structurées autour de problèmes historiques améliorent simultanément la compréhension conceptuelle et l'intérêt pour la discipline~\citep{delfavero2007classroom}. L'étude, portant sur la Première Guerre mondiale et le miracle économique italien, a montré que la confrontation des interprétations entre pairs activait des processus de réélaboration cognitive absents de l'apprentissage individuel.

\subsubsection*{Approches aux résultats mitigés ou négatifs}

Certaines technologies n'ont pas tenu leurs promesses en contexte historique.

Les agents pédagogiques scriptés présentent des résultats décevants en Histoire. Les méta-analyses montrent un effet négatif sur l'apprentissage dans ce domaine~\citep{davis2019effectiveness}. Cette contre-performance pourrait s'expliquer par l'inadéquation entre la rigidité des réponses préprogrammées et la nature interprétative de la discipline~: un agent incapable de nuancer son discours ou de reconnaître la pluralité des interprétations peut apparaître comme une source d'autorité inappropriée.

Les interfaces conversationnelles textuelles avec figures historiques produisent des résultats quantitatifs mixtes malgré des gains motivationnels~\citep{pataranutaporn2023living}. La comparaison entre dialogues interactifs et lecture traditionnelle indique une amélioration de la motivation mais des effets incertains sur l'apprentissage objectif. L'absence de modalité orale pourrait limiter l'activation des mécanismes de présence sociale.

La simple introduction d'outils technologiques sans repensée pédagogique ne garantit pas l'engagement~\citep{yarema2002using}. Les technologies qui reproduisent le format transmissif sous une forme numérique échouent à transformer la relation des élèves au contenu historique.

\subsubsection*{Conditions de succès transversales}

L'analyse comparative de ces expérimentations suggère trois conditions nécessaires à l'efficacité des technologies en enseignement de l'Histoire~: l'activation de l'agentivité de l'apprenant (les dispositifs efficaces offrent des choix significatifs qui engagent la réflexion), la dimension sociale de l'interaction (les approches les plus prometteuses impliquent une confrontation des perspectives), et la cohérence avec l'épistémologie disciplinaire (les technologies qui respectent la nature interprétative de l'Histoire surpassent celles qui importent un modèle transmissif adapté aux STIM).

Ces constats informent directement la conception de notre dispositif expérimental~: l'agent conversationnel incarnant un personnage historique vise précisément à satisfaire ces trois conditions en offrant une interaction dialogique authentique, socialement engageante, et épistémologiquement cohérente avec la discipline.
