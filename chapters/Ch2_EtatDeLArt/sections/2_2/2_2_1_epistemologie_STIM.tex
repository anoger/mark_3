% ============================================================================
% Sous-section 2.2.1 : Position de l'Histoire dans l'Écosystème Scolaire
% ============================================================================
% Sources : État de l'Art Histoire (Harris & Haydn 2006, Haydn & Harris 2010,
%           Van Straaten et al. 2015, Grever et al. 2011)
% Calibrage : ~550 mots
% Type : C (Rédaction originale)
% ============================================================================

\subsection{Position de l'Histoire dans l'Écosystème Scolaire}
\label{subsec:epistemologie_STIM}

Contrairement à une idée répandue, l'histoire n'est pas rejetée par les élèves. Avec 69,8\% d'opinions favorables auprès de 1740 élèves britanniques, la discipline se classe en cinquième position des matières appréciées~\citep{harris2006pupils}. Ce constat, corroboré dans d'autres contextes nationaux, invite à nuancer le diagnostic d'une discipline en crise~: l'histoire occupe une position intermédiaire qui révèle moins un rejet qu'un déficit de médiation.

Cette position se caractérise par un décalage entre deux formes de valeur perçue. Les élèves reconnaissent à l'histoire une utilité \textit{cognitive}~--- comprendre le présent, éviter de répéter les erreurs du passé, développer un regard critique sur le monde~--- tout en peinant à lui attribuer une utilité \textit{instrumentale} comparable à celle des disciplines scientifiques et techniques. Sur une échelle d'importance perçue, les mathématiques obtiennent 4,46 et l'anglais 4,42, contre 3,26 pour l'histoire~\citep{haydn2010pupil}. Cette asymétrie s'explique par la lisibilité des débouchés professionnels~: les filières scientifiques offrent des trajectoires clairement identifiées là où l'histoire semble cantonnée à l'enseignement ou aux métiers du patrimoine~\citep{grever2011high}.

Cette hiérarchie implicite s'enracine dans des différences épistémologiques que l'école rend rarement explicites. Les disciplines STIM reposent sur des savoirs cumulatifs, universels et vérifiables par l'expérimentation~: la progression y suit une logique d'accumulation où chaque concept s'appuie sur les précédents, et le rapport à la vérité s'établit par démonstration~\citep{vanstraaten2015making}. Le feedback y est immédiat~--- une équation est correctement résolue ou ne l'est pas. L'histoire, en revanche, produit des savoirs interprétatifs et contextuels où la vérité émerge de l'argumentation fondée sur des sources, processus moins définitif où plusieurs interprétations peuvent coexister pour un même événement. Cette nature interprétative requiert une tolérance à l'ambiguïté que l'enseignement scolaire cultive rarement de manière explicite~--- d'où le sentiment de certains élèves que \og l'histoire est morte et n'a rien à voir avec [leur] vie présente\fg{}, une proportion qui atteint 14\% dans les enquêtes européennes~\citep{vanstraaten2015making}.

Le contraste méthodologique est tout aussi marqué. Les STIM privilégient l'expérimentation contrôlée et la modélisation mathématique, offrant des procédures reproductibles dont la rigueur est immédiatement perceptible. L'histoire utilise l'analyse critique de sources, la contextualisation et la mise en perspective~--- méthodes moins standardisées dont les élèves saisissent moins aisément l'exigence intellectuelle. Cette différence nourrit une perception de moindre scientificité, alors même que l'histoire développe des compétences critiques équivalentes~\citep{harris2006pupils}.

Ces différences épistémologiques ne constituent pas des défauts~; elles définissent des compétences distinctives. L'analyse critique de discours, la capacité à peser des arguments contradictoires, la compréhension des motivations humaines dans leur contexte, l'empathie historique~: autant de capacités que les STIM ne cultivent pas avec la même intensité. Le potentiel narratif et dramatique de l'histoire autorise un engagement émotionnel que les disciplines formelles peinent à susciter~\citep{harris2006pupils}. Sa contribution à la formation citoyenne et à la compréhension interculturelle répond à des besoins sociétaux croissants~\citep{grever2011high}. L'enjeu n'est donc pas d'imiter les STIM, mais de valoriser ces spécificités tout en explicitant les compétences qu'elles développent~--- un travail de médiation que l'enseignement traditionnel n'accomplit qu'imparfaitement.
