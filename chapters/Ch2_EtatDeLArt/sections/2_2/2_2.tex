% ============================================================================
% Section 2.2 : Le Contexte Spécifique de l'Enseignement de l'Histoire
% ============================================================================
% Objectif : Établir le diagnostic du terrain disciplinaire — position de
%            l'histoire dans l'écosystème scolaire, perception des élèves,
%            leviers pédagogiques identifiés par la recherche
% Sources : État de l'Art Histoire (Harris & Haydn, Haydn & Harris, Grever,
%           Henkaline, Van Straaten, Gómez Carrasco, Cairns & Garrard, Bergin)
% Calibrage : ~1 800 mots
% ============================================================================

\section{Le Contexte Spécifique de l'Enseignement de l'Histoire}
\label{sec:contexte_histoire}

Les cadres théoriques présentés dans la section précédente éclairent les mécanismes généraux de l'engagement dans l'apprentissage. Leur application à l'enseignement de l'histoire révèle un terrain singulier. La discipline n'est pas rejetée par les élèves~; elle souffre d'un déficit de médiation entre ses spécificités épistémologiques et les attentes de son public. Ce diagnostic structure l'analyse qui suit~: la position de l'histoire dans l'écosystème scolaire (\S\ref{subsec:epistemologie_STIM}), le paradoxe de la perception des élèves (\S\ref{subsec:perception_eleves}), et les leviers pédagogiques identifiés par la recherche (\S\ref{subsec:pratiques_pedagogiques}).

% ----------------------------------------------------------------------------
% Sous-sections
% ----------------------------------------------------------------------------
% ============================================================================
% Sous-section 2.2.1 : Position de l'Histoire dans l'Écosystème Scolaire
% ============================================================================
% Sources : État de l'Art Histoire (Harris & Haydn 2006, Haydn & Harris 2010,
%           Van Straaten et al. 2015, Grever et al. 2011)
% Calibrage : ~550 mots
% Type : C (Rédaction originale)
% ============================================================================

\subsection{Position de l'Histoire dans l'Écosystème Scolaire}
\label{subsec:epistemologie_STIM}

Contrairement à une idée répandue, l'histoire n'est pas rejetée par les élèves. Avec 69,8\% d'opinions favorables auprès de 1740 élèves britanniques, la discipline se classe en cinquième position des matières appréciées~\citep{harris2006pupils}. Ce constat, corroboré dans d'autres contextes nationaux, invite à nuancer le diagnostic d'une discipline en crise~: l'histoire occupe une position intermédiaire qui révèle moins un rejet qu'un déficit de médiation.

Cette position se caractérise par un décalage entre deux formes de valeur perçue. Les élèves reconnaissent à l'histoire une utilité \textit{cognitive}~--- comprendre le présent, éviter de répéter les erreurs du passé, développer un regard critique sur le monde~--- tout en peinant à lui attribuer une utilité \textit{instrumentale} comparable à celle des disciplines scientifiques et techniques. Sur une échelle d'importance perçue, les mathématiques obtiennent 4,46 et l'anglais 4,42, contre 3,26 pour l'histoire~\citep{haydn2010pupil}. Cette asymétrie s'explique par la lisibilité des débouchés professionnels~: les filières scientifiques offrent des trajectoires clairement identifiées là où l'histoire semble cantonnée à l'enseignement ou aux métiers du patrimoine~\citep{grever2011high}.

Cette hiérarchie implicite s'enracine dans des différences épistémologiques que l'école rend rarement explicites. Les disciplines STIM reposent sur des savoirs cumulatifs, universels et vérifiables par l'expérimentation~: la progression y suit une logique d'accumulation où chaque concept s'appuie sur les précédents, et le rapport à la vérité s'établit par démonstration~\citep{vanstraaten2015making}. Le feedback y est immédiat~--- une équation est correctement résolue ou ne l'est pas. L'histoire, en revanche, produit des savoirs interprétatifs et contextuels où la vérité émerge de l'argumentation fondée sur des sources, processus moins définitif où plusieurs interprétations peuvent coexister pour un même événement. Cette nature interprétative requiert une tolérance à l'ambiguïté que l'enseignement scolaire cultive rarement de manière explicite~--- d'où le sentiment de certains élèves que \og l'histoire est morte et n'a rien à voir avec [leur] vie présente\fg{}, une proportion qui atteint 14\% dans les enquêtes européennes~\citep{vanstraaten2015making}.

Le contraste méthodologique est tout aussi marqué. Les STIM privilégient l'expérimentation contrôlée et la modélisation mathématique, offrant des procédures reproductibles dont la rigueur est immédiatement perceptible. L'histoire utilise l'analyse critique de sources, la contextualisation et la mise en perspective~--- méthodes moins standardisées dont les élèves saisissent moins aisément l'exigence intellectuelle. Cette différence nourrit une perception de moindre scientificité, alors même que l'histoire développe des compétences critiques équivalentes~\citep{harris2006pupils}.

Ces différences épistémologiques ne constituent pas des défauts~; elles définissent des compétences distinctives. L'analyse critique de discours, la capacité à peser des arguments contradictoires, la compréhension des motivations humaines dans leur contexte, l'empathie historique~: autant de capacités que les STIM ne cultivent pas avec la même intensité. Le potentiel narratif et dramatique de l'histoire autorise un engagement émotionnel que les disciplines formelles peinent à susciter~\citep{harris2006pupils}. Sa contribution à la formation citoyenne et à la compréhension interculturelle répond à des besoins sociétaux croissants~\citep{grever2011high}. L'enjeu n'est donc pas d'imiter les STIM, mais de valoriser ces spécificités tout en explicitant les compétences qu'elles développent~--- un travail de médiation que l'enseignement traditionnel n'accomplit qu'imparfaitement.

\subsection{Les Pratiques Actives en Histoire : Au-Delà du Cours Magistral}
\label{subsec:pratiques_actives}

Les recherches en didactique de l'histoire ont identifié des formats pédagogiques qui déplacent l'élève de la réception passive d'un récit vers la construction active de sa compréhension. La discussion collaborative et la résolution de problèmes en groupe augmentent la compréhension des concepts historiques et l'intérêt pour la discipline par rapport à l'apprentissage individuel : les échanges entre pairs obligent chaque participant à expliciter son raisonnement, à confronter ses interprétations et à intégrer des perspectives divergentes \citep{delfavero2007}. Le récit interactif numérique, dans lequel les apprenants rencontrent des points de décision stratégiques au fil d'une narration historique, produit un effet convergent : les choix narratifs stimulent la discussion de groupe et favorisent une compréhension en profondeur des enjeux historiques, parce qu'ils contraignent l'apprenant à évaluer les conséquences de décisions prises dans un contexte passé \citep{petousi2022}.

L'immersion dans des environnements virtuels renforce ces effets en ajoutant une dimension spatiale et sensorielle à l'expérience historique. La reconstruction tridimensionnelle de cités antiques améliore la performance et l'engagement des élèves, qui consacrent davantage de temps à l'exploration du contenu historique que les groupes exposés au même contenu par le texte ou la vidéo \citep{ijaz2017}. Les dispositifs muséaux interactifs, qui combinent objets tangibles et récits à forte charge émotionnelle, favorisent l'empathie historique et la pensée critique chez les adolescents : la manipulation physique d'artefacts liés à des récits personnels du passé crée une connexion affective que le texte seul ne parvient pas à produire \citep{roussou2024}. Ces résultats convergent vers un principe commun : l'apprentissage de l'histoire gagne en profondeur lorsque l'élève passe de la réception d'un récit clos à la construction active de sa propre compréhension du passé --- un déplacement qui correspond au passage du mode passif vers les modes constructif et interactif du cadre ICAP (cf.~\ref{subsec:icap}). Trois formats illustrent ce principe avec une force particulière : la simulation, l'incarnation par le jeu de rôle et la multimodalité du jeu historique.

% ============================================================================
% Sous-section 2.2.3 : Leviers Pédagogiques de l'Engagement
% ============================================================================
% Sources : État de l'Art Histoire (Harris & Haydn 2006, Gómez Carrasco et al. 2021,
%           Cairns & Garrard 2024, Van Straaten et al. 2015, Bergin 1999,
%           Henkaline 2023)
% Calibrage : ~650 mots
% Type : C (Rédaction originale)
% ============================================================================

\subsection{Leviers Pédagogiques de l'Engagement}
\label{subsec:pratiques_pedagogiques}

Si le paradoxe \og pertinent mais ennuyeux\fg{} désigne le problème, la recherche en didactique de l'histoire identifie avec une précision croissante les conditions de sa résolution. Les données empiriques convergent vers un constat structurant~: l'efficacité pédagogique dépend moins du contenu enseigné que des modalités d'enseignement.

Les méthodes interactives recueillent massivement l'adhésion des élèves. Sur 1740 répondants britanniques, le jeu de rôle et le théâtre totalisent 295 mentions positives, les discussions et débats 108, le travail de groupe 56~\citep{harris2006pupils}. À l'inverse, le travail écrit excessif concentre 394 mentions négatives, les tests fréquents 151, la sur-utilisation des manuels et fiches 57. Ce pattern n'est pas propre au contexte britannique~: les élèves américains valorisent de même les méthodes expérientielles~--- simulations de procès, vidéos immersives~--- et critiquent la prise de notes passive~\citep{henkaline2023eighth}. La convergence transculturelle de ces préférences suggère qu'elles reflètent des invariants cognitifs plutôt que des spécificités locales.

Ces préférences ne relèvent pas du simple confort~; elles produisent des gains mesurables sur l'apprentissage. Un programme de formation combinant méthodes actives et réflexion épistémologique, évalué auprès de 467 élèves répartis en 18 classes, révèle des améliorations significatives~: $r = 0.522$ pour l'évaluation de la méthodologie, $r = 0.443$ pour la motivation, $r = 0.335$ pour l'apprentissage perçu~\citep{gomezcarrasco2021motivation}. L'intensité de l'intervention module les effets~: les enseignants formés de manière approfondie produisent des résultats supérieurs à ceux formés superficiellement. La transformation pédagogique requiert donc un investissement substantiel en formation.

Le cadre ICAP (cf.~\S\ref{subsec:ICAP}) éclaire les mécanismes sous-jacents à cette efficacité. Les méthodes interactives~--- débats, jeux de rôle, confrontation des interprétations~--- activent les modes \textit{Constructif} et \textit{Interactif}, où l'élève génère des inférences et co-construit du sens avec ses pairs. Le cours magistral cantonne l'élève au mode \textit{Passif}, où l'attention ne garantit pas le traitement profond. La supériorité des méthodes actives n'est donc pas une question de préférence subjective mais de fonctionnement cognitif~: l'engagement émotionnel améliore la mémorisation, l'expérience directe développe l'empathie historique, l'interaction sociale répond aux besoins d'appartenance~\citep{bergin1999influences}.

Parmi les leviers identifiés, la connexion explicite entre temporalités occupe une place centrale. Un cadre théorique articulant passé, présent et futur comme condition de la pertinence perçue montre que les élèves trouvent les tâches de connexion temporelle plus difficiles mais plus intéressantes que les exercices traditionnels~\citep{vanstraaten2015making}. Cette observation trouve un écho dans les données australiennes~: les élèves qui poursuivent l'histoire valorisent précisément sa capacité à éclairer le présent~\citep{cairns2024learning}. L'explicitation de ces connexions~--- plutôt que leur découverte supposée spontanée~--- constitue une condition de l'engagement.

L'enseignant demeure la variable déterminante. Les écarts d'appréciation entre établissements (cf.~\S\ref{subsec:perception_eleves}) ne s'expliquent ni par les contenus enseignés ni par les caractéristiques des publics, mais par les pratiques professorales et départementales. Les enseignants \og enthousiastes qui respectent les élèves\fg{} obtiennent de meilleurs résultats indépendamment des méthodes utilisées~\citep{harris2006pupils}. Cette observation s'inscrit dans un modèle plus général de l'intérêt situationnel~: nouveauté modérée, interaction sociale, narration, prise en compte de l'appartenance culturelle constituent autant de facteurs que l'enseignant contrôle et qui peuvent déclencher l'intérêt même chez des élèves initialement peu motivés~\citep{bergin1999influences}.

Ces résultats dessinent le cahier des charges implicite d'une innovation pédagogique en histoire. L'efficacité suppose l'activation de l'élève plutôt que sa réception passive, la confrontation des perspectives plutôt que la transmission d'un récit unique, la connexion explicite aux préoccupations contemporaines plutôt que l'enfermement dans le passé. Les sections suivantes examinent dans quelle mesure les technologies d'interaction~--- et singulièrement les agents conversationnels~--- peuvent satisfaire ces exigences.

% Note : Le contenu sur les technologies éducatives en histoire est déplacé
%        en section 2.4 pour respecter la progression argumentative du chapitre
% % ============================================================================
% Sous-section 2.2.4 : Technologies Éducatives en Histoire — Bilan Empirique
% ============================================================================
% Sources : IJCCI_extraction, État de l'Art Histoire
% Calibrage : ~650 mots
% Type : C (Rédaction originale)
% ============================================================================

\subsection{Technologies Éducatives en Histoire~: Bilan Empirique}
\label{subsec:technologies_histoire}

L'intégration des technologies numériques dans l'enseignement de l'Histoire a fait l'objet d'expérimentations variées, dont les résultats permettent d'identifier les approches prometteuses et celles qui se sont révélées décevantes.

\subsubsection*{Approches ayant produit des résultats probants}

Plusieurs dispositifs technologiques ont démontré une efficacité mesurable sur l'engagement ou l'apprentissage en contexte historique.

La narration numérique interactive constitue l'approche la mieux documentée. L'intégration de points de décision stratégiques dans des récits historiques stimule des discussions de groupe significatives et une compréhension approfondie~\citep{petousi2022interactive}. Dans une étude sur l'Athènes antique, les élèves confrontés à des choix narratifs ont développé une réflexion historique plus élaborée que ceux exposés à un récit linéaire. L'efficacité repose sur l'activation du mode \textit{Constructif} du cadre ICAP~: les apprenants génèrent des hypothèses sur les conséquences de leurs choix plutôt que de recevoir passivement l'information.

Les environnements virtuels immersifs produisent des gains d'apprentissage et d'engagement lorsqu'ils permettent une exploration autonome. Une reconstruction 3D de l'ancienne Uruk a conduit à une amélioration des performances aux tests et à un temps d'exploration accru comparé aux méthodes traditionnelles~\citep{ijaz2017immersion}. L'immersion spatiale semble activer des processus de mémorisation épisodique qui renforcent la rétention des informations contextuelles.

Les installations muséales combinant interactions tangibles et récits émotionnels favorisent le développement de l'empathie historique~\citep{roussou2024emotions}. L'exploration de perspectives émotionnelles multiples sur des personnages historiques permet aux adolescents de dépasser la connaissance factuelle pour développer une compréhension des motivations et des contextes. Cette approche répond au déficit de connexion personnelle identifié dans la perception de l'Histoire par les élèves.

Les discussions de groupe structurées autour de problèmes historiques améliorent simultanément la compréhension conceptuelle et l'intérêt pour la discipline~\citep{delfavero2007classroom}. L'étude, portant sur la Première Guerre mondiale et le miracle économique italien, a montré que la confrontation des interprétations entre pairs activait des processus de réélaboration cognitive absents de l'apprentissage individuel.

\subsubsection*{Approches aux résultats mitigés ou négatifs}

Certaines technologies n'ont pas tenu leurs promesses en contexte historique.

Les agents pédagogiques scriptés présentent des résultats décevants en Histoire. Les méta-analyses montrent un effet négatif sur l'apprentissage dans ce domaine~\citep{davis2019effectiveness}. Cette contre-performance pourrait s'expliquer par l'inadéquation entre la rigidité des réponses préprogrammées et la nature interprétative de la discipline~: un agent incapable de nuancer son discours ou de reconnaître la pluralité des interprétations peut apparaître comme une source d'autorité inappropriée.

Les interfaces conversationnelles textuelles avec figures historiques produisent des résultats quantitatifs mixtes malgré des gains motivationnels~\citep{pataranutaporn2023living}. La comparaison entre dialogues interactifs et lecture traditionnelle indique une amélioration de la motivation mais des effets incertains sur l'apprentissage objectif. L'absence de modalité orale pourrait limiter l'activation des mécanismes de présence sociale.

La simple introduction d'outils technologiques sans repensée pédagogique ne garantit pas l'engagement~\citep{yarema2002using}. Les technologies qui reproduisent le format transmissif sous une forme numérique échouent à transformer la relation des élèves au contenu historique.

\subsubsection*{Conditions de succès transversales}

L'analyse comparative de ces expérimentations suggère trois conditions nécessaires à l'efficacité des technologies en enseignement de l'Histoire~: l'activation de l'agentivité de l'apprenant (les dispositifs efficaces offrent des choix significatifs qui engagent la réflexion), la dimension sociale de l'interaction (les approches les plus prometteuses impliquent une confrontation des perspectives), et la cohérence avec l'épistémologie disciplinaire (les technologies qui respectent la nature interprétative de l'Histoire surpassent celles qui importent un modèle transmissif adapté aux STIM).

Ces constats informent directement la conception de notre dispositif expérimental~: l'agent conversationnel incarnant un personnage historique vise précisément à satisfaire ces trois conditions en offrant une interaction dialogique authentique, socialement engageante, et épistémologiquement cohérente avec la discipline.

