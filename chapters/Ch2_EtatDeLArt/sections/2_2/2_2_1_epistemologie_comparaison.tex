\subsection{Le Triple Décalage de l'Histoire Scolaire}
\label{subsec:triple_decalage}

L'histoire en tant que discipline académique repose sur un processus d'enquête interprétative : le chercheur confronte des sources de natures diverses, reconstruit des contextes disparus et produit des récits qui rendent compte de la complexité du passé \citep{wineburg1991}. L'enseignement scolaire de cette discipline opère cependant une réduction systématique de cette complexité. Les manuels présentent un récit unique, linéaire, sans auteur apparent --- comme si les événements se racontaient d'eux-mêmes. Les sources primaires, lorsqu'elles apparaissent, servent d'illustrations plutôt que de matériaux d'enquête. Les élèves ne sont pas formés à lire un document comme un \textit{site d'interprétation} où se croisent intentions, contextes de production et silences \citep{wineburg1991}. Ce premier décalage, entre la nature épistémologique de la discipline et sa transposition scolaire, affecte directement la perception que les élèves construisent de la matière : l'histoire apparaît comme un répertoire de faits à mémoriser plutôt que comme un processus de construction de sens.

Le deuxième décalage concerne la valeur perçue de la discipline. Les élèves reconnaissent généralement l'importance sociale de l'histoire --- formation du citoyen, compréhension du présent, construction de l'identité collective --- mais perçoivent son enseignement comme déconnecté de leur expérience personnelle \citep{audigier2010}. Ce paradoxe s'analyse par le prisme de la théorie de l'attente-valeur (cf.~\ref{subsec:pertinence_valeur}) : la valeur d'atteinte (l'importance de l'histoire pour l'identité sociale) coexiste avec un déficit de valeur d'utilité (aucune application professionnelle immédiate perçue) et de valeur intrinsèque (peu de plaisir retiré de l'activité scolaire). Les disciplines scientifiques et technologiques bénéficient d'une utilité professionnelle visible qui facilite la construction de la pertinence ; l'histoire exige un travail de connexion personnelle que ni le programme ni le format pédagogique dominant ne facilitent. Ce décalage perceptif s'accentue au secondaire, au moment où le besoin d'autonomie de l'adolescent entre en conflit avec un enseignement prescriptif qui laisse peu de place aux questions de l'élève (cf.~\ref{subsec:sdt_cet}).

Le troisième décalage oppose le potentiel narratif de la discipline à la pauvreté des formats pédagogiques qui la transmettent. L'histoire possède une richesse intrinsèque --- personnages, conflits, dilemmes moraux, tournants imprévisibles --- qui la prédispose à l'engagement émotionnel et à la construction de sens. Pourtant, le format transmissif dominant (cours magistral, lecture de manuels, prise de notes) neutralise ce potentiel en plaçant l'élève en mode passif au sens du cadre ICAP (cf.~\ref{subsec:icap}). L'enseignement reste dominé par la monomodalité textuelle : les enseignants eux-mêmes peinent à concevoir les sources historiques autrement que comme des textes écrits \citep{wineburg1991}. La personnalisation, identifiée comme levier de pertinence (cf.~\ref{subsec:pertinence_valeur}), se heurte aux contraintes logistiques de classes nombreuses et de programmes denses. Des pratiques pédagogiques émergentes cherchent à résoudre simultanément ces trois décalages en combinant engagement actif, connexion personnelle et accès multimodal au passé --- elles font l'objet des sections suivantes.
