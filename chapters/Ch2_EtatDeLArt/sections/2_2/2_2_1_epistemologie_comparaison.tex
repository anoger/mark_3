\subsection{Le Triple Décalage de l'Histoire Scolaire}
\label{subsec:triple_decalage}

L'histoire en tant que discipline académique repose sur un processus d'enquête interprétative : l'historien confronte des sources de natures diverses, reconstruit des contextes disparus et produit des récits argumentés qui rendent compte de la complexité du passé \citep{wineburg1991}. Ce régime épistémologique diffère structurellement de celui des disciplines scientifiques et technologiques (STIM). Là où les STIM produisent des savoirs cumulatifs, vérifiables par l'expérimentation et sanctionnés par un feedback immédiat (correct/incorrect), l'histoire produit des savoirs interprétatifs, contextuels et provisoires, fondés sur l'argumentation à partir de traces fragmentaires \citep{audigier2010}. Cette asymétrie affecte directement la perception des élèves : les STIM offrent la sécurité de réponses définitives, tandis que l'histoire exige une tolérance à l'ambiguïté que l'enseignement scolaire ne forme pas à reconnaître. L'enquête historique, telle que la pratiquent les chercheurs, repose sur l'évaluation critique de documents et la reconstruction de contextes disparus --- un processus que les manuels scolaires occultent en présentant un récit unique, linéaire, sans auteur apparent, comme si les événements se racontaient d'eux-mêmes \citep{wineburg1991}.

Ce premier décalage --- entre la nature épistémologique de la discipline et sa transposition scolaire --- se double d'un déficit de pertinence perçue empiriquement documenté. Les données convergent à travers plusieurs contextes nationaux : sur une échelle d'importance perçue, les mathématiques (4,46) et l'anglais (4,42) devancent nettement l'histoire (3,26), et les matières impliquant une activité concrète --- EPS, technologie, arts --- dominent systématiquement les préférences des élèves \citep{haydn2010, harris2006}. Cette hiérarchie s'analyse par le prisme de la théorie de l'attente-valeur (cf.~\ref{subsec:pertinence_valeur}) : les STIM bénéficient d'une valeur d'utilité professionnelle visible --- débouchés clairement identifiés, applications concrètes --- tandis que l'histoire souffre simultanément d'un déficit de valeur d'utilité (débouchés perçus comme limités à l'enseignement ou à l'archéologie) et de valeur intrinsèque (plaisir réduit par le format transmissif). L'histoire se classe cinquième parmi les matières préférées, derrière toutes celles requérant manipulation physique ou activité concrète \citep{harris2006}.

Ce déficit de pertinence ne signifie pas que l'histoire est intrinsèquement inadaptée à l'engagement. L'écart considérable entre établissements --- de 51\% à 86\% d'appréciation selon les contextes \citep{harris2006} --- et la capacité de l'histoire à rivaliser avec les STIM lorsque les méthodes sont adaptées \citep{henkaline2023} suggèrent que le problème réside dans la modalité pédagogique, non dans le contenu disciplinaire. L'histoire possède un potentiel narratif et dramatique unique --- personnages, conflits, dilemmes moraux, tournants imprévisibles --- qui la prédispose à l'engagement émotionnel. Pourtant, le format transmissif dominant neutralise ce potentiel en plaçant l'élève en mode passif au sens du cadre ICAP (cf.~\ref{subsec:icap}). Ce troisième décalage --- entre le potentiel narratif de la discipline et la pauvreté des formats qui la transmettent --- appelle un diagnostic plus fin de la perception des élèves, objet de la section suivante.
