\subsection{Nature épistémologique de l'histoire et conséquences pédagogiques}
\label{subsec:epistemologie_histoire}

L'histoire se distingue des disciplines scientifiques par la nature de ses savoirs et les opérations cognitives qu'elle exige. Les sciences expérimentales produisent des connaissances cumulatives, vérifiables par reproduction et exprimées dans un langage formel : un résultat en physique ou en chimie est soit confirmé soit réfuté par l'expérience, et le feedback est immédiat \citep{vanStraaten2016}. L'histoire produit des savoirs interprétatifs, fondés sur l'analyse critique de sources lacunaires et contradictoires, dont la validité repose sur la qualité de l'argumentation plutôt que sur la démonstration expérimentale \citep{grever2011}. La compréhension historique ne consiste pas à mémoriser une séquence de faits, mais à construire un récit argumenté qui rend compte de la complexité des situations passées en intégrant des perspectives multiples \citep{miguelrevilla2021}. Cette différence épistémologique a deux conséquences directes pour l'enseignement : l'une concerne la perception de valeur, l'autre les exigences cognitives.

La première conséquence porte sur la valeur d'utilité perçue (cf.~\ref{subsec:pertinence_valeur}). Les disciplines scientifiques bénéficient d'une perception d'utilité professionnelle immédiate : les débouchés en ingénierie, en médecine ou en technologie sont identifiables et socialement valorisés. Les données empiriques convergent : les élèves britanniques évaluent l'importance perçue des mathématiques et de l'anglais significativement au-dessus de celle de l'histoire, et les élèves néerlandais jugent l'histoire moins utile que l'anglais, l'économie et les mathématiques \citep{haydn2010, vanStraaten2016}. Ce déficit ne résulte pas d'un manque de valeur intrinsèque : les mêmes enquêtes montrent que les élèves reconnaissent l'importance de l'histoire pour comprendre le monde contemporain \citep{cairns2024}. L'asymétrie porte spécifiquement sur la valeur d'utilité --- la perception que le contenu sert des objectifs concrets et identifiables --- qui constitue, dans le cadre de la théorie de l'attente-valeur, l'un des quatre déterminants de l'engagement (cf.~\ref{subsec:pertinence_valeur}). L'absence de feedback objectif et immédiat, caractéristique des sciences expérimentales, renforce ce déficit : en histoire, les critères de réussite sont plus diffus et les élèves peinent à évaluer leur propre progression \citep{henkaline2023}.

La seconde conséquence concerne les opérations cognitives requises. La compréhension historique mobilise la prise de perspective (se représenter les motivations et contraintes d'acteurs situés dans un contexte différent), la construction narrative (organiser des informations fragmentaires en un récit cohérent) et l'évaluation critique de sources (confronter des témoignages contradictoires pour en extraire une interprétation fondée) \citep{gomez2021}. Ces opérations relèvent des modes constructif et interactif du cadre ICAP (cf.~\ref{subsec:icap}) : elles exigent que l'apprenant génère des inférences qui dépassent le matériel présenté, et non qu'il reproduise une information reçue. L'enseignement de l'histoire se trouve ainsi face à un paradoxe : la discipline exige, par sa nature épistémologique, les formes les plus élaborées d'engagement cognitif, mais sa pédagogie dominante reste organisée autour de la transmission d'un récit à mémoriser \citep{audigier2010}. Le format transmissif --- cours magistral, prise de notes, restitution factuelle --- réduit l'activité de l'élève au mode passif, en contradiction directe avec les exigences cognitives de la discipline.

