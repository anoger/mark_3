\subsection{Nature de l'histoire et conséquences pour l'enseignement}
\label{subsec:epistemologie_histoire}

L'histoire tend à produire un type de savoir distinct de celui des disciplines scientifiques. En physique ou en mathématiques, un résultat est confirmé ou réfuté par la démonstration, le feedback tend à être immédiat et les critères de réussite plus explicites. L'histoire, à l'inverse, produit des savoirs interprétatifs fondés sur l'analyse de sources lacunaires et contradictoires : la validité d'une interprétation repose sur la qualité de l'argumentation, non sur la reproduction d'un résultat \citep{vanStraaten2016}. Comprendre un événement historique, c'est construire un récit argumenté qui intègre des perspectives multiples, et non mémoriser une séquence de faits \citep{grever2011}. Cette différence a deux conséquences pour l'enseignement : l'une porte sur la perception de valeur, l'autre sur les exigences cognitives.

La première conséquence concerne la valeur d'utilité perçue (cf.~\ref{subsec:pertinence_valeur}). Les disciplines scientifiques tendent à offrir des débouchés professionnels plus identifiables --- ingénierie, médecine, technologie --- ce qui peut rendre leur utilité plus visible. Les données suggèrent un déficit perçu de l'histoire : les élèves britanniques et néerlandais la jugent moins utile que les mathématiques, l'anglais ou l'économie \citep{haydn2010}. Ce déficit ne traduit pas un rejet : les élèves reconnaissent l'importance de l'histoire pour comprendre le monde contemporain \citep{cairns2024}. L'asymétrie porte sur la valeur d'utilité au sens de la théorie de l'attente-valeur (cf.~\ref{subsec:pertinence_valeur}) --- la perception que le contenu sert des objectifs concrets. L'absence de feedback objectif et immédiat, propre aux disciplines scientifiques, tend à aggraver ce déficit : en histoire, les critères de réussite restent diffus et les élèves peinent à évaluer leur propre progression \citep{kristinhenkaline2023}.

La seconde conséquence porte sur les opérations cognitives que la discipline exige. Comprendre l'histoire mobilise la prise de perspective (se représenter les contraintes d'acteurs situés dans un autre contexte), la construction narrative (organiser des informations fragmentaires en un récit cohérent) et l'évaluation critique de sources (confronter des témoignages contradictoires) \citep{gomez2021}. Ces opérations semblent relever des modes constructif et interactif du cadre ICAP (cf.~\ref{subsec:icap}) : elles exigent que l'apprenant produise des inférences qui dépassent le matériel présenté. Or la pédagogie dominante en histoire reste organisée autour de la transmission d'un récit à mémoriser \citep{audigier2010}. Le cours magistral, la prise de notes et la restitution factuelle tendent à réduire l'activité de l'élève au mode passif --- en contradiction directe avec ce que la discipline requiert.

