\subsection{Activer le potentiel latent : interventions actives et interactives en histoire}
\label{subsec:pratiques_pedagogiques_histoire}

Les interventions qui réussissent à engager les élèves en histoire partagent un mécanisme commun : elles déplacent l'activité cognitive de l'apprenant vers les modes constructif ou interactif du cadre ICAP (cf.~\ref{subsec:icap}) en activant le potentiel narratif, empathique et argumentatif de la discipline. Trois catégories d'interventions illustrent ce déplacement, chacune mobilisant un levier distinct.

La première catégorie repose sur la \textit{prise de perspective historique} par le jeu de rôle et l'histoire orale. Un dispositif d'entretiens fictifs dans lequel les étudiants incarnent des personnages historiques et échangent avec leurs pairs produit un engagement qui dépasse la restitution factuelle : les participants rapportent avoir développé une compréhension de l'ambiguïté inhérente aux sources historiques et une empathie pour les acteurs du passé \citep{stevens2015}. Ce type d'activité active le mode interactif au sens strict du cadre ICAP : chaque contribution d'un participant s'appuie sur la précédente et la transforme, dans un processus de co-construction qui va au-delà de l'alternance questions-réponses. L'analyse de jeux vidéo historiques en tant que sources multimodales --- combinant narration, iconographie et mécaniques ludiques --- mobilise un mécanisme comparable : l'apprenant construit activement une interprétation du passé en confrontant la représentation ludique à ses connaissances historiques \citep{redder2024}. La dimension constructive de cette activité réside dans la production d'inférences sur les écarts entre la représentation et la réalité historique.

La deuxième catégorie exploite l'\textit{immersion et la narration interactive}. L'exploration d'une reconstruction virtuelle en 3D de la cité antique d'Uruk produit des résultats supérieurs en termes de performance et d'engagement par rapport à un format d'apprentissage traditionnel : les apprenants passent plus de temps à explorer les éléments historiques et à interagir avec les personnages virtuels de l'environnement \citep{ijaz2017}. Un dispositif de narration numérique interactive situé dans l'Athènes antique, intégrant des points de décision stratégiques, stimule les discussions de groupe et approfondit la compréhension historique : les élèves s'engagent dans une délibération collective sur les choix des acteurs historiques, ce qui active simultanément la prise de perspective et le mode interactif \citep{petousi2022}. Dans un contexte muséal, un dispositif combinant interaction tangible et récits émotionnels autour de personnages historiques encourage les adolescents à explorer des perspectives émotionnelles multiples, développant l'empathie historique au-delà de la connaissance factuelle \citep{roussou2024}.

La troisième catégorie concerne la \textit{résolution collaborative de problèmes historiques}. Un protocole comparant travail individuel et travail en groupe sur des problématiques liées à la Première Guerre mondiale montre que la discussion collaborative améliore à la fois la compréhension des concepts historiques et l'intérêt pour la discipline \citep{delfavero2007}. Le mécanisme est celui de la co-construction identifié par le cadre ICAP : l'échange entre pairs oblige chaque participant à expliciter son raisonnement, à confronter ses interprétations à celles des autres et à réviser ses représentations. La formation des enseignants aux méthodes actives et à l'épistémologie historique amplifie ces effets : un programme combinant apprentissage par projets, études de cas et travail sur les compétences de pensée historique produit des améliorations significatives de la motivation et de l'apprentissage perçu chez 467~élèves du secondaire \citep{gomez2021}.

Ces interventions partagent cependant un ensemble de limites. Les jeux de rôle et les simulations exigent une préparation lourde et un accompagnement expert. Les environnements virtuels immersifs requièrent des infrastructures techniques coûteuses. Les dispositifs narratifs interactifs reposent sur des scénarios prédéterminés qui ne s'adaptent pas aux questions individuelles des apprenants. La résolution collaborative de problèmes dépend de la dynamique de groupe et du niveau de formation de l'enseignant. Aucune de ces interventions ne résout simultanément trois exigences : soutenir un dialogue co-constructif en temps réel, s'adapter aux intérêts et au niveau de chaque apprenant, et se déployer à l'échelle d'une classe entière sans surcharge de préparation pour l'enseignant.

