\subsection{Trois Formats d'Engagement Actif : Simulation, Incarnation, Multimodalité}
\label{subsec:formats_actifs}

La simulation sur table (\textit{tabletop wargaming}) constitue un dispositif d'échec productif appliqué à l'enseignement de l'histoire. Dans un protocole universitaire portant sur la Première Guerre mondiale, les étudiants confrontés à un wargame sur table découvrent par l'expérience directe que les décisions stratégiques du passé ne résultent pas de l'incompétence des acteurs mais de contraintes situationnelles que le récit rétrospectif efface \citep{reynaud2015}. L'échec dans la simulation --- une offensive qui échoue, une position devenue intenable --- provoque une dissonance cognitive qui force la révision des représentations initiales. Les étudiants passent d'une conception de l'histoire comme contenu à retenir à une compréhension de l'histoire comme enchaînement de décisions prises sous contrainte. Ce mécanisme, que le cours magistral ne parvient pas à activer, opère précisément parce que la simulation place l'apprenant en position de décideur plutôt que de spectateur.

Le jeu de rôle oral engage un mécanisme distinct : la construction de l'empathie historique par l'incarnation d'un personnage. Un dispositif de \textit{speed dating} fondé sur des récits oraux permet aux élèves d'incarner des témoins de la partition de l'Inde et de dialoguer entre eux en adoptant la perspective de leur personnage \citep{stevens2015}. L'exercice développe la multiperspectivité --- la capacité à appréhender un même événement depuis des positions sociales différentes --- et donne accès à des voix marginalisées absentes du manuel scolaire. Les élèves rapportent un sentiment accru d'engagement et de compréhension après l'activité. La portée du dispositif connaît toutefois des limites : une minorité d'élèves tire peu de bénéfice de l'exercice, ce qui indique que l'efficacité du jeu de rôle dépend de la préparation, du format et de la disposition individuelle à entrer dans la fiction historique.

Le jeu vidéo historique opère une reconfiguration épistémologique plus radicale. Les jeux historiques ne se réduisent pas à des illustrations ludiques du passé : ils constituent des représentations multimodales qui combinent exploration spatiale, narration, interaction et sources primaires intégrées dans l'environnement de jeu \citep{redder2024}. Cette multimodalité repositionne l'élève en \textit{joueur-historien} qui accède au passé non par la lecture d'un texte mais par la navigation dans un monde reconstruit. Trois modalités historiques coexistent dans ces environnements : l'histoire documentée (\textit{lore history}), l'histoire imaginée à partir de traces fragmentaires (\textit{imaginative history}) et l'histoire alternative qui explore des possibles non réalisés (\textit{alternate history}). Le défi réside dans la résistance institutionnelle : les enseignants formés à la monomodalité textuelle peinent à reconnaître le jeu vidéo comme une source historique légitime, ce qui limite l'intégration de ces dispositifs dans les pratiques scolaires.

Ces trois formats résolvent chacun un aspect du triple décalage identifié en~\ref{subsec:triple_decalage}. La simulation corrige le décalage épistémologique : l'élève découvre que comprendre le passé ne se réduit pas à mémoriser des faits mais implique de reconstruire les contraintes qui pesaient sur les acteurs. L'incarnation par le jeu de rôle comble le décalage perceptif : le contenu historique acquiert une pertinence émotionnelle lorsque l'apprenant adopte la perspective d'un témoin. La multimodalité du jeu historique dépasse le décalage pédagogique : l'accès au passé ne se limite plus au texte écrit mais s'étend à l'exploration spatiale, sonore et interactive. Cependant, aucun de ces dispositifs ne combine les trois résolutions : la simulation reste abstraite sur le plan émotionnel, le jeu de rôle dépend d'une logistique de classe difficile à généraliser et le jeu vidéo pose des problèmes de validation disciplinaire. La section suivante examine un type d'interface --- l'agent conversationnel incarné --- susceptible d'intégrer ces trois dimensions : l'interactivité de la simulation, la connexion personnelle de l'incarnation et la multimodalité de l'environnement numérique.
