\subsection{Trois Formats d'Engagement Actif : Simulation, Incarnation, Multimodalité}
\label{subsec:formats_actifs}

La simulation sur table (\textit{tabletop wargaming}) constitue un dispositif d'échec productif appliqué à l'enseignement de l'histoire. Dans un protocole universitaire portant sur la Première Guerre mondiale, les étudiants confrontés à un wargame découvrent par l'expérience directe que les décisions stratégiques du passé ne résultent pas de l'incompétence des acteurs mais de contraintes situationnelles que le récit rétrospectif efface \citep{reynaud2015}. L'échec dans la simulation --- une offensive qui échoue, une position devenue intenable --- provoque une dissonance cognitive qui force la révision des représentations initiales. Les étudiants passent d'une conception de l'histoire comme contenu à retenir à une compréhension de l'histoire comme enchaînement de décisions prises sous contrainte. Ce mécanisme active le mode constructif du cadre ICAP (cf.~\ref{subsec:icap}) : l'apprenant génère des inférences qui dépassent le matériel présenté. Cependant, la simulation reste abstraite sur le plan de la connexion émotionnelle --- elle modélise des contraintes stratégiques sans incarner les individus qui les subissent.

Le jeu de rôle oral engage un mécanisme distinct : la construction de l'empathie historique par l'incarnation d'un personnage. Un dispositif de \textit{speed dating} fondé sur des récits oraux permet aux élèves d'incarner des témoins de la partition de l'Inde et de dialoguer entre eux en adoptant la perspective de leur personnage \citep{stevens2015}. L'exercice développe la multiperspectivité --- la capacité à appréhender un même événement depuis des positions sociales différentes --- et donne accès à des voix marginalisées absentes du manuel scolaire. L'activité active le mode interactif du cadre ICAP par la co-construction entre pairs. Les élèves rapportent un sentiment accru d'engagement et de compréhension. Toutefois, une minorité d'élèves tire peu de bénéfice de l'exercice, ce qui indique que l'efficacité du jeu de rôle dépend de la préparation, du format et de la disposition individuelle à entrer dans la fiction historique. Le dispositif se heurte en outre à une contrainte logistique majeure : il exige une organisation de classe spécifique, difficile à généraliser à l'échelle d'un système éducatif.

Le jeu vidéo historique opère une reconfiguration épistémologique plus radicale. Les jeux historiques ne se réduisent pas à des illustrations ludiques du passé : ils constituent des représentations multimodales qui combinent exploration spatiale, narration, interaction et sources primaires intégrées dans l'environnement de jeu \citep{redder2024}. Cette multimodalité repositionne l'élève en \textit{joueur-historien} qui accède au passé non par la lecture d'un texte mais par la navigation dans un monde reconstruit. Trois modalités historiques coexistent dans ces environnements : l'histoire documentée (\textit{lore history}), l'histoire imaginée à partir de traces fragmentaires (\textit{imaginative history}) et l'histoire alternative qui explore des possibles non réalisés (\textit{alternate history}). Le défi réside dans la résistance institutionnelle : les enseignants formés à la monomodalité textuelle peinent à reconnaître le jeu vidéo comme une source historique légitime, ce qui limite son intégration dans les pratiques scolaires.

Ces trois formats répondent au paradoxe identifié en~\ref{subsec:perception_eleves} en activant les modes constructif et interactif du cadre ICAP. La simulation produit des inférences par l'échec productif, le jeu de rôle génère une co-construction par les échanges entre pairs, le jeu vidéo combine exploration autonome et analyse critique. Cependant, chaque dispositif présente des limites structurelles qui en restreignent la portée. La simulation reste abstraite sur le plan de la connexion personnelle : elle modélise des contraintes mais ne crée pas de relation affective avec un individu du passé. Le jeu de rôle dépend d'une logistique de classe difficile à généraliser et ne s'adapte pas au profil individuel de l'apprenant. Le jeu vidéo pose des problèmes de validation disciplinaire et ne propose pas d'interaction conversationnelle avec un interlocuteur historique. Aucun de ces dispositifs ne combine interactivité adaptative, personnalisation au profil de l'apprenant et incarnation d'un interlocuteur capable de répondre aux questions de l'élève --- trois dimensions que les agents conversationnels alimentés par l'intelligence artificielle pourraient intégrer. La section suivante examine le cadre théorique de cette interaction.
