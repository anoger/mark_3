\subsection{Interventions actives en histoire : effets et limites}
\label{subsec:pratiques_pedagogiques_histoire}

Les interventions qui parviennent à engager les élèves en histoire partagent un trait commun : elles déplacent l'activité cognitive vers les modes constructif ou interactif du cadre ICAP (cf.~\ref{subsec:icap}). Trois leviers d'intervention illustrent ce déplacement.

Le premier levier est la \textit{prise de perspective historique}. Un dispositif d'entretiens fictifs dans lequel les élèves incarnent des personnages historiques produit un engagement qui dépasse la restitution factuelle : les participants développent une compréhension de l'ambiguïté des sources et une empathie pour les acteurs du passé \citep{stevens2015}. Chaque contribution s'appuie sur la précédente et la transforme, ce qui correspond au mode interactif du cadre ICAP. L'analyse de jeux vidéo historiques comme sources multimodales mobilise un mécanisme comparable : l'apprenant construit une interprétation en confrontant la représentation ludique à ses connaissances \citep{redder2024}.

Le deuxième levier est l'\textit{immersion narrative}. L'exploration d'une reconstruction virtuelle en 3D d'une cité antique améliore la performance et l'engagement par rapport à un format traditionnel \citep{ijaz2017}. Un dispositif de narration interactive situé dans l'Athènes antique, intégrant des points de décision, stimule la délibération collective sur les choix des acteurs historiques \citep{petousi2022}. Dans un contexte muséal, un dispositif combinant interaction tangible et récits autour de personnages historiques développe l'empathie historique chez les adolescents \citep{roussou2024}. Dans ces trois cas, l'apprenant ne reçoit pas un récit : il s'y projette, ce qui crée la connexion personnelle nécessaire au maintien de l'intérêt situationnel (cf.~\ref{subsec:architecture_interet}).

Le troisième levier est la \textit{résolution collaborative de problèmes}. La discussion en groupe sur des problèmes liés à la Première Guerre mondiale améliore à la fois la compréhension des concepts et l'intérêt pour la discipline \citep{delfavero2007}. L'échange entre pairs oblige chaque participant à expliciter son raisonnement et à réviser ses représentations --- le mécanisme de co-construction du cadre ICAP. La formation des enseignants aux méthodes actives amplifie ces effets : un programme combinant apprentissage par projets et travail sur les compétences de pensée historique produit des améliorations significatives de la motivation chez des élèves du secondaire \citep{gomez2021}.

Ces interventions partagent cependant des limites. Les jeux de rôle exigent une préparation lourde. Les environnements virtuels immersifs requièrent des infrastructures coûteuses. Les dispositifs narratifs reposent sur des scénarios figés qui ne s'adaptent pas aux questions individuelles. La résolution collaborative dépend de la dynamique de groupe et de la formation de l'enseignant. Aucune ne résout simultanément trois exigences : soutenir un dialogue co-constructif en temps réel, s'adapter au niveau et aux intérêts de chaque apprenant, et se déployer à l'échelle d'une classe entière sans surcharge pour l'enseignant.

