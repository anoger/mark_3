% ============================================================================
% Sous-section 2.6.2 : Les Lacunes de la Littérature Actuelle
% ============================================================================
% Sources : rapport_mapper (Research Paper Mapper)
% Calibrage : ~500 mots
% Type : C (Rédaction originale)
% ============================================================================

\subsection{Les Lacunes de la Littérature Actuelle}
\label{subsec:lacunes_litterature}

L'analyse systématique de la littérature sur les agents pédagogiques révèle plusieurs lacunes que ce travail doctoral vise à combler.

\textbf{Sous-représentation des populations K-12.} Les méta-analyses récentes~\citep{dai2022systematic, schroeder2025designing} convergent sur un constat~: la majorité des études portent sur des populations universitaires, avec une sous-représentation significative des élèves du secondaire. Cette lacune est problématique à double titre. D'une part, les mécanismes motivationnels et métacognitifs varient avec le développement~; les résultats obtenus avec des adultes ne sont pas directement transférables aux adolescents. D'autre part, c'est précisément durant l'adolescence que se cristallisent les attitudes envers les disciplines scolaires~; intervenir à ce stade présente un potentiel d'impact maximal.

\textbf{Absence d'études dans le domaine de l'Histoire.} La littérature se concentre massivement sur les domaines STIM, avec des effets particulièrement marqués en biologie et en informatique~\citep{davis2019effectiveness}. L'Histoire reste un terrain quasi inexploré. Une méta-analyse a identifié un effet négatif des agents pédagogiques en histoire ($g^+ = -0.80$), mais ce résultat isolé repose sur un nombre limité d'études et nécessite une investigation approfondie. La spécificité épistémologique de l'Histoire --- discipline interprétative où la narrativité joue un rôle central --- justifie une attention particulière.

\textbf{Études antérieures à l'ère des LLM.} La quasi-totalité des travaux fondateurs~\citep{schroeder2013examining, johnson2016face} a été conduite avec des agents scriptés. L'avènement des grands modèles de langage en 2023 a changé la donne~: les agents peuvent désormais générer des réponses contextuellement adaptées avec une fluidité conversationnelle sans précédent. Les conclusions des études antérieures, notamment sur l'importance relative de l'apparence visuelle versus les capacités fonctionnelles, méritent d'être réexaminées dans ce contexte.

\textbf{Mesures focalisées sur les niveaux cognitifs inférieurs.} Les protocoles d'évaluation se concentrent sur la mémorisation et la compréhension de surface~\citep{dai2022systematic}. Les niveaux cognitifs supérieurs de la taxonomie de Bloom restent peu explorés. Plus fondamentalement, les mesures métacognitives sont rares~: peu d'études évaluent explicitement la calibration de la confiance ou l'illusion de compréhension. Cette lacune est d'autant plus problématique que l'illusion de maîtrise constitue potentiellement le risque majeur des agents génératifs.

\textbf{Absence d'études longitudinales.} La durée médiane des interventions est de l'ordre de 30 minutes à quelques heures~\citep{veletsianos2013pedagogical}. Cette brièveté empêche d'évaluer les effets d'apprentissage durables, la rétention à long terme, et l'évolution de la relation élève-agent. L'effet de nouveauté n'est jamais contrôlé, confondant enthousiasme initial et efficacité durable. Des études longitudinales sont nécessaires pour établir la valeur pédagogique réelle des agents conversationnels.

Ce travail doctoral vise à combler ces lacunes en conduisant des études expérimentales auprès d'élèves du secondaire français, dans le domaine de l'Histoire, avec des agents alimentés par IA générative, et en intégrant des mesures métacognitives explicites (protocole IOED). Les limites temporelles (sessions uniques) constituent une contrainte assumée, appelant des travaux complémentaires ultérieurs.
