\subsection{Lacunes de la littérature actuelle}
\label{subsec:lacunes_litterature}

La première lacune concerne l'interactivité orale avec un agent génératif en contexte écologique. Les méta-analyses sur les agents pédagogiques portent principalement sur des agents scriptés dont les réponses sont sélectionnées dans un répertoire fixe (cf.~\ref{subsec:incarnation_agence_sociale}). Les LLM modifient qualitativement l'interaction en permettant une adaptation sémantique en temps réel (cf.~\ref{subsec:agents_scriptes_generatifs}), mais les preuves empiriques de leur efficacité en contexte scolaire réel restent rares. Les études existantes reposent sur des échantillons restreints, des prototypes et des mesures immédiates, ce qui limite leur validité écologique (cf.~\ref{subsec:agents_scriptes_generatifs}). Le cadre ICAP distingue les modes passif et interactif (cf.~\ref{subsec:icap}), mais cette distinction n'a pas été testée avec des agents génératifs déployés en classe. On ne sait pas si l'avantage interactif documenté avec les agents scriptés se maintient avec des agents alimentés par des LLM.

La deuxième lacune porte sur l'alignement thématique de l'agent et le stade développemental. La congruence entre l'apparence de l'agent et le domaine enseigné produit des effets variables selon le type d'alignement (cf.~\ref{subsec:incarnation_agence_sociale}). Aucune étude ne teste cependant l'incarnation d'un personnage historique comme variable d'alignement dans l'enseignement de l'histoire. Le modèle de l'intérêt prédit que la qualité du déclencheur dépend du stade développemental de l'apprenant (cf.~\ref{subsec:architecture_interet}), mais le rôle de l'âge dans la réponse à l'alignement thématique n'a pas été étudié avec des agents génératifs. Les effets des agents sont plus marqués chez les 10-14~ans (cf.~\ref{subsec:incarnation_agence_sociale}), mais la littérature ne couvre pas le spectre complet de l'adolescence. On ne sait pas si l'alignement thématique de l'agent module l'intérêt de manière différenciée selon l'âge, ni si le style de présentation --- formel ou accessible --- interagit avec le stade développemental.

La troisième lacune concerne l'illusion de compréhension induite par l'interaction avec un agent génératif. L'illusion de compréhension (cf.~\ref{subsec:metacognition_calibration}) et l'heuristique de fluidité (cf.~\ref{subsec:heuristique_fluidite}) sont documentées dans des contextes multimédia classiques --- vidéos, animations, présentations passives. Les agents conversationnels génératifs combinent toutefois fluence linguistique, incarnation anthropomorphe et adaptabilité en temps réel, une combinaison inédite dont les effets métacognitifs n'ont pas été mesurés. La littérature sur la surconfiance envers l'IA porte sur des adultes ou des collégiens interagissant avec des chatbots textuels (cf.~\ref{subsec:autorite_agent_vigilance}). L'effet d'une incarnation visuelle --- humanoïde ou abstraite --- sur l'illusion de compréhension n'a pas été testé. Le protocole IOED, conçu pour mesurer la profondeur illusoire des explications causales \citep{rozenblit2002}, n'a pas été adapté aux contextes d'interaction avec des agents génératifs incarnés. On ne sait pas si l'apparence de l'agent amplifie l'illusion de compréhension, ni si la fluence de l'interaction suffit à induire une surévaluation de la compréhension indépendamment du design visuel.

