% ============================================================================
% Sous-section 2.6.1 : Convergence et Tensions Théoriques (V3)
% ============================================================================
% Corrections appliquées :
% - Lacune 6 : Synthèse par références croisées (évite répétition ICAP)
% - Lacune 18 : Apport nouveau explicité (hypothèse équilibre symétrique)
% ============================================================================

\subsection{Convergence et Tensions Théoriques}
\label{subsec:convergences_tensions}

L'analyse de la littérature révèle une tension fondamentale~: les mêmes caractéristiques qui rendent les agents conversationnels pédagogiquement prometteurs constituent simultanément leurs principaux risques.

Du côté des convergences positives, les cadres théoriques présentés dans ce chapitre s'articulent en une prédiction cohérente. L'interactivité dialogique active les modes d'engagement les plus profonds selon ICAP (\S\ref{subsec:ICAP}). La présence d'un agent personnifié déclenche un contrat de partenariat qui augmente l'effort cognitif selon la théorie de l'agence sociale (\S\ref{subsec:presence_sociale}). Le paradigme CASA explique pourquoi des indices sociaux minimaux suffisent à activer nos scripts relationnels (\S\ref{subsec:presence_sociale}). La théorie de l'autodétermination identifie l'interactivité comme vecteur de satisfaction du besoin d'autonomie (\S\ref{subsec:SDT_CET}). Ces cadres convergent vers une prédiction commune~: un agent conversationnel interactif devrait favoriser l'engagement et l'intérêt des apprenants.

Du côté des tensions, les caractéristiques mêmes qui génèrent l'engagement peuvent simultanément compromettre la qualité de l'apprentissage. La fluidité conversationnelle active l'heuristique qui conduit l'apprenant à confondre facilité de traitement et maîtrise du contenu (\S\ref{subsec:fluency_heuristic}). L'incarnation réaliste peut agir comme un détail séduisant qui détourne l'attention du contenu. L'autorité perçue de l'agent peut désactiver la vigilance épistémique nécessaire à un apprentissage critique (\S\ref{subsec:autorite_vigilance}).

Cette tension suggère ce que nous proposons d'appeler l'\textit{hypothèse de l'équilibre des défauts symétriques}. Dans cette perspective, l'interactivité et l'incarnation constituent des leviers robustes pour l'engagement --- mais chaque gain sur cette dimension s'accompagne potentiellement d'un risque métacognitif. L'agent idéal ne serait pas celui qui maximise l'engagement, mais celui qui optimise le rapport entre engagement généré et risque d'illusion induit. Cette hypothèse constitue l'apport spécifique de notre analyse~: elle reformule la question du design des agents pédagogiques comme un problème d'optimisation multi-objectif plutôt que de maximisation unidimensionnelle.

Cette hypothèse trouve un écho dans les recommandations de la littérature. Plutôt que de concevoir des outils \og transparents\fg{} qui s'effacent pour laisser place à la tâche~\citep{heidegger2016being}, certains auteurs suggèrent de \og scripter la rupture\fg{}~: concevoir des agents qui, loin de fournir des réponses fluides et directes, exposent délibérément leurs limites, admettent leur incertitude, ou forcent l'apprenant à valider l'information~\citep{solyst2024generative}. L'objectif n'est plus de maximiser la confiance, mais de la \textit{calibrer} en maintenant active la vigilance critique.

Ces tensions théoriques se traduisent en questions de design concrètes. Comment concevoir un agent suffisamment engageant pour maintenir l'attention, sans que cette qualité ne désactive les mécanismes de vigilance? L'incarnation visuelle amplifie-t-elle le risque métacognitif, ou la fluidité conversationnelle constitue-t-elle à elle seule le facteur déterminant? L'alignement thématique entre le personnage et le contenu peut-il servir de levier d'engagement sans les coûts attentionnels d'une incarnation réaliste? Ces questions structurent directement notre programme de recherche.
