% ============================================================================
% Sous-section 2.6.3 : Questions et Hypothèses de Recherche
% ============================================================================
% Calibrage : ~500 mots
% Type : C (Rédaction originale)
% ============================================================================

\subsection{Questions et Hypothèses de Recherche}
\label{subsec:questions_hypotheses}

Sur la base des convergences théoriques, des tensions identifiées et des lacunes de la littérature, nous formulons trois questions de recherche qui structurent ce travail doctoral.

\subsubsection*{Question de Recherche 1 (QR1)}
\textit{Dans quelle mesure l'interactivité d'un agent conversationnel incarné influence-t-elle l'intérêt des élèves pour l'activité, le contenu historique et le personnage?}

Cette question interroge le potentiel d'engagement des agents conversationnels. La littérature prédit un effet positif de l'interactivité sur l'intérêt, mais cet effet reste à démontrer empiriquement dans le contexte spécifique de l'enseignement de l'Histoire auprès d'adolescents français. Nous formulons l'hypothèse suivante~:

\begin{hypothesis}[H1]
L'interactivité de l'agent génère un effet positif significatif sur l'intérêt des élèves, mesurable sur les trois dimensions (activité, contenu, personnage), indépendamment du niveau scolaire.
\end{hypothesis}

\subsubsection*{Question de Recherche 2 (QR2)}
\textit{L'alignement thématique entre le personnage historique et le contenu pédagogique module-t-il l'effet de l'interactivité sur l'intérêt?}

Cette question explore l'hypothèse selon laquelle l'efficacité des agents dépend de leur cohérence thématique avec le contenu. Un personnage aligné (Napoléon enseignant l'Empire) pourrait amplifier l'engagement par rapport à un agent neutre, en activant des mécanismes d'empathie historique et de pertinence perçue. Nous formulons l'hypothèse suivante~:

\begin{hypothesis}[H2]
L'alignement thématique entre le personnage et le contenu amplifie l'effet positif de l'interactivité sur l'intérêt, avec une interaction significative entre les deux facteurs.
\end{hypothesis}

\subsubsection*{Question de Recherche 3 (QR3)}
\textit{L'apparence de l'agent (humanoïde vs. abstrait) influence-t-elle la propension des élèves à l'illusion de compréhension, leur confiance et la crédibilité perçue des informations délivrées?}

Cette question aborde le risque métacognitif central de la thèse. La littérature sur le paradigme CASA et la théorie de l'agence sociale prédit qu'un agent humanoïde devrait générer une confiance et une crédibilité accrues. L'hypothèse concurrente, issue de l'analyse de l'heuristique de fluidité, suggère que la fluidité conversationnelle pourrait constituer le signal dominant, éclipsant l'effet de l'apparence. Nous formulons les hypothèses suivantes~:

\begin{hypothesis}[H3a]
L'agent humanoïde induit une confiance et une persuasion perçue supérieures à l'agent abstrait.
\end{hypothesis}

\begin{hypothesis}[H3b]
L'agent humanoïde amplifie l'illusion de compréhension par rapport à l'agent abstrait, mesurable par une augmentation de l'écart entre confiance subjective (IOED T3) et performance objective.
\end{hypothesis}

\subsubsection*{Articulation des études}
Ces questions de recherche sont traitées par deux études empiriques complémentaires. L'\textbf{Étude 1} (Chapitre~\ref{chap:etude1}) examine les effets de l'interactivité et de l'alignement thématique sur l'intérêt (QR1, QR2) à travers trois expérimentations auprès de populations d'âges différents (4\textsuperscript{ème}, 6\textsuperscript{ème}, Terminale). L'\textbf{Étude 2} (Chapitre~\ref{chap:etude2}) examine l'effet du design visuel sur la confiance et l'illusion de compréhension (QR3) auprès d'élèves de 5\textsuperscript{ème}. Cette articulation permet d'explorer séparément les dimensions d'engagement et de risque métacognitif, avant de les confronter dans la discussion générale.
