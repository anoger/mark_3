\subsection{Questions de recherche et hypothèses}
\label{subsec:questions_hypotheses}

Ces lacunes motivent trois questions de recherche complémentaires. Elles suivent une progression dialectique : de l'exploration des bénéfices potentiels vers l'examen des risques associés.

\textbf{QR1 : Dans quelle mesure l'interactivité orale directe avec un agent historique alimenté par l'IA influence-t-elle l'intérêt des élèves par rapport à une présentation vidéo passive ?}

Cette question opérationnalise la distinction entre les modes passif et interactif du cadre ICAP (cf.~\ref{subsec:icap}) en contexte d'agent génératif. La théorie de l'agence sociale prédit que les indices sociaux réactifs activent la présence sociale et augmentent l'effort cognitif (cf.~\ref{subsec:incarnation_agence_sociale}). Le dépassement du plafond des agents scriptés (cf.~\ref{subsec:agents_scriptes_generatifs}) permet de tester cette prédiction avec une adaptation sémantique en temps réel. L'hypothèse H1 prédit que l'interaction orale avec l'agent génère un intérêt supérieur à la présentation vidéo passive, sur les trois dimensions mesurées : intérêt pour l'activité, pour le contenu et pour le personnage.

\textbf{QR2 : Comment l'alignement thématique de l'agent (personnage historique vs. neutre ; pair vs. autorité) et son style de présentation (formel vs. accessible) modulent-ils cet intérêt en fonction du stade développemental des élèves ?}

Cette question explore les conditions aux limites de l'effet d'interactivité. Le modèle de l'intérêt identifie le personnage historique comme déclencheur potentiel d'intérêt situationnel et la pertinence personnelle comme facteur de maintien (cf.~\ref{subsec:architecture_interet},~\ref{subsec:pertinence_valeur}). La congruence thématique entre l'agent et le domaine enseigné peut renforcer l'engagement (cf.~\ref{subsec:incarnation_agence_sociale}). Les effets des agents étant plus marqués chez les 10-14~ans (cf.~\ref{subsec:incarnation_agence_sociale}), l'âge constitue un modérateur à tester sur un spectre développemental étendu. L'hypothèse H2 prédit que l'incarnation d'un personnage historique aligné avec le contenu génère un intérêt supérieur à celui d'un agent neutre. L'hypothèse H3, testée auprès d'élèves de terminale, prédit qu'un personnage pair génère un intérêt supérieur à un personnage d'autorité.

\textbf{QR3 : L'apparence de l'agent (humanoïde vs. abstrait) influence-t-elle la propension des élèves à l'illusion de compréhension, leur confiance et la crédibilité perçue des informations délivrées ?}

Cette question déplace l'analyse du versant motivationnel vers le versant métacognitif. L'heuristique de fluidité (cf.~\ref{subsec:heuristique_fluidite}) et les mécanismes d'autorité de l'agent (cf.~\ref{subsec:autorite_agent_vigilance}) prédisent qu'une incarnation humanoïde amplifie la crédibilité perçue et réduit la vigilance critique. Le protocole IOED, adapté au contexte d'interaction avec un agent génératif, permet de mesurer l'écart entre confiance subjective et performance objective (cf.~\ref{subsec:metacognition_calibration}). L'hypothèse H4 prédit que l'agent humanoïde suscite une confiance et une crédibilité perçue supérieures à celles de l'agent abstrait. L'hypothèse H5 prédit que l'agent humanoïde amplifie l'illusion de compréhension.

Le chapitre suivant présente le dispositif expérimental conçu pour tester ces hypothèses : la plateforme MemorIA, les quatre études et les instruments de mesure.

