\subsection{Questions de recherche}
\label{subsec:questions_hypotheses}

Ces lacunes motivent trois questions de recherche complémentaires. Elles suivent une progression dialectique : de l'exploration des bénéfices potentiels vers l'examen des risques associés.

La revue a établi que le cadre ICAP prédit un avantage du mode interactif sur le mode passif (cf.~\ref{subsec:icap}) et que la théorie de l'agence sociale postule que les indices sociaux réactifs activent la présence sociale et augmentent l'effort cognitif (cf.~\ref{subsec:incarnation_agence_sociale}). Le dépassement du plafond des agents scriptés par les LLM (cf.~\ref{subsec:agents_scriptes_generatifs}) permet de tester ces prédictions avec une adaptation sémantique en temps réel. Or cette configuration n'a pas été évaluée en contexte scolaire écologique. C'est cette lacune qui motive notre première question de recherche :

\textbf{QR1 : Dans quelle mesure l'interactivité orale directe avec un agent historique alimenté par l'IA influence-t-elle l'intérêt des élèves par rapport à une présentation vidéo passive ?}

La série d'études 1, composée de trois expérimentations menées auprès d'élèves de 6\textsuperscript{e}, 4\textsuperscript{e} et terminale, permettra de déterminer si l'avantage interactif documenté avec les agents scriptés se maintient avec des agents génératifs déployés en classe, et d'identifier l'ampleur de cet effet sur les trois dimensions de l'intérêt (activité, contenu, personnage).

La revue a également montré que le modèle de l'intérêt distingue le déclenchement du maintien et que la pertinence personnelle joue un rôle déterminant dans la transition entre ces phases (cf.~\ref{subsec:architecture_interet},~\ref{subsec:pertinence_valeur}). La congruence thématique entre l'agent et le domaine enseigné peut renforcer l'engagement (cf.~\ref{subsec:incarnation_agence_sociale}), et les effets des agents sont plus marqués chez les 10-14 ans. Or aucune étude ne teste l'incarnation d'un personnage historique comme variable d'alignement, ni ne mesure l'interaction entre cet alignement et le stade développemental. Ce constat motive notre deuxième question :

\textbf{QR2 : Comment l'alignement thématique de l'agent (personnage historique vs. neutre ; pair vs. autorité) et son style de présentation (formel vs. accessible) modulent-ils cet intérêt en fonction du stade développemental des élèves ?}

Les trois expérimentations de la série d'études 1 permettront de déterminer si l'alignement thématique produit un effet différencié selon l'âge, et si le style de présentation du personnage --- formel ou accessible --- interagit avec le stade développemental pour moduler l'intérêt.

Enfin, la revue a identifié une convergence de mécanismes pouvant réduire la vigilance critique : l'heuristique de fluidité (cf.~\ref{subsec:heuristique_fluidite}), les scripts sociaux du paradigme CASA (cf.~\ref{subsec:presence_sociale_CASA}) et la perception d'autorité de l'agent (cf.~\ref{subsec:autorite_agent_vigilance}). Ces mécanismes prédisent qu'une incarnation humanoïde amplifie la crédibilité perçue et réduit la vigilance. Le protocole IOED, adapté au contexte d'interaction avec un agent génératif, permet de mesurer l'écart entre confiance subjective et performance objective (cf.~\ref{subsec:metacognition_calibration}). L'effet d'une incarnation visuelle sur cette illusion n'a cependant pas été testé. Ce constat motive notre troisième question :

\textbf{QR3 : L'apparence de l'agent (humanoïde vs. abstrait) influence-t-elle la propension des élèves à l'illusion de compréhension, leur confiance et la crédibilité perçue des informations délivrées ?}

L'étude 2, menée auprès d'élèves de 5\textsuperscript{e}, permettra de déterminer si le design visuel de l'agent amplifie l'illusion de compréhension indépendamment de la fluence conversationnelle, et d'évaluer dans quelle mesure l'incarnation humanoïde module la confiance et la crédibilité perçue.

Pour répondre à ces questions, la suite du document s'organise en trois chapitres. Le chapitre~3 présente la plateforme MemorIA, conçue pour déployer des agents conversationnels historiques en contexte scolaire. Il décrit son architecture technique et sa phase pilote en classe réelle. Les retours recueillis lors de cette phase permettront de valider les choix de conception et d'ajuster le protocole en vue de l'évaluation expérimentale. Le chapitre~4 rapporte la série d'études~1, composée de trois expérimentations menées auprès d'élèves de 6\textsuperscript{e}, 4\textsuperscript{e} et terminale (QR1 et QR2). Le chapitre~5 rapporte l'étude~2, qui examine l'influence du design visuel de l'agent sur la confiance, la crédibilité perçue et l'illusion de compréhension (QR3).
