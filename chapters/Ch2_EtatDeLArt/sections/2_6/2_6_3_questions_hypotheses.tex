\subsection{Questions de recherche}
\label{subsec:questions_hypotheses}

Ces lacunes motivent trois questions de recherche complémentaires. Elles suivent une progression dialectique : de l'exploration des bénéfices potentiels vers l'examen des risques associés.

\textbf{QR1 : Dans quelle mesure l'interactivité orale directe avec un agent historique alimenté par l'IA influence-t-elle l'intérêt des élèves par rapport à une présentation vidéo passive ?}

Cette question opérationnalise la distinction entre les modes passif et interactif du cadre ICAP (cf.~\ref{subsec:icap}) en contexte d'agent génératif. La théorie de l'agence sociale prédit que les indices sociaux réactifs activent la présence sociale et augmentent l'effort cognitif (cf.~\ref{subsec:incarnation_agence_sociale}). Le dépassement du plafond des agents scriptés (cf.~\ref{subsec:agents_scriptes_generatifs}) permet de tester cette prédiction avec une adaptation sémantique en temps réel. Ces cadres prédisent que l'interaction orale avec l'agent devrait générer un intérêt supérieur à la présentation vidéo passive, sur les trois dimensions mesurées : intérêt pour l'activité, pour le contenu et pour le personnage. La série d'études~1 teste cette prédiction.

\textbf{QR2 : Comment l'alignement thématique de l'agent (personnage historique vs. neutre ; pair vs. autorité) et son style de présentation (formel vs. accessible) modulent-ils cet intérêt en fonction du stade développemental des élèves ?}

Cette question explore les conditions aux limites de l'effet d'interactivité. Le modèle de l'intérêt identifie le personnage historique comme déclencheur potentiel d'intérêt situationnel et la pertinence personnelle comme facteur de maintien (cf.~\ref{subsec:architecture_interet},~\ref{subsec:pertinence_valeur}). La congruence thématique entre l'agent et le domaine enseigné peut renforcer l'engagement (cf.~\ref{subsec:incarnation_agence_sociale}). Les effets des agents étant plus marqués chez les 10-14~ans (cf.~\ref{subsec:incarnation_agence_sociale}), l'âge constitue un modérateur à tester sur un spectre développemental étendu. La littérature prédit que l'incarnation d'un personnage historique aligné avec le contenu devrait générer un intérêt supérieur à celui d'un agent neutre. La question se pose également de savoir si un personnage pair produit un effet différent d'un personnage d'autorité chez les élèves plus âgés. Ces questions sont examinées dans la série d'études~1.

\textbf{QR3 : L'apparence de l'agent (humanoïde vs. abstrait) influence-t-elle la propension des élèves à l'illusion de compréhension, leur confiance et la crédibilité perçue des informations délivrées ?}

Cette question déplace l'analyse du versant motivationnel vers le versant métacognitif. L'heuristique de fluidité (cf.~\ref{subsec:heuristique_fluidite}) et les mécanismes d'autorité de l'agent (cf.~\ref{subsec:autorite_agent_vigilance}) prédisent qu'une incarnation humanoïde amplifie la crédibilité perçue et réduit la vigilance critique. Le protocole IOED, adapté au contexte d'interaction avec un agent génératif, permet de mesurer l'écart entre confiance subjective et performance objective (cf.~\ref{subsec:metacognition_calibration}). Ces mécanismes prédisent qu'une incarnation humanoïde devrait susciter une confiance et une crédibilité perçue supérieures à celles d'un agent abstrait, et amplifier l'illusion de compréhension. L'étude~2 teste ces prédictions.

Pour répondre à ces questions, la suite du document s'organise en trois chapitres. Le chapitre~3 présente la plateforme MemorIA, conçue pour déployer des agents conversationnels historiques en contexte scolaire. Il décrit son architecture technique, sa phase pilote en classe réelle et le feedback recueilli en vue de l'évaluation expérimentale. Le chapitre~4 rapporte la série d'études~1, composée de trois expérimentations menées auprès d'élèves de 6\textsuperscript{e}, 4\textsuperscript{e} et terminale. Ces expérimentations testent l'effet de l'interactivité orale et de l'alignement thématique de l'agent sur l'intérêt des élèves (QR1 et QR2). Le chapitre~5 rapporte l'étude~2, qui examine l'influence du design visuel de l'agent --- humanoïde ou abstrait --- sur la confiance, la crédibilité perçue et l'illusion de compréhension (QR3).

