\subsection{Convergence et tensions théoriques}
\label{subsec:convergence_tensions}

La revue de littérature fait émerger une double dynamique. D'un côté, plusieurs cadres théoriques convergent pour prédire qu'un agent conversationnel historique alimenté par un LLM devrait engager l'apprenant. Le cadre ICAP situe le dialogue oral avec un agent au niveau interactif, supérieur au mode passif que constitue le visionnage d'une vidéo (cf.~\ref{subsec:icap}). La théorie de l'agence sociale prédit que les indices sociaux émis par l'agent --- voix, visage, gestes --- activent la présence sociale et augmentent l'effort cognitif (cf.~\ref{subsec:incarnation_agence_sociale}). Le modèle de l'intérêt identifie le personnage historique comme déclencheur potentiel d'intérêt situationnel et la personnalisation narrative comme facteur de maintien (cf.~\ref{subsec:architecture_interet}). La pertinence perçue est renforcée lorsque l'agent incarne le contenu disciplinaire plutôt qu'un intermédiaire générique (cf.~\ref{subsec:pertinence_valeur}). Les LLM lèvent le plafond des agents scriptés en permettant une adaptation sémantique en temps réel, condition de la co-construction au sens du cadre ICAP (cf.~\ref{subsec:agents_scriptes_generatifs}).

Ces mêmes propriétés génèrent cependant des risques. La fluence linguistique des LLM active l'heuristique de fluidité : l'apprenant tend à confondre la facilité de traitement avec sa propre maîtrise du contenu (cf.~\ref{subsec:heuristique_fluidite}). L'incarnation anthropomorphe amplifie ce risque en activant les scripts sociaux du paradigme CASA et la perception d'autorité (cf.~\ref{subsec:presence_sociale_CASA},~\ref{subsec:autorite_agent_vigilance}). Les hallucinations des LLM, indétectables par un apprenant novice, sont rendues particulièrement dangereuses en histoire par la plausibilité narrative qui masque l'inexactitude factuelle (cf.~\ref{subsec:fiabilite_hallucinations}). La délégation métacognitive à l'agent tend à supprimer les \textit{desirable difficulties} nécessaires à l'apprentissage profond (cf.~\ref{subsec:autorite_agent_vigilance}). La dissociation entre perception et performance traverse l'ensemble de la revue : les formats les plus appréciés ne produisent pas nécessairement les meilleurs apprentissages (cf.~\ref{subsec:icap},~\ref{subsec:heuristique_fluidite}).

Le dispositif qui maximise l'engagement risque ainsi de minimiser la vigilance critique. Cette tension est structurelle : les propriétés qui rendent l'agent convaincant --- fluence, incarnation, adaptation --- sont les mêmes qui activent les heuristiques de crédibilité et réduisent l'effort métacognitif. La question n'est pas de choisir entre engagement et vigilance, mais d'identifier les conditions de conception qui permettent de maintenir les deux. Les lacunes de la littérature existante sur ce point font l'objet de la section suivante.

