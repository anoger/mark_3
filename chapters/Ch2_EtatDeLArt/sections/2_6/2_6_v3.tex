% ============================================================================
% Section 2.6 : Synthèse et Problématique (V3)
% ============================================================================
% Corrections appliquées :
% - 2_6_1 : Synthèse par références, apport nouveau explicité
% ============================================================================

\section{Synthèse et Problématique}
\label{sec:synthese_problematique}

Les sections précédentes ont établi le cadre théorique nécessaire à l'analyse de l'interaction entre élèves et agents conversationnels historiques. Il convient maintenant de synthétiser ces apports pour faire émerger les tensions qui structurent notre problématique. Cette section identifie d'abord les convergences et tensions théoriques (\S\ref{subsec:convergences_tensions}), puis analyse les lacunes de la littérature actuelle (\S\ref{subsec:lacunes_litterature}), avant de formuler les questions et hypothèses de recherche qui guident ce travail doctoral (\S\ref{subsec:questions_hypotheses}).

% ----------------------------------------------------------------------------
% Sous-sections : v3 pour 2_6_1, originaux pour 2_6_2 et 2_6_3
% ----------------------------------------------------------------------------
% ============================================================================
% Sous-section 2.6.1 : Convergence et Tensions Théoriques (V3)
% ============================================================================
% Corrections appliquées :
% - Lacune 6 : Synthèse par références croisées (évite répétition ICAP)
% - Lacune 18 : Apport nouveau explicité (hypothèse équilibre symétrique)
% ============================================================================

\subsection{Convergence et Tensions Théoriques}
\label{subsec:convergences_tensions}

L'analyse de la littérature révèle une tension fondamentale~: les mêmes caractéristiques qui rendent les agents conversationnels pédagogiquement prometteurs constituent simultanément leurs principaux risques.

Du côté des convergences positives, les cadres théoriques présentés dans ce chapitre s'articulent en une prédiction cohérente. L'interactivité dialogique active les modes d'engagement les plus profonds selon ICAP (\S\ref{subsec:ICAP}). La présence d'un agent personnifié déclenche un contrat de partenariat qui augmente l'effort cognitif selon la théorie de l'agence sociale (\S\ref{subsec:presence_sociale}). Le paradigme CASA explique pourquoi des indices sociaux minimaux suffisent à activer nos scripts relationnels (\S\ref{subsec:presence_sociale}). La théorie de l'autodétermination identifie l'interactivité comme vecteur de satisfaction du besoin d'autonomie (\S\ref{subsec:SDT_CET}). Ces cadres convergent vers une prédiction commune~: un agent conversationnel interactif devrait favoriser l'engagement et l'intérêt des apprenants.

Du côté des tensions, les caractéristiques mêmes qui génèrent l'engagement peuvent simultanément compromettre la qualité de l'apprentissage. La fluidité conversationnelle active l'heuristique qui conduit l'apprenant à confondre facilité de traitement et maîtrise du contenu (\S\ref{subsec:fluency_heuristic}). L'incarnation réaliste peut agir comme un détail séduisant qui détourne l'attention du contenu. L'autorité perçue de l'agent peut désactiver la vigilance épistémique nécessaire à un apprentissage critique (\S\ref{subsec:autorite_vigilance}).

Cette tension suggère ce que nous proposons d'appeler l'\textit{hypothèse de l'équilibre des défauts symétriques}. Dans cette perspective, l'interactivité et l'incarnation constituent des leviers robustes pour l'engagement --- mais chaque gain sur cette dimension s'accompagne potentiellement d'un risque métacognitif. L'agent idéal ne serait pas celui qui maximise l'engagement, mais celui qui optimise le rapport entre engagement généré et risque d'illusion induit. Cette hypothèse constitue l'apport spécifique de notre analyse~: elle reformule la question du design des agents pédagogiques comme un problème d'optimisation multi-objectif plutôt que de maximisation unidimensionnelle.

Cette hypothèse trouve un écho dans les recommandations de la littérature. Plutôt que de concevoir des outils \og transparents\fg{} qui s'effacent pour laisser place à la tâche~\citep{heidegger2016being}, certains auteurs suggèrent de \og scripter la rupture\fg{}~: concevoir des agents qui, loin de fournir des réponses fluides et directes, exposent délibérément leurs limites, admettent leur incertitude, ou forcent l'apprenant à valider l'information~\citep{solyst2024generative}. L'objectif n'est plus de maximiser la confiance, mais de la \textit{calibrer} en maintenant active la vigilance critique.

Ces tensions théoriques se traduisent en questions de design concrètes. Comment concevoir un agent suffisamment engageant pour maintenir l'attention, sans que cette qualité ne désactive les mécanismes de vigilance? L'incarnation visuelle amplifie-t-elle le risque métacognitif, ou la fluidité conversationnelle constitue-t-elle à elle seule le facteur déterminant? L'alignement thématique entre le personnage et le contenu peut-il servir de levier d'engagement sans les coûts attentionnels d'une incarnation réaliste? Ces questions structurent directement notre programme de recherche.

\subsection{Lacunes de la littérature actuelle}
\label{subsec:lacunes_litterature}

La première lacune concerne l'interactivité orale avec un agent génératif en contexte écologique. Les méta-analyses sur les agents pédagogiques portent principalement sur des agents scriptés dont les réponses sont sélectionnées dans un répertoire fixe (cf.~\ref{subsec:incarnation_agence_sociale}). Les LLM modifient qualitativement l'interaction en permettant une adaptation sémantique en temps réel (cf.~\ref{subsec:agents_scriptes_generatifs}), mais les preuves empiriques de leur efficacité en contexte scolaire réel restent rares. Les études existantes reposent sur des échantillons restreints, des prototypes et des mesures immédiates, ce qui limite leur validité écologique (cf.~\ref{subsec:agents_scriptes_generatifs}). Le cadre ICAP distingue les modes passif et interactif (cf.~\ref{subsec:icap}), mais cette distinction n'a pas été testée avec des agents génératifs déployés en classe. On ne sait pas si l'avantage interactif documenté avec les agents scriptés se maintient avec des agents alimentés par des LLM.

La deuxième lacune porte sur l'alignement thématique de l'agent et le stade développemental. La congruence entre l'apparence de l'agent et le domaine enseigné produit des effets variables selon le type d'alignement (cf.~\ref{subsec:incarnation_agence_sociale}). Aucune étude ne teste cependant l'incarnation d'un personnage historique comme variable d'alignement dans l'enseignement de l'histoire. Le modèle de l'intérêt prédit que la qualité du déclencheur dépend du stade développemental de l'apprenant (cf.~\ref{subsec:architecture_interet}), mais le rôle de l'âge dans la réponse à l'alignement thématique n'a pas été étudié avec des agents génératifs. Les effets des agents sont plus marqués chez les 10-14~ans (cf.~\ref{subsec:incarnation_agence_sociale}), mais la littérature ne couvre pas le spectre complet de l'adolescence. On ne sait pas si l'alignement thématique de l'agent module l'intérêt de manière différenciée selon l'âge, ni si le style de présentation --- formel ou accessible --- interagit avec le stade développemental.

La troisième lacune concerne l'illusion de compréhension induite par l'interaction avec un agent génératif. L'illusion de compréhension (cf.~\ref{subsec:metacognition_calibration}) et l'heuristique de fluidité (cf.~\ref{subsec:heuristique_fluidite}) sont documentées dans des contextes multimédia classiques --- vidéos, animations, présentations passives. Les agents conversationnels génératifs combinent toutefois fluence linguistique, incarnation anthropomorphe et adaptabilité en temps réel, une combinaison inédite dont les effets métacognitifs n'ont pas été mesurés. La littérature sur la surconfiance envers l'IA porte sur des adultes ou des collégiens interagissant avec des chatbots textuels (cf.~\ref{subsec:autorite_agent_vigilance}). L'effet d'une incarnation visuelle --- humanoïde ou abstraite --- sur l'illusion de compréhension n'a pas été testé. Le protocole IOED, conçu pour mesurer la profondeur illusoire des explications causales \citep{rozenblit2002}, n'a pas été adapté aux contextes d'interaction avec des agents génératifs incarnés. On ne sait pas si l'apparence de l'agent amplifie l'illusion de compréhension, ni si la fluence de l'interaction suffit à induire une surévaluation de la compréhension indépendamment du design visuel.


\subsection{Questions de recherche et hypothèses}
\label{subsec:questions_hypotheses}

Ces lacunes motivent trois questions de recherche complémentaires. Elles suivent une progression dialectique : de l'exploration des bénéfices potentiels vers l'examen des risques associés.

\textbf{QR1 : Dans quelle mesure l'interactivité orale directe avec un agent historique alimenté par l'IA influence-t-elle l'intérêt des élèves par rapport à une présentation vidéo passive ?}

Cette question opérationnalise la distinction entre les modes passif et interactif du cadre ICAP (cf.~\ref{subsec:icap}) en contexte d'agent génératif. La théorie de l'agence sociale prédit que les indices sociaux réactifs activent la présence sociale et augmentent l'effort cognitif (cf.~\ref{subsec:incarnation_agence_sociale}). Le dépassement du plafond des agents scriptés (cf.~\ref{subsec:agents_scriptes_generatifs}) permet de tester cette prédiction avec une adaptation sémantique en temps réel. L'hypothèse H1 prédit que l'interaction orale avec l'agent génère un intérêt supérieur à la présentation vidéo passive, sur les trois dimensions mesurées : intérêt pour l'activité, pour le contenu et pour le personnage.

\textbf{QR2 : Comment l'alignement thématique de l'agent (personnage historique vs. neutre ; pair vs. autorité) et son style de présentation (formel vs. accessible) modulent-ils cet intérêt en fonction du stade développemental des élèves ?}

Cette question explore les conditions aux limites de l'effet d'interactivité. Le modèle de l'intérêt identifie le personnage historique comme déclencheur potentiel d'intérêt situationnel et la pertinence personnelle comme facteur de maintien (cf.~\ref{subsec:architecture_interet},~\ref{subsec:pertinence_valeur}). La congruence thématique entre l'agent et le domaine enseigné peut renforcer l'engagement (cf.~\ref{subsec:incarnation_agence_sociale}). Les effets des agents étant plus marqués chez les 10-14~ans (cf.~\ref{subsec:incarnation_agence_sociale}), l'âge constitue un modérateur à tester sur un spectre développemental étendu. L'hypothèse H2 prédit que l'incarnation d'un personnage historique aligné avec le contenu génère un intérêt supérieur à celui d'un agent neutre. L'hypothèse H3, testée auprès d'élèves de terminale, prédit qu'un personnage pair génère un intérêt supérieur à un personnage d'autorité.

\textbf{QR3 : L'apparence de l'agent (humanoïde vs. abstrait) influence-t-elle la propension des élèves à l'illusion de compréhension, leur confiance et la crédibilité perçue des informations délivrées ?}

Cette question déplace l'analyse du versant motivationnel vers le versant métacognitif. L'heuristique de fluidité (cf.~\ref{subsec:heuristique_fluidite}) et les mécanismes d'autorité de l'agent (cf.~\ref{subsec:autorite_agent_vigilance}) prédisent qu'une incarnation humanoïde amplifie la crédibilité perçue et réduit la vigilance critique. Le protocole IOED, adapté au contexte d'interaction avec un agent génératif, permet de mesurer l'écart entre confiance subjective et performance objective (cf.~\ref{subsec:metacognition_calibration}). L'hypothèse H4 prédit que l'agent humanoïde suscite une confiance et une crédibilité perçue supérieures à celles de l'agent abstrait. L'hypothèse H5 prédit que l'agent humanoïde amplifie l'illusion de compréhension.

Le chapitre suivant présente le dispositif expérimental conçu pour tester ces hypothèses : la plateforme MemorIA, les quatre études et les instruments de mesure.



% Transition vers Chapitre 3
Les questions de recherche ainsi formulées appellent une méthodologie expérimentale rigoureuse, capable de manipuler les variables de design identifiées tout en mesurant leurs effets sur l'engagement et les biais métacognitifs. Avant de présenter les études empiriques, il convient de décrire la plateforme technique développée pour les conduire~: MemorIA, dont l'architecture et la validation font l'objet du chapitre suivant.
