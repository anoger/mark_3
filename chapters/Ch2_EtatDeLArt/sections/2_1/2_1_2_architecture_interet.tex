\subsection{Architecture de l'Intérêt}
\label{subsec:architecture_interet}

L'intérêt, tel que défini dans la recherche en psychologie de l'éducation, combine deux composantes de nature distincte \citep{hidi2006}. La composante affective se manifeste par une expérience positive associée à l'activité --- attention accrue, affect favorable, envie spontanée de poursuivre. La composante cognitive se traduit par une orientation vers la compréhension du contenu : l'individu cherche à approfondir, pose des questions, établit des connexions entre les informations. Cette double nature distingue l'intérêt des construits apparentés. Là où la motivation intrinsèque (cf. \ref{subsec:sdt_cet}) désigne un processus général applicable à toute activité, l'intérêt est toujours dirigé vers un contenu spécifique \citep{renninger2015}. Un même élève peut manifester un intérêt soutenu pour la biologie et un désintérêt marqué pour l'histoire, alors que son niveau de motivation intrinsèque globale reste stable. Cette spécificité de contenu confère à l'intérêt une valeur explicative particulière pour l'apprentissage disciplinaire : il contribue à expliquer pourquoi un dispositif pédagogique efficace dans une matière échoue dans une autre \citep{hidi2000}.

Le développement de l'intérêt suit une séquence en quatre phases dont chacune se caractérise par un équilibre distinct entre soutien externe et engagement autonome \citep{hidi2006}. La première phase, l'intérêt situationnel déclenché, désigne une réponse attentionnelle à un stimulus environnemental --- nouveauté, surprise, incongruité avec les attentes. Cette réponse est brève et dépend entièrement du déclencheur externe. La deuxième phase, l'intérêt situationnel maintenu, apparaît lorsque l'engagement avec le contenu se prolonge au-delà de la réaction initiale : l'apprenant commence à traiter l'information en profondeur, mais le soutien de l'environnement reste nécessaire pour maintenir son attention. La troisième phase, l'intérêt individuel émergent, marque un changement qualitatif : l'apprenant développe une prédisposition à se réengager avec le contenu de sa propre initiative, génère des questions et recherche activement des informations complémentaires. La quatrième phase, l'intérêt individuel développé, correspond à une disposition stable : l'apprenant tolère la frustration, autorégule son apprentissage et produit des questions de curiosité qui alimentent sa progression \citep{renninger2015}. La transition de la deuxième à la troisième phase constitue le point de basculement du modèle : l'intérêt tend à ne plus dépendre du soutien de l'environnement pour devenir auto-entretenu.

Les facteurs qui déclenchent l'intérêt situationnel ne sont pas ceux qui le maintiennent. La nouveauté, le caractère inattendu d'une information et l'intensité perceptive d'un stimulus suffisent à capter l'attention (phase 1), mais leur effet tend à se dissiper si aucune connexion significative ne s'établit avec l'apprenant \citep{hidi2006}. Le maintien de l'intérêt (phase 2) repose sur des facteurs différents : la pertinence personnelle perçue, l'engagement actif dans une tâche signifiante et le sentiment de compétence dans l'interaction avec le contenu \citep{bergin1999}. Le contexte social joue un rôle transversal. L'identité du locuteur, la qualité de la relation avec l'enseignant et l'influence des pairs peuvent à la fois déclencher et maintenir l'intérêt \citep{bergin2016}. Un interlocuteur perçu comme signifiant --- qu'il soit enseignant, pair ou figure de référence --- peut affecter le développement de l'intérêt indépendamment du contenu transmis \citep{renninger2009}. Cette dissociation entre déclenchement et maintien pose un problème de conception : un dispositif qui mise exclusivement sur la nouveauté sans créer de connexion personnelle avec le contenu ne dépasse pas la première phase.

La curiosité et l'intérêt entretiennent une relation de renforcement mutuel qui s'intensifie au fil du développement \citep{hidi2020}. Dans les phases avancées de l'intérêt (phases 3 et 4), l'apprenant ne se contente plus de répondre aux stimuli externes : il génère spontanément des questions qui orientent son exploration du domaine. Ces questions de curiosité constituent le moteur interne de la progression vers un intérêt individuel stable. Les environnements éducatifs qui accueillent ces questions --- plutôt que de les canaliser vers des objectifs prédéterminés --- facilitent la transition entre les phases \citep{renninger2015}. La progression de l'intérêt situationnel vers l'intérêt individuel dépend toutefois d'un facteur que le modèle identifie sans l'approfondir : la capacité de l'apprenant à percevoir le contenu comme personnellement pertinent. Ce mécanisme, par lequel un savoir disciplinaire acquiert une valeur aux yeux de l'apprenant, relève d'un cadre théorique distinct : la théorie de la valeur et de la pertinence, qui fait l'objet de la section suivante.
