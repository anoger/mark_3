% ============================================================================
% Sous-section 2.1.2 : Architecture de l'Intérêt
% ============================================================================
% Sources : IJCCI_extraction (§2.1), Vault.xlsx (Hidi & Renninger, 2006),
%           Harackiewicz (2014), Bergin (1999)
% Calibrage : ~1000 mots
% Type : C (Rédaction originale)
% ============================================================================

\subsection{Architecture de l'Intérêt}
\label{subsec:architecture_interet}

Si la motivation intrinsèque décrit une orientation générale vers l'action autodéterminée, l'intérêt désigne un état psychologique plus spécifique, focalisé sur un contenu particulier et caractérisé par une double composante~: affective (l'expérience émotionnelle positive associée à l'activité) et cognitive (la motivation à approfondir la compréhension du sujet)~\citep{renninger2015power}. Cette distinction conceptuelle s'avère cruciale~: contrairement à la motivation, qui peut être extrinsèque ou intrinsèque selon sa source, l'intérêt implique toujours une relation significative entre l'individu et un domaine de connaissance.

L'intérêt n'est pas un état binaire mais un construit développemental qui évolue sur un continuum~\citep{hidi2006four}. À une extrémité, l'\textit{intérêt situationnel} émerge comme réponse transitoire à des déclencheurs environnementaux --- nouveauté, surprise, discordance avec les attentes. À l'autre extrémité, l'\textit{intérêt individuel} constitue une disposition stable, internalisée, qui pousse à rechercher activement des occasions de réengagement avec le domaine~\citep{priniski2018making}. La transition de l'un à l'autre représente le passage d'un engagement soutenu de l'extérieur vers un engagement auto-entretenu.

\subsubsection*{Trajectoire développementale de l'intérêt}

Cette transition s'opère selon une séquence en quatre phases qui permet de comprendre comment un intérêt éphémère peut se cristalliser en disposition durable~\citep{hidi2006four}. La première phase, l'\textit{intérêt situationnel déclenché}, correspond à une capture attentionnelle initiale provoquée par des stimuli saillants --- informations inattendues, présentations nouvelles, activités qui remettent en question les présupposés. Cette attention reste temporaire et peut s'évanouir rapidement si elle n'est pas nourrie.

La deuxième phase, l'\textit{intérêt situationnel maintenu}, marque un premier approfondissement~: l'attention devient soutenue à travers un engagement significatif avec le contenu. L'environnement d'apprentissage joue ici un rôle déterminant~: un équilibre entre structure et autonomie~\citep{jang2010engaging}, combiné à des opportunités de participation active~\citep{centsboonstra2021promoting}, fournit le soutien nécessaire au maintien de l'intérêt. Sans ces conditions, l'intérêt déclenché retombe.

Les troisième et quatrième phases --- \textit{intérêt individuel émergent} puis \textit{développé} --- marquent l'internalisation progressive de l'intérêt. L'élève commence à générer ses propres questions, à rechercher des informations en dehors du contexte scolaire, à développer une expertise qui, à son tour, alimente l'intérêt. À ce stade, le soutien externe devient moins nécessaire~: l'individu assume un rôle actif dans le maintien de son propre intérêt.

\subsubsection*{Déclencheurs de l'intérêt~: facteurs individuels et situationnels}

Si le modèle développemental décrit la trajectoire de l'intérêt, il reste à identifier les leviers qui permettent de le déclencher et de le maintenir. Une taxonomie distingue à cet effet les \textit{facteurs individuels}, caractéristiques propres à l'apprenant, des \textit{facteurs situationnels}, éléments de l'environnement d'apprentissage~\citep{bergin1999classroom}.

Les facteurs individuels déterminent la réceptivité de l'élève aux stimuli environnementaux. Parmi ceux-ci, l'\textit{appartenance} (\textit{belongingness}) --- incluant la valeur culturelle accordée au contenu, l'identification à des modèles, et le soutien social perçu --- résonne avec le besoin d'affiliation identifié par la SDT. De même, le \textit{sentiment de compétence} et la \textit{pertinence par rapport aux objectifs personnels} prolongent les dimensions d'autonomie et de compétence. Les \textit{connaissances préalables} permettent d'ancrer les nouvelles informations dans des schémas existants, facilitant l'engagement cognitif. Les \textit{émotions} associées au sujet, enfin, rappellent la dimension affective constitutive de l'intérêt.

Les facteurs situationnels concernent les caractéristiques de l'environnement susceptibles de déclencher l'intérêt~: les \textit{activités pratiques}, la présence d'un \textit{auteur visible} qui personnalise le contenu, la \textit{modélisation} par des pairs ou des figures d'autorité, les \textit{jeux et puzzles}, la \textit{discordance} avec les attentes préalables, la \textit{nouveauté}, l'\textit{interaction sociale}, et --- particulièrement pertinent pour notre propos --- le \textit{narratif}. La structure narrative, en engageant les processus d'identification et en créant une tension dramatique, constitue un puissant déclencheur d'intérêt situationnel.

Cette articulation entre facteurs individuels et situationnels éclaire le potentiel des agents conversationnels incarnant des personnages historiques. Un tel dispositif combine plusieurs facteurs situationnels --- nouveauté, interaction sociale, narratif, auteur visible --- tout en pouvant activer des facteurs individuels à travers le dialogue --- identification au personnage, pertinence personnelle, connexion aux connaissances préalables. La question devient alors~: comment orchestrer ces facteurs pour favoriser la transition vers un intérêt maintenu, puis individuel?

\subsubsection*{Le rôle pivot de la valeur perçue}

Un mécanisme clé de cette transition réside dans l'évolution de la \textit{valeur perçue} du contenu~\citep{hidi2006four}. Pour progresser de l'intérêt situationnel vers l'intérêt individuel, l'apprenant doit percevoir une valeur croissante dans le domaine --- valeur qui motive la poursuite de l'engagement au-delà de l'attrait initial de la nouveauté. Cette proposition établit un pont conceptuel entre la théorie de l'intérêt et la théorie de l'attente-valeur (voir section~\ref{subsec:pertinence_personnalisation})~: les deux cadres convergent pour prédire qu'augmenter la valeur perçue constitue un levier viable pour promouvoir l'intérêt et la motivation~\citep{harackiewicz2014harnessing}.

Cette convergence théorique suggère une stratégie d'intervention~: si la valeur perçue est le médiateur de la progression développementale de l'intérêt, alors les interventions visant à augmenter cette valeur --- notamment par la pertinence personnelle et la personnalisation --- pourraient faciliter la transition de l'intérêt situationnel vers l'intérêt individuel. C'est cette hypothèse que nous explorons dans la section suivante.

