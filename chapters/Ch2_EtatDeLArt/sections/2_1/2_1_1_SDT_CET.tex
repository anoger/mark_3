% ============================================================================
% Sous-section 2.1.1 : Dynamiques Motivationnelles — SDT et CET
% ============================================================================
% Sources : IJCCI_extraction (§2.1), Vault.xlsx
% Calibrage : ~600 mots
% Type : C (Rédaction originale, synthèse analytique)
% ============================================================================

\subsection{Dynamiques Motivationnelles~: SDT et CET}
\label{subsec:SDT_CET}

La motivation intrinsèque --- cette propension à s'engager dans une activité pour le plaisir et la satisfaction qu'elle procure --- constitue un prédicteur robuste de la qualité de l'apprentissage~\citep{deci2000what}. Sa compréhension nécessite d'articuler deux niveaux d'analyse~: les mécanismes par lesquels l'environnement affecte cette motivation, et les besoins psychologiques sous-jacents qui la conditionnent.

Au premier niveau, les facteurs environnementaux n'exercent pas d'effet direct et uniforme sur la motivation~; leur impact dépend de l'interprétation qu'en fait l'individu~\citep{deci1985intrinsic}. Un même feedback peut ainsi être vécu comme \textit{informationnel} --- fournissant un retour constructif sur la compétence et soutenant la motivation --- ou comme \textit{contrôlant} --- exerçant une pression sur le comportement et sapant l'autodétermination. Cette distinction, issue de la Théorie de l'Évaluation Cognitive (CET), s'articule autour de deux dimensions psychologiques~: le \textit{locus de causalité perçu}, soit le sentiment que ses actions émanent de soi plutôt que de contraintes externes, et le \textit{sentiment de compétence}, soit la conviction de pouvoir atteindre les résultats souhaités. Un troisième aspect, \textit{amotivant}, peut signaler l'incompétence et conduire au désengagement total.

Au second niveau, trois besoins psychologiques fondamentaux conditionnent le bien-être et l'engagement~: l'\textit{autonomie}, la \textit{compétence} et l'\textit{affiliation}~\citep{ryan2017self}. La Théorie de l'Autodétermination (SDT) postule que les environnements d'apprentissage favorisent l'intérêt lorsqu'ils satisfont ces besoins à travers des choix significatifs, des défis appropriés, un feedback constructif et des interactions soutenantes. L'ajout du besoin d'affiliation --- le sentiment de connexion aux autres --- étend la portée explicative du modèle au-delà des seules dimensions cognitives pour intégrer la dimension sociale de l'apprentissage.

L'articulation de ces deux niveaux révèle une dynamique complexe~: un environnement qui offre des choix (autonomie), propose des défis calibrés avec un feedback informatif (compétence), et crée des opportunités d'interaction authentique (affiliation), réunit les conditions propices au développement de la motivation intrinsèque. À l'inverse, un environnement perçu comme contrôlant, qui impose des tâches sans en expliciter le sens et isole l'apprenant, tend à éroder cette motivation --- un phénomène particulièrement documenté lors de la transition vers le secondaire~\citep{gnambs2016decline}. Cette période, marquée par une pression accrue sur la performance et des structures scolaires qui peuvent limiter la curiosité naturelle~\citep{engel2011children}, voit la motivation intrinsèque décliner significativement.

Ce déclin revêt une importance particulière pour l'enseignement de l'Histoire, discipline souvent perçue comme distante et déconnectée des expériences personnelles~\citep{audigier2010histoire}. Le défi consiste alors à concevoir des environnements qui, tout en respectant les contraintes curriculaires, créent les conditions de satisfaction des besoins identifiés. Les agents conversationnels offrent une piste prometteuse~: l'interactivité dialogique peut soutenir l'autonomie en permettant à l'élève de diriger l'échange, les réponses adaptatives peuvent nourrir le sentiment de compétence, et les indices sociaux de l'agent peuvent répondre au besoin d'affiliation. Cette hypothèse, qui guide notre programme expérimental, nécessite cependant de comprendre comment l'intérêt se développe et se maintient --- objet de la section suivante.
