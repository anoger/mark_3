\subsection{Dynamiques Motivationnelles : SDT et CET}
\label{subsec:sdt_cet}

La motivation intrinsèque --- l'engagement dans une activité pour la satisfaction inhérente qu'elle procure, indépendamment de toute récompense externe --- constitue le prédicteur le plus stable de la persistance et de la qualité de l'apprentissage \citep{deci2000-hk}. Ce construit se distingue de l'engagement, qui désigne la manifestation comportementale observable de l'implication dans une tâche, et de l'intérêt, qui renvoie à un état psychologique orienté vers un contenu spécifique (cf. \ref{subsec:architecture_interet}). La Théorie de l'Autodétermination (\textit{Self-Determination Theory}, SDT) offre le cadre macro-théorique le plus largement mobilisé pour expliquer les conditions d'émergence et de maintien de cette motivation \citep{ryan2000}. La SDT se compose de plusieurs mini-théories articulées, dont la Théorie de l'Évaluation Cognitive (\textit{Cognitive Evaluation Theory}, CET) traite spécifiquement de l'impact des événements contextuels sur la motivation intrinsèque \citep{deci1985}. L'analyse qui suit examine d'abord ce mécanisme contextuel (CET), puis le cadre des besoins psychologiques qui le sous-tend (SDT).

La CET repose sur un postulat : les événements externes --- récompenses, feedback, évaluations, contraintes temporelles --- n'exercent pas d'effet mécanique sur la motivation. Leur impact dépend de la signification fonctionnelle que l'individu leur attribue \citep{deci1985}. Trois significations sont possibles. L'aspect informationnel fournit un retour sur la compétence sans exercer de pression : un feedback qui précise ce que l'apprenant maîtrise et ce qui reste à acquérir tend à soutenir la motivation intrinsèque. L'aspect contrôlant, à l'inverse, est perçu comme une pression vers un comportement prescrit : il déplace le locus de causalité perçu de l'interne vers l'externe, ce qui réduit le sentiment d'autodétermination. À l'inverse, la possibilité de choisir entre plusieurs options lors d'une tâche suffit à maintenir la motivation intrinsèque et à augmenter le temps consacré à l'activité, même lorsque les options sont fonctionnellement équivalentes \citep{zuckerman1978}. L'aspect amotivant signale l'incompétence et mine le sentiment d'efficacité. La méta-analyse portant sur 128 études confirme ce mécanisme : les récompenses tangibles conditionnelles réduisent la motivation intrinsèque chez l'enfant comme chez l'adulte, précisément parce qu'elles sont interprétées comme contrôlantes \citep{deci1999}. En contexte éducatif, cette grille d'analyse implique qu'un même feedback --- qu'il provienne d'un enseignant ou d'un agent virtuel --- peut soutenir ou compromettre la motivation selon la manière dont l'apprenant interprète sa fonction.

Au niveau macro, la SDT identifie trois besoins psychologiques dont la satisfaction conditionne le développement de la motivation intrinsèque : l'autonomie (se percevoir comme l'agent causal de ses actions), la compétence (se sentir efficace dans ses interactions avec l'environnement) et l'affiliation (\textit{relatedness} --- se sentir connecté à autrui et appartenir à un groupe social) \citep{ryan2000}. Ces besoins ne sont pas des préférences acquises mais des nécessités psychologiques innées et universelles dont la frustration systématique entraîne une diminution du bien-être et de la motivation, indépendamment du contexte culturel \citep{ryan2017self}. Les environnements d'apprentissage qui offrent des choix signifiants, proposent des défis calibrés au niveau de l'apprenant et favorisent des interactions sociales de qualité tendent à satisfaire ces trois besoins simultanément \citep{deci2000-hk}. Le besoin d'affiliation revêt une pertinence particulière pour les dispositifs impliquant des agents conversationnels : la nature sociale de l'interaction --- tour de parole, adaptation du discours, indices paraverbaux --- pourrait activer ce besoin même en l'absence d'un interlocuteur humain, une hypothèse que le paradigme CASA examinera en détail (cf. \ref{subsec:presence_sociale_casa}).

La trajectoire développementale de la motivation intrinsèque en contexte scolaire met en lumière l'enjeu pratique de ce cadre théorique. Les données longitudinales indiquent un déclin régulier de la motivation intrinsèque tout au long de la scolarité, avec une accélération lors de la transition vers le secondaire \citep{gnambs2016}. Cette période combine une augmentation de la pression évaluative, une réduction des marges d'autonomie accordées aux élèves et des transformations biologiques qui amplifient la sensibilité aux signaux sociaux. La CET fournit une explication parcimonieuse de ce déclin : les structures scolaires du secondaire accentuent les aspects contrôlants (notation chiffrée, classement, programme prescriptif) au détriment des aspects informationnels du feedback. Les institutions éducatives tendent par ailleurs à restreindre les occasions d'exploration spontanée, limitant la curiosité naturelle des élèves au profit d'objectifs curriculaires standardisés \citep{engel2009, engel2011}. Si la SDT et la CET expliquent les dynamiques motivationnelles générales, la question de savoir comment un contenu disciplinaire spécifique capte et maintient l'attention de l'apprenant relève d'un construit distinct : l'intérêt, dont l'architecture développementale fait l'objet de la section suivante.
