\subsection{Pertinence, Valeur et Personnalisation}
\label{subsec:pertinence_valeur}

La section précédente a identifié la pertinence personnelle comme levier de la progression de l'intérêt situationnel vers l'intérêt individuel. La théorie de l'attente-valeur (\textit{Expectancy-Value Theory}, EVT) fournit le cadre qui explique ce mécanisme. L'engagement d'un apprenant dans une tâche dépend de deux évaluations simultanées : l'attente de succès (la croyance de pouvoir réussir) et la valeur subjective attribuée à la tâche \citep{wigfield1998}. La valeur se décompose en quatre dimensions : la valeur intrinsèque (le plaisir retiré de l'activité, qui correspond à l'intérêt tel que défini en \ref{subsec:architecture_interet}), la valeur d'utilité (l'utilité perçue du contenu pour des buts futurs), la valeur d'atteinte (l'importance de la réussite pour l'image de soi) et le coût (l'effort requis, l'anxiété générée, les activités alternatives sacrifiées). La pertinence personnelle constitue le mécanisme transversal par lequel ces dimensions s'activent : elle désigne le degré auquel l'apprenant perçoit une connexion entre le contenu et sa propre vie, ses buts ou son identité \citep{albrecht2018}. Ce construit ne se réduit pas à l'utilité perçue. Il forme un continuum qui va de l'association ponctuelle (un exemple qui rappelle une expérience vécue) à l'identification profonde (un contenu qui touche aux valeurs ou aux aspirations de l'individu) \citep{priniski2018}.

Les interventions qui visent à renforcer la valeur d'utilité perçue illustrent comment la pertinence peut être activée en contexte scolaire. L'approche consiste à demander aux élèves de rédiger un texte reliant le contenu du cours à leur vie quotidienne --- non pas à leur transmettre des arguments sur l'utilité de la matière, mais à les amener à construire eux-mêmes cette connexion. En sciences, cette intervention tend à augmenter l'intérêt et la performance, avec un effet plus marqué chez les élèves dont les attentes de succès sont initialement faibles \citep{hulleman2009}. Le mécanisme est actif : l'apprenant ne reçoit pas passivement une information sur la valeur du contenu, il la construit par un effort de mise en relation. Quatre leviers d'intervention ont été identifiés pour soutenir ce processus : adapter les caractéristiques structurelles de l'environnement d'apprentissage, personnaliser le contexte, proposer un apprentissage par problèmes et souligner l'utilité du contenu \citep{harackiewicz2016}. Ces leviers opèrent sur le passage de l'intérêt situationnel déclenché à l'intérêt situationnel maintenu (cf. \ref{subsec:architecture_interet}) : ils transforment une réaction attentionnelle initiale en engagement prolongé en ancrant le contenu dans l'expérience de l'apprenant.

La personnalisation de l'apprentissage constitue la traduction pédagogique du principe de pertinence. Trois mécanismes distincts semblent contribuer indépendamment à la motivation : la contextualisation (ancrer le contenu dans des situations familières à l'apprenant), la personnalisation proprement dite (intégrer des éléments liés à l'identité de l'élève --- son prénom, ses centres d'intérêt) et le choix (offrir des options qui permettent à l'apprenant de connecter le contenu à ses préférences) \citep{cordova1996}. Le contrôle sur des éléments même non essentiels à la tâche --- choix d'un avatar, d'un thème visuel --- tend à augmenter la motivation, ce qui rejoint le besoin d'autonomie identifié par la SDT (cf. \ref{subsec:sdt_cet}). Les interventions ciblant les intérêts individuels tendent à augmenter l'engagement dans des matières perçues comme difficiles \citep{reber2018}. Le choix d'exemples pertinents renforce l'engagement, en particulier chez les élèves dont l'intérêt initial est faible \citep{hogheim2015}. L'intégration des questions formulées par les élèves dans le curriculum produit des effets comparables, en plaçant la curiosité de l'apprenant au centre du dispositif pédagogique \citep{hagay2015}.

La personnalisation se heurte toutefois à des contraintes qui limitent sa mise en œuvre à grande échelle. Trois obstacles ont été identifiés : la diversité des intérêts au sein d'une même classe, l'évolution de ces intérêts dans le temps et le coût de production de contenus personnalisés de qualité pour chaque élève \citep{walkington2018}. Ces contraintes rendent la personnalisation manuelle difficilement généralisable, ce qui justifie la recherche de dispositifs capables d'adapter le contenu de manière automatisée. La pertinence et la personnalisation expliquent \textit{pourquoi} un apprenant s'engage avec un contenu donné. La question complémentaire --- \textit{comment} la modalité d'interaction avec ce contenu détermine la profondeur de l'apprentissage --- relève d'un cadre distinct : le modèle ICAP, qui fait l'objet de la section suivante.
