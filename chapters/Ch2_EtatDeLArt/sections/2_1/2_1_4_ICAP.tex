\subsection{L'Apprentissage Actif et le Cadre ICAP}
\label{subsec:icap}

La distinction entre apprentissage passif et actif structure depuis longtemps le discours pédagogique, mais elle reste trop grossière pour analyser les différences de traitement cognitif entre dispositifs. Le cadre ICAP (\textit{Interactive, Constructive, Active, Passive}) propose une taxonomie plus fine en distinguant quatre modes d'engagement cognitif ordonnés par profondeur de traitement \citep{chi2009}. Le mode passif désigne la réception de l'information sans transformation : écouter un exposé, regarder une vidéo. Le mode actif implique une manipulation du matériel sans production de contenu nouveau : surligner un texte, prendre des notes verbatim, répéter une procédure. Le mode constructif apparaît lorsque l'apprenant génère des inférences qui dépassent le matériel présenté : expliquer un concept dans ses propres mots, poser des questions, établir des connexions entre des idées. Le mode interactif engage deux interlocuteurs dans une co-construction de connaissances par des contributions mutuellement réactives : chaque intervention s'appuie sur la précédente et la transforme \citep{chi2014}. L'hypothèse centrale du cadre est que ces modes produisent des résultats d'apprentissage croissants --- Interactif > Constructif > Actif > Passif --- parce qu'ils sollicitent des processus cognitifs de plus en plus élaborés. Le cadre porte sur l'engagement cognitif, non sur l'engagement comportemental : un élève physiquement actif (qui manipule du matériel) peut rester cognitivement passif si aucune inférence n'est générée.

Les données empiriques soutiennent cette hiérarchie au-delà des disciplines scientifiques où elle a d'abord été validée. La méta-analyse portant sur 225 études en sciences, ingénierie et mathématiques indique que les formats d'apprentissage actif augmentent la performance aux examens et réduisent les taux d'échec par rapport aux cours magistraux \citep{freeman2014}. Une méta-analyse distincte portant sur 104 études en sciences humaines et sociales (sociologie, psychologie, économie, éducation, langues) produit un résultat convergent : les formats actifs génèrent un avantage de performance de $g = 0{,}49$ par rapport au cours magistral, une taille d'effet comparable à celle observée dans les disciplines scientifiques \citep{kozanitis2023}. Le cadre ICAP affine ces résultats en montrant que les activités constructives et interactives produisent des gains supérieurs aux activités simplement actives \citep{chi2014}. Cette distinction entre niveaux d'engagement cognitif converge avec la définition de l'apprentissage actif comme combinaison de deux dimensions : l'engagement corporel (\textit{hands-on}) et l'engagement cognitif (\textit{minds-on}), le second étant la condition nécessaire du gain d'apprentissage \citep{yannier2021}. La manipulation physique d'un matériel --- déplacer des pièces, interagir avec une interface --- ne produit un bénéfice que si elle s'accompagne d'un traitement inférentiel du contenu.

Un paradoxe perceptif limite cependant l'adoption des formats actifs. Les apprenants jugent parfois les cours magistraux plus efficaces que les formats constructifs ou interactifs, alors que les mesures objectives indiquent le contraire. Des données expérimentales en physique confirment ce décalage : dans un protocole contrôlé, les étudiants exposés à un cours actif obtiennent des scores d'apprentissage supérieurs à ceux du groupe exposé au cours magistral, mais évaluent subjectivement leur propre apprentissage comme inférieur \citep{deslauriers2019}. L'effort cognitif supplémentaire requis par les modes constructif et interactif est interprété comme un signe de difficulté plutôt que comme un indicateur de traitement approfondi. Ce paradoxe perceptif rejoint l'heuristique de fluidité qui sera examinée en section~\ref{sec:illusion_comprehension} : la facilité de traitement est confondue avec la qualité de l'apprentissage. Le cadre ICAP converge par ailleurs avec la SDT (cf.~\ref{subsec:sdt_cet}) : les modes constructif et interactif favorisent le sentiment de compétence (l'apprenant reçoit un retour sur sa propre production) et d'autonomie (il participe à la construction du savoir plutôt que de le recevoir).

Le mode interactif du cadre ICAP se caractérise par la co-construction mutuelle entre deux interlocuteurs. Les conditions qui rendent cette co-construction effective dans un environnement médiatisé relèvent de l'apprentissage multimédia. Trois leviers d'interactivité ont été identifiés dans les environnements multimodaux : le dialogue (l'échange verbal entre l'apprenant et le système), le contrôle (la possibilité pour l'apprenant de réguler le rythme et la séquence du contenu) et la manipulation (l'action directe sur les éléments de l'interface) \citep{moreno2007}. L'effet d'interactivité désigne le gain d'apprentissage produit lorsque ces leviers sont activés : les apprenants qui contrôlent le rythme de présentation, explorent les concepts selon leurs propres questions et reçoivent un feedback immédiat sur leurs actions obtiennent de meilleurs résultats que les apprenants exposés au même contenu en format linéaire \citep{evans2007}. Le feedback immédiat joue un rôle particulier : il permet à l'apprenant de corriger ses représentations en temps réel, ce qui soutient le processus de génération d'inférences propre au mode constructif. La combinaison de ces trois leviers --- dialogue, contrôle, feedback --- définit les conditions minimales pour qu'un dispositif médiatisé active les processus cognitifs associés au mode interactif du cadre ICAP.

Le cadre ICAP fournit une grille d'analyse directement applicable aux dispositifs éducatifs intégrant des agents conversationnels. Regarder la vidéo d'un agent qui présente un contenu historique relève du mode passif : l'apprenant reçoit l'information sans la transformer. Dialoguer avec ce même agent, en posant des questions et en recevant des réponses adaptées, se situe potentiellement au niveau interactif --- à condition que l'échange implique des contributions mutuellement réactives et non une simple alternance de questions-réponses factuelles. La distinction est déterminante : la possibilité technique de poser des questions ne garantit pas un engagement interactif au sens du cadre ICAP. Seul un échange où chaque contribution transforme la précédente active les processus de co-construction qui caractérisent le mode interactif. Cette distinction entre formats passif et interactif constitue l'une des variables centrales de la présente recherche.

Les quatre sous-sections précédentes ont établi les mécanismes psychologiques qui régissent l'engagement de l'apprenant : les dynamiques motivationnelles, le développement de l'intérêt, le rôle de la pertinence et les niveaux d'engagement cognitif. La section suivante examine comment ces mécanismes se manifestent dans un contexte disciplinaire spécifique : l'enseignement de l'histoire.
