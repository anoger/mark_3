% ============================================================================
% Sous-section 2.1.4 : L'Apprentissage Actif et le Cadre ICAP
% ============================================================================
% Sources : IJCCI_extraction (§2.3), Chi (2009), Freeman et al. (2014)
% Calibrage : ~550 mots
% Type : C (Rédaction originale)
% ============================================================================

\subsection{L'Apprentissage Actif et le Cadre ICAP}
\label{subsec:ICAP}

Les sections précédentes ont montré que l'effort cognitif investi dans la génération de connexions personnelles renforce leur impact motivationnel. Ce constat s'inscrit dans un cadre plus général~: l'apprentissage actif, qui dépasse la simple activité physique pour désigner le fait de \og penser activement à ce que l'on fait\fg{}~\citep{mayer2014cambridge, yannier2021active}. La question devient alors~: comment caractériser les différents niveaux d'engagement cognitif et prédire leurs effets sur l'apprentissage?

Le cadre ICAP (\textit{Interactive, Constructive, Active, Passive}) propose une taxonomie hiérarchisée des activités d'apprentissage selon leur niveau d'engagement cognitif~\citep{chi2009active}. Le mode \textit{Passif} correspond à la réception d'information sans comportement observable au-delà de l'attention --- écouter un cours, regarder une vidéo. Le mode \textit{Actif} implique une manipulation ou une attention focalisée sans production de nouvelles idées --- prendre des notes verbatim, surligner. Le mode \textit{Constructif} requiert la génération d'idées qui dépassent l'information présentée --- formuler des hypothèses, élaborer des explications, connecter le contenu à son expérience personnelle. Le mode \textit{Interactif} ajoute une dimension dialogique~: les partenaires co-construisent des connaissances à travers un échange où chacun contribue substantiellement.

La prédiction centrale du modèle --- Interactif $>$ Constructif $>$ Actif $>$ Passif en termes de gains d'apprentissage --- a reçu un soutien empirique substantiel. Une méta-analyse portant sur 225 études montre que l'apprentissage actif augmente significativement la performance des étudiants en STIM comparé aux cours magistraux~\citep{freeman2014active}. L'interactivité permet de réguler le rythme d'apprentissage, d'explorer les concepts selon ses intérêts, de formuler des questions et de recevoir un feedback immédiat~\citep{domagk2010pedagogical, evans2007interactivity}.

Cette hiérarchie éclaire les résultats sur les interventions de pertinence~: les approches à effort élevé (auto-génération, réévaluation dirigée) relèvent du mode \textit{Constructif}, tandis que les approches à faible effort (communication directe) maintiennent l'apprenant en mode \textit{Passif}. L'efficacité supérieure des premières s'explique ainsi par leur niveau d'engagement cognitif plus profond. Un paradoxe mérite cependant attention~: l'apprentissage actif, bien que conduisant à de meilleurs résultats, est souvent perçu comme moins efficace par les apprenants eux-mêmes~\citep{deslauriers2019measuring}. L'effort cognitif accru est interprété comme un signe de difficulté alors qu'il indique un traitement plus profond.

Dans l'enseignement de l'Histoire, les approches interactives produisent des effets positifs sur la compréhension et l'intérêt. Les discussions de groupe favorisent une meilleure compréhension des concepts historiques~\citep{delfavero2007classroom}. La narration numérique interactive, avec des points de décision stratégiques, stimule des discussions significatives et une compréhension plus profonde~\citep{petousi2022interactive}. La combinaison d'interactions tangibles avec des récits émotionnels encourage les adolescents à s'engager avec les figures historiques au-delà de la connaissance factuelle~\citep{roussou2024emotions}.

Un agent conversationnel incarnant un personnage historique se situe naturellement au niveau \textit{Interactif}~: l'élève formule des questions (activité constructive), l'agent répond, et l'échange peut conduire à une co-construction de sens. Cette position contraste avec la vidéo (mode \textit{Passif}) et le texte (mode \textit{Actif} si l'élève surligne ou prend des notes). Le cadre ICAP prédit ainsi que l'agent dialogique devrait produire des gains d'engagement et d'apprentissage supérieurs aux formats traditionnels --- une prédiction que notre programme expérimental vise à tester.

Cette convergence des cadres théoriques --- SDT, développement de l'intérêt, théorie de l'attente-valeur, ICAP --- dessine les contours d'une intervention prometteuse~: un agent conversationnel qui satisfait les besoins d'autonomie, de compétence et d'affiliation, déclenche et maintient l'intérêt par la nouveauté et l'interaction sociale, permet la génération de connexions personnelles, et engage l'apprenant au niveau interactif du cadre ICAP.
