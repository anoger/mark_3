% ============================================================================
% Section 2.1 : Cadre Théorique de l'Apprenant
% ============================================================================
% Objectif : Définir les moteurs psychologiques internes de l'apprentissage
% Calibrage : ~150 mots (introduction)
% ============================================================================

\section{Cadre Théorique de l'Apprenant~: Motivation, Engagement et Pertinence}
\label{sec:cadre_theorique_apprenant}

Cette première section établit les fondations théoriques nécessaires à l'analyse des mécanismes psychologiques qui sous-tendent l'engagement des élèves dans les apprentissages. Nous examinons successivement les théories motivationnelles qui éclairent les conditions de l'engagement intrinsèque (\S\ref{subsec:SDT_CET}), l'architecture développementale de l'intérêt (\S\ref{subsec:architecture_interet}), les processus par lesquels les élèves attribuent de la valeur aux contenus d'apprentissage et les stratégies de personnalisation (\S\ref{subsec:pertinence_personnalisation}), et les niveaux d'engagement cognitif différenciés par le cadre ICAP (\S\ref{subsec:ICAP}).

Ces cadres théoriques, issus de la psychologie de l'éducation, fournissent les outils conceptuels qui seront mobilisés dans les sections suivantes pour analyser les spécificités de l'enseignement de l'Histoire (\S\ref{sec:contexte_histoire}) et pour comprendre les mécanismes d'interaction avec les agents conversationnels (\S\ref{sec:cadre_interaction}).

% Inclusion des sous-sections
% ============================================================================
% Sous-section 2.1.1 : Dynamiques Motivationnelles — SDT et CET
% ============================================================================
% Sources : IJCCI_extraction (§2.1), Vault.xlsx
% Calibrage : ~600 mots
% Type : C (Rédaction originale, synthèse analytique)
% ============================================================================

\subsection{Dynamiques Motivationnelles~: SDT et CET}
\label{subsec:SDT_CET}

La motivation intrinsèque --- cette propension à s'engager dans une activité pour le plaisir et la satisfaction qu'elle procure --- constitue un prédicteur robuste de la qualité de l'apprentissage~\citep{deci2000what}. Sa compréhension nécessite d'articuler deux niveaux d'analyse~: les mécanismes par lesquels l'environnement affecte cette motivation, et les besoins psychologiques sous-jacents qui la conditionnent.

Au premier niveau, les facteurs environnementaux n'exercent pas d'effet direct et uniforme sur la motivation~; leur impact dépend de l'interprétation qu'en fait l'individu~\citep{deci1985intrinsic}. Un même feedback peut ainsi être vécu comme \textit{informationnel} --- fournissant un retour constructif sur la compétence et soutenant la motivation --- ou comme \textit{contrôlant} --- exerçant une pression sur le comportement et sapant l'autodétermination. Cette distinction, issue de la Théorie de l'Évaluation Cognitive (CET), s'articule autour de deux dimensions psychologiques~: le \textit{locus de causalité perçu}, soit le sentiment que ses actions émanent de soi plutôt que de contraintes externes, et le \textit{sentiment de compétence}, soit la conviction de pouvoir atteindre les résultats souhaités. Un troisième aspect, \textit{amotivant}, peut signaler l'incompétence et conduire au désengagement total.

Au second niveau, trois besoins psychologiques fondamentaux conditionnent le bien-être et l'engagement~: l'\textit{autonomie}, la \textit{compétence} et l'\textit{affiliation}~\citep{ryan2017self}. La Théorie de l'Autodétermination (SDT) postule que les environnements d'apprentissage favorisent l'intérêt lorsqu'ils satisfont ces besoins à travers des choix significatifs, des défis appropriés, un feedback constructif et des interactions soutenantes. L'ajout du besoin d'affiliation --- le sentiment de connexion aux autres --- étend la portée explicative du modèle au-delà des seules dimensions cognitives pour intégrer la dimension sociale de l'apprentissage.

L'articulation de ces deux niveaux révèle une dynamique complexe~: un environnement qui offre des choix (autonomie), propose des défis calibrés avec un feedback informatif (compétence), et crée des opportunités d'interaction authentique (affiliation), réunit les conditions propices au développement de la motivation intrinsèque. À l'inverse, un environnement perçu comme contrôlant, qui impose des tâches sans en expliciter le sens et isole l'apprenant, tend à éroder cette motivation --- un phénomène particulièrement documenté lors de la transition vers le secondaire~\citep{gnambs2016decline}. Cette période, marquée par une pression accrue sur la performance et des structures scolaires qui peuvent limiter la curiosité naturelle~\citep{engel2011children}, voit la motivation intrinsèque décliner significativement.

Ce déclin revêt une importance particulière pour l'enseignement de l'Histoire, discipline souvent perçue comme distante et déconnectée des expériences personnelles~\citep{audigier2010histoire}. Le défi consiste alors à concevoir des environnements qui, tout en respectant les contraintes curriculaires, créent les conditions de satisfaction des besoins identifiés. Les agents conversationnels offrent une piste prometteuse~: l'interactivité dialogique peut soutenir l'autonomie en permettant à l'élève de diriger l'échange, les réponses adaptatives peuvent nourrir le sentiment de compétence, et les indices sociaux de l'agent peuvent répondre au besoin d'affiliation. Cette hypothèse, qui guide notre programme expérimental, nécessite cependant de comprendre comment l'intérêt se développe et se maintient --- objet de la section suivante.

\subsection{Architecture de l'Intérêt}
\label{subsec:architecture_interet}

L'intérêt, tel que défini dans la recherche en psychologie de l'éducation, combine deux composantes de nature distincte \citep{hidi2006}. La composante affective se manifeste par une expérience positive associée à l'activité --- attention accrue, affect favorable, envie spontanée de poursuivre. La composante cognitive se traduit par une orientation vers la compréhension du contenu : l'individu cherche à approfondir, pose des questions, établit des connexions entre les informations. Cette double nature distingue l'intérêt des construits apparentés. Là où la motivation intrinsèque (cf. \ref{subsec:sdt_cet}) désigne un processus général applicable à toute activité, l'intérêt est toujours dirigé vers un contenu spécifique \citep{renninger2015}. Un même élève peut manifester un intérêt soutenu pour la biologie et un désintérêt marqué pour l'histoire, alors que son niveau de motivation intrinsèque globale reste stable. Cette spécificité de contenu confère à l'intérêt une valeur explicative particulière pour l'apprentissage disciplinaire : il contribue à expliquer pourquoi un dispositif pédagogique efficace dans une matière échoue dans une autre \citep{hidi2000}.

Le développement de l'intérêt suit une séquence en quatre phases dont chacune se caractérise par un équilibre distinct entre soutien externe et engagement autonome \citep{hidi2006}. La première phase, l'intérêt situationnel déclenché, désigne une réponse attentionnelle à un stimulus environnemental --- nouveauté, surprise, incongruité avec les attentes. Cette réponse est brève et dépend entièrement du déclencheur externe. La deuxième phase, l'intérêt situationnel maintenu, apparaît lorsque l'engagement avec le contenu se prolonge au-delà de la réaction initiale : l'apprenant commence à traiter l'information en profondeur, mais le soutien de l'environnement reste nécessaire pour maintenir son attention. La troisième phase, l'intérêt individuel émergent, marque un changement qualitatif : l'apprenant développe une prédisposition à se réengager avec le contenu de sa propre initiative, génère des questions et recherche activement des informations complémentaires. La quatrième phase, l'intérêt individuel développé, correspond à une disposition stable : l'apprenant tolère la frustration, autorégule son apprentissage et produit des questions de curiosité qui alimentent sa progression \citep{renninger2015}. La transition de la deuxième à la troisième phase constitue le point de basculement du modèle : l'intérêt tend à ne plus dépendre du soutien de l'environnement pour devenir auto-entretenu.

Les facteurs qui déclenchent l'intérêt situationnel ne sont pas ceux qui le maintiennent. La nouveauté, le caractère inattendu d'une information et l'intensité perceptive d'un stimulus suffisent à capter l'attention (phase 1), mais leur effet tend à se dissiper si aucune connexion significative ne s'établit avec l'apprenant \citep{hidi2006}. Le maintien de l'intérêt (phase 2) repose sur des facteurs différents : la pertinence personnelle perçue, l'engagement actif dans une tâche signifiante et le sentiment de compétence dans l'interaction avec le contenu \citep{bergin1999}. Le contexte social joue un rôle transversal. L'identité du locuteur, la qualité de la relation avec l'enseignant et l'influence des pairs peuvent à la fois déclencher et maintenir l'intérêt \citep{bergin2016}. Un interlocuteur perçu comme signifiant --- qu'il soit enseignant, pair ou figure de référence --- peut affecter le développement de l'intérêt indépendamment du contenu transmis \citep{renninger2009}. Cette dissociation entre déclenchement et maintien pose un problème de conception : un dispositif qui mise exclusivement sur la nouveauté sans créer de connexion personnelle avec le contenu ne dépasse pas la première phase.

La curiosité et l'intérêt entretiennent une relation de renforcement mutuel qui s'intensifie au fil du développement \citep{hidi2020}. Dans les phases avancées de l'intérêt (phases 3 et 4), l'apprenant ne se contente plus de répondre aux stimuli externes : il génère spontanément des questions qui orientent son exploration du domaine. Ces questions de curiosité constituent le moteur interne de la progression vers un intérêt individuel stable. Les environnements éducatifs qui accueillent ces questions --- plutôt que de les canaliser vers des objectifs prédéterminés --- facilitent la transition entre les phases \citep{renninger2015}. La progression de l'intérêt situationnel vers l'intérêt individuel dépend toutefois d'un facteur que le modèle identifie sans l'approfondir : la capacité de l'apprenant à percevoir le contenu comme personnellement pertinent. Ce mécanisme, par lequel un savoir disciplinaire acquiert une valeur aux yeux de l'apprenant, relève d'un cadre théorique distinct : la théorie de la valeur et de la pertinence, qui fait l'objet de la section suivante.

% ============================================================================
% Sous-section 2.1.3 : Pertinence, Valeur et Personnalisation de l'Apprentissage
% ============================================================================
% Sources : IJCCI_extraction (§2.1-2.2), Harackiewicz (2014, 2016), Vault.xlsx,
%           Albrecht & Karabenick (2018)
% Calibrage : ~1100-1200 mots
% Type : C (Rédaction originale)
% ============================================================================

\subsection{Pertinence, Valeur et Personnalisation de l'Apprentissage}
\label{subsec:pertinence_personnalisation}

La section précédente a établi que la valeur perçue constitue un mécanisme pivot dans la transition de l'intérêt situationnel vers l'intérêt individuel. Cette valeur ne relève pas d'une propriété intrinsèque du contenu~; elle émerge de la relation que l'apprenant établit entre ce contenu et ses préoccupations personnelles. La théorie de l'attente-valeur formalise cette intuition en postulant que la motivation à s'engager dans une tâche dépend de deux facteurs multiplicatifs~: les \textit{attentes de succès} (croyance en sa capacité de réussir) et la \textit{valeur} attribuée à la tâche~\citep{eccles1983expectancies, wigfield2000expectancy}.

La valeur subjective se décompose en quatre dimensions qui éclairent différentes facettes de la pertinence. La \textit{valeur d'accomplissement} renvoie à l'importance de la tâche pour l'identité personnelle --- réussir confirme une image de soi valorisée. La \textit{valeur intrinsèque} correspond au plaisir tiré de l'activité elle-même, rejoignant la dimension affective de l'intérêt. La \textit{valeur d'utilité} concerne la pertinence pour les objectifs futurs --- le contenu est perçu comme un moyen vers une fin désirée. Enfin, le \textit{coût} perçu (effort, anxiété, opportunités sacrifiées) vient moduler négativement ces valeurs positives. Cette décomposition révèle pourquoi l'enseignement de l'Histoire souffre souvent d'un déficit d'engagement~: la discipline peine à démontrer sa valeur d'utilité immédiate, contrairement aux disciplines STIM dont les applications apparaissent plus évidentes~\citep{harackiewicz2016importance}.

\subsubsection*{De la valeur d'utilité à la pertinence personnelle}

La valeur d'utilité, si elle constitue un levier d'intervention accessible, ne représente qu'une facette de la pertinence. Au-delà de la question instrumentale \og à quoi ça sert?\fg{}, une question plus profonde émerge~: \og en quoi cela me concerne-t-il?\fg{}. Cette \textit{pertinence personnelle} (\textit{self-relevance}) implique que le contenu entre en résonance avec l'identité, les expériences et les préoccupations de l'apprenant~\citep{priniski2018making}.

Une conceptualisation multidimensionnelle distingue ainsi la pertinence \textit{personnelle} (connexion au soi) de la pertinence \textit{impersonnelle} (utilité pour des entités externes), et la pertinence \textit{appliquée} (utilité pour des actions concrètes) de la pertinence \textit{conceptuelle} (aide à la compréhension du monde)~\citep{albrecht2018relevance}. Cette taxonomie explique pourquoi une même intervention peut fonctionner pour certains élèves et échouer pour d'autres~: un élève sensible à la pertinence appliquée (\og cela m'aidera dans mon métier\fg{}) et un autre répondant à la pertinence conceptuelle (\og cela m'aide à comprendre l'actualité\fg{}) nécessitent des approches différenciées.

\subsubsection*{Stratégies d'intervention sur la pertinence}

Les interventions visant à augmenter la pertinence perçue se distinguent par le niveau d'effort cognitif qu'elles requièrent de l'apprenant~\citep{albrecht2018relevance}. À faible effort, la \textit{communication directe} --- l'enseignant explique pourquoi le contenu est pertinent --- et la \textit{personnalisation} --- le contenu est adapté aux intérêts déclarés --- représentent des approches accessibles mais dont l'impact reste limité. L'information fournie de l'extérieur peut être ignorée ou rejetée si elle ne résonne pas avec les schémas existants de l'apprenant.

Les approches à effort modéré, comme la \textit{réflexion critique}, invitent les élèves à examiner leurs propres croyances sur la pertinence du contenu. Cette introspection peut révéler des connexions insoupçonnées, mais elle reste circonscrite au répertoire cognitif préexistant de l'apprenant.

Les approches à effort élevé --- \textit{auto-génération} et \textit{réévaluation dirigée} --- demandent aux élèves de produire activement des connexions entre le contenu et leur vie, ou de reconsidérer la valeur du contenu à la lumière de nouvelles perspectives. Les interventions de valeur d'utilité, où les élèves rédigent des essais connectant le contenu académique à leur expérience personnelle, illustrent cette approche~\citep{harackiewicz2014harnessing}. L'effort cognitif investi dans la génération de connexions semble renforcer leur impact motivationnel --- un résultat qui résonne avec la hiérarchie des modes d'engagement que nous examinerons dans la section suivante (cadre ICAP).

Une nuance importante émerge cependant~: ces interventions bénéficient particulièrement aux élèves présentant de faibles attentes de succès initiales, pour qui la découverte de connexions personnelles peut transformer la perception du contenu. Pour les élèves avec de fortes attentes de succès, l'intervention peut paradoxalement détourner l'attention de stratégies d'apprentissage déjà efficaces~\citep{harackiewicz2014harnessing}. Cette interaction entre intervention et profil de l'apprenant souligne la nécessité d'une approche différenciée.

\subsubsection*{La personnalisation comme vecteur de pertinence}

La personnalisation de l'apprentissage --- l'adaptation du contenu et des activités aux intérêts, connaissances préalables et préférences individuels --- constitue une stratégie prometteuse pour développer la pertinence à grande échelle~\citep{walkington2018personalization}. L'intégration d'informations personnelles dans les contextes d'apprentissage (prénom de l'élève, centres d'intérêt déclarés) peut augmenter la motivation~\citep{cordova1996intrinsic}. Plus remarquable encore, offrir un contrôle sur des aspects même non essentiels de l'apprentissage (choix d'un avatar, d'un thème visuel) produit des effets positifs --- suggérant que le sentiment d'appropriation personnelle contribue à l'engagement indépendamment du contenu lui-même. Ce résultat établit un lien avec le besoin d'autonomie identifié par la SDT~: le choix, même symbolique, nourrit le sentiment d'autodétermination.

En mathématiques et en sciences, permettre aux élèves de choisir des exemples connectant les concepts abstraits à leurs intérêts renforce l'engagement, particulièrement chez ceux présentant initialement un faible intérêt~\citep{hogheim2015, reber2018personalized}. Cette stratégie crée des ponts entre concepts abstraits et expériences concrètes, facilitant l'ancrage des nouvelles connaissances dans les schémas existants --- retrouvant ainsi le facteur \og connaissances préalables\fg{} de la taxonomie de l'intérêt. Placer la curiosité des élèves au centre de l'apprentissage peut également améliorer l'engagement et les relations pédagogiques~\citep{hagay2015incorporating}.

\subsubsection*{Défis et perspectives technologiques}

La personnalisation se heurte cependant à des obstacles pratiques~: diversité des intérêts, évolution constante des préférences, complexité de créer du contenu adapté pour chaque profil~\citep{walkington2018personalization}. Ces contraintes ont longtemps limité le déploiement à grande échelle d'approches véritablement individualisées.

Les grands modèles de langage (LLM) offrent de nouvelles perspectives en permettant une adaptation dynamique et contextuelle~\citep{kasneci2023chatgpt, labadze2023generative}. Un agent conversationnel incarnant un personnage historique peut ainsi devenir un vecteur naturel de pertinence personnelle. Le dialogue direct --- où l'élève pose ses propres questions et reçoit des réponses adaptées à son niveau de compréhension et à ses préoccupations --- crée une connexion entre le contenu historique et la curiosité individuelle. Chaque élève explore les aspects qui l'intriguent, génère ses propres connexions à travers ses questions, et découvre ainsi la pertinence personnelle du savoir historique.

Cette \textit{personnalisation émergente} --- qui naît de l'interaction plutôt que d'une programmation préalable --- combine plusieurs mécanismes identifiés~: elle permet l'auto-génération de connexions personnelles (effort élevé), offre un sentiment de contrôle sur l'exploration (autonomie), et peut activer différentes dimensions de pertinence selon les questions posées (personnelle/impersonnelle, appliquée/conceptuelle). L'alignement thématique entre le personnage et le contenu de la leçon peut amplifier ces effets~\citep{schmidt2019effects}. Cette hypothèse constitue l'un des axes centraux de notre programme expérimental.

Reste à comprendre comment la forme de l'interaction --- et non seulement son contenu --- affecte l'engagement cognitif. Le cadre ICAP, que nous examinons maintenant, offre une grille d'analyse pour distinguer différents niveaux d'engagement et prédire leurs effets sur l'apprentissage.


% ============================================================================
% Sous-section 2.1.4 : L'Apprentissage Actif et le Cadre ICAP
% ============================================================================
% Sources : IJCCI_extraction (§2.3), Chi (2009), Freeman et al. (2014)
% Calibrage : ~550 mots
% Type : C (Rédaction originale)
% ============================================================================

\subsection{L'Apprentissage Actif et le Cadre ICAP}
\label{subsec:ICAP}

Les sections précédentes ont montré que l'effort cognitif investi dans la génération de connexions personnelles renforce leur impact motivationnel. Ce constat s'inscrit dans un cadre plus général~: l'apprentissage actif, qui dépasse la simple activité physique pour désigner le fait de \og penser activement à ce que l'on fait\fg{}~\citep{mayer2014cambridge, yannier2021active}. La question devient alors~: comment caractériser les différents niveaux d'engagement cognitif et prédire leurs effets sur l'apprentissage?

Le cadre ICAP (\textit{Interactive, Constructive, Active, Passive}) propose une taxonomie hiérarchisée des activités d'apprentissage selon leur niveau d'engagement cognitif~\citep{chi2009active}. Le mode \textit{Passif} correspond à la réception d'information sans comportement observable au-delà de l'attention --- écouter un cours, regarder une vidéo. Le mode \textit{Actif} implique une manipulation ou une attention focalisée sans production de nouvelles idées --- prendre des notes verbatim, surligner. Le mode \textit{Constructif} requiert la génération d'idées qui dépassent l'information présentée --- formuler des hypothèses, élaborer des explications, connecter le contenu à son expérience personnelle. Le mode \textit{Interactif} ajoute une dimension dialogique~: les partenaires co-construisent des connaissances à travers un échange où chacun contribue substantiellement.

La prédiction centrale du modèle --- Interactif $>$ Constructif $>$ Actif $>$ Passif en termes de gains d'apprentissage --- a reçu un soutien empirique substantiel. Une méta-analyse portant sur 225 études montre que l'apprentissage actif augmente significativement la performance des étudiants en STIM comparé aux cours magistraux~\citep{freeman2014active}. L'interactivité permet de réguler le rythme d'apprentissage, d'explorer les concepts selon ses intérêts, de formuler des questions et de recevoir un feedback immédiat~\citep{domagk2010pedagogical, evans2007interactivity}.

Cette hiérarchie éclaire les résultats sur les interventions de pertinence~: les approches à effort élevé (auto-génération, réévaluation dirigée) relèvent du mode \textit{Constructif}, tandis que les approches à faible effort (communication directe) maintiennent l'apprenant en mode \textit{Passif}. L'efficacité supérieure des premières s'explique ainsi par leur niveau d'engagement cognitif plus profond. Un paradoxe mérite cependant attention~: l'apprentissage actif, bien que conduisant à de meilleurs résultats, est souvent perçu comme moins efficace par les apprenants eux-mêmes~\citep{deslauriers2019measuring}. L'effort cognitif accru est interprété comme un signe de difficulté alors qu'il indique un traitement plus profond.

Dans l'enseignement de l'Histoire, les approches interactives produisent des effets positifs sur la compréhension et l'intérêt. Les discussions de groupe favorisent une meilleure compréhension des concepts historiques~\citep{delfavero2007classroom}. La narration numérique interactive, avec des points de décision stratégiques, stimule des discussions significatives et une compréhension plus profonde~\citep{petousi2022interactive}. La combinaison d'interactions tangibles avec des récits émotionnels encourage les adolescents à s'engager avec les figures historiques au-delà de la connaissance factuelle~\citep{roussou2024emotions}.

Un agent conversationnel incarnant un personnage historique se situe naturellement au niveau \textit{Interactif}~: l'élève formule des questions (activité constructive), l'agent répond, et l'échange peut conduire à une co-construction de sens. Cette position contraste avec la vidéo (mode \textit{Passif}) et le texte (mode \textit{Actif} si l'élève surligne ou prend des notes). Le cadre ICAP prédit ainsi que l'agent dialogique devrait produire des gains d'engagement et d'apprentissage supérieurs aux formats traditionnels --- une prédiction que notre programme expérimental vise à tester.

Cette convergence des cadres théoriques --- SDT, développement de l'intérêt, théorie de l'attente-valeur, ICAP --- dessine les contours d'une intervention prometteuse~: un agent conversationnel qui satisfait les besoins d'autonomie, de compétence et d'affiliation, déclenche et maintient l'intérêt par la nouveauté et l'interaction sociale, permet la génération de connexions personnelles, et engage l'apprenant au niveau interactif du cadre ICAP.

