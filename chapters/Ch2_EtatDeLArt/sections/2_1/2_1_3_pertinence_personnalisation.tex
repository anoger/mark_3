% ============================================================================
% Sous-section 2.1.3 : Pertinence, Valeur et Personnalisation de l'Apprentissage
% ============================================================================
% Sources : IJCCI_extraction (§2.1-2.2), Harackiewicz (2014, 2016), Vault.xlsx,
%           Albrecht & Karabenick (2018)
% Calibrage : ~1100-1200 mots
% Type : C (Rédaction originale)
% ============================================================================

\subsection{Pertinence, Valeur et Personnalisation de l'Apprentissage}
\label{subsec:pertinence_personnalisation}

La section précédente a établi que la valeur perçue constitue un mécanisme pivot dans la transition de l'intérêt situationnel vers l'intérêt individuel. Cette valeur ne relève pas d'une propriété intrinsèque du contenu~; elle émerge de la relation que l'apprenant établit entre ce contenu et ses préoccupations personnelles. La théorie de l'attente-valeur formalise cette intuition en postulant que la motivation à s'engager dans une tâche dépend de deux facteurs multiplicatifs~: les \textit{attentes de succès} (croyance en sa capacité de réussir) et la \textit{valeur} attribuée à la tâche~\citep{eccles1983expectancies, wigfield2000expectancy}.

La valeur subjective se décompose en quatre dimensions qui éclairent différentes facettes de la pertinence. La \textit{valeur d'accomplissement} renvoie à l'importance de la tâche pour l'identité personnelle --- réussir confirme une image de soi valorisée. La \textit{valeur intrinsèque} correspond au plaisir tiré de l'activité elle-même, rejoignant la dimension affective de l'intérêt. La \textit{valeur d'utilité} concerne la pertinence pour les objectifs futurs --- le contenu est perçu comme un moyen vers une fin désirée. Enfin, le \textit{coût} perçu (effort, anxiété, opportunités sacrifiées) vient moduler négativement ces valeurs positives. Cette décomposition révèle pourquoi l'enseignement de l'Histoire souffre souvent d'un déficit d'engagement~: la discipline peine à démontrer sa valeur d'utilité immédiate, contrairement aux disciplines STIM dont les applications apparaissent plus évidentes~\citep{harackiewicz2016importance}.

\subsubsection*{De la valeur d'utilité à la pertinence personnelle}

La valeur d'utilité, si elle constitue un levier d'intervention accessible, ne représente qu'une facette de la pertinence. Au-delà de la question instrumentale \og à quoi ça sert?\fg{}, une question plus profonde émerge~: \og en quoi cela me concerne-t-il?\fg{}. Cette \textit{pertinence personnelle} (\textit{self-relevance}) implique que le contenu entre en résonance avec l'identité, les expériences et les préoccupations de l'apprenant~\citep{priniski2018making}.

Une conceptualisation multidimensionnelle distingue ainsi la pertinence \textit{personnelle} (connexion au soi) de la pertinence \textit{impersonnelle} (utilité pour des entités externes), et la pertinence \textit{appliquée} (utilité pour des actions concrètes) de la pertinence \textit{conceptuelle} (aide à la compréhension du monde)~\citep{albrecht2018relevance}. Cette taxonomie explique pourquoi une même intervention peut fonctionner pour certains élèves et échouer pour d'autres~: un élève sensible à la pertinence appliquée (\og cela m'aidera dans mon métier\fg{}) et un autre répondant à la pertinence conceptuelle (\og cela m'aide à comprendre l'actualité\fg{}) nécessitent des approches différenciées.

\subsubsection*{Stratégies d'intervention sur la pertinence}

Les interventions visant à augmenter la pertinence perçue se distinguent par le niveau d'effort cognitif qu'elles requièrent de l'apprenant~\citep{albrecht2018relevance}. À faible effort, la \textit{communication directe} --- l'enseignant explique pourquoi le contenu est pertinent --- et la \textit{personnalisation} --- le contenu est adapté aux intérêts déclarés --- représentent des approches accessibles mais dont l'impact reste limité. L'information fournie de l'extérieur peut être ignorée ou rejetée si elle ne résonne pas avec les schémas existants de l'apprenant.

Les approches à effort modéré, comme la \textit{réflexion critique}, invitent les élèves à examiner leurs propres croyances sur la pertinence du contenu. Cette introspection peut révéler des connexions insoupçonnées, mais elle reste circonscrite au répertoire cognitif préexistant de l'apprenant.

Les approches à effort élevé --- \textit{auto-génération} et \textit{réévaluation dirigée} --- demandent aux élèves de produire activement des connexions entre le contenu et leur vie, ou de reconsidérer la valeur du contenu à la lumière de nouvelles perspectives. Les interventions de valeur d'utilité, où les élèves rédigent des essais connectant le contenu académique à leur expérience personnelle, illustrent cette approche~\citep{harackiewicz2014harnessing}. L'effort cognitif investi dans la génération de connexions semble renforcer leur impact motivationnel --- un résultat qui résonne avec la hiérarchie des modes d'engagement que nous examinerons dans la section suivante (cadre ICAP).

Une nuance importante émerge cependant~: ces interventions bénéficient particulièrement aux élèves présentant de faibles attentes de succès initiales, pour qui la découverte de connexions personnelles peut transformer la perception du contenu. Pour les élèves avec de fortes attentes de succès, l'intervention peut paradoxalement détourner l'attention de stratégies d'apprentissage déjà efficaces~\citep{harackiewicz2014harnessing}. Cette interaction entre intervention et profil de l'apprenant souligne la nécessité d'une approche différenciée.

\subsubsection*{La personnalisation comme vecteur de pertinence}

La personnalisation de l'apprentissage --- l'adaptation du contenu et des activités aux intérêts, connaissances préalables et préférences individuels --- constitue une stratégie prometteuse pour développer la pertinence à grande échelle~\citep{walkington2018personalization}. L'intégration d'informations personnelles dans les contextes d'apprentissage (prénom de l'élève, centres d'intérêt déclarés) peut augmenter la motivation~\citep{cordova1996intrinsic}. Plus remarquable encore, offrir un contrôle sur des aspects même non essentiels de l'apprentissage (choix d'un avatar, d'un thème visuel) produit des effets positifs --- suggérant que le sentiment d'appropriation personnelle contribue à l'engagement indépendamment du contenu lui-même. Ce résultat établit un lien avec le besoin d'autonomie identifié par la SDT~: le choix, même symbolique, nourrit le sentiment d'autodétermination.

En mathématiques et en sciences, permettre aux élèves de choisir des exemples connectant les concepts abstraits à leurs intérêts renforce l'engagement, particulièrement chez ceux présentant initialement un faible intérêt~\citep{hogheim2015, reber2018personalized}. Cette stratégie crée des ponts entre concepts abstraits et expériences concrètes, facilitant l'ancrage des nouvelles connaissances dans les schémas existants --- retrouvant ainsi le facteur \og connaissances préalables\fg{} de la taxonomie de l'intérêt. Placer la curiosité des élèves au centre de l'apprentissage peut également améliorer l'engagement et les relations pédagogiques~\citep{hagay2015incorporating}.

\subsubsection*{Défis et perspectives technologiques}

La personnalisation se heurte cependant à des obstacles pratiques~: diversité des intérêts, évolution constante des préférences, complexité de créer du contenu adapté pour chaque profil~\citep{walkington2018personalization}. Ces contraintes ont longtemps limité le déploiement à grande échelle d'approches véritablement individualisées.

Les grands modèles de langage (LLM) offrent de nouvelles perspectives en permettant une adaptation dynamique et contextuelle~\citep{kasneci2023chatgpt, labadze2023generative}. Un agent conversationnel incarnant un personnage historique peut ainsi devenir un vecteur naturel de pertinence personnelle. Le dialogue direct --- où l'élève pose ses propres questions et reçoit des réponses adaptées à son niveau de compréhension et à ses préoccupations --- crée une connexion entre le contenu historique et la curiosité individuelle. Chaque élève explore les aspects qui l'intriguent, génère ses propres connexions à travers ses questions, et découvre ainsi la pertinence personnelle du savoir historique.

Cette \textit{personnalisation émergente} --- qui naît de l'interaction plutôt que d'une programmation préalable --- combine plusieurs mécanismes identifiés~: elle permet l'auto-génération de connexions personnelles (effort élevé), offre un sentiment de contrôle sur l'exploration (autonomie), et peut activer différentes dimensions de pertinence selon les questions posées (personnelle/impersonnelle, appliquée/conceptuelle). L'alignement thématique entre le personnage et le contenu de la leçon peut amplifier ces effets~\citep{schmidt2019effects}. Cette hypothèse constitue l'un des axes centraux de notre programme expérimental.

Reste à comprendre comment la forme de l'interaction --- et non seulement son contenu --- affecte l'engagement cognitif. Le cadre ICAP, que nous examinons maintenant, offre une grille d'analyse pour distinguer différents niveaux d'engagement et prédire leurs effets sur l'apprentissage.

