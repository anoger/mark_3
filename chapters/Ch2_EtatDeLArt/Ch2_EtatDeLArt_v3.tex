% ============================================================================
% Chapitre 2 : État de l'Art (VERSION 3)
% Fichier Maître — Structure modulaire avec corrections des lacunes
% ============================================================================
% Corrections appliquées :
%   - Section 2.2 : Consolidation citations (lacunes 4, 20, 23)
%   - Section 2.3 : Aération, transitions ICAP, regroupement citations (lacunes 5, 19, 21, 25)
%   - Section 2.4 : Simplification fluidité, ref forward (lacunes 7, 17)
%   - Section 2.6 : Synthèse par références, apport explicité (lacunes 6, 18)
% ============================================================================

\chapter{État de l'Art}
\label{chap:etat_art}

% Sommaire de chapitre
\sommairechapitre

% Introduction du chapitre
Ce chapitre établit le cadre théorique nécessaire à l'analyse de l'interaction entre élèves et agents virtuels historiques. La structure suit une progression logique~: elle part des mécanismes internes de l'apprenant (section~\ref{sec:cadre_apprenant}), examine le contexte disciplinaire spécifique de l'enseignement de l'histoire (section~\ref{sec:contexte_histoire}), analyse la nature de l'interaction avec l'agent (section~\ref{sec:agents_pedagogiques}), explore la rupture technologique des IA génératives (section~\ref{sec:IA_generatives}), pour aboutir au risque métacognitif central de la thèse (section~\ref{sec:illusion_comprehension}). Une synthèse finale articule ces éléments en une problématique de recherche cohérente (section~\ref{sec:synthese_problematique}).

% ============================================================================
% Section 2.1 : Cadre Théorique de l'Apprenant (original - pas de lacunes)
% ============================================================================
% ============================================================================
% Section 2.1 : Cadre Théorique de l'Apprenant
% ============================================================================
% Objectif : Définir les moteurs psychologiques internes de l'apprentissage
% Calibrage : ~150 mots (introduction)
% ============================================================================

\section{Cadre Théorique de l'Apprenant~: Motivation, Engagement et Pertinence}
\label{sec:cadre_theorique_apprenant}

Cette première section établit les fondations théoriques nécessaires à l'analyse des mécanismes psychologiques qui sous-tendent l'engagement des élèves dans les apprentissages. Nous examinons successivement les théories motivationnelles qui éclairent les conditions de l'engagement intrinsèque (\S\ref{subsec:SDT_CET}), l'architecture développementale de l'intérêt (\S\ref{subsec:architecture_interet}), les processus par lesquels les élèves attribuent de la valeur aux contenus d'apprentissage et les stratégies de personnalisation (\S\ref{subsec:pertinence_personnalisation}), et les niveaux d'engagement cognitif différenciés par le cadre ICAP (\S\ref{subsec:ICAP}).

Ces cadres théoriques, issus de la psychologie de l'éducation, fournissent les outils conceptuels qui seront mobilisés dans les sections suivantes pour analyser les spécificités de l'enseignement de l'Histoire (\S\ref{sec:contexte_histoire}) et pour comprendre les mécanismes d'interaction avec les agents conversationnels (\S\ref{sec:cadre_interaction}).

% Inclusion des sous-sections
% ============================================================================
% Sous-section 2.1.1 : Dynamiques Motivationnelles — SDT et CET
% ============================================================================
% Sources : IJCCI_extraction (§2.1), Vault.xlsx
% Calibrage : ~600 mots
% Type : C (Rédaction originale, synthèse analytique)
% ============================================================================

\subsection{Dynamiques Motivationnelles~: SDT et CET}
\label{subsec:SDT_CET}

La motivation intrinsèque --- cette propension à s'engager dans une activité pour le plaisir et la satisfaction qu'elle procure --- constitue un prédicteur robuste de la qualité de l'apprentissage~\citep{deci2000what}. Sa compréhension nécessite d'articuler deux niveaux d'analyse~: les mécanismes par lesquels l'environnement affecte cette motivation, et les besoins psychologiques sous-jacents qui la conditionnent.

Au premier niveau, les facteurs environnementaux n'exercent pas d'effet direct et uniforme sur la motivation~; leur impact dépend de l'interprétation qu'en fait l'individu~\citep{deci1985intrinsic}. Un même feedback peut ainsi être vécu comme \textit{informationnel} --- fournissant un retour constructif sur la compétence et soutenant la motivation --- ou comme \textit{contrôlant} --- exerçant une pression sur le comportement et sapant l'autodétermination. Cette distinction, issue de la Théorie de l'Évaluation Cognitive (CET), s'articule autour de deux dimensions psychologiques~: le \textit{locus de causalité perçu}, soit le sentiment que ses actions émanent de soi plutôt que de contraintes externes, et le \textit{sentiment de compétence}, soit la conviction de pouvoir atteindre les résultats souhaités. Un troisième aspect, \textit{amotivant}, peut signaler l'incompétence et conduire au désengagement total.

Au second niveau, trois besoins psychologiques fondamentaux conditionnent le bien-être et l'engagement~: l'\textit{autonomie}, la \textit{compétence} et l'\textit{affiliation}~\citep{ryan2017self}. La Théorie de l'Autodétermination (SDT) postule que les environnements d'apprentissage favorisent l'intérêt lorsqu'ils satisfont ces besoins à travers des choix significatifs, des défis appropriés, un feedback constructif et des interactions soutenantes. L'ajout du besoin d'affiliation --- le sentiment de connexion aux autres --- étend la portée explicative du modèle au-delà des seules dimensions cognitives pour intégrer la dimension sociale de l'apprentissage.

L'articulation de ces deux niveaux révèle une dynamique complexe~: un environnement qui offre des choix (autonomie), propose des défis calibrés avec un feedback informatif (compétence), et crée des opportunités d'interaction authentique (affiliation), réunit les conditions propices au développement de la motivation intrinsèque. À l'inverse, un environnement perçu comme contrôlant, qui impose des tâches sans en expliciter le sens et isole l'apprenant, tend à éroder cette motivation --- un phénomène particulièrement documenté lors de la transition vers le secondaire~\citep{gnambs2016decline}. Cette période, marquée par une pression accrue sur la performance et des structures scolaires qui peuvent limiter la curiosité naturelle~\citep{engel2011children}, voit la motivation intrinsèque décliner significativement.

Ce déclin revêt une importance particulière pour l'enseignement de l'Histoire, discipline souvent perçue comme distante et déconnectée des expériences personnelles~\citep{audigier2010histoire}. Le défi consiste alors à concevoir des environnements qui, tout en respectant les contraintes curriculaires, créent les conditions de satisfaction des besoins identifiés. Les agents conversationnels offrent une piste prometteuse~: l'interactivité dialogique peut soutenir l'autonomie en permettant à l'élève de diriger l'échange, les réponses adaptatives peuvent nourrir le sentiment de compétence, et les indices sociaux de l'agent peuvent répondre au besoin d'affiliation. Cette hypothèse, qui guide notre programme expérimental, nécessite cependant de comprendre comment l'intérêt se développe et se maintient --- objet de la section suivante.

\subsection{Architecture de l'Intérêt}
\label{subsec:architecture_interet}

L'intérêt, tel que défini dans la recherche en psychologie de l'éducation, combine deux composantes de nature distincte \citep{hidi2006}. La composante affective se manifeste par une expérience positive associée à l'activité --- attention accrue, affect favorable, envie spontanée de poursuivre. La composante cognitive se traduit par une orientation vers la compréhension du contenu : l'individu cherche à approfondir, pose des questions, établit des connexions entre les informations. Cette double nature distingue l'intérêt des construits apparentés. Là où la motivation intrinsèque (cf. \ref{subsec:sdt_cet}) désigne un processus général applicable à toute activité, l'intérêt est toujours dirigé vers un contenu spécifique \citep{renninger2015}. Un même élève peut manifester un intérêt soutenu pour la biologie et un désintérêt marqué pour l'histoire, alors que son niveau de motivation intrinsèque globale reste stable. Cette spécificité de contenu confère à l'intérêt une valeur explicative particulière pour l'apprentissage disciplinaire : il contribue à expliquer pourquoi un dispositif pédagogique efficace dans une matière échoue dans une autre \citep{hidi2000}.

Le développement de l'intérêt suit une séquence en quatre phases dont chacune se caractérise par un équilibre distinct entre soutien externe et engagement autonome \citep{hidi2006}. La première phase, l'intérêt situationnel déclenché, désigne une réponse attentionnelle à un stimulus environnemental --- nouveauté, surprise, incongruité avec les attentes. Cette réponse est brève et dépend entièrement du déclencheur externe. La deuxième phase, l'intérêt situationnel maintenu, apparaît lorsque l'engagement avec le contenu se prolonge au-delà de la réaction initiale : l'apprenant commence à traiter l'information en profondeur, mais le soutien de l'environnement reste nécessaire pour maintenir son attention. La troisième phase, l'intérêt individuel émergent, marque un changement qualitatif : l'apprenant développe une prédisposition à se réengager avec le contenu de sa propre initiative, génère des questions et recherche activement des informations complémentaires. La quatrième phase, l'intérêt individuel développé, correspond à une disposition stable : l'apprenant tolère la frustration, autorégule son apprentissage et produit des questions de curiosité qui alimentent sa progression \citep{renninger2015}. La transition de la deuxième à la troisième phase constitue le point de basculement du modèle : l'intérêt tend à ne plus dépendre du soutien de l'environnement pour devenir auto-entretenu.

Les facteurs qui déclenchent l'intérêt situationnel ne sont pas ceux qui le maintiennent. La nouveauté, le caractère inattendu d'une information et l'intensité perceptive d'un stimulus suffisent à capter l'attention (phase 1), mais leur effet tend à se dissiper si aucune connexion significative ne s'établit avec l'apprenant \citep{hidi2006}. Le maintien de l'intérêt (phase 2) repose sur des facteurs différents : la pertinence personnelle perçue, l'engagement actif dans une tâche signifiante et le sentiment de compétence dans l'interaction avec le contenu \citep{bergin1999}. Le contexte social joue un rôle transversal. L'identité du locuteur, la qualité de la relation avec l'enseignant et l'influence des pairs peuvent à la fois déclencher et maintenir l'intérêt \citep{bergin2016}. Un interlocuteur perçu comme signifiant --- qu'il soit enseignant, pair ou figure de référence --- peut affecter le développement de l'intérêt indépendamment du contenu transmis \citep{renninger2009}. Cette dissociation entre déclenchement et maintien pose un problème de conception : un dispositif qui mise exclusivement sur la nouveauté sans créer de connexion personnelle avec le contenu ne dépasse pas la première phase.

La curiosité et l'intérêt entretiennent une relation de renforcement mutuel qui s'intensifie au fil du développement \citep{hidi2020}. Dans les phases avancées de l'intérêt (phases 3 et 4), l'apprenant ne se contente plus de répondre aux stimuli externes : il génère spontanément des questions qui orientent son exploration du domaine. Ces questions de curiosité constituent le moteur interne de la progression vers un intérêt individuel stable. Les environnements éducatifs qui accueillent ces questions --- plutôt que de les canaliser vers des objectifs prédéterminés --- facilitent la transition entre les phases \citep{renninger2015}. La progression de l'intérêt situationnel vers l'intérêt individuel dépend toutefois d'un facteur que le modèle identifie sans l'approfondir : la capacité de l'apprenant à percevoir le contenu comme personnellement pertinent. Ce mécanisme, par lequel un savoir disciplinaire acquiert une valeur aux yeux de l'apprenant, relève d'un cadre théorique distinct : la théorie de la valeur et de la pertinence, qui fait l'objet de la section suivante.

% ============================================================================
% Sous-section 2.1.3 : Pertinence, Valeur et Personnalisation de l'Apprentissage
% ============================================================================
% Sources : IJCCI_extraction (§2.1-2.2), Harackiewicz (2014, 2016), Vault.xlsx,
%           Albrecht & Karabenick (2018)
% Calibrage : ~1100-1200 mots
% Type : C (Rédaction originale)
% ============================================================================

\subsection{Pertinence, Valeur et Personnalisation de l'Apprentissage}
\label{subsec:pertinence_personnalisation}

La section précédente a établi que la valeur perçue constitue un mécanisme pivot dans la transition de l'intérêt situationnel vers l'intérêt individuel. Cette valeur ne relève pas d'une propriété intrinsèque du contenu~; elle émerge de la relation que l'apprenant établit entre ce contenu et ses préoccupations personnelles. La théorie de l'attente-valeur formalise cette intuition en postulant que la motivation à s'engager dans une tâche dépend de deux facteurs multiplicatifs~: les \textit{attentes de succès} (croyance en sa capacité de réussir) et la \textit{valeur} attribuée à la tâche~\citep{eccles1983expectancies, wigfield2000expectancy}.

La valeur subjective se décompose en quatre dimensions qui éclairent différentes facettes de la pertinence. La \textit{valeur d'accomplissement} renvoie à l'importance de la tâche pour l'identité personnelle --- réussir confirme une image de soi valorisée. La \textit{valeur intrinsèque} correspond au plaisir tiré de l'activité elle-même, rejoignant la dimension affective de l'intérêt. La \textit{valeur d'utilité} concerne la pertinence pour les objectifs futurs --- le contenu est perçu comme un moyen vers une fin désirée. Enfin, le \textit{coût} perçu (effort, anxiété, opportunités sacrifiées) vient moduler négativement ces valeurs positives. Cette décomposition révèle pourquoi l'enseignement de l'Histoire souffre souvent d'un déficit d'engagement~: la discipline peine à démontrer sa valeur d'utilité immédiate, contrairement aux disciplines STIM dont les applications apparaissent plus évidentes~\citep{harackiewicz2016importance}.

\subsubsection*{De la valeur d'utilité à la pertinence personnelle}

La valeur d'utilité, si elle constitue un levier d'intervention accessible, ne représente qu'une facette de la pertinence. Au-delà de la question instrumentale \og à quoi ça sert?\fg{}, une question plus profonde émerge~: \og en quoi cela me concerne-t-il?\fg{}. Cette \textit{pertinence personnelle} (\textit{self-relevance}) implique que le contenu entre en résonance avec l'identité, les expériences et les préoccupations de l'apprenant~\citep{priniski2018making}.

Une conceptualisation multidimensionnelle distingue ainsi la pertinence \textit{personnelle} (connexion au soi) de la pertinence \textit{impersonnelle} (utilité pour des entités externes), et la pertinence \textit{appliquée} (utilité pour des actions concrètes) de la pertinence \textit{conceptuelle} (aide à la compréhension du monde)~\citep{albrecht2018relevance}. Cette taxonomie explique pourquoi une même intervention peut fonctionner pour certains élèves et échouer pour d'autres~: un élève sensible à la pertinence appliquée (\og cela m'aidera dans mon métier\fg{}) et un autre répondant à la pertinence conceptuelle (\og cela m'aide à comprendre l'actualité\fg{}) nécessitent des approches différenciées.

\subsubsection*{Stratégies d'intervention sur la pertinence}

Les interventions visant à augmenter la pertinence perçue se distinguent par le niveau d'effort cognitif qu'elles requièrent de l'apprenant~\citep{albrecht2018relevance}. À faible effort, la \textit{communication directe} --- l'enseignant explique pourquoi le contenu est pertinent --- et la \textit{personnalisation} --- le contenu est adapté aux intérêts déclarés --- représentent des approches accessibles mais dont l'impact reste limité. L'information fournie de l'extérieur peut être ignorée ou rejetée si elle ne résonne pas avec les schémas existants de l'apprenant.

Les approches à effort modéré, comme la \textit{réflexion critique}, invitent les élèves à examiner leurs propres croyances sur la pertinence du contenu. Cette introspection peut révéler des connexions insoupçonnées, mais elle reste circonscrite au répertoire cognitif préexistant de l'apprenant.

Les approches à effort élevé --- \textit{auto-génération} et \textit{réévaluation dirigée} --- demandent aux élèves de produire activement des connexions entre le contenu et leur vie, ou de reconsidérer la valeur du contenu à la lumière de nouvelles perspectives. Les interventions de valeur d'utilité, où les élèves rédigent des essais connectant le contenu académique à leur expérience personnelle, illustrent cette approche~\citep{harackiewicz2014harnessing}. L'effort cognitif investi dans la génération de connexions semble renforcer leur impact motivationnel --- un résultat qui résonne avec la hiérarchie des modes d'engagement que nous examinerons dans la section suivante (cadre ICAP).

Une nuance importante émerge cependant~: ces interventions bénéficient particulièrement aux élèves présentant de faibles attentes de succès initiales, pour qui la découverte de connexions personnelles peut transformer la perception du contenu. Pour les élèves avec de fortes attentes de succès, l'intervention peut paradoxalement détourner l'attention de stratégies d'apprentissage déjà efficaces~\citep{harackiewicz2014harnessing}. Cette interaction entre intervention et profil de l'apprenant souligne la nécessité d'une approche différenciée.

\subsubsection*{La personnalisation comme vecteur de pertinence}

La personnalisation de l'apprentissage --- l'adaptation du contenu et des activités aux intérêts, connaissances préalables et préférences individuels --- constitue une stratégie prometteuse pour développer la pertinence à grande échelle~\citep{walkington2018personalization}. L'intégration d'informations personnelles dans les contextes d'apprentissage (prénom de l'élève, centres d'intérêt déclarés) peut augmenter la motivation~\citep{cordova1996intrinsic}. Plus remarquable encore, offrir un contrôle sur des aspects même non essentiels de l'apprentissage (choix d'un avatar, d'un thème visuel) produit des effets positifs --- suggérant que le sentiment d'appropriation personnelle contribue à l'engagement indépendamment du contenu lui-même. Ce résultat établit un lien avec le besoin d'autonomie identifié par la SDT~: le choix, même symbolique, nourrit le sentiment d'autodétermination.

En mathématiques et en sciences, permettre aux élèves de choisir des exemples connectant les concepts abstraits à leurs intérêts renforce l'engagement, particulièrement chez ceux présentant initialement un faible intérêt~\citep{hogheim2015, reber2018personalized}. Cette stratégie crée des ponts entre concepts abstraits et expériences concrètes, facilitant l'ancrage des nouvelles connaissances dans les schémas existants --- retrouvant ainsi le facteur \og connaissances préalables\fg{} de la taxonomie de l'intérêt. Placer la curiosité des élèves au centre de l'apprentissage peut également améliorer l'engagement et les relations pédagogiques~\citep{hagay2015incorporating}.

\subsubsection*{Défis et perspectives technologiques}

La personnalisation se heurte cependant à des obstacles pratiques~: diversité des intérêts, évolution constante des préférences, complexité de créer du contenu adapté pour chaque profil~\citep{walkington2018personalization}. Ces contraintes ont longtemps limité le déploiement à grande échelle d'approches véritablement individualisées.

Les grands modèles de langage (LLM) offrent de nouvelles perspectives en permettant une adaptation dynamique et contextuelle~\citep{kasneci2023chatgpt, labadze2023generative}. Un agent conversationnel incarnant un personnage historique peut ainsi devenir un vecteur naturel de pertinence personnelle. Le dialogue direct --- où l'élève pose ses propres questions et reçoit des réponses adaptées à son niveau de compréhension et à ses préoccupations --- crée une connexion entre le contenu historique et la curiosité individuelle. Chaque élève explore les aspects qui l'intriguent, génère ses propres connexions à travers ses questions, et découvre ainsi la pertinence personnelle du savoir historique.

Cette \textit{personnalisation émergente} --- qui naît de l'interaction plutôt que d'une programmation préalable --- combine plusieurs mécanismes identifiés~: elle permet l'auto-génération de connexions personnelles (effort élevé), offre un sentiment de contrôle sur l'exploration (autonomie), et peut activer différentes dimensions de pertinence selon les questions posées (personnelle/impersonnelle, appliquée/conceptuelle). L'alignement thématique entre le personnage et le contenu de la leçon peut amplifier ces effets~\citep{schmidt2019effects}. Cette hypothèse constitue l'un des axes centraux de notre programme expérimental.

Reste à comprendre comment la forme de l'interaction --- et non seulement son contenu --- affecte l'engagement cognitif. Le cadre ICAP, que nous examinons maintenant, offre une grille d'analyse pour distinguer différents niveaux d'engagement et prédire leurs effets sur l'apprentissage.


% ============================================================================
% Sous-section 2.1.4 : L'Apprentissage Actif et le Cadre ICAP
% ============================================================================
% Sources : IJCCI_extraction (§2.3), Chi (2009), Freeman et al. (2014)
% Calibrage : ~550 mots
% Type : C (Rédaction originale)
% ============================================================================

\subsection{L'Apprentissage Actif et le Cadre ICAP}
\label{subsec:ICAP}

Les sections précédentes ont montré que l'effort cognitif investi dans la génération de connexions personnelles renforce leur impact motivationnel. Ce constat s'inscrit dans un cadre plus général~: l'apprentissage actif, qui dépasse la simple activité physique pour désigner le fait de \og penser activement à ce que l'on fait\fg{}~\citep{mayer2014cambridge, yannier2021active}. La question devient alors~: comment caractériser les différents niveaux d'engagement cognitif et prédire leurs effets sur l'apprentissage?

Le cadre ICAP (\textit{Interactive, Constructive, Active, Passive}) propose une taxonomie hiérarchisée des activités d'apprentissage selon leur niveau d'engagement cognitif~\citep{chi2009active}. Le mode \textit{Passif} correspond à la réception d'information sans comportement observable au-delà de l'attention --- écouter un cours, regarder une vidéo. Le mode \textit{Actif} implique une manipulation ou une attention focalisée sans production de nouvelles idées --- prendre des notes verbatim, surligner. Le mode \textit{Constructif} requiert la génération d'idées qui dépassent l'information présentée --- formuler des hypothèses, élaborer des explications, connecter le contenu à son expérience personnelle. Le mode \textit{Interactif} ajoute une dimension dialogique~: les partenaires co-construisent des connaissances à travers un échange où chacun contribue substantiellement.

La prédiction centrale du modèle --- Interactif $>$ Constructif $>$ Actif $>$ Passif en termes de gains d'apprentissage --- a reçu un soutien empirique substantiel. Une méta-analyse portant sur 225 études montre que l'apprentissage actif augmente significativement la performance des étudiants en STIM comparé aux cours magistraux~\citep{freeman2014active}. L'interactivité permet de réguler le rythme d'apprentissage, d'explorer les concepts selon ses intérêts, de formuler des questions et de recevoir un feedback immédiat~\citep{domagk2010pedagogical, evans2007interactivity}.

Cette hiérarchie éclaire les résultats sur les interventions de pertinence~: les approches à effort élevé (auto-génération, réévaluation dirigée) relèvent du mode \textit{Constructif}, tandis que les approches à faible effort (communication directe) maintiennent l'apprenant en mode \textit{Passif}. L'efficacité supérieure des premières s'explique ainsi par leur niveau d'engagement cognitif plus profond. Un paradoxe mérite cependant attention~: l'apprentissage actif, bien que conduisant à de meilleurs résultats, est souvent perçu comme moins efficace par les apprenants eux-mêmes~\citep{deslauriers2019measuring}. L'effort cognitif accru est interprété comme un signe de difficulté alors qu'il indique un traitement plus profond.

Dans l'enseignement de l'Histoire, les approches interactives produisent des effets positifs sur la compréhension et l'intérêt. Les discussions de groupe favorisent une meilleure compréhension des concepts historiques~\citep{delfavero2007classroom}. La narration numérique interactive, avec des points de décision stratégiques, stimule des discussions significatives et une compréhension plus profonde~\citep{petousi2022interactive}. La combinaison d'interactions tangibles avec des récits émotionnels encourage les adolescents à s'engager avec les figures historiques au-delà de la connaissance factuelle~\citep{roussou2024emotions}.

Un agent conversationnel incarnant un personnage historique se situe naturellement au niveau \textit{Interactif}~: l'élève formule des questions (activité constructive), l'agent répond, et l'échange peut conduire à une co-construction de sens. Cette position contraste avec la vidéo (mode \textit{Passif}) et le texte (mode \textit{Actif} si l'élève surligne ou prend des notes). Le cadre ICAP prédit ainsi que l'agent dialogique devrait produire des gains d'engagement et d'apprentissage supérieurs aux formats traditionnels --- une prédiction que notre programme expérimental vise à tester.

Cette convergence des cadres théoriques --- SDT, développement de l'intérêt, théorie de l'attente-valeur, ICAP --- dessine les contours d'une intervention prometteuse~: un agent conversationnel qui satisfait les besoins d'autonomie, de compétence et d'affiliation, déclenche et maintient l'intérêt par la nouveauté et l'interaction sociale, permet la génération de connexions personnelles, et engage l'apprenant au niveau interactif du cadre ICAP.



% ============================================================================
% Section 2.2 : Le Contexte Disciplinaire — Enseigner et Apprendre l'Histoire (V3)
% ============================================================================
% ============================================================================
% Section 2.2 : Le Contexte Spécifique de l'Enseignement de l'Histoire (V3)
% ============================================================================
% Corrections appliquées :
% - 2_2_2 : Consolidation citations haydn2010pupil
% - 2_2_3 : Consolidation citations harris2006pupils
% ============================================================================

\section{Le Contexte Spécifique de l'Enseignement de l'Histoire}
\label{sec:contexte_histoire}

Les cadres théoriques présentés dans la section précédente éclairent les mécanismes généraux de l'engagement dans l'apprentissage. Leur application à l'enseignement de l'histoire révèle un terrain singulier. La discipline n'est pas rejetée par les élèves~; elle souffre d'un déficit de médiation entre ses spécificités épistémologiques et les attentes de son public. Ce diagnostic structure l'analyse qui suit~: la position de l'histoire dans l'écosystème scolaire (\S\ref{subsec:epistemologie_STIM}), le paradoxe de la perception des élèves (\S\ref{subsec:perception_eleves}), et les leviers pédagogiques identifiés par la recherche (\S\ref{subsec:pratiques_pedagogiques}).

% ----------------------------------------------------------------------------
% Sous-sections : 2_2_1 original, 2_2_2 et 2_2_3 en version v3
% ----------------------------------------------------------------------------
% ============================================================================
% Sous-section 2.2.1 : Position de l'Histoire dans l'Écosystème Scolaire
% ============================================================================
% Sources : État de l'Art Histoire (Harris & Haydn 2006, Haydn & Harris 2010,
%           Van Straaten et al. 2015, Grever et al. 2011)
% Calibrage : ~550 mots
% Type : C (Rédaction originale)
% ============================================================================

\subsection{Position de l'Histoire dans l'Écosystème Scolaire}
\label{subsec:epistemologie_STIM}

Contrairement à une idée répandue, l'histoire n'est pas rejetée par les élèves. Avec 69,8\% d'opinions favorables auprès de 1740 élèves britanniques, la discipline se classe en cinquième position des matières appréciées~\citep{harris2006pupils}. Ce constat, corroboré dans d'autres contextes nationaux, invite à nuancer le diagnostic d'une discipline en crise~: l'histoire occupe une position intermédiaire qui révèle moins un rejet qu'un déficit de médiation.

Cette position se caractérise par un décalage entre deux formes de valeur perçue. Les élèves reconnaissent à l'histoire une utilité \textit{cognitive}~--- comprendre le présent, éviter de répéter les erreurs du passé, développer un regard critique sur le monde~--- tout en peinant à lui attribuer une utilité \textit{instrumentale} comparable à celle des disciplines scientifiques et techniques. Sur une échelle d'importance perçue, les mathématiques obtiennent 4,46 et l'anglais 4,42, contre 3,26 pour l'histoire~\citep{haydn2010pupil}. Cette asymétrie s'explique par la lisibilité des débouchés professionnels~: les filières scientifiques offrent des trajectoires clairement identifiées là où l'histoire semble cantonnée à l'enseignement ou aux métiers du patrimoine~\citep{grever2011high}.

Cette hiérarchie implicite s'enracine dans des différences épistémologiques que l'école rend rarement explicites. Les disciplines STIM reposent sur des savoirs cumulatifs, universels et vérifiables par l'expérimentation~: la progression y suit une logique d'accumulation où chaque concept s'appuie sur les précédents, et le rapport à la vérité s'établit par démonstration~\citep{vanstraaten2015making}. Le feedback y est immédiat~--- une équation est correctement résolue ou ne l'est pas. L'histoire, en revanche, produit des savoirs interprétatifs et contextuels où la vérité émerge de l'argumentation fondée sur des sources, processus moins définitif où plusieurs interprétations peuvent coexister pour un même événement. Cette nature interprétative requiert une tolérance à l'ambiguïté que l'enseignement scolaire cultive rarement de manière explicite~--- d'où le sentiment de certains élèves que \og l'histoire est morte et n'a rien à voir avec [leur] vie présente\fg{}, une proportion qui atteint 14\% dans les enquêtes européennes~\citep{vanstraaten2015making}.

Le contraste méthodologique est tout aussi marqué. Les STIM privilégient l'expérimentation contrôlée et la modélisation mathématique, offrant des procédures reproductibles dont la rigueur est immédiatement perceptible. L'histoire utilise l'analyse critique de sources, la contextualisation et la mise en perspective~--- méthodes moins standardisées dont les élèves saisissent moins aisément l'exigence intellectuelle. Cette différence nourrit une perception de moindre scientificité, alors même que l'histoire développe des compétences critiques équivalentes~\citep{harris2006pupils}.

Ces différences épistémologiques ne constituent pas des défauts~; elles définissent des compétences distinctives. L'analyse critique de discours, la capacité à peser des arguments contradictoires, la compréhension des motivations humaines dans leur contexte, l'empathie historique~: autant de capacités que les STIM ne cultivent pas avec la même intensité. Le potentiel narratif et dramatique de l'histoire autorise un engagement émotionnel que les disciplines formelles peinent à susciter~\citep{harris2006pupils}. Sa contribution à la formation citoyenne et à la compréhension interculturelle répond à des besoins sociétaux croissants~\citep{grever2011high}. L'enjeu n'est donc pas d'imiter les STIM, mais de valoriser ces spécificités tout en explicitant les compétences qu'elles développent~--- un travail de médiation que l'enseignement traditionnel n'accomplit qu'imparfaitement.

\input{chapters/Ch2_EtatDeLArt/sections/2_2/2_2_2_perception_eleves_v3.tex}
\input{chapters/Ch2_EtatDeLArt/sections/2_2/2_2_3_pratiques_pedagogiques_v3.tex}


% ============================================================================
% Section 2.3 : Les Agents Pédagogiques Virtuels (V3)
% ============================================================================
% ============================================================================
% Section 2.3 : Les Agents Pédagogiques Virtuels (V3)
% ============================================================================
% Corrections appliquées :
% - 2_3_1 : Aération citations, consolidation effet persona
% - 2_3_2 : Ajout transition vers ICAP
% - 2_3_4 : Regroupement citations davis2018impact
% ============================================================================

\section{Les Agents Pédagogiques Virtuels~: Typologie, Signaux Sociaux et Efficacité}
\label{sec:agents_pedagogiques}

% ----------------------------------------------------------------------------
% Introduction de section (~200 mots)
% ----------------------------------------------------------------------------
Depuis le tuteur SCHOLAR de Carbonell en 1970, les agents pédagogiques virtuels ont connu trois générations de développement. La première (2000--2011) a exploré les possibilités des agents animés avec des systèmes comme Steve et Herman the Bug. La deuxième (2012--2019) a consolidé les bases empiriques à travers de nombreuses méta-analyses. La troisième (2020--présent) intègre les capacités des grands modèles de langage, transformant radicalement les possibilités d'interaction.

Cette section dresse un état de l'art systématique du domaine. Nous proposons d'abord une cartographie des agents pédagogiques à travers leur évolution historique et leurs caractéristiques de design (\S\ref{subsec:cartographie_agents}). Nous analysons ensuite les fondements cognitifs qui contraignent leur efficacité (\S\ref{subsec:fondements_cognitifs}), puis les mécanismes par lesquels ils génèrent une présence sociale (\S\ref{subsec:presence_sociale}). Une taxonomie des signaux sociaux et de leur efficacité empirique est ensuite présentée (\S\ref{subsec:taxonomie_signaux}). Nous examinons enfin les limites du réalisme (\S\ref{subsec:limites_realisme}).

% ----------------------------------------------------------------------------
% Sous-sections : v3 pour 2_3_1, 2_3_2, 2_3_4 ; originaux pour 2_3_3, 2_3_5
% ----------------------------------------------------------------------------
\input{chapters/Ch2_EtatDeLArt/sections/2_3/2_3_1_cartographie_agents_v3.tex}
% ============================================================================
% Sous-section 2.3.2 : Fondements Cognitifs de l'Apprentissage avec Agents (V3)
% ============================================================================
% Corrections appliquées :
% - Lacune 19 : Ajout transition explicite vers ICAP (cf. section 2.1.4)
% ============================================================================

\subsection{Fondements Cognitifs~: Contraintes et Principes de Design}
\label{subsec:fondements_cognitifs}

L'efficacité des agents pédagogiques est contrainte par l'architecture cognitive humaine. Deux cadres théoriques informent directement leur conception~: la Théorie de la Charge Cognitive et la Théorie Cognitive de l'Apprentissage Multimédia. Ces cadres complètent la hiérarchie d'engagement du cadre ICAP présenté en section~\ref{subsec:ICAP}~: si ICAP décrit \textit{ce que fait} l'apprenant, les théories qui suivent expliquent \textit{pourquoi} certaines activités sont plus efficaces que d'autres.

La mémoire de travail constitue le goulot d'étranglement de l'apprentissage~\citep{sweller2011cognitive}. Sa capacité limitée impose de distinguer trois types de charge~: la charge \textit{intrinsèque} (complexité du contenu), la charge \textit{extrinsèque} (inefficacités de présentation), et la charge \textit{pertinente} (effort d'apprentissage productif). Pour les agents pédagogiques, cette distinction a une implication directe~: les éléments de design non fonctionnels peuvent constituer une charge extrinsèque susceptible de compromettre l'apprentissage.

Un agent visuellement complexe --- animations élaborées, environnement 3D détaillé, expressions faciales sophistiquées --- peut paradoxalement nuire à l'apprentissage s'il détourne des ressources cognitives du contenu éducatif. Les méta-analyses confirment ce risque~: les agents 2D surpassent les agents 3D sur les mesures d'apprentissage~\citep{castroalonso2021effectiveness}. Ce résultat contre-intuitif s'explique par le coût cognitif du réalisme~: le traitement d'un environnement 3D immersif consomme des ressources au détriment du contenu.

La Théorie Cognitive de l'Apprentissage Multimédia~\citep{mayer2014cambridge} postule l'existence de deux canaux de traitement distincts~: visuel-pictural et auditif-verbal. Le \textit{principe de modalité} qui en découle stipule que l'apprentissage est favorisé lorsque les explications verbales sont présentées sous forme audio plutôt que textuelle. En déléguant l'information verbale au canal auditif, le concepteur libère le canal visuel pour les éléments graphiques pertinents.

Ce principe fournit la justification théorique des agents vocaux~: un agent qui \textit{parle} plutôt qu'il n'affiche du texte permet une répartition optimale de la charge entre les canaux. Les voix humaines s'avèrent plus efficaces que les voix synthétiques, bien que cet écart se réduise avec les progrès technologiques. Le \textit{principe de personnalisation} complète cette recommandation~: un style conversationnel surpasse un style formel.

Les gestes de l'agent constituent un cas particulier~: ils peuvent soit faciliter l'apprentissage en guidant l'attention, soit le compromettre en ajoutant une charge extrinsèque. Les données empiriques révèlent des effets modérés des gestes sur le transfert proche et la rétention~\citep{davis2018impact}. Ces effets sont conditionnés par la \textit{congruence sémantique}~: les gestes doivent être alignés avec le contenu verbal. Un geste déictique pointant vers un élément pertinent du graphique facilite l'intégration~; un geste générique non relié au contenu constitue une distraction.

Ces résultats convergent vers un paradoxe apparent~: les agents les plus efficaces ne sont pas les plus réalistes, mais les plus parcimonieux. Un agent doit fournir suffisamment d'indices pour activer l'engagement social (voir section suivante), sans surcharger le système cognitif par des éléments non fonctionnels. Cette recommandation de parcimonie entre en tension avec l'évolution technologique~: les capacités croissantes de rendu réaliste ne garantissent pas une efficacité pédagogique accrue. La question n'est pas \og que peut-on techniquement réaliser?\fg{} mais \og qu'est-ce qui sert effectivement l'apprentissage?\fg{}.

% ============================================================================
% Sous-section 2.3.3 : De l'Artefact au Partenaire — Présence Sociale
% ============================================================================
% Sources : IJCCI_extraction (§2.5), illusion_extraction, Protocole CER
% Calibrage : ~600 mots
% Type : Révision fusionnant 2_3_2_CASA.tex et 2_3_3_agence_sociale.tex
% ============================================================================

\subsection{De l'Artefact au Partenaire~: Mécanismes de la Présence Sociale}
\label{subsec:presence_sociale}

La section précédente a établi les contraintes cognitives qui encadrent l'efficacité des agents. Mais l'impact d'un agent ne se réduit pas à sa fonction de canal d'information~: il dépend aussi de sa capacité à être perçu comme un partenaire social. Cette perception transforme l'interaction technique en expérience relationnelle.

Le paradigme CASA (\textit{Computers Are Social Actors}) constitue le cadre explicatif central de ce phénomène~\citep{nass2000machines}. Les travaux fondateurs de Nass et ses collaborateurs ont démontré que les utilisateurs appliquent inconsciemment aux ordinateurs les règles sociales qu'ils utiliseraient avec des humains~: politesse, réciprocité, attribution de personnalité. Ce traitement social tend à s'opérer même lorsque l'utilisateur \textit{sait} qu'il interagit avec une machine --- un processus qualifié d'\textit{ethopoeia}.

Des indices sociaux minimaux suffisent à déclencher ces scripts relationnels. Une voix, qu'elle soit synthétique ou humaine, active des réponses sociales. Les utilisateurs évaluent différemment un ordinateur selon le genre de sa voix, reproduisant les stéréotypes sociaux~: une voix masculine est perçue comme plus compétente sur les sujets techniques. Ces attributions automatiques expliquent pourquoi même des agents rudimentaires peuvent générer un engagement significatif.

La présence sociale désigne le sentiment d'être \og avec\fg{} une autre intelligence dans un environnement médiatisé~\citep{biocca2003toward}. Ce sentiment ne requiert pas un interlocuteur humain~; il peut émerger de l'interaction avec un agent virtuel pourvu que certaines conditions soient réunies. L'utilisateur doit percevoir l'agent non comme un outil passif, mais comme une entité dotée d'une forme d'intentionnalité et de réactivité.

Cette présence sociale constitue la condition préalable à l'activation des mécanismes d'apprentissage social. Sans elle, l'interaction tend à rester instrumentale~: l'utilisateur traite l'information sans s'engager pleinement dans la relation. Avec elle, l'agent devient un partenaire dont on respecte implicitement le \og contrat de communication\fg{}.

La théorie de l'agence sociale (\textit{Social Agency Theory}) explicite le lien entre présence sociale et apprentissage~\citep{moreno2001case, mayer2012embodiment}. Lorsque l'apprenant perçoit l'agent comme un partenaire social, un contrat implicite s'établit~: l'apprenant s'engage à \og honorer\fg{} cette relation par un effort cognitif accru. Cet effort supplémentaire se traduit par un traitement plus profond du contenu.

Les méta-analyses confirment ce mécanisme~: les agents anthropomorphisés produisent des effets supérieurs aux agents non anthropomorphisés, mais ces effets concernent davantage les mesures affectives et motivationnelles que les mesures cognitives pures~\citep{schroeder2025designing, dai2022meta}.


% ============================================================================
% Sous-section 2.3.4 : Taxonomie et Efficacité des Signaux Sociaux (V3)
% ============================================================================
% Corrections appliquées :
% - Lacune 21 : Regroupement des citations davis2018impact (3x -> 1x)
% ============================================================================

\subsection{Taxonomie et Efficacité des Signaux Sociaux}
\label{subsec:taxonomie_signaux}

Les agents pédagogiques mobilisent une variété de signaux sociaux pour établir la présence et faciliter l'apprentissage. Cette section propose une taxonomie systématique de ces signaux, fondée sur les résultats empiriques des méta-analyses du domaine. Le tableau~\ref{tab:signaux_sociaux} synthétise les principaux effets documentés.

\begin{table}[htbp]
\centering
\caption{Efficacité des signaux sociaux dans les agents pédagogiques. Synthèse des résultats issus des méta-analyses.}
\label{tab:signaux_sociaux}
\small
\begin{tabular}{p{2.5cm}p{3.5cm}p{3cm}p{4cm}}
\toprule
\textbf{Catégorie} & \textbf{Signal} & \textbf{Effet} & \textbf{Source} \\
\midrule
\multirow{3}{*}{\textit{Représentation}}
& Présence vs absence & Effet positif & \citet{castroalonso2021effectiveness} \\
& 2D vs 3D & 2D > 3D & \citet{castroalonso2021effectiveness} \\
& Animé vs statique & Animé > Statique & \citet{dai2024effects} \\
\addlinespace
\multirow{2}{*}{\textit{Voix}}
& Voix humaine vs synthétique & Humaine > Synthétique & \citet{davis2018impact} \\
& Style conversationnel & Effet positif modéré & \citet{wang2023effects} \\
\addlinespace
\multirow{2}{*}{\textit{Gestes}}
& Gestes déictiques & Effet positif (transfert) & \citet{davis2018impact} \\
& Gestes génériques & Effet modéré (rétention) & \citet{davis2018impact} \\
\addlinespace
\multirow{2}{*}{\textit{Affect}}
& Expression émotionnelle & Effet positif modéré & \citet{wang2023effects} \\
& Affect positif & Corrélation positive & \citet{guo2015affect} \\
\addlinespace
\multirow{2}{*}{\textit{Interaction}}
& Chatbots IA & Effet positif important & \citet{wu2024chatbots} \\
& Feedback adaptatif & Effet positif & \citet{martha2018design} \\
\bottomrule
\end{tabular}
\end{table}

La \textbf{représentation visuelle} constitue le premier niveau de signaux. La présence d'un agent produit un effet global positif, mais ce résultat masque des variations importantes selon le type de représentation~\citep{castroalonso2021effectiveness}. Les agents 2D surpassent les agents 3D, un écart qui s'explique par le coût cognitif du réalisme~: le traitement d'environnements 3D immersifs consomme des ressources au détriment du contenu pédagogique. L'animation constitue néanmoins un atout~: les agents animés surpassent les agents statiques, l'animation permettant de guider l'attention et de maintenir l'engagement~\citep{dai2024effects}.

La \textbf{voix} représente un signal social particulièrement puissant. Les voix humaines conservent un avantage sur les voix synthétiques, bien que cet écart se réduise avec les progrès technologiques~: les voix synthétiques de haute qualité produisent désormais des résultats d'apprentissage équivalents, voire supérieurs, pour le transfert de connaissances~\citep{craig2017schroeder, davis2018impact}. La qualité prosodique s'avère plus déterminante que l'incarnation visuelle pour la perception de naturalité~: un agent réaliste avec une prosodie inadéquate sera jugé moins naturel qu'un agent désincarné avec une prosodie correcte~\citep{ehret2021}. Au-delà de la qualité vocale, le style discursif influence l'apprentissage~: le style conversationnel produit un effet positif par rapport au style formel~\citep{wang2023effects}. Ce résultat s'aligne avec le principe de personnalisation de la théorie cognitive de l'apprentissage multimédia~: un discours adressé directement à l'apprenant (``vous'') favorise l'engagement par rapport à un discours impersonnel.

Les \textbf{gestes} de l'agent constituent une catégorie de signaux aux effets différenciés. Une méta-analyse distingue deux types de gestes~\citep{davis2018impact}~: les gestes déictiques --- pointage vers des éléments pertinents de l'interface --- produisent un effet important sur le transfert en guidant explicitement l'attention visuelle et en facilitant l'intégration des informations verbales et graphiques~; les gestes génériques --- mouvements non liés au contenu --- produisent un effet plus modeste sur la rétention. Cette différence confirme l'importance de la congruence sémantique~: les gestes doivent être alignés avec le contenu pour maximiser leur efficacité.

La combinaison de plusieurs types de communication non verbale n'amplifie pas nécessairement les effets~: un seul type de signal approprié au résultat d'apprentissage visé (gestes \textit{ou} expressions faciales) surpasse la combinaison des deux~\citep{baylor2009design}. Les expressions faciales favorisent l'apprentissage attitudinal, tandis que les gestes facilitent l'apprentissage procédural. Ce résultat contre-intuitif s'explique par la théorie de la charge cognitive~: la multiplication des signaux peut surcharger l'apprenant plutôt que l'assister. L'inconsistance des résultats dans la littérature confirme que l'efficacité des signaux non verbaux dépend davantage de leur appropriation au contexte que de leur quantité~\citep{wang2021examining}.

L'\textbf{affect} de l'agent influence l'apprentissage à travers plusieurs mécanismes. Une corrélation positive existe entre affect de l'agent et apprentissage~\citep{guo2015affect}. L'expression émotionnelle se distingue de la simple présence affective~\citep{wang2023effects}. Les expressions faciales doivent être alignées avec le contexte pour produire leurs effets~\citep{liew2016}~; lorsqu'elles sont coordonnées avec le discours, elles influencent positivement la motivation~\citep{liew2017}. Toutefois, l'expressivité émotionnelle positive peut s'avérer contre-productive~: elle peut sembler inappropriée et nuire à la confiance lors de tâches critiques où la neutralité est attendue~\citep{torre2019}. Les effets affectifs sont également plus marqués sur les mesures motivationnelles que sur les mesures cognitives pures~\citep{tao2022exploring}.

La dimension \textbf{interactive} distingue les agents conversationnels des agents unidirectionnels. Les chatbots IA en éducation produisent des effets positifs importants~\citep{wu2024chatbots}, suggérant que la capacité de dialogue --- recevoir les questions de l'apprenant et y répondre de manière contextuelle --- constitue un amplificateur majeur de l'efficacité pédagogique.

La \textbf{motivation} constitue une variable médiatrice importante. Les effets des chatbots éducatifs varient selon le type de motivation mesurée~: l'auto-efficacité montre des effets importants, tandis que la motivation intrinsèque présente des effets plus modérés~\citep{gladstone2025motivation}. Ces résultats suggèrent que les agents agissent principalement en renforçant la confiance de l'apprenant dans ses capacités plutôt qu'en modifiant son intérêt intrinsèque pour le contenu.

L'\textbf{alignement thématique} entre l'agent et le contenu constitue un facteur modulateur. Les apprenants interagissant avec un agent thématiquement aligné (un astronaute enseignant des concepts spatiaux) rapportent des niveaux supérieurs de présence sociale et d'engagement comparés à ceux interagissant avec un agent neutre~\citep{schmidt2019}. Ce résultat suggère que la cohérence entre l'identité de l'agent et le domaine enseigné peut amplifier les effets des signaux sociaux.

Une analyse transversale révèle un patron cohérent~: les effets des signaux sociaux sont systématiquement plus importants sur les mesures affectives et motivationnelles que sur les mesures cognitives. Sur 15 études avec groupe contrôle, 9 ne montrent aucune différence d'apprentissage significative, alors que les effets sur l'engagement sont généralement positifs~\citep{heidig2011pedagogical}. Une dissociation frappante émerge entre perception et performance~: les apprenants apprécient davantage la leçon, la jugent plus intéressante et ont l'impression de mieux apprendre, mais leurs résultats de compréhension objective diminuent --- un phénomène résumé par l'expression ``\textit{more gaze, but less learning}''~\citep{wilson2018}.

La figure~\ref{fig:signaux_efficacite} visualise ces relations entre signaux sociaux et types d'effets.

\begin{figure}[htbp]
\centering
\fbox{\parbox{0.9\textwidth}{\centering\vspace{2cm}
\textit{Figure à insérer~: Modèle des effets des signaux sociaux}\\[0.5em]
Schéma montrant~:\\
(1) Les catégories de signaux (représentation, voix, gestes, affect, interaction)\\
(2) Les mécanismes médiateurs (présence sociale, engagement, charge cognitive)\\
(3) Les types d'effets (affectif, motivationnel, cognitif)\\
Avec indication des tailles d'effet pour chaque relation
\vspace{2cm}}}
\caption{Modèle intégratif des effets des signaux sociaux des agents pédagogiques. Les signaux influencent l'apprentissage à travers des mécanismes médiateurs, avec des effets différenciés selon le type de mesure.}
\label{fig:signaux_efficacite}
\end{figure}

Ces résultats convergent vers une recommandation de parcimonie stratégique~: multiplier les signaux sociaux n'amplifie pas nécessairement l'efficacité pédagogique. Chaque signal doit être évalué selon son rapport bénéfice (engagement, guidage attentionnel) / coût (charge cognitive, distraction). Cette tension entre richesse des signaux et surcharge potentielle conduit naturellement à examiner les limites du réalisme dans la conception des agents.

% ============================================================================
% Sous-section 2.3.5 : Limites du Réalisme dans les Agents Pédagogiques
% ============================================================================
% Sources : Rapport Analyse Research Paper Mapper
% Calibrage : ~500 mots
% Type : Révision de 2_3_4_uncanny_valley.tex
% ============================================================================

\subsection{Limites du Réalisme~: Le Paradoxe de l'Hyperréalisme}
\label{subsec:limites_realisme}

Les résultats empiriques présentés dans les sections précédentes convergent vers un constat contre-intuitif~: le réalisme accru des agents ne garantit pas une efficacité pédagogique supérieure. Cette section examine les mécanismes qui expliquent ce paradoxe et ses implications pour le design.

Les comparaisons directes entre agents réalistes et agents stylisés révèlent une absence de différence significative sur les mesures d'apprentissage~\citep{shiban2015appearance}. Les agents stylisés génèrent parfois un engagement supérieur, possiblement parce qu'ils évitent les attentes implicites associées à l'apparence humaine. Lorsqu'un agent ressemble étroitement à un humain, l'apprenant s'attend à des comportements pleinement humains~; les déviations --- latence de réponse, expressions faciales limitées, erreurs de compréhension --- peuvent générer une dissonance qui compromet l'interaction.

L'écart entre agents 2D et agents 3D illustre ce phénomène~\citep{castroalonso2021effectiveness}. Cette différence ne s'explique pas uniquement par la charge cognitive du réalisme visuel~: elle reflète également une inadéquation entre les capacités comportementales des agents et les attentes générées par leur apparence. Un agent 3D photoréaliste qui ne maintient pas un contact visuel approprié ou dont les expressions faciales restent rigides crée une impression d'étrangeté absente chez un agent 2D dont les limitations sont explicitement acceptées.

Ce décalage constitue un défi persistant du domaine~\citep{johnson2016face}. Les progrès technologiques permettent désormais des rendus visuels sophistiqués, mais les comportements interactifs --- synchronisation multimodale, gestion des tours de parole, réactivité émotionnelle --- n'ont pas progressé au même rythme. Cette asymétrie crée une vallée de l'étrange (\textit{uncanny valley}) où l'agent est suffisamment réaliste pour activer les attentes sociales, mais insuffisamment capable pour les satisfaire.

Les agents hyperréalistes peuvent également réduire l'engagement par un mécanisme psychologique distinct~\citep{dai2022systematic}~: face à un agent perçu comme ``intelligent'' et ``compétent'', certains apprenants développent une anxiété de performance qui inhibe leur participation. Les agents stylisés, perçus comme moins menaçants, favoriseraient une interaction plus détendue et un engagement plus authentique.

Ces résultats ont des implications directes pour le design des agents historiques. Un personnage historique rendu de manière hyperréaliste risque de générer des attentes impossibles à satisfaire~: l'apprenant s'attend à une interaction ``comme avec un vrai humain'', alors que l'agent reste limité par ses capacités techniques et la fiabilité de ses connaissances. Une représentation stylisée, explicitement non réaliste, pourrait paradoxalement favoriser une interaction plus productive en établissant d'emblée les limites de l'échange.

Cette analyse conduit à reformuler la question du design~: plutôt que de maximiser le réalisme, il s'agit d'optimiser la congruence entre l'apparence de l'agent, ses capacités comportementales et les attentes de l'apprenant. Ces principes, établis pour les agents traditionnels à scripts prédéfinis, doivent être réexaminés à la lumière de la rupture technologique introduite par les IA génératives, dont les capacités conversationnelles modifient substantiellement les modalités d'interaction.




% ============================================================================
% Section 2.4 : La Rupture Technologique des IA Génératives en Éducation (V3)
% ============================================================================
% ============================================================================
% Section 2.4 : La Rupture Technologique des IA Génératives en Éducation (V3)
% ============================================================================
% Corrections appliquées :
% - 2_4_3 : Simplification fluidité, ref forward explicite
% ============================================================================

\section{La Rupture Technologique des IA Génératives en Éducation}
\label{sec:IA_generatives}

Les cadres théoriques présentés jusqu'ici ont été élaborés dans un contexte technologique où les agents pédagogiques fonctionnaient sur la base de scripts prédéfinis. L'émergence des Grands Modèles de Langage (LLM) et des technologies de synthèse multimédia a transformé ce paysage, créant de nouvelles possibilités mais aussi de nouveaux risques.

Cette section examine le saut qualitatif représenté par les agents génératifs (\S\ref{subsec:agents_generatifs}), leurs capacités de personnalisation en temps réel (\S\ref{subsec:personnalisation_temps_reel}), et le défi épistémique posé par leur propension aux hallucinations (\S\ref{subsec:hallucinations}).

% Inclusion des sous-sections : originaux pour 2_4_1 et 2_4_2, v3 pour 2_4_3
% ============================================================================
% Sous-section 2.4.1 : Des Agents Scriptés aux Agents Génératifs
% ============================================================================
% Sources : IJCCI_extraction (§2.4), illusion_extraction (§2.2.3)
% Calibrage : ~500 mots
% Type : C (Rédaction originale)
% ============================================================================

\subsection{Des Agents Scriptés aux Agents Génératifs~: le Saut Qualitatif}
\label{subsec:agents_generatifs}

Les agents pédagogiques ont évolué au cours des dernières décennies, passant de simples outils de diffusion d'information à des partenaires d'apprentissage plus sophistiqués~\citep{johnson2016face}. L'émergence des Grands Modèles de Langage (\textit{Large Language Models}, LLM) a particulièrement accéléré cette évolution.

Les agents pédagogiques traditionnels fonctionnaient sur la base de scripts prédéfinis et d'arbres de décision. Leur comportement était entièrement déterminé par les anticipations de leurs concepteurs~: chaque question possible devait être prévue, chaque réponse pré-rédigée. Cette architecture présentait des avantages --- prévisibilité, contrôle du contenu, absence d'erreurs factuelles --- mais aussi des limites fondamentales. La rigidité constituait le principal écueil~: l'agent ne pouvait répondre qu'aux questions anticipées, dans les formulations anticipées. Toute déviation se heurtait à des réponses génériques. Cette rigidité contrastait avec la fluidité du dialogue humain et limitait le sentiment de présence sociale.

Les LLM ont transformé cette équation. Ces modèles peuvent générer des réponses personnalisées, s'adapter aux besoins des élèves en temps réel, et produire du contenu éducatif contextuellement pertinent~\citep{kasneci2023chatgpt, labadze2023role}. Ces avancées s'inscrivent dans la continuité des travaux sur les environnements d'apprentissage multimodaux interactifs~\citep{moreno2007interactive}. La fluidité conversationnelle des LLM constitue leur caractéristique la plus distinctive~: ils génèrent un discours structuré, linguistiquement cohérent, et adapté au contexte de l'échange. Cette fluidité peut activer les mécanismes d'agence sociale décrits en \S\ref{subsec:agence_sociale}~: l'apprenant perçoit l'agent comme un interlocuteur plutôt qu'un système automatique.

Cette évolution s'accompagne d'avancées parallèles en synthèse multimédia. Les réseaux antagonistes génératifs (GAN) permettent de créer des représentations visuelles hyperréalistes, brouillant la frontière entre réel et artificiel~\citep{whittaker2020deepfakes}. L'animation faciale par apprentissage profond, le clonage vocal, et la génération vidéo atteignent des niveaux de réalisme inédits. Ces technologies peuvent atténuer l'effet de vallée de l'étrange (cf. \S\ref{subsec:uncanny_valley}), produisant des agents synthétiques perçus comme attractifs~\citep{xu2025recorded}. Certains travaux indiquent que ces agents peuvent atteindre des niveaux de performance et de perception comparables à ceux d'instructeurs humains~\citep{leiker2023generative, lim2024potential}.

C'est précisément cette convergence --- moteurs conversationnels fluides et interfaces visuelles réalistes --- qui multiplie les enjeux. D'un côté, des \og moteurs\fg{} capables de produire un discours éloquent~; de l'autre, des \og interfaces\fg{} visuelles pour les incarner. Cette combinaison ouvre des possibilités pédagogiques inédites, mais comporte également des risques que les sections suivantes examineront.

% ============================================================================
% Sous-section 2.4.2 : La Personnalisation en Temps Réel
% ============================================================================
% Sources : IJCCI_extraction (§2.4), Leong et al., Pataranutaporn et al.
% Calibrage : ~450 mots
% Type : C (Rédaction originale)
% ============================================================================

\subsection{La Personnalisation en Temps Réel}
\label{subsec:personnalisation_temps_reel}

L'une des capacités distinctives des agents alimentés par LLM réside dans leur aptitude à personnaliser le contenu d'apprentissage en temps réel, sans nécessiter de programmation préalable pour chaque profil d'apprenant. Cette capacité répond aux défis de mise en œuvre identifiés en \S\ref{subsec:personnalisation}.

L'apprentissage adaptatif traditionnel reposait sur des algorithmes prédéfinis~: en fonction des réponses de l'élève à des questions diagnostiques, le système orientait vers des parcours prédéterminés~\citep{walkington2018personalization}. Les LLM permettent une forme d'adaptation plus fine~: l'agent peut moduler son vocabulaire, ses exemples, son niveau de détail en fonction du flux de la conversation elle-même.

L'étude de la personnalisation dans l'apprentissage du vocabulaire illustre ce potentiel~: un système développant des exemples et des récits adaptés aux intérêts individuels conduit à une augmentation de la motivation intrinsèque et à un sentiment renforcé de compétence et d'autonomie~\citep{leong2024putting}. Ces résultats démontrent la faisabilité d'une personnalisation automatique à grande échelle. L'exploration des interactions conversationnelles avec des figures historiques à travers des interfaces textuelles indique une amélioration de la motivation et des résultats d'apprentissage comparés à la lecture traditionnelle~\citep{pataranutaporn2023living}. L'évaluation de l'interactivité textuelle en éducation financière confirme que permettre aux étudiants de dialoguer avec l'instructeur virtuel conduit à une motivation et un engagement accrus comparés à l'instruction vidéo passive~\citep{prasongpongchai2024influence}.

Ce qui distingue la personnalisation par LLM est son caractère émergent. L'adaptation n'est pas programmée explicitement~: elle émerge de la capacité du modèle à générer des réponses contextuellement appropriées. L'élève pose une question selon ses propres termes, l'agent répond en s'adaptant. Si l'élève manifeste une incompréhension, l'agent peut reformuler spontanément. Cette fluidité adaptative présente un avantage pédagogique~: chaque interaction devient unique, calibrée sur les besoins du moment.

Elle présente également un risque~: l'adaptation peut masquer l'absence de compréhension réelle. L'agent qui reformule efficacement peut donner l'impression à l'élève qu'il a compris, alors que c'est l'agent qui a simplifié son discours au point de ne plus transmettre le concept dans sa complexité. Ce mécanisme constitue l'une des sources potentielles de l'illusion de compréhension.

L'examen des agents conversationnels conçus pour favoriser la curiosité chez les enfants du primaire révèle des résultats prometteurs~: un agent encourageant le questionnement divergent conduit à une amélioration de la qualité des questions et à des activités exploratoires soutenues~\citep{abdelghani2024exploring}. Ces résultats suggèrent que la personnalisation peut être mise au service de l'engagement cognitif authentique plutôt que de la simple facilitation.

% ============================================================================
% Sous-section 2.4.3 : Fiabilité et Hallucinations — Le Défi Épistémique (V3)
% ============================================================================
% Corrections appliquées :
% - Lacune 7 : Simplification du passage sur fluidité (détail en 2.5.2)
% - Lacune 17 : Ref forward déjà explicite - conservée
% ============================================================================

\subsection{Fiabilité et Hallucinations~: le Défi Épistémique}
\label{subsec:hallucinations}

La fluidité conversationnelle des LLM masque une faille intrinsèque~: leur propension à générer des informations factuellement incorrectes présentées avec une confiance apparente. Ce phénomène, qualifié d'\og hallucination\fg{}, représente un défi épistémique majeur pour les applications éducatives.

Les LLM sont des modèles probabilistes qui prédisent le mot suivant le plus probable étant donné le contexte. Cette nature stochastique implique qu'ils ne \og connaissent\fg{} pas les faits au sens humain~: ils génèrent des séquences statistiquement plausibles~\citep{zhang2025sirens}. Une taxonomie des hallucinations distingue les \textit{hallucinations factuelles} (assertions fausses sur le monde), les \textit{hallucinations de fidélité} (déformations de l'information source), et les \textit{hallucinations d'entrée} (fabrication d'éléments non présents dans la requête). Dans un contexte éducatif, chaque type présente des risques spécifiques~: date erronée, citation inventée, personnage historique attribué à la mauvaise période.

L'enseignement de l'Histoire présente une vulnérabilité particulière à ces hallucinations. La discipline repose sur des faits précis --- dates, lieux, protagonistes --- dont l'exactitude est vérifiable. Or, les LLM excellent dans la production de récits plausibles et cohérents~; ils peuvent générer une narration parfaitement fluide qui contient néanmoins des erreurs factuelles. Cette tension est problématique car la fluidité du discours constitue un signal de crédibilité --- mécanisme que nous analysons en détail dans la section suivante (cf. \S\ref{subsec:fluency_heuristic}). Le LLM, en produisant un discours maximalement fluide, maximise cette heuristique --- indépendamment de la véracité de ce qu'il affirme.

Les avancées technologiques soulèvent plusieurs défis connexes. L'engagement des élèves peut fluctuer en raison d'effets de nouveauté~\citep{fryer2019bots}, tandis que les questions de fiabilité, de biais, de confidentialité et d'intégrité académique nécessitent une attention particulière~\citep{labadze2023role, dempere2023impact, berson2024childrens}. Au-delà de l'établissement de lignes directrices pour l'intégration de l'IA en éducation, des opportunités de recherche existent pour examiner comment l'interaction verbale avec des agents pourrait compléter les pratiques pédagogiques actuelles. La modalité orale, en tirant parti des dynamiques naturelles de la classe, pourrait créer des patterns d'engagement différents comparés aux interactions textuelles individuelles~\citep{moreno2007interactive}.

Ces considérations informent directement notre programme expérimental. L'Étude~2, en exposant les participants à un agent capable de produire des informations historiques incorrectes mais présentées avec fluidité, teste précisément ce risque~: la fluidité de l'agent conduit-elle les élèves à accepter des informations fausses et à surestimer leur propre compréhension?



% ============================================================================
% Section 2.5 : Le Phénomène de l'Illusion de Compréhension (original - pas de lacunes)
% ============================================================================
% ============================================================================
% Section 2.5 : Le Phénomène de l'Illusion de Compréhension (IoU)
% ============================================================================
% Objectif : Définir le risque majeur exploré par la thèse,
%            revers de la médaille de la fluidité
% Sources : illusion_extraction (§2.2.1, §2.2.2), Vault.xlsx
% Calibrage : ~2 000 mots
% ============================================================================

\section{Le Phénomène de l'Illusion de Compréhension}
\label{sec:illusion_comprehension}

La fluidité des agents conversationnels génératifs constitue leur force pédagogique principale --- mais aussi leur risque le plus insidieux. Lorsqu'un agent présente l'information de manière parfaitement articulée et convaincante, l'apprenant peut confondre la facilité de réception avec la qualité de sa propre compréhension. Ce phénomène, que nous désignons par \textit{illusion de compréhension} (ou \textit{Illusion of Understanding}, IoU), représente le risque métacognitif central examiné par cette thèse. Cette section analyse d'abord les fondements cognitifs de ce phénomène à travers l'Illusion de Profondeur Explicative (\S\ref{subsec:metacognition_IOED}), puis examine le mécanisme de l'heuristique de fluidité (\S\ref{subsec:fluency_heuristic}), avant d'explorer comment l'autorité perçue de l'agent peut compromettre la vigilance épistémique (\S\ref{subsec:autorite_vigilance}).

% ----------------------------------------------------------------------------
% Sous-sections
% ----------------------------------------------------------------------------
% ============================================================================
% Sous-section 2.5.1 : Métacognition et Calibration de la Confiance
% ============================================================================
% Sources : illusion_extraction (§2.2.1), illusion_ai_agent_draft
% Calibrage : ~550 mots
% Type : C (Rédaction originale)
% ============================================================================

\subsection{Métacognition et Calibration de la Confiance}
\label{subsec:metacognition_IOED}

La métacognition désigne la cognition sur la cognition~: la capacité d'un individu à surveiller et réguler ses propres processus cognitifs~\citep{flavell1979metacognition}. Dans le contexte de l'apprentissage, cette capacité se manifeste par deux fonctions distinctes~: le \textit{monitoring}, qui consiste à évaluer sa propre compréhension, et le \textit{control}, qui permet d'ajuster ses stratégies en conséquence. L'auto-évaluation de la compréhension constitue un processus intrinsèquement faillible~\citep{glenberg1982automatic}~: les apprenants croient fréquemment avoir compris un contenu alors même qu'ils échouent à détecter des contradictions évidentes.

Ce déficit métacognitif trouve sa formalisation dans le concept d'\textit{Illusion of Explanatory Depth} (IOED). Ce phénomène désigne la tendance des individus à surestimer la profondeur de leur compréhension des systèmes causaux complexes --- jusqu'au moment où ils sont contraints de produire une explication détaillée~\citep{rozenblit2002misunderstood}. Le protocole expérimental classique procède en trois phases~: une auto-évaluation initiale (T1), la production d'une explication écrite, puis une seconde auto-évaluation (T2). La révélation est systématique~: confrontés à l'exercice d'explication, les participants découvrent que leur compréhension était moins profonde qu'ils ne le croyaient, se traduisant par une chute significative entre T1 et T2.

Ce phénomène s'avère pertinent pour l'étude des interactions avec les agents conversationnels. Certaines modalités de présentation peuvent donner l'impression que le matériel est facile à traiter~\citep{paik2013effects}. Cette facilité apparente conduit les apprenants à sous-estimer la difficulté de la tâche et à développer une métacompréhension excessivement optimiste. Le résultat est une dissociation entre la confiance subjective, qui augmente, et l'apprentissage objectif, qui stagne.

Cette illusion s'inscrit dans un déficit métacognitif plus large. Le phénomène décrit par l'effet Dunning-Kruger met en lumière un \og double fardeau\fg{}~: les individus les moins compétents manquent également des compétences métacognitives nécessaires pour reconnaître leur propre incompétence~\citep{kruger1999unskilled}. Dans le contexte d'une interaction avec un agent conversationnel fluide, ce déficit devient particulièrement problématique~: la facilité apparente de l'échange peut renforcer une confiance injustifiée.

La notion de \textit{calibration} désigne l'écart entre la confiance subjective et la performance objective~\citep{koriat2008subjective}. Les études révèlent une tendance systématique à la \textit{surconfiance}~: les apprenants surestiment leur niveau de compréhension. Cette surconfiance constitue un prédicteur robuste de mauvais apprentissage, car elle conduit à un désengagement prématuré de l'effort cognitif.

Dans le contexte des agents alimentés par IA générative, ce problème prend une dimension nouvelle. Un \og paradoxe métacognitif\fg{} émerge~: bien que l'assistance de l'IA puisse améliorer la performance immédiate, elle dégrade la capacité de l'utilisateur à évaluer cette même performance~\citep{fernandes2026metacognitive}. L'agent, en fournissant des réponses instantanées et fluides, prive l'apprenant des \og difficultés désirables\fg{} --- l'effort de construction et de réorganisation des connaissances --- pourtant essentielles à un apprentissage durable~\citep{bjork2013self}.

% ============================================================================
% Sous-section 2.5.2 : L'Heuristique de Fluidité (Fluency Heuristic)
% ============================================================================
% Sources : illusion_extraction (§2.2.1, §2.2.2), illusion_ai_agent_draft
% Calibrage : ~550 mots
% Type : C (Rédaction originale)
% ============================================================================

\subsection{L'Heuristique de Fluidité}
\label{subsec:fluency_heuristic}

Le mécanisme central de l'illusion de compréhension réside dans ce que la littérature désigne par \textit{processing fluency}~: l'information facile à percevoir et à traiter cognitivement est vécue comme familière~\citep{reber1999effects}. Cette facilité de traitement est alors attribuée par erreur à la propre maîtrise du sujet par l'apprenant plutôt qu'aux qualités de la présentation. L'heuristique de fluidité constitue un raccourci cognitif par lequel nous jugeons plus vraie, plus crédible et mieux comprise l'information qui \og passe bien\fg{}.

La simple facilité de perception d'un énoncé --- police lisible, articulation claire, formulation syntaxiquement simple --- augmente sa crédibilité perçue, indépendamment de sa véracité objective. L'individu interprète la facilité cognitive comme un signal de familiarité, et la familiarité comme un indice de vérité. Dans le contexte des agents conversationnels alimentés par LLM, cette heuristique prend une dimension particulière~: ces modèles possèdent des caractéristiques techniques qui maximisent la fluidité de traitement --- discours instantané, parfaitement structuré et linguistiquement fluide~\citep{shanahan2024talking}. Cette double fluidité --- linguistique et auditive lorsque couplée à une synthèse vocale --- crée les conditions idéales pour que l'heuristique opère.

Une variante pertinente est l'\textit{instructor fluency effect}, qui décrit comment les comportements non-verbaux de l'enseignant --- dynamisme, contact visuel, fluidité verbale --- peuvent biaiser les jugements d'apprentissage~\citep{toftness2018instructor}. Les apprenants confondent la qualité de la délivrance pédagogique avec la qualité de leur propre apprentissage, rapportant une confiance élevée sans gains de performance correspondants. La présence visuelle d'un instructeur peut augmenter la satisfaction et l'apprentissage perçu tout en détournant l'attention du contenu~\citep{wilson2018instructor}. L'apprenant se trouve dans une situation paradoxale où son expérience subjective positive masque un apprentissage objectivement dégradé.

Au-delà de la compréhension conceptuelle, l'heuristique de fluidité peut générer une \og illusion d'acquisition de compétence\fg{}~: l'observation passive d'une démonstration peut conduire l'observateur à confondre la fluidité de traitement visuel avec sa propre capacité à exécuter la tâche~\citep{kardas2018illusion}. Par analogie, un apprenant qui observe un agent expliquer un phénomène avec aisance peut confondre la clarté de la présentation avec sa propre maîtrise du sujet.

La convergence de ces mécanismes crée un risque métacognitif majeur. L'interaction conversationnelle tend à augmenter la crédibilité perçue et à réduire la détection des inexactitudes par rapport au texte statique, car elle active des heuristiques sociales qui diminuent la vigilance critique~\citep{anderl2024conversational}. Même lorsque le contenu n'est pas jugé globalement plus crédible, il est souvent perçu comme plus clair et plus engageant --- une qualité qui peut conduire à une acceptation non critique~\citep{huschens2023unambiguous}. Cette configuration favorise une forme de \og paresse métacognitive\fg{}, où les apprenants délèguent les processus cognitifs coûteux et renoncent à l'effort de construction personnelle des connaissances~\citep{fan2023metacognitive}.

% ============================================================================
% Sous-section 2.5.3 : L'Autorité de l'Agent et la Vigilance Épistémique
% ============================================================================
% Sources : illusion_extraction (§2.2.1, §2.2.2), illusion_ai_agent_draft
% Calibrage : ~550 mots
% Type : C (Rédaction originale)
% ============================================================================

\subsection{L'Autorité de l'Agent et la Vigilance Épistémique}
\label{subsec:autorite_vigilance}

Au-delà de l'illusion de compréhension --- qui constitue une erreur d'auto-évaluation ---, les agents conversationnels génèrent un second risque épistémique~: la surconfiance accordée à la source elle-même. Cette distinction conceptuelle mérite une analyse séparée.

La surconfiance épistémique ne désigne pas une erreur d'auto-évaluation, mais une erreur concernant l'information externe~\citep{kulgemeyer2023epistemic}. Ce biais se manifeste lorsque les apprenants accordent une confiance excessive à des informations factuellement incorrectes parce que la source apparaît autoritaire ou que l'explication est intuitivement séduisante. Les explications fondées sur des conceptions erronées, parce qu'elles s'alignent avec l'expérience quotidienne, sont souvent jugées plus convaincantes que des explications scientifiquement correctes mais contre-intuitives. Cette dynamique prend une dimension critique face aux \og hallucinations\fg{} des LLM --- la génération d'informations incorrectes présentées avec assurance~\citep{zhang2025hallucination}. La fluidité discursive de l'agent crée les conditions d'une acceptation non critique d'informations potentiellement fausses.

L'\textit{illusory truth effect} constitue un mécanisme cognitif qui contribue à cette surconfiance~\citep{fazio2015knowledge}. La simple répétition d'un énoncé, même faux, augmente sa crédibilité perçue en accroissant sa familiarité. Dans le contexte d'interactions répétées avec un agent, ce mécanisme peut consolider des croyances erronées initialement introduites par une hallucination. Il convient toutefois de distinguer cet effet de l'illusion de maîtrise mesurée par le protocole IOED~: l'effet de vérité illusoire constitue une variable explicative de la persistance des fausses croyances plutôt qu'une mesure de l'auto-évaluation. Les deux phénomènes peuvent néanmoins se renforcer~: un apprenant qui croit maîtriser un sujet sera moins enclin à questionner les informations reçues.

L'anthropomorphisme --- notre tendance à attribuer des caractéristiques humaines à des entités non-humaines~\citep{epley2007seeing} --- joue un rôle central dans l'attribution d'autorité aux agents. L'apparence visuelle déclenche ce processus, mais les travaux récents suggèrent que la fluidité conversationnelle pourrait constituer un signal social de premier ordre~\citep{nass1994computers}. Cette dynamique s'avère préoccupante pour les jeunes apprenants, dont la tendance naturelle à l'anthropomorphisme est plus prononcée~\citep{kidd2023anthropomorphism}. Les adolescents, dont les modèles mentaux de la technologie sont en développement, sont plus dépendants des indices de surface pour évaluer la crédibilité~\citep{andries2023trust}.

La convergence de ces mécanismes --- fluidité de traitement, autorité perçue, anthropomorphisme --- tend à désactiver la vigilance épistémique de l'apprenant. Les explications trompeuses générées par l'IA peuvent être plus persuasives que les explications honnêtes~\citep{danry2024don}. Le \og halo de confiance\fg{} observé dans les interactions avec les agents --- où les élèves généralisent la compétence perçue à l'ensemble du domaine --- agit comme un \og bouclier de crédibilité\fg{}. L'apprenant, ayant catégorisé l'agent comme une source fiable, cesse d'appliquer les filtres critiques qu'il mobiliserait face à une source dont l'autorité n'est pas établie. Ce phénomène de \og négligence épistémique\fg{} opère de manière inconsciente, échappant à la régulation métacognitive de l'individu.


% Transition vers Section 2.6
L'analyse de l'illusion de compréhension révèle une tension fondamentale~: les mêmes caractéristiques qui rendent les agents conversationnels engageants --- fluidité, réalisme, crédibilité --- peuvent simultanément compromettre la profondeur de l'apprentissage en désactivant les mécanismes de vigilance critique. Cette tension entre engagement et vigilance constitue le fil conducteur de notre problématique de recherche, qu'il convient maintenant de formuler explicitement.


% ============================================================================
% Section 2.6 : Synthèse et Problématique (V3)
% ============================================================================
% ============================================================================
% Section 2.6 : Synthèse et Problématique (V3)
% ============================================================================
% Corrections appliquées :
% - 2_6_1 : Synthèse par références, apport nouveau explicité
% ============================================================================

\section{Synthèse et Problématique}
\label{sec:synthese_problematique}

Les sections précédentes ont établi le cadre théorique nécessaire à l'analyse de l'interaction entre élèves et agents conversationnels historiques. Il convient maintenant de synthétiser ces apports pour faire émerger les tensions qui structurent notre problématique. Cette section identifie d'abord les convergences et tensions théoriques (\S\ref{subsec:convergences_tensions}), puis analyse les lacunes de la littérature actuelle (\S\ref{subsec:lacunes_litterature}), avant de formuler les questions et hypothèses de recherche qui guident ce travail doctoral (\S\ref{subsec:questions_hypotheses}).

% ----------------------------------------------------------------------------
% Sous-sections : v3 pour 2_6_1, originaux pour 2_6_2 et 2_6_3
% ----------------------------------------------------------------------------
% ============================================================================
% Sous-section 2.6.1 : Convergence et Tensions Théoriques (V3)
% ============================================================================
% Corrections appliquées :
% - Lacune 6 : Synthèse par références croisées (évite répétition ICAP)
% - Lacune 18 : Apport nouveau explicité (hypothèse équilibre symétrique)
% ============================================================================

\subsection{Convergence et Tensions Théoriques}
\label{subsec:convergences_tensions}

L'analyse de la littérature révèle une tension fondamentale~: les mêmes caractéristiques qui rendent les agents conversationnels pédagogiquement prometteurs constituent simultanément leurs principaux risques.

Du côté des convergences positives, les cadres théoriques présentés dans ce chapitre s'articulent en une prédiction cohérente. L'interactivité dialogique active les modes d'engagement les plus profonds selon ICAP (\S\ref{subsec:ICAP}). La présence d'un agent personnifié déclenche un contrat de partenariat qui augmente l'effort cognitif selon la théorie de l'agence sociale (\S\ref{subsec:presence_sociale}). Le paradigme CASA explique pourquoi des indices sociaux minimaux suffisent à activer nos scripts relationnels (\S\ref{subsec:presence_sociale}). La théorie de l'autodétermination identifie l'interactivité comme vecteur de satisfaction du besoin d'autonomie (\S\ref{subsec:SDT_CET}). Ces cadres convergent vers une prédiction commune~: un agent conversationnel interactif devrait favoriser l'engagement et l'intérêt des apprenants.

Du côté des tensions, les caractéristiques mêmes qui génèrent l'engagement peuvent simultanément compromettre la qualité de l'apprentissage. La fluidité conversationnelle active l'heuristique qui conduit l'apprenant à confondre facilité de traitement et maîtrise du contenu (\S\ref{subsec:fluency_heuristic}). L'incarnation réaliste peut agir comme un détail séduisant qui détourne l'attention du contenu. L'autorité perçue de l'agent peut désactiver la vigilance épistémique nécessaire à un apprentissage critique (\S\ref{subsec:autorite_vigilance}).

Cette tension suggère ce que nous proposons d'appeler l'\textit{hypothèse de l'équilibre des défauts symétriques}. Dans cette perspective, l'interactivité et l'incarnation constituent des leviers robustes pour l'engagement --- mais chaque gain sur cette dimension s'accompagne potentiellement d'un risque métacognitif. L'agent idéal ne serait pas celui qui maximise l'engagement, mais celui qui optimise le rapport entre engagement généré et risque d'illusion induit. Cette hypothèse constitue l'apport spécifique de notre analyse~: elle reformule la question du design des agents pédagogiques comme un problème d'optimisation multi-objectif plutôt que de maximisation unidimensionnelle.

Cette hypothèse trouve un écho dans les recommandations de la littérature. Plutôt que de concevoir des outils \og transparents\fg{} qui s'effacent pour laisser place à la tâche~\citep{heidegger2016being}, certains auteurs suggèrent de \og scripter la rupture\fg{}~: concevoir des agents qui, loin de fournir des réponses fluides et directes, exposent délibérément leurs limites, admettent leur incertitude, ou forcent l'apprenant à valider l'information~\citep{solyst2024generative}. L'objectif n'est plus de maximiser la confiance, mais de la \textit{calibrer} en maintenant active la vigilance critique.

Ces tensions théoriques se traduisent en questions de design concrètes. Comment concevoir un agent suffisamment engageant pour maintenir l'attention, sans que cette qualité ne désactive les mécanismes de vigilance? L'incarnation visuelle amplifie-t-elle le risque métacognitif, ou la fluidité conversationnelle constitue-t-elle à elle seule le facteur déterminant? L'alignement thématique entre le personnage et le contenu peut-il servir de levier d'engagement sans les coûts attentionnels d'une incarnation réaliste? Ces questions structurent directement notre programme de recherche.

\subsection{Lacunes de la littérature actuelle}
\label{subsec:lacunes_litterature}

La première lacune concerne l'interactivité orale avec un agent génératif en contexte écologique. Les méta-analyses sur les agents pédagogiques portent principalement sur des agents scriptés dont les réponses sont sélectionnées dans un répertoire fixe (cf.~\ref{subsec:incarnation_agence_sociale}). Les LLM modifient qualitativement l'interaction en permettant une adaptation sémantique en temps réel (cf.~\ref{subsec:agents_scriptes_generatifs}), mais les preuves empiriques de leur efficacité en contexte scolaire réel restent rares. Les études existantes reposent sur des échantillons restreints, des prototypes et des mesures immédiates, ce qui limite leur validité écologique (cf.~\ref{subsec:agents_scriptes_generatifs}). Le cadre ICAP distingue les modes passif et interactif (cf.~\ref{subsec:icap}), mais cette distinction n'a pas été testée avec des agents génératifs déployés en classe. On ne sait pas si l'avantage interactif documenté avec les agents scriptés se maintient avec des agents alimentés par des LLM.

La deuxième lacune porte sur l'alignement thématique de l'agent et le stade développemental. La congruence entre l'apparence de l'agent et le domaine enseigné produit des effets variables selon le type d'alignement (cf.~\ref{subsec:incarnation_agence_sociale}). Aucune étude ne teste cependant l'incarnation d'un personnage historique comme variable d'alignement dans l'enseignement de l'histoire. Le modèle de l'intérêt prédit que la qualité du déclencheur dépend du stade développemental de l'apprenant (cf.~\ref{subsec:architecture_interet}), mais le rôle de l'âge dans la réponse à l'alignement thématique n'a pas été étudié avec des agents génératifs. Les effets des agents sont plus marqués chez les 10-14~ans (cf.~\ref{subsec:incarnation_agence_sociale}), mais la littérature ne couvre pas le spectre complet de l'adolescence. On ne sait pas si l'alignement thématique de l'agent module l'intérêt de manière différenciée selon l'âge, ni si le style de présentation --- formel ou accessible --- interagit avec le stade développemental.

La troisième lacune concerne l'illusion de compréhension induite par l'interaction avec un agent génératif. L'illusion de compréhension (cf.~\ref{subsec:metacognition_calibration}) et l'heuristique de fluidité (cf.~\ref{subsec:heuristique_fluidite}) sont documentées dans des contextes multimédia classiques --- vidéos, animations, présentations passives. Les agents conversationnels génératifs combinent toutefois fluence linguistique, incarnation anthropomorphe et adaptabilité en temps réel, une combinaison inédite dont les effets métacognitifs n'ont pas été mesurés. La littérature sur la surconfiance envers l'IA porte sur des adultes ou des collégiens interagissant avec des chatbots textuels (cf.~\ref{subsec:autorite_agent_vigilance}). L'effet d'une incarnation visuelle --- humanoïde ou abstraite --- sur l'illusion de compréhension n'a pas été testé. Le protocole IOED, conçu pour mesurer la profondeur illusoire des explications causales \citep{rozenblit2002}, n'a pas été adapté aux contextes d'interaction avec des agents génératifs incarnés. On ne sait pas si l'apparence de l'agent amplifie l'illusion de compréhension, ni si la fluence de l'interaction suffit à induire une surévaluation de la compréhension indépendamment du design visuel.


\subsection{Questions de recherche et hypothèses}
\label{subsec:questions_hypotheses}

Ces lacunes motivent trois questions de recherche complémentaires. Elles suivent une progression dialectique : de l'exploration des bénéfices potentiels vers l'examen des risques associés.

\textbf{QR1 : Dans quelle mesure l'interactivité orale directe avec un agent historique alimenté par l'IA influence-t-elle l'intérêt des élèves par rapport à une présentation vidéo passive ?}

Cette question opérationnalise la distinction entre les modes passif et interactif du cadre ICAP (cf.~\ref{subsec:icap}) en contexte d'agent génératif. La théorie de l'agence sociale prédit que les indices sociaux réactifs activent la présence sociale et augmentent l'effort cognitif (cf.~\ref{subsec:incarnation_agence_sociale}). Le dépassement du plafond des agents scriptés (cf.~\ref{subsec:agents_scriptes_generatifs}) permet de tester cette prédiction avec une adaptation sémantique en temps réel. L'hypothèse H1 prédit que l'interaction orale avec l'agent génère un intérêt supérieur à la présentation vidéo passive, sur les trois dimensions mesurées : intérêt pour l'activité, pour le contenu et pour le personnage.

\textbf{QR2 : Comment l'alignement thématique de l'agent (personnage historique vs. neutre ; pair vs. autorité) et son style de présentation (formel vs. accessible) modulent-ils cet intérêt en fonction du stade développemental des élèves ?}

Cette question explore les conditions aux limites de l'effet d'interactivité. Le modèle de l'intérêt identifie le personnage historique comme déclencheur potentiel d'intérêt situationnel et la pertinence personnelle comme facteur de maintien (cf.~\ref{subsec:architecture_interet},~\ref{subsec:pertinence_valeur}). La congruence thématique entre l'agent et le domaine enseigné peut renforcer l'engagement (cf.~\ref{subsec:incarnation_agence_sociale}). Les effets des agents étant plus marqués chez les 10-14~ans (cf.~\ref{subsec:incarnation_agence_sociale}), l'âge constitue un modérateur à tester sur un spectre développemental étendu. L'hypothèse H2 prédit que l'incarnation d'un personnage historique aligné avec le contenu génère un intérêt supérieur à celui d'un agent neutre. L'hypothèse H3, testée auprès d'élèves de terminale, prédit qu'un personnage pair génère un intérêt supérieur à un personnage d'autorité.

\textbf{QR3 : L'apparence de l'agent (humanoïde vs. abstrait) influence-t-elle la propension des élèves à l'illusion de compréhension, leur confiance et la crédibilité perçue des informations délivrées ?}

Cette question déplace l'analyse du versant motivationnel vers le versant métacognitif. L'heuristique de fluidité (cf.~\ref{subsec:heuristique_fluidite}) et les mécanismes d'autorité de l'agent (cf.~\ref{subsec:autorite_agent_vigilance}) prédisent qu'une incarnation humanoïde amplifie la crédibilité perçue et réduit la vigilance critique. Le protocole IOED, adapté au contexte d'interaction avec un agent génératif, permet de mesurer l'écart entre confiance subjective et performance objective (cf.~\ref{subsec:metacognition_calibration}). L'hypothèse H4 prédit que l'agent humanoïde suscite une confiance et une crédibilité perçue supérieures à celles de l'agent abstrait. L'hypothèse H5 prédit que l'agent humanoïde amplifie l'illusion de compréhension.

Le chapitre suivant présente le dispositif expérimental conçu pour tester ces hypothèses : la plateforme MemorIA, les quatre études et les instruments de mesure.



% Transition vers Chapitre 3
Les questions de recherche ainsi formulées appellent une méthodologie expérimentale rigoureuse, capable de manipuler les variables de design identifiées tout en mesurant leurs effets sur l'engagement et les biais métacognitifs. Avant de présenter les études empiriques, il convient de décrire la plateforme technique développée pour les conduire~: MemorIA, dont l'architecture et la validation font l'objet du chapitre suivant.

