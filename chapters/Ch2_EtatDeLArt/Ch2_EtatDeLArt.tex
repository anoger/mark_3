% ============================================================================
% Chapitre 2 : État de l'Art
% Fichier Maître — Structure modulaire
% ============================================================================
% Organisation : 6 sections, 22 sous-sections
% Sources principales :
%   - IJCCI_extraction.txt (Related Work)
%   - illusion_extraction.txt (Related Work)
%   - Rapport État de l'Art Histoire
%   - Rapport Technologies Éducatives
%   - Rapport Research Paper Mapper
%   - Vault.xlsx (références bibliographiques)
% ============================================================================

\chapter{État de l'Art}
\label{chap:etat_art}

% Sommaire de chapitre
\sommairechapitre

% Introduction du chapitre
Ce chapitre établit le cadre théorique nécessaire à l'analyse de l'interaction entre élèves et agents virtuels historiques. La structure suit une progression logique~: elle part des mécanismes internes de l'apprenant (section~\ref{sec:cadre_apprenant}), examine le contexte disciplinaire spécifique de l'enseignement de l'histoire (section~\ref{sec:contexte_histoire}), analyse la nature de l'interaction avec l'agent (section~\ref{sec:interaction_cognition}), explore la rupture technologique des IA génératives (section~\ref{sec:ia_generatives}), pour aboutir au risque métacognitif central de la thèse (section~\ref{sec:illusion_comprehension}). Une synthèse finale articule ces éléments en une problématique de recherche cohérente (section~\ref{sec:synthese}).

% ============================================================================
% Section 2.1 : Cadre Théorique de l'Apprenant
% ============================================================================
% ============================================================================
% Section 2.1 : Cadre Théorique de l'Apprenant
% ============================================================================
% Objectif : Définir les moteurs psychologiques internes de l'apprentissage
% Calibrage : ~150 mots (introduction)
% ============================================================================

\section{Cadre Théorique de l'Apprenant~: Motivation, Engagement et Pertinence}
\label{sec:cadre_theorique_apprenant}

Cette première section établit les fondations théoriques nécessaires à l'analyse des mécanismes psychologiques qui sous-tendent l'engagement des élèves dans les apprentissages. Nous examinons successivement les théories motivationnelles qui éclairent les conditions de l'engagement intrinsèque (\S\ref{subsec:SDT_CET}), l'architecture développementale de l'intérêt (\S\ref{subsec:architecture_interet}), les processus par lesquels les élèves attribuent de la valeur aux contenus d'apprentissage et les stratégies de personnalisation (\S\ref{subsec:pertinence_personnalisation}), et les niveaux d'engagement cognitif différenciés par le cadre ICAP (\S\ref{subsec:ICAP}).

Ces cadres théoriques, issus de la psychologie de l'éducation, fournissent les outils conceptuels qui seront mobilisés dans les sections suivantes pour analyser les spécificités de l'enseignement de l'Histoire (\S\ref{sec:contexte_histoire}) et pour comprendre les mécanismes d'interaction avec les agents conversationnels (\S\ref{sec:cadre_interaction}).

% Inclusion des sous-sections
% ============================================================================
% Sous-section 2.1.1 : Dynamiques Motivationnelles — SDT et CET
% ============================================================================
% Sources : IJCCI_extraction (§2.1), Vault.xlsx
% Calibrage : ~600 mots
% Type : C (Rédaction originale, synthèse analytique)
% ============================================================================

\subsection{Dynamiques Motivationnelles~: SDT et CET}
\label{subsec:SDT_CET}

La motivation intrinsèque --- cette propension à s'engager dans une activité pour le plaisir et la satisfaction qu'elle procure --- constitue un prédicteur robuste de la qualité de l'apprentissage~\citep{deci2000what}. Sa compréhension nécessite d'articuler deux niveaux d'analyse~: les mécanismes par lesquels l'environnement affecte cette motivation, et les besoins psychologiques sous-jacents qui la conditionnent.

Au premier niveau, les facteurs environnementaux n'exercent pas d'effet direct et uniforme sur la motivation~; leur impact dépend de l'interprétation qu'en fait l'individu~\citep{deci1985intrinsic}. Un même feedback peut ainsi être vécu comme \textit{informationnel} --- fournissant un retour constructif sur la compétence et soutenant la motivation --- ou comme \textit{contrôlant} --- exerçant une pression sur le comportement et sapant l'autodétermination. Cette distinction, issue de la Théorie de l'Évaluation Cognitive (CET), s'articule autour de deux dimensions psychologiques~: le \textit{locus de causalité perçu}, soit le sentiment que ses actions émanent de soi plutôt que de contraintes externes, et le \textit{sentiment de compétence}, soit la conviction de pouvoir atteindre les résultats souhaités. Un troisième aspect, \textit{amotivant}, peut signaler l'incompétence et conduire au désengagement total.

Au second niveau, trois besoins psychologiques fondamentaux conditionnent le bien-être et l'engagement~: l'\textit{autonomie}, la \textit{compétence} et l'\textit{affiliation}~\citep{ryan2017self}. La Théorie de l'Autodétermination (SDT) postule que les environnements d'apprentissage favorisent l'intérêt lorsqu'ils satisfont ces besoins à travers des choix significatifs, des défis appropriés, un feedback constructif et des interactions soutenantes. L'ajout du besoin d'affiliation --- le sentiment de connexion aux autres --- étend la portée explicative du modèle au-delà des seules dimensions cognitives pour intégrer la dimension sociale de l'apprentissage.

L'articulation de ces deux niveaux révèle une dynamique complexe~: un environnement qui offre des choix (autonomie), propose des défis calibrés avec un feedback informatif (compétence), et crée des opportunités d'interaction authentique (affiliation), réunit les conditions propices au développement de la motivation intrinsèque. À l'inverse, un environnement perçu comme contrôlant, qui impose des tâches sans en expliciter le sens et isole l'apprenant, tend à éroder cette motivation --- un phénomène particulièrement documenté lors de la transition vers le secondaire~\citep{gnambs2016decline}. Cette période, marquée par une pression accrue sur la performance et des structures scolaires qui peuvent limiter la curiosité naturelle~\citep{engel2011children}, voit la motivation intrinsèque décliner significativement.

Ce déclin revêt une importance particulière pour l'enseignement de l'Histoire, discipline souvent perçue comme distante et déconnectée des expériences personnelles~\citep{audigier2010histoire}. Le défi consiste alors à concevoir des environnements qui, tout en respectant les contraintes curriculaires, créent les conditions de satisfaction des besoins identifiés. Les agents conversationnels offrent une piste prometteuse~: l'interactivité dialogique peut soutenir l'autonomie en permettant à l'élève de diriger l'échange, les réponses adaptatives peuvent nourrir le sentiment de compétence, et les indices sociaux de l'agent peuvent répondre au besoin d'affiliation. Cette hypothèse, qui guide notre programme expérimental, nécessite cependant de comprendre comment l'intérêt se développe et se maintient --- objet de la section suivante.

\subsection{Architecture de l'Intérêt}
\label{subsec:architecture_interet}

L'intérêt, tel que défini dans la recherche en psychologie de l'éducation, combine deux composantes de nature distincte \citep{hidi2006}. La composante affective se manifeste par une expérience positive associée à l'activité --- attention accrue, affect favorable, envie spontanée de poursuivre. La composante cognitive se traduit par une orientation vers la compréhension du contenu : l'individu cherche à approfondir, pose des questions, établit des connexions entre les informations. Cette double nature distingue l'intérêt des construits apparentés. Là où la motivation intrinsèque (cf. \ref{subsec:sdt_cet}) désigne un processus général applicable à toute activité, l'intérêt est toujours dirigé vers un contenu spécifique \citep{renninger2015}. Un même élève peut manifester un intérêt soutenu pour la biologie et un désintérêt marqué pour l'histoire, alors que son niveau de motivation intrinsèque globale reste stable. Cette spécificité de contenu confère à l'intérêt une valeur explicative particulière pour l'apprentissage disciplinaire : il contribue à expliquer pourquoi un dispositif pédagogique efficace dans une matière échoue dans une autre \citep{hidi2000}.

Le développement de l'intérêt suit une séquence en quatre phases dont chacune se caractérise par un équilibre distinct entre soutien externe et engagement autonome \citep{hidi2006}. La première phase, l'intérêt situationnel déclenché, désigne une réponse attentionnelle à un stimulus environnemental --- nouveauté, surprise, incongruité avec les attentes. Cette réponse est brève et dépend entièrement du déclencheur externe. La deuxième phase, l'intérêt situationnel maintenu, apparaît lorsque l'engagement avec le contenu se prolonge au-delà de la réaction initiale : l'apprenant commence à traiter l'information en profondeur, mais le soutien de l'environnement reste nécessaire pour maintenir son attention. La troisième phase, l'intérêt individuel émergent, marque un changement qualitatif : l'apprenant développe une prédisposition à se réengager avec le contenu de sa propre initiative, génère des questions et recherche activement des informations complémentaires. La quatrième phase, l'intérêt individuel développé, correspond à une disposition stable : l'apprenant tolère la frustration, autorégule son apprentissage et produit des questions de curiosité qui alimentent sa progression \citep{renninger2015}. La transition de la deuxième à la troisième phase constitue le point de basculement du modèle : l'intérêt tend à ne plus dépendre du soutien de l'environnement pour devenir auto-entretenu.

Les facteurs qui déclenchent l'intérêt situationnel ne sont pas ceux qui le maintiennent. La nouveauté, le caractère inattendu d'une information et l'intensité perceptive d'un stimulus suffisent à capter l'attention (phase 1), mais leur effet tend à se dissiper si aucune connexion significative ne s'établit avec l'apprenant \citep{hidi2006}. Le maintien de l'intérêt (phase 2) repose sur des facteurs différents : la pertinence personnelle perçue, l'engagement actif dans une tâche signifiante et le sentiment de compétence dans l'interaction avec le contenu \citep{bergin1999}. Le contexte social joue un rôle transversal. L'identité du locuteur, la qualité de la relation avec l'enseignant et l'influence des pairs peuvent à la fois déclencher et maintenir l'intérêt \citep{bergin2016}. Un interlocuteur perçu comme signifiant --- qu'il soit enseignant, pair ou figure de référence --- peut affecter le développement de l'intérêt indépendamment du contenu transmis \citep{renninger2009}. Cette dissociation entre déclenchement et maintien pose un problème de conception : un dispositif qui mise exclusivement sur la nouveauté sans créer de connexion personnelle avec le contenu ne dépasse pas la première phase.

La curiosité et l'intérêt entretiennent une relation de renforcement mutuel qui s'intensifie au fil du développement \citep{hidi2020}. Dans les phases avancées de l'intérêt (phases 3 et 4), l'apprenant ne se contente plus de répondre aux stimuli externes : il génère spontanément des questions qui orientent son exploration du domaine. Ces questions de curiosité constituent le moteur interne de la progression vers un intérêt individuel stable. Les environnements éducatifs qui accueillent ces questions --- plutôt que de les canaliser vers des objectifs prédéterminés --- facilitent la transition entre les phases \citep{renninger2015}. La progression de l'intérêt situationnel vers l'intérêt individuel dépend toutefois d'un facteur que le modèle identifie sans l'approfondir : la capacité de l'apprenant à percevoir le contenu comme personnellement pertinent. Ce mécanisme, par lequel un savoir disciplinaire acquiert une valeur aux yeux de l'apprenant, relève d'un cadre théorique distinct : la théorie de la valeur et de la pertinence, qui fait l'objet de la section suivante.

% ============================================================================
% Sous-section 2.1.3 : Pertinence, Valeur et Personnalisation de l'Apprentissage
% ============================================================================
% Sources : IJCCI_extraction (§2.1-2.2), Harackiewicz (2014, 2016), Vault.xlsx,
%           Albrecht & Karabenick (2018)
% Calibrage : ~1100-1200 mots
% Type : C (Rédaction originale)
% ============================================================================

\subsection{Pertinence, Valeur et Personnalisation de l'Apprentissage}
\label{subsec:pertinence_personnalisation}

La section précédente a établi que la valeur perçue constitue un mécanisme pivot dans la transition de l'intérêt situationnel vers l'intérêt individuel. Cette valeur ne relève pas d'une propriété intrinsèque du contenu~; elle émerge de la relation que l'apprenant établit entre ce contenu et ses préoccupations personnelles. La théorie de l'attente-valeur formalise cette intuition en postulant que la motivation à s'engager dans une tâche dépend de deux facteurs multiplicatifs~: les \textit{attentes de succès} (croyance en sa capacité de réussir) et la \textit{valeur} attribuée à la tâche~\citep{eccles1983expectancies, wigfield2000expectancy}.

La valeur subjective se décompose en quatre dimensions qui éclairent différentes facettes de la pertinence. La \textit{valeur d'accomplissement} renvoie à l'importance de la tâche pour l'identité personnelle --- réussir confirme une image de soi valorisée. La \textit{valeur intrinsèque} correspond au plaisir tiré de l'activité elle-même, rejoignant la dimension affective de l'intérêt. La \textit{valeur d'utilité} concerne la pertinence pour les objectifs futurs --- le contenu est perçu comme un moyen vers une fin désirée. Enfin, le \textit{coût} perçu (effort, anxiété, opportunités sacrifiées) vient moduler négativement ces valeurs positives. Cette décomposition révèle pourquoi l'enseignement de l'Histoire souffre souvent d'un déficit d'engagement~: la discipline peine à démontrer sa valeur d'utilité immédiate, contrairement aux disciplines STIM dont les applications apparaissent plus évidentes~\citep{harackiewicz2016importance}.

\subsubsection*{De la valeur d'utilité à la pertinence personnelle}

La valeur d'utilité, si elle constitue un levier d'intervention accessible, ne représente qu'une facette de la pertinence. Au-delà de la question instrumentale \og à quoi ça sert?\fg{}, une question plus profonde émerge~: \og en quoi cela me concerne-t-il?\fg{}. Cette \textit{pertinence personnelle} (\textit{self-relevance}) implique que le contenu entre en résonance avec l'identité, les expériences et les préoccupations de l'apprenant~\citep{priniski2018making}.

Une conceptualisation multidimensionnelle distingue ainsi la pertinence \textit{personnelle} (connexion au soi) de la pertinence \textit{impersonnelle} (utilité pour des entités externes), et la pertinence \textit{appliquée} (utilité pour des actions concrètes) de la pertinence \textit{conceptuelle} (aide à la compréhension du monde)~\citep{albrecht2018relevance}. Cette taxonomie explique pourquoi une même intervention peut fonctionner pour certains élèves et échouer pour d'autres~: un élève sensible à la pertinence appliquée (\og cela m'aidera dans mon métier\fg{}) et un autre répondant à la pertinence conceptuelle (\og cela m'aide à comprendre l'actualité\fg{}) nécessitent des approches différenciées.

\subsubsection*{Stratégies d'intervention sur la pertinence}

Les interventions visant à augmenter la pertinence perçue se distinguent par le niveau d'effort cognitif qu'elles requièrent de l'apprenant~\citep{albrecht2018relevance}. À faible effort, la \textit{communication directe} --- l'enseignant explique pourquoi le contenu est pertinent --- et la \textit{personnalisation} --- le contenu est adapté aux intérêts déclarés --- représentent des approches accessibles mais dont l'impact reste limité. L'information fournie de l'extérieur peut être ignorée ou rejetée si elle ne résonne pas avec les schémas existants de l'apprenant.

Les approches à effort modéré, comme la \textit{réflexion critique}, invitent les élèves à examiner leurs propres croyances sur la pertinence du contenu. Cette introspection peut révéler des connexions insoupçonnées, mais elle reste circonscrite au répertoire cognitif préexistant de l'apprenant.

Les approches à effort élevé --- \textit{auto-génération} et \textit{réévaluation dirigée} --- demandent aux élèves de produire activement des connexions entre le contenu et leur vie, ou de reconsidérer la valeur du contenu à la lumière de nouvelles perspectives. Les interventions de valeur d'utilité, où les élèves rédigent des essais connectant le contenu académique à leur expérience personnelle, illustrent cette approche~\citep{harackiewicz2014harnessing}. L'effort cognitif investi dans la génération de connexions semble renforcer leur impact motivationnel --- un résultat qui résonne avec la hiérarchie des modes d'engagement que nous examinerons dans la section suivante (cadre ICAP).

Une nuance importante émerge cependant~: ces interventions bénéficient particulièrement aux élèves présentant de faibles attentes de succès initiales, pour qui la découverte de connexions personnelles peut transformer la perception du contenu. Pour les élèves avec de fortes attentes de succès, l'intervention peut paradoxalement détourner l'attention de stratégies d'apprentissage déjà efficaces~\citep{harackiewicz2014harnessing}. Cette interaction entre intervention et profil de l'apprenant souligne la nécessité d'une approche différenciée.

\subsubsection*{La personnalisation comme vecteur de pertinence}

La personnalisation de l'apprentissage --- l'adaptation du contenu et des activités aux intérêts, connaissances préalables et préférences individuels --- constitue une stratégie prometteuse pour développer la pertinence à grande échelle~\citep{walkington2018personalization}. L'intégration d'informations personnelles dans les contextes d'apprentissage (prénom de l'élève, centres d'intérêt déclarés) peut augmenter la motivation~\citep{cordova1996intrinsic}. Plus remarquable encore, offrir un contrôle sur des aspects même non essentiels de l'apprentissage (choix d'un avatar, d'un thème visuel) produit des effets positifs --- suggérant que le sentiment d'appropriation personnelle contribue à l'engagement indépendamment du contenu lui-même. Ce résultat établit un lien avec le besoin d'autonomie identifié par la SDT~: le choix, même symbolique, nourrit le sentiment d'autodétermination.

En mathématiques et en sciences, permettre aux élèves de choisir des exemples connectant les concepts abstraits à leurs intérêts renforce l'engagement, particulièrement chez ceux présentant initialement un faible intérêt~\citep{hogheim2015, reber2018personalized}. Cette stratégie crée des ponts entre concepts abstraits et expériences concrètes, facilitant l'ancrage des nouvelles connaissances dans les schémas existants --- retrouvant ainsi le facteur \og connaissances préalables\fg{} de la taxonomie de l'intérêt. Placer la curiosité des élèves au centre de l'apprentissage peut également améliorer l'engagement et les relations pédagogiques~\citep{hagay2015incorporating}.

\subsubsection*{Défis et perspectives technologiques}

La personnalisation se heurte cependant à des obstacles pratiques~: diversité des intérêts, évolution constante des préférences, complexité de créer du contenu adapté pour chaque profil~\citep{walkington2018personalization}. Ces contraintes ont longtemps limité le déploiement à grande échelle d'approches véritablement individualisées.

Les grands modèles de langage (LLM) offrent de nouvelles perspectives en permettant une adaptation dynamique et contextuelle~\citep{kasneci2023chatgpt, labadze2023generative}. Un agent conversationnel incarnant un personnage historique peut ainsi devenir un vecteur naturel de pertinence personnelle. Le dialogue direct --- où l'élève pose ses propres questions et reçoit des réponses adaptées à son niveau de compréhension et à ses préoccupations --- crée une connexion entre le contenu historique et la curiosité individuelle. Chaque élève explore les aspects qui l'intriguent, génère ses propres connexions à travers ses questions, et découvre ainsi la pertinence personnelle du savoir historique.

Cette \textit{personnalisation émergente} --- qui naît de l'interaction plutôt que d'une programmation préalable --- combine plusieurs mécanismes identifiés~: elle permet l'auto-génération de connexions personnelles (effort élevé), offre un sentiment de contrôle sur l'exploration (autonomie), et peut activer différentes dimensions de pertinence selon les questions posées (personnelle/impersonnelle, appliquée/conceptuelle). L'alignement thématique entre le personnage et le contenu de la leçon peut amplifier ces effets~\citep{schmidt2019effects}. Cette hypothèse constitue l'un des axes centraux de notre programme expérimental.

Reste à comprendre comment la forme de l'interaction --- et non seulement son contenu --- affecte l'engagement cognitif. Le cadre ICAP, que nous examinons maintenant, offre une grille d'analyse pour distinguer différents niveaux d'engagement et prédire leurs effets sur l'apprentissage.


% ============================================================================
% Sous-section 2.1.4 : L'Apprentissage Actif et le Cadre ICAP
% ============================================================================
% Sources : IJCCI_extraction (§2.3), Chi (2009), Freeman et al. (2014)
% Calibrage : ~550 mots
% Type : C (Rédaction originale)
% ============================================================================

\subsection{L'Apprentissage Actif et le Cadre ICAP}
\label{subsec:ICAP}

Les sections précédentes ont montré que l'effort cognitif investi dans la génération de connexions personnelles renforce leur impact motivationnel. Ce constat s'inscrit dans un cadre plus général~: l'apprentissage actif, qui dépasse la simple activité physique pour désigner le fait de \og penser activement à ce que l'on fait\fg{}~\citep{mayer2014cambridge, yannier2021active}. La question devient alors~: comment caractériser les différents niveaux d'engagement cognitif et prédire leurs effets sur l'apprentissage?

Le cadre ICAP (\textit{Interactive, Constructive, Active, Passive}) propose une taxonomie hiérarchisée des activités d'apprentissage selon leur niveau d'engagement cognitif~\citep{chi2009active}. Le mode \textit{Passif} correspond à la réception d'information sans comportement observable au-delà de l'attention --- écouter un cours, regarder une vidéo. Le mode \textit{Actif} implique une manipulation ou une attention focalisée sans production de nouvelles idées --- prendre des notes verbatim, surligner. Le mode \textit{Constructif} requiert la génération d'idées qui dépassent l'information présentée --- formuler des hypothèses, élaborer des explications, connecter le contenu à son expérience personnelle. Le mode \textit{Interactif} ajoute une dimension dialogique~: les partenaires co-construisent des connaissances à travers un échange où chacun contribue substantiellement.

La prédiction centrale du modèle --- Interactif $>$ Constructif $>$ Actif $>$ Passif en termes de gains d'apprentissage --- a reçu un soutien empirique substantiel. Une méta-analyse portant sur 225 études montre que l'apprentissage actif augmente significativement la performance des étudiants en STIM comparé aux cours magistraux~\citep{freeman2014active}. L'interactivité permet de réguler le rythme d'apprentissage, d'explorer les concepts selon ses intérêts, de formuler des questions et de recevoir un feedback immédiat~\citep{domagk2010pedagogical, evans2007interactivity}.

Cette hiérarchie éclaire les résultats sur les interventions de pertinence~: les approches à effort élevé (auto-génération, réévaluation dirigée) relèvent du mode \textit{Constructif}, tandis que les approches à faible effort (communication directe) maintiennent l'apprenant en mode \textit{Passif}. L'efficacité supérieure des premières s'explique ainsi par leur niveau d'engagement cognitif plus profond. Un paradoxe mérite cependant attention~: l'apprentissage actif, bien que conduisant à de meilleurs résultats, est souvent perçu comme moins efficace par les apprenants eux-mêmes~\citep{deslauriers2019measuring}. L'effort cognitif accru est interprété comme un signe de difficulté alors qu'il indique un traitement plus profond.

Dans l'enseignement de l'Histoire, les approches interactives produisent des effets positifs sur la compréhension et l'intérêt. Les discussions de groupe favorisent une meilleure compréhension des concepts historiques~\citep{delfavero2007classroom}. La narration numérique interactive, avec des points de décision stratégiques, stimule des discussions significatives et une compréhension plus profonde~\citep{petousi2022interactive}. La combinaison d'interactions tangibles avec des récits émotionnels encourage les adolescents à s'engager avec les figures historiques au-delà de la connaissance factuelle~\citep{roussou2024emotions}.

Un agent conversationnel incarnant un personnage historique se situe naturellement au niveau \textit{Interactif}~: l'élève formule des questions (activité constructive), l'agent répond, et l'échange peut conduire à une co-construction de sens. Cette position contraste avec la vidéo (mode \textit{Passif}) et le texte (mode \textit{Actif} si l'élève surligne ou prend des notes). Le cadre ICAP prédit ainsi que l'agent dialogique devrait produire des gains d'engagement et d'apprentissage supérieurs aux formats traditionnels --- une prédiction que notre programme expérimental vise à tester.

Cette convergence des cadres théoriques --- SDT, développement de l'intérêt, théorie de l'attente-valeur, ICAP --- dessine les contours d'une intervention prometteuse~: un agent conversationnel qui satisfait les besoins d'autonomie, de compétence et d'affiliation, déclenche et maintient l'intérêt par la nouveauté et l'interaction sociale, permet la génération de connexions personnelles, et engage l'apprenant au niveau interactif du cadre ICAP.



% ============================================================================
% Section 2.2 : Le Contexte Disciplinaire — Enseigner et Apprendre l'Histoire
% ============================================================================
% ============================================================================
% Section 2.2 : Le Contexte Spécifique de l'Enseignement de l'Histoire
% ============================================================================
% Objectif : Établir le diagnostic du terrain disciplinaire — position de
%            l'histoire dans l'écosystème scolaire, perception des élèves,
%            leviers pédagogiques identifiés par la recherche
% Sources : État de l'Art Histoire (Harris & Haydn, Haydn & Harris, Grever,
%           Henkaline, Van Straaten, Gómez Carrasco, Cairns & Garrard, Bergin)
% Calibrage : ~1 800 mots
% ============================================================================

\section{Le Contexte Spécifique de l'Enseignement de l'Histoire}
\label{sec:contexte_histoire}

Les cadres théoriques présentés dans la section précédente éclairent les mécanismes généraux de l'engagement dans l'apprentissage. Leur application à l'enseignement de l'histoire révèle un terrain singulier. La discipline n'est pas rejetée par les élèves~; elle souffre d'un déficit de médiation entre ses spécificités épistémologiques et les attentes de son public. Ce diagnostic structure l'analyse qui suit~: la position de l'histoire dans l'écosystème scolaire (\S\ref{subsec:epistemologie_STIM}), le paradoxe de la perception des élèves (\S\ref{subsec:perception_eleves}), et les leviers pédagogiques identifiés par la recherche (\S\ref{subsec:pratiques_pedagogiques}).

% ----------------------------------------------------------------------------
% Sous-sections
% ----------------------------------------------------------------------------
% ============================================================================
% Sous-section 2.2.1 : Position de l'Histoire dans l'Écosystème Scolaire
% ============================================================================
% Sources : État de l'Art Histoire (Harris & Haydn 2006, Haydn & Harris 2010,
%           Van Straaten et al. 2015, Grever et al. 2011)
% Calibrage : ~550 mots
% Type : C (Rédaction originale)
% ============================================================================

\subsection{Position de l'Histoire dans l'Écosystème Scolaire}
\label{subsec:epistemologie_STIM}

Contrairement à une idée répandue, l'histoire n'est pas rejetée par les élèves. Avec 69,8\% d'opinions favorables auprès de 1740 élèves britanniques, la discipline se classe en cinquième position des matières appréciées~\citep{harris2006pupils}. Ce constat, corroboré dans d'autres contextes nationaux, invite à nuancer le diagnostic d'une discipline en crise~: l'histoire occupe une position intermédiaire qui révèle moins un rejet qu'un déficit de médiation.

Cette position se caractérise par un décalage entre deux formes de valeur perçue. Les élèves reconnaissent à l'histoire une utilité \textit{cognitive}~--- comprendre le présent, éviter de répéter les erreurs du passé, développer un regard critique sur le monde~--- tout en peinant à lui attribuer une utilité \textit{instrumentale} comparable à celle des disciplines scientifiques et techniques. Sur une échelle d'importance perçue, les mathématiques obtiennent 4,46 et l'anglais 4,42, contre 3,26 pour l'histoire~\citep{haydn2010pupil}. Cette asymétrie s'explique par la lisibilité des débouchés professionnels~: les filières scientifiques offrent des trajectoires clairement identifiées là où l'histoire semble cantonnée à l'enseignement ou aux métiers du patrimoine~\citep{grever2011high}.

Cette hiérarchie implicite s'enracine dans des différences épistémologiques que l'école rend rarement explicites. Les disciplines STIM reposent sur des savoirs cumulatifs, universels et vérifiables par l'expérimentation~: la progression y suit une logique d'accumulation où chaque concept s'appuie sur les précédents, et le rapport à la vérité s'établit par démonstration~\citep{vanstraaten2015making}. Le feedback y est immédiat~--- une équation est correctement résolue ou ne l'est pas. L'histoire, en revanche, produit des savoirs interprétatifs et contextuels où la vérité émerge de l'argumentation fondée sur des sources, processus moins définitif où plusieurs interprétations peuvent coexister pour un même événement. Cette nature interprétative requiert une tolérance à l'ambiguïté que l'enseignement scolaire cultive rarement de manière explicite~--- d'où le sentiment de certains élèves que \og l'histoire est morte et n'a rien à voir avec [leur] vie présente\fg{}, une proportion qui atteint 14\% dans les enquêtes européennes~\citep{vanstraaten2015making}.

Le contraste méthodologique est tout aussi marqué. Les STIM privilégient l'expérimentation contrôlée et la modélisation mathématique, offrant des procédures reproductibles dont la rigueur est immédiatement perceptible. L'histoire utilise l'analyse critique de sources, la contextualisation et la mise en perspective~--- méthodes moins standardisées dont les élèves saisissent moins aisément l'exigence intellectuelle. Cette différence nourrit une perception de moindre scientificité, alors même que l'histoire développe des compétences critiques équivalentes~\citep{harris2006pupils}.

Ces différences épistémologiques ne constituent pas des défauts~; elles définissent des compétences distinctives. L'analyse critique de discours, la capacité à peser des arguments contradictoires, la compréhension des motivations humaines dans leur contexte, l'empathie historique~: autant de capacités que les STIM ne cultivent pas avec la même intensité. Le potentiel narratif et dramatique de l'histoire autorise un engagement émotionnel que les disciplines formelles peinent à susciter~\citep{harris2006pupils}. Sa contribution à la formation citoyenne et à la compréhension interculturelle répond à des besoins sociétaux croissants~\citep{grever2011high}. L'enjeu n'est donc pas d'imiter les STIM, mais de valoriser ces spécificités tout en explicitant les compétences qu'elles développent~--- un travail de médiation que l'enseignement traditionnel n'accomplit qu'imparfaitement.

\subsection{Les Pratiques Actives en Histoire : Au-Delà du Cours Magistral}
\label{subsec:pratiques_actives}

Les recherches en didactique de l'histoire ont identifié des formats pédagogiques qui déplacent l'élève de la réception passive d'un récit vers la construction active de sa compréhension. La discussion collaborative et la résolution de problèmes en groupe augmentent la compréhension des concepts historiques et l'intérêt pour la discipline par rapport à l'apprentissage individuel : les échanges entre pairs obligent chaque participant à expliciter son raisonnement, à confronter ses interprétations et à intégrer des perspectives divergentes \citep{delfavero2007}. Le récit interactif numérique, dans lequel les apprenants rencontrent des points de décision stratégiques au fil d'une narration historique, produit un effet convergent : les choix narratifs stimulent la discussion de groupe et favorisent une compréhension en profondeur des enjeux historiques, parce qu'ils contraignent l'apprenant à évaluer les conséquences de décisions prises dans un contexte passé \citep{petousi2022}.

L'immersion dans des environnements virtuels renforce ces effets en ajoutant une dimension spatiale et sensorielle à l'expérience historique. La reconstruction tridimensionnelle de cités antiques améliore la performance et l'engagement des élèves, qui consacrent davantage de temps à l'exploration du contenu historique que les groupes exposés au même contenu par le texte ou la vidéo \citep{ijaz2017}. Les dispositifs muséaux interactifs, qui combinent objets tangibles et récits à forte charge émotionnelle, favorisent l'empathie historique et la pensée critique chez les adolescents : la manipulation physique d'artefacts liés à des récits personnels du passé crée une connexion affective que le texte seul ne parvient pas à produire \citep{roussou2024}. Ces résultats convergent vers un principe commun : l'apprentissage de l'histoire gagne en profondeur lorsque l'élève passe de la réception d'un récit clos à la construction active de sa propre compréhension du passé --- un déplacement qui correspond au passage du mode passif vers les modes constructif et interactif du cadre ICAP (cf.~\ref{subsec:icap}). Trois formats illustrent ce principe avec une force particulière : la simulation, l'incarnation par le jeu de rôle et la multimodalité du jeu historique.

% ============================================================================
% Sous-section 2.2.3 : Leviers Pédagogiques de l'Engagement
% ============================================================================
% Sources : État de l'Art Histoire (Harris & Haydn 2006, Gómez Carrasco et al. 2021,
%           Cairns & Garrard 2024, Van Straaten et al. 2015, Bergin 1999,
%           Henkaline 2023)
% Calibrage : ~650 mots
% Type : C (Rédaction originale)
% ============================================================================

\subsection{Leviers Pédagogiques de l'Engagement}
\label{subsec:pratiques_pedagogiques}

Si le paradoxe \og pertinent mais ennuyeux\fg{} désigne le problème, la recherche en didactique de l'histoire identifie avec une précision croissante les conditions de sa résolution. Les données empiriques convergent vers un constat structurant~: l'efficacité pédagogique dépend moins du contenu enseigné que des modalités d'enseignement.

Les méthodes interactives recueillent massivement l'adhésion des élèves. Sur 1740 répondants britanniques, le jeu de rôle et le théâtre totalisent 295 mentions positives, les discussions et débats 108, le travail de groupe 56~\citep{harris2006pupils}. À l'inverse, le travail écrit excessif concentre 394 mentions négatives, les tests fréquents 151, la sur-utilisation des manuels et fiches 57. Ce pattern n'est pas propre au contexte britannique~: les élèves américains valorisent de même les méthodes expérientielles~--- simulations de procès, vidéos immersives~--- et critiquent la prise de notes passive~\citep{henkaline2023eighth}. La convergence transculturelle de ces préférences suggère qu'elles reflètent des invariants cognitifs plutôt que des spécificités locales.

Ces préférences ne relèvent pas du simple confort~; elles produisent des gains mesurables sur l'apprentissage. Un programme de formation combinant méthodes actives et réflexion épistémologique, évalué auprès de 467 élèves répartis en 18 classes, révèle des améliorations significatives~: $r = 0.522$ pour l'évaluation de la méthodologie, $r = 0.443$ pour la motivation, $r = 0.335$ pour l'apprentissage perçu~\citep{gomezcarrasco2021motivation}. L'intensité de l'intervention module les effets~: les enseignants formés de manière approfondie produisent des résultats supérieurs à ceux formés superficiellement. La transformation pédagogique requiert donc un investissement substantiel en formation.

Le cadre ICAP (cf.~\S\ref{subsec:ICAP}) éclaire les mécanismes sous-jacents à cette efficacité. Les méthodes interactives~--- débats, jeux de rôle, confrontation des interprétations~--- activent les modes \textit{Constructif} et \textit{Interactif}, où l'élève génère des inférences et co-construit du sens avec ses pairs. Le cours magistral cantonne l'élève au mode \textit{Passif}, où l'attention ne garantit pas le traitement profond. La supériorité des méthodes actives n'est donc pas une question de préférence subjective mais de fonctionnement cognitif~: l'engagement émotionnel améliore la mémorisation, l'expérience directe développe l'empathie historique, l'interaction sociale répond aux besoins d'appartenance~\citep{bergin1999influences}.

Parmi les leviers identifiés, la connexion explicite entre temporalités occupe une place centrale. Un cadre théorique articulant passé, présent et futur comme condition de la pertinence perçue montre que les élèves trouvent les tâches de connexion temporelle plus difficiles mais plus intéressantes que les exercices traditionnels~\citep{vanstraaten2015making}. Cette observation trouve un écho dans les données australiennes~: les élèves qui poursuivent l'histoire valorisent précisément sa capacité à éclairer le présent~\citep{cairns2024learning}. L'explicitation de ces connexions~--- plutôt que leur découverte supposée spontanée~--- constitue une condition de l'engagement.

L'enseignant demeure la variable déterminante. Les écarts d'appréciation entre établissements (cf.~\S\ref{subsec:perception_eleves}) ne s'expliquent ni par les contenus enseignés ni par les caractéristiques des publics, mais par les pratiques professorales et départementales. Les enseignants \og enthousiastes qui respectent les élèves\fg{} obtiennent de meilleurs résultats indépendamment des méthodes utilisées~\citep{harris2006pupils}. Cette observation s'inscrit dans un modèle plus général de l'intérêt situationnel~: nouveauté modérée, interaction sociale, narration, prise en compte de l'appartenance culturelle constituent autant de facteurs que l'enseignant contrôle et qui peuvent déclencher l'intérêt même chez des élèves initialement peu motivés~\citep{bergin1999influences}.

Ces résultats dessinent le cahier des charges implicite d'une innovation pédagogique en histoire. L'efficacité suppose l'activation de l'élève plutôt que sa réception passive, la confrontation des perspectives plutôt que la transmission d'un récit unique, la connexion explicite aux préoccupations contemporaines plutôt que l'enfermement dans le passé. Les sections suivantes examinent dans quelle mesure les technologies d'interaction~--- et singulièrement les agents conversationnels~--- peuvent satisfaire ces exigences.

% Note : Le contenu sur les technologies éducatives en histoire est déplacé
%        en section 2.4 pour respecter la progression argumentative du chapitre
% % ============================================================================
% Sous-section 2.2.4 : Technologies Éducatives en Histoire — Bilan Empirique
% ============================================================================
% Sources : IJCCI_extraction, État de l'Art Histoire
% Calibrage : ~650 mots
% Type : C (Rédaction originale)
% ============================================================================

\subsection{Technologies Éducatives en Histoire~: Bilan Empirique}
\label{subsec:technologies_histoire}

L'intégration des technologies numériques dans l'enseignement de l'Histoire a fait l'objet d'expérimentations variées, dont les résultats permettent d'identifier les approches prometteuses et celles qui se sont révélées décevantes.

\subsubsection*{Approches ayant produit des résultats probants}

Plusieurs dispositifs technologiques ont démontré une efficacité mesurable sur l'engagement ou l'apprentissage en contexte historique.

La narration numérique interactive constitue l'approche la mieux documentée. L'intégration de points de décision stratégiques dans des récits historiques stimule des discussions de groupe significatives et une compréhension approfondie~\citep{petousi2022interactive}. Dans une étude sur l'Athènes antique, les élèves confrontés à des choix narratifs ont développé une réflexion historique plus élaborée que ceux exposés à un récit linéaire. L'efficacité repose sur l'activation du mode \textit{Constructif} du cadre ICAP~: les apprenants génèrent des hypothèses sur les conséquences de leurs choix plutôt que de recevoir passivement l'information.

Les environnements virtuels immersifs produisent des gains d'apprentissage et d'engagement lorsqu'ils permettent une exploration autonome. Une reconstruction 3D de l'ancienne Uruk a conduit à une amélioration des performances aux tests et à un temps d'exploration accru comparé aux méthodes traditionnelles~\citep{ijaz2017immersion}. L'immersion spatiale semble activer des processus de mémorisation épisodique qui renforcent la rétention des informations contextuelles.

Les installations muséales combinant interactions tangibles et récits émotionnels favorisent le développement de l'empathie historique~\citep{roussou2024emotions}. L'exploration de perspectives émotionnelles multiples sur des personnages historiques permet aux adolescents de dépasser la connaissance factuelle pour développer une compréhension des motivations et des contextes. Cette approche répond au déficit de connexion personnelle identifié dans la perception de l'Histoire par les élèves.

Les discussions de groupe structurées autour de problèmes historiques améliorent simultanément la compréhension conceptuelle et l'intérêt pour la discipline~\citep{delfavero2007classroom}. L'étude, portant sur la Première Guerre mondiale et le miracle économique italien, a montré que la confrontation des interprétations entre pairs activait des processus de réélaboration cognitive absents de l'apprentissage individuel.

\subsubsection*{Approches aux résultats mitigés ou négatifs}

Certaines technologies n'ont pas tenu leurs promesses en contexte historique.

Les agents pédagogiques scriptés présentent des résultats décevants en Histoire. Les méta-analyses montrent un effet négatif sur l'apprentissage dans ce domaine~\citep{davis2019effectiveness}. Cette contre-performance pourrait s'expliquer par l'inadéquation entre la rigidité des réponses préprogrammées et la nature interprétative de la discipline~: un agent incapable de nuancer son discours ou de reconnaître la pluralité des interprétations peut apparaître comme une source d'autorité inappropriée.

Les interfaces conversationnelles textuelles avec figures historiques produisent des résultats quantitatifs mixtes malgré des gains motivationnels~\citep{pataranutaporn2023living}. La comparaison entre dialogues interactifs et lecture traditionnelle indique une amélioration de la motivation mais des effets incertains sur l'apprentissage objectif. L'absence de modalité orale pourrait limiter l'activation des mécanismes de présence sociale.

La simple introduction d'outils technologiques sans repensée pédagogique ne garantit pas l'engagement~\citep{yarema2002using}. Les technologies qui reproduisent le format transmissif sous une forme numérique échouent à transformer la relation des élèves au contenu historique.

\subsubsection*{Conditions de succès transversales}

L'analyse comparative de ces expérimentations suggère trois conditions nécessaires à l'efficacité des technologies en enseignement de l'Histoire~: l'activation de l'agentivité de l'apprenant (les dispositifs efficaces offrent des choix significatifs qui engagent la réflexion), la dimension sociale de l'interaction (les approches les plus prometteuses impliquent une confrontation des perspectives), et la cohérence avec l'épistémologie disciplinaire (les technologies qui respectent la nature interprétative de l'Histoire surpassent celles qui importent un modèle transmissif adapté aux STIM).

Ces constats informent directement la conception de notre dispositif expérimental~: l'agent conversationnel incarnant un personnage historique vise précisément à satisfaire ces trois conditions en offrant une interaction dialogique authentique, socialement engageante, et épistémologiquement cohérente avec la discipline.



% ============================================================================
% Section 2.3 : Cadre Théorique de l'Interaction — Cognition, Multimédia, Présence Sociale
% ============================================================================
% ============================================================================
% Section 2.3 : Les Agents Pédagogiques Virtuels
% ============================================================================
% Objectif : État de l'art des agents pédagogiques : typologie, signaux sociaux,
%            efficacité et limites
% Sources : Rapport Analyse Research Paper Mapper (17 méta-analyses),
%           IJCCI_extraction, illusion_extraction, Vault.xlsx
% Calibrage : ~4 200 mots
% ============================================================================

\section{Les Agents Pédagogiques Virtuels~: Typologie, Signaux Sociaux et Efficacité}
\label{sec:agents_pedagogiques}

% ----------------------------------------------------------------------------
% Introduction de section (~200 mots)
% ----------------------------------------------------------------------------
Depuis le tuteur SCHOLAR de Carbonell en 1970, les agents pédagogiques virtuels ont connu trois générations de développement. La première (2000--2011) a exploré les possibilités des agents animés avec des systèmes comme Steve et Herman the Bug. La deuxième (2012--2019) a consolidé les bases empiriques à travers de nombreuses méta-analyses. La troisième (2020--présent) intègre les capacités des grands modèles de langage, transformant radicalement les possibilités d'interaction.

Cette section dresse un état de l'art systématique du domaine. Nous proposons d'abord une cartographie des agents pédagogiques à travers leur évolution historique et leurs caractéristiques de design (\S\ref{subsec:cartographie_agents}). Nous analysons ensuite les fondements cognitifs qui contraignent leur efficacité (\S\ref{subsec:fondements_cognitifs}), puis les mécanismes par lesquels ils génèrent une présence sociale (\S\ref{subsec:presence_sociale}). Une taxonomie des signaux sociaux et de leur efficacité empirique est ensuite présentée (\S\ref{subsec:taxonomie_signaux}). Nous examinons enfin les limites du réalisme (\S\ref{subsec:limites_realisme}).

% ----------------------------------------------------------------------------
% Sous-sections
% ----------------------------------------------------------------------------
% ============================================================================
% Sous-section 2.3.1 : Cartographie des Agents Pédagogiques Virtuels
% ============================================================================
% Sources : Rapport Analyse Research Paper Mapper (17 méta-analyses)
% Calibrage : ~800 mots
% Type : NOUVEAU
% ============================================================================

\subsection{Cartographie des Agents Pédagogiques~: 25 Ans d'Évolution}
\label{subsec:cartographie_agents}

L'histoire des agents pédagogiques virtuels peut être structurée en trois générations distinctes, chacune caractérisée par des paradigmes technologiques et des objectifs de recherche spécifiques~\citep{johnson2016face, alfaro2020new}.

La \textbf{première génération (1997--2011)} correspond à l'ère des pionniers. Les premiers agents pédagogiques animés émergent à la fin des années 1990 avec des systèmes comme Steve (\textit{Soar Training Expert for Virtual Environments})~\citep{johnson1998steve}, agent 3D guidant les marins dans l'apprentissage de procédures techniques, ou Herman the Bug~\citep{lester1997lifelike}, agent 2D anthropomorphisé sous forme d'insecte qui a démontré l'\textit{effet de persona}~: sa seule présence améliorait la motivation des apprenants, indépendamment de la qualité du feedback~\citep{heidig2011pedagogical}. AutoTutor~\citep{graesser2004autotutor} a introduit le dialogue socratique automatisé, posant les bases des agents conversationnels éducatifs. Ces systèmes pionniers reposaient sur des scripts prédéfinis et des arbres de décision, ce qui limitait leur flexibilité conversationnelle.

La \textbf{deuxième génération (2012--2019)} se caractérise par la consolidation empirique. Les méta-analyses se multiplient pour évaluer l'efficacité réelle des agents selon différentes variables de design~: effets de l'affect, impact des gestes, comparaisons entre apparences 2D et 3D, voix humaines et synthétiques, niveaux d'anthropomorphisme~\citep{guo2015affect, davis2018impact}. Un constat émerge de ces synthèses~: sur 15 études avec groupe contrôle, 9 ne montrent aucune différence d'apprentissage significative, appelant à une approche plus rigoureuse~\citep{heidig2011pedagogical}. Cette génération établit les bases empiriques du domaine, mais les agents restent contraints par leurs scripts préprogrammés.

La \textbf{troisième génération (2020--présent)} correspond à l'ère générative. L'émergence des grands modèles de langage transforme radicalement le paysage~: les agents peuvent désormais générer des réponses contextuellement appropriées sans scripts prédéfinis. Les chatbots IA en éducation produisent des effets positifs sur les résultats d'apprentissage, mais cette flexibilité conversationnelle accrue s'accompagne de nouveaux défis liés à l'imprévisibilité des réponses~\citep{wu2024chatbots, schroeder2025designing}. Cette génération, à laquelle appartient notre agent historique, hérite des acquis des générations précédentes tout en introduisant des risques inédits --- notamment celui des hallucinations analysé en section~\ref{sec:IA_generatives}.

Au-delà de la chronologie, les agents se différencient selon plusieurs dimensions de design. Le tableau~\ref{tab:typologie_agents} propose une typologie fondée sur les caractéristiques empiriquement étudiées dans la littérature, synthétisée à partir de 17 méta-analyses et revues systématiques.

\begin{table}[htbp]
\centering
\caption{Typologie des agents pédagogiques selon leurs caractéristiques de design. Synthèse établie à partir des méta-analyses de \citet{castroalonso2021effectiveness}, \citet{dai2024effects} et \citet{martha2018design}.}
\label{tab:typologie_agents}
\small
\begin{tabular}{p{2.8cm}p{4cm}p{4cm}p{3cm}}
\toprule
\textbf{Dimension} & \textbf{Modalités} & \textbf{Exemples} & \textbf{Effet sur apprentissage} \\
\midrule
\textit{Représentation visuelle} &
Absent / Statique / Animé 2D / Animé 3D &
Texte seul / Image fixe / Cartoon / Avatar 3D &
Animé > Statique~; 2D $\geq$ 3D \\
\addlinespace
\textit{Niveau d'anthropomorphisme} &
Non-humain / Stylisé / Réaliste / Hyperréaliste &
Robot / Cartoon / Avatar / Deepfake &
Stylisé souvent optimal \\
\addlinespace
\textit{Modalité vocale} &
Texte / Voix synthétique / Voix humaine &
Chat / TTS / Enregistrement &
Humaine > Synthétique \\
\addlinespace
\textit{Rôle pédagogique} &
Tuteur / Compagnon / Teachable agent &
Expert / Assistant / Élève virtuel &
Dépend du contexte \\
\addlinespace
\textit{Moteur conversationnel} &
Script / Règles / NLP-ML / LLM &
Arbre décision / AIML / Seq2Seq / GPT &
LLM~: flexibilité $\uparrow$, contrôle $\downarrow$ \\
\bottomrule
\end{tabular}
\end{table}

L'application des agents pédagogiques à l'enseignement de l'Histoire reste peu explorée et les résultats disponibles sont préoccupants~: l'effet mesuré pour l'histoire est négatif, contre des effets positifs en biologie et en informatique~\citep{castroalonso2021effectiveness}. Cette disparité disciplinaire s'explique par un défi épistémologique spécifique~: contrairement aux domaines STIM où l'agent peut s'appuyer sur des connaissances vérifiables, l'agent historique doit naviguer entre faits documentés, interprétations historiographiques et zones d'incertitude. Cette spécificité justifie une attention particulière aux risques métacognitifs que nous analysons dans cette thèse.

La figure~\ref{fig:evolution_agents} illustre cette évolution à travers deux exemples représentatifs des deux premières générations.

\begin{figure}[htbp]
\centering
% Placeholder pour figure composite
\fbox{\parbox{0.9\textwidth}{\centering\vspace{2cm}
\textit{Figure à insérer~: Deux agents représentatifs}\\[0.5em]
(a) Steve~\citep{johnson1998steve} --- Génération 1\\
(b) Agent expressif de \citet{torre2019} --- Génération 2
\vspace{2cm}}}
\caption{Évolution des agents pédagogiques à travers les deux premières générations. (a)~Steve, agent 3D pionnier pour l'entraînement naval~\citep{johnson1998steve}. (b)~Agent expressif utilisé dans les études sur l'alignement comportemental et contextuel~\citep{torre2019}.}
\label{fig:evolution_agents}
\end{figure}


\input{chapters/Ch2_EtatDeLArt/sections/2_3/2_3_2_fondements_cognitifs.tex}
% ============================================================================
% Sous-section 2.3.3 : De l'Artefact au Partenaire — Présence Sociale
% ============================================================================
% Sources : IJCCI_extraction (§2.5), illusion_extraction, Protocole CER
% Calibrage : ~600 mots
% Type : Révision fusionnant 2_3_2_CASA.tex et 2_3_3_agence_sociale.tex
% ============================================================================

\subsection{De l'Artefact au Partenaire~: Mécanismes de la Présence Sociale}
\label{subsec:presence_sociale}

La section précédente a établi les contraintes cognitives qui encadrent l'efficacité des agents. Mais l'impact d'un agent ne se réduit pas à sa fonction de canal d'information~: il dépend aussi de sa capacité à être perçu comme un partenaire social. Cette perception transforme l'interaction technique en expérience relationnelle.

Le paradigme CASA (\textit{Computers Are Social Actors}) constitue le cadre explicatif central de ce phénomène~\citep{nass2000machines}. Les travaux fondateurs de Nass et ses collaborateurs ont démontré que les utilisateurs appliquent inconsciemment aux ordinateurs les règles sociales qu'ils utiliseraient avec des humains~: politesse, réciprocité, attribution de personnalité. Ce traitement social tend à s'opérer même lorsque l'utilisateur \textit{sait} qu'il interagit avec une machine --- un processus qualifié d'\textit{ethopoeia}.

Des indices sociaux minimaux suffisent à déclencher ces scripts relationnels. Une voix, qu'elle soit synthétique ou humaine, active des réponses sociales. Les utilisateurs évaluent différemment un ordinateur selon le genre de sa voix, reproduisant les stéréotypes sociaux~: une voix masculine est perçue comme plus compétente sur les sujets techniques. Ces attributions automatiques expliquent pourquoi même des agents rudimentaires peuvent générer un engagement significatif.

La présence sociale désigne le sentiment d'être \og avec\fg{} une autre intelligence dans un environnement médiatisé~\citep{biocca2003toward}. Ce sentiment ne requiert pas un interlocuteur humain~; il peut émerger de l'interaction avec un agent virtuel pourvu que certaines conditions soient réunies. L'utilisateur doit percevoir l'agent non comme un outil passif, mais comme une entité dotée d'une forme d'intentionnalité et de réactivité.

Cette présence sociale constitue la condition préalable à l'activation des mécanismes d'apprentissage social. Sans elle, l'interaction tend à rester instrumentale~: l'utilisateur traite l'information sans s'engager pleinement dans la relation. Avec elle, l'agent devient un partenaire dont on respecte implicitement le \og contrat de communication\fg{}.

La théorie de l'agence sociale (\textit{Social Agency Theory}) explicite le lien entre présence sociale et apprentissage~\citep{moreno2001case, mayer2012embodiment}. Lorsque l'apprenant perçoit l'agent comme un partenaire social, un contrat implicite s'établit~: l'apprenant s'engage à \og honorer\fg{} cette relation par un effort cognitif accru. Cet effort supplémentaire se traduit par un traitement plus profond du contenu.

Les méta-analyses confirment ce mécanisme~: les agents anthropomorphisés produisent des effets supérieurs aux agents non anthropomorphisés, mais ces effets concernent davantage les mesures affectives et motivationnelles que les mesures cognitives pures~\citep{schroeder2025designing, dai2022meta}.


\input{chapters/Ch2_EtatDeLArt/sections/2_3/2_3_4_taxonomie_signaux.tex}
% ============================================================================
% Sous-section 2.3.5 : Limites du Réalisme dans les Agents Pédagogiques
% ============================================================================
% Sources : Rapport Analyse Research Paper Mapper
% Calibrage : ~500 mots
% Type : Révision de 2_3_4_uncanny_valley.tex
% ============================================================================

\subsection{Limites du Réalisme~: Le Paradoxe de l'Hyperréalisme}
\label{subsec:limites_realisme}

Les résultats empiriques présentés dans les sections précédentes convergent vers un constat contre-intuitif~: le réalisme accru des agents ne garantit pas une efficacité pédagogique supérieure. Cette section examine les mécanismes qui expliquent ce paradoxe et ses implications pour le design.

Les comparaisons directes entre agents réalistes et agents stylisés révèlent une absence de différence significative sur les mesures d'apprentissage~\citep{shiban2015appearance}. Les agents stylisés génèrent parfois un engagement supérieur, possiblement parce qu'ils évitent les attentes implicites associées à l'apparence humaine. Lorsqu'un agent ressemble étroitement à un humain, l'apprenant s'attend à des comportements pleinement humains~; les déviations --- latence de réponse, expressions faciales limitées, erreurs de compréhension --- peuvent générer une dissonance qui compromet l'interaction.

L'écart entre agents 2D et agents 3D illustre ce phénomène~\citep{castroalonso2021effectiveness}. Cette différence ne s'explique pas uniquement par la charge cognitive du réalisme visuel~: elle reflète également une inadéquation entre les capacités comportementales des agents et les attentes générées par leur apparence. Un agent 3D photoréaliste qui ne maintient pas un contact visuel approprié ou dont les expressions faciales restent rigides crée une impression d'étrangeté absente chez un agent 2D dont les limitations sont explicitement acceptées.

Ce décalage constitue un défi persistant du domaine~\citep{johnson2016face}. Les progrès technologiques permettent désormais des rendus visuels sophistiqués, mais les comportements interactifs --- synchronisation multimodale, gestion des tours de parole, réactivité émotionnelle --- n'ont pas progressé au même rythme. Cette asymétrie crée une vallée de l'étrange (\textit{uncanny valley}) où l'agent est suffisamment réaliste pour activer les attentes sociales, mais insuffisamment capable pour les satisfaire.

Les agents hyperréalistes peuvent également réduire l'engagement par un mécanisme psychologique distinct~\citep{dai2022systematic}~: face à un agent perçu comme ``intelligent'' et ``compétent'', certains apprenants développent une anxiété de performance qui inhibe leur participation. Les agents stylisés, perçus comme moins menaçants, favoriseraient une interaction plus détendue et un engagement plus authentique.

Ces résultats ont des implications directes pour le design des agents historiques. Un personnage historique rendu de manière hyperréaliste risque de générer des attentes impossibles à satisfaire~: l'apprenant s'attend à une interaction ``comme avec un vrai humain'', alors que l'agent reste limité par ses capacités techniques et la fiabilité de ses connaissances. Une représentation stylisée, explicitement non réaliste, pourrait paradoxalement favoriser une interaction plus productive en établissant d'emblée les limites de l'échange.

Cette analyse conduit à reformuler la question du design~: plutôt que de maximiser le réalisme, il s'agit d'optimiser la congruence entre l'apparence de l'agent, ses capacités comportementales et les attentes de l'apprenant. Ces principes, établis pour les agents traditionnels à scripts prédéfinis, doivent être réexaminés à la lumière de la rupture technologique introduite par les IA génératives, dont les capacités conversationnelles modifient substantiellement les modalités d'interaction.




% ============================================================================
% Section 2.4 : La Rupture Technologique des IA Génératives en Éducation
% ============================================================================
% ============================================================================
% Section 2.4 : La Rupture Technologique des IA Génératives en Éducation
% ============================================================================
% Objectif : Contextualiser l'étude dans l'ère post-2023 et les nouvelles
%            affordances des LLM
% Calibrage : ~100 mots (introduction)
% ============================================================================

\section{La Rupture Technologique des IA Génératives en Éducation}
\label{sec:IA_generatives}

Les cadres théoriques présentés jusqu'ici ont été élaborés dans un contexte technologique où les agents pédagogiques fonctionnaient sur la base de scripts prédéfinis. L'émergence des Grands Modèles de Langage (LLM) et des technologies de synthèse multimédia a transformé ce paysage, créant de nouvelles possibilités mais aussi de nouveaux risques.

Cette section examine le saut qualitatif représenté par les agents génératifs (\S\ref{subsec:agents_generatifs}), leurs capacités de personnalisation en temps réel (\S\ref{subsec:personnalisation_temps_reel}), et le défi épistémique posé par leur propension aux hallucinations (\S\ref{subsec:hallucinations}).

% Inclusion des sous-sections
% ============================================================================
% Sous-section 2.4.1 : Des Agents Scriptés aux Agents Génératifs
% ============================================================================
% Sources : IJCCI_extraction (§2.4), illusion_extraction (§2.2.3)
% Calibrage : ~500 mots
% Type : C (Rédaction originale)
% ============================================================================

\subsection{Des Agents Scriptés aux Agents Génératifs~: le Saut Qualitatif}
\label{subsec:agents_generatifs}

Les agents pédagogiques ont évolué au cours des dernières décennies, passant de simples outils de diffusion d'information à des partenaires d'apprentissage plus sophistiqués~\citep{johnson2016face}. L'émergence des Grands Modèles de Langage (\textit{Large Language Models}, LLM) a particulièrement accéléré cette évolution.

Les agents pédagogiques traditionnels fonctionnaient sur la base de scripts prédéfinis et d'arbres de décision. Leur comportement était entièrement déterminé par les anticipations de leurs concepteurs~: chaque question possible devait être prévue, chaque réponse pré-rédigée. Cette architecture présentait des avantages --- prévisibilité, contrôle du contenu, absence d'erreurs factuelles --- mais aussi des limites fondamentales. La rigidité constituait le principal écueil~: l'agent ne pouvait répondre qu'aux questions anticipées, dans les formulations anticipées. Toute déviation se heurtait à des réponses génériques. Cette rigidité contrastait avec la fluidité du dialogue humain et limitait le sentiment de présence sociale.

Les LLM ont transformé cette équation. Ces modèles peuvent générer des réponses personnalisées, s'adapter aux besoins des élèves en temps réel, et produire du contenu éducatif contextuellement pertinent~\citep{kasneci2023chatgpt, labadze2023role}. Ces avancées s'inscrivent dans la continuité des travaux sur les environnements d'apprentissage multimodaux interactifs~\citep{moreno2007interactive}. La fluidité conversationnelle des LLM constitue leur caractéristique la plus distinctive~: ils génèrent un discours structuré, linguistiquement cohérent, et adapté au contexte de l'échange. Cette fluidité peut activer les mécanismes d'agence sociale décrits en \S\ref{subsec:agence_sociale}~: l'apprenant perçoit l'agent comme un interlocuteur plutôt qu'un système automatique.

Cette évolution s'accompagne d'avancées parallèles en synthèse multimédia. Les réseaux antagonistes génératifs (GAN) permettent de créer des représentations visuelles hyperréalistes, brouillant la frontière entre réel et artificiel~\citep{whittaker2020deepfakes}. L'animation faciale par apprentissage profond, le clonage vocal, et la génération vidéo atteignent des niveaux de réalisme inédits. Ces technologies peuvent atténuer l'effet de vallée de l'étrange (cf. \S\ref{subsec:uncanny_valley}), produisant des agents synthétiques perçus comme attractifs~\citep{xu2025recorded}. Certains travaux indiquent que ces agents peuvent atteindre des niveaux de performance et de perception comparables à ceux d'instructeurs humains~\citep{leiker2023generative, lim2024potential}.

C'est précisément cette convergence --- moteurs conversationnels fluides et interfaces visuelles réalistes --- qui multiplie les enjeux. D'un côté, des \og moteurs\fg{} capables de produire un discours éloquent~; de l'autre, des \og interfaces\fg{} visuelles pour les incarner. Cette combinaison ouvre des possibilités pédagogiques inédites, mais comporte également des risques que les sections suivantes examineront.

% ============================================================================
% Sous-section 2.4.2 : La Personnalisation en Temps Réel
% ============================================================================
% Sources : IJCCI_extraction (§2.4), Leong et al., Pataranutaporn et al.
% Calibrage : ~450 mots
% Type : C (Rédaction originale)
% ============================================================================

\subsection{La Personnalisation en Temps Réel}
\label{subsec:personnalisation_temps_reel}

L'une des capacités distinctives des agents alimentés par LLM réside dans leur aptitude à personnaliser le contenu d'apprentissage en temps réel, sans nécessiter de programmation préalable pour chaque profil d'apprenant. Cette capacité répond aux défis de mise en œuvre identifiés en \S\ref{subsec:personnalisation}.

L'apprentissage adaptatif traditionnel reposait sur des algorithmes prédéfinis~: en fonction des réponses de l'élève à des questions diagnostiques, le système orientait vers des parcours prédéterminés~\citep{walkington2018personalization}. Les LLM permettent une forme d'adaptation plus fine~: l'agent peut moduler son vocabulaire, ses exemples, son niveau de détail en fonction du flux de la conversation elle-même.

L'étude de la personnalisation dans l'apprentissage du vocabulaire illustre ce potentiel~: un système développant des exemples et des récits adaptés aux intérêts individuels conduit à une augmentation de la motivation intrinsèque et à un sentiment renforcé de compétence et d'autonomie~\citep{leong2024putting}. Ces résultats démontrent la faisabilité d'une personnalisation automatique à grande échelle. L'exploration des interactions conversationnelles avec des figures historiques à travers des interfaces textuelles indique une amélioration de la motivation et des résultats d'apprentissage comparés à la lecture traditionnelle~\citep{pataranutaporn2023living}. L'évaluation de l'interactivité textuelle en éducation financière confirme que permettre aux étudiants de dialoguer avec l'instructeur virtuel conduit à une motivation et un engagement accrus comparés à l'instruction vidéo passive~\citep{prasongpongchai2024influence}.

Ce qui distingue la personnalisation par LLM est son caractère émergent. L'adaptation n'est pas programmée explicitement~: elle émerge de la capacité du modèle à générer des réponses contextuellement appropriées. L'élève pose une question selon ses propres termes, l'agent répond en s'adaptant. Si l'élève manifeste une incompréhension, l'agent peut reformuler spontanément. Cette fluidité adaptative présente un avantage pédagogique~: chaque interaction devient unique, calibrée sur les besoins du moment.

Elle présente également un risque~: l'adaptation peut masquer l'absence de compréhension réelle. L'agent qui reformule efficacement peut donner l'impression à l'élève qu'il a compris, alors que c'est l'agent qui a simplifié son discours au point de ne plus transmettre le concept dans sa complexité. Ce mécanisme constitue l'une des sources potentielles de l'illusion de compréhension.

L'examen des agents conversationnels conçus pour favoriser la curiosité chez les enfants du primaire révèle des résultats prometteurs~: un agent encourageant le questionnement divergent conduit à une amélioration de la qualité des questions et à des activités exploratoires soutenues~\citep{abdelghani2024exploring}. Ces résultats suggèrent que la personnalisation peut être mise au service de l'engagement cognitif authentique plutôt que de la simple facilitation.

% ============================================================================
% Sous-section 2.4.3 : Fiabilité et Hallucinations — Le Défi Épistémique
% ============================================================================
% Sources : illusion_extraction (§2.2.2), Zhang et al. (2025)
% Calibrage : ~450 mots
% Type : C (Rédaction originale)
% ============================================================================

\subsection{Fiabilité et Hallucinations~: le Défi Épistémique}
\label{subsec:hallucinations}

La fluidité conversationnelle des LLM masque une faille intrinsèque~: leur propension à générer des informations factuellement incorrectes présentées avec une confiance apparente. Ce phénomène, qualifié d'\og hallucination\fg{}, représente un défi épistémique majeur pour les applications éducatives.

Les LLM sont des modèles probabilistes qui prédisent le mot suivant le plus probable étant donné le contexte. Cette nature stochastique implique qu'ils ne \og connaissent\fg{} pas les faits au sens humain~: ils génèrent des séquences statistiquement plausibles~\citep{zhang2025sirens}. Une taxonomie des hallucinations distingue les \textit{hallucinations factuelles} (assertions fausses sur le monde), les \textit{hallucinations de fidélité} (déformations de l'information source), et les \textit{hallucinations d'entrée} (fabrication d'éléments non présents dans la requête). Dans un contexte éducatif, chaque type présente des risques spécifiques~: date erronée, citation inventée, personnage historique attribué à la mauvaise période.

L'enseignement de l'Histoire présente une vulnérabilité particulière à ces hallucinations. La discipline repose sur des faits précis --- dates, lieux, protagonistes --- dont l'exactitude est vérifiable. Or, les LLM excellent dans la production de récits plausibles et cohérents~; ils peuvent générer une narration parfaitement fluide qui contient néanmoins des erreurs factuelles. Cette tension est d'autant plus problématique que la fluidité du discours constitue un signal de crédibilité (cf. \S\ref{subsec:fluency_heuristic})~: un récit bien construit est intuitivement perçu comme plus vrai qu'un récit hésitant~\citep{reber1999effects}. Le LLM, en produisant un discours maximalement fluide, maximise cette heuristique --- indépendamment de la véracité de ce qu'il affirme.

Les avancées technologiques soulèvent plusieurs défis connexes. L'engagement des élèves peut fluctuer en raison d'effets de nouveauté~\citep{fryer2019bots}, tandis que les questions de fiabilité, de biais, de confidentialité et d'intégrité académique nécessitent une attention particulière~\citep{labadze2023role, dempere2023impact, berson2024childrens}. Au-delà de l'établissement de lignes directrices pour l'intégration de l'IA en éducation, des opportunités de recherche existent pour examiner comment l'interaction verbale avec des agents pourrait compléter les pratiques pédagogiques actuelles. La modalité orale, en tirant parti des dynamiques naturelles de la classe, pourrait créer des patterns d'engagement différents comparés aux interactions textuelles individuelles~\citep{moreno2007interactive}.

Ces considérations informent directement notre programme expérimental. L'Étude~2, en exposant les participants à un agent capable de produire des informations historiques incorrectes mais présentées avec fluidité, teste précisément ce risque~: la fluidité de l'agent conduit-elle les élèves à accepter des informations fausses et à surestimer leur propre compréhension?



% ============================================================================
% Section 2.5 : Le Phénomène de l'Illusion de Compréhension
% ============================================================================
% ============================================================================
% Section 2.5 : Le Phénomène de l'Illusion de Compréhension (IoU)
% ============================================================================
% Objectif : Définir le risque majeur exploré par la thèse,
%            revers de la médaille de la fluidité
% Sources : illusion_extraction (§2.2.1, §2.2.2), Vault.xlsx
% Calibrage : ~2 000 mots
% ============================================================================

\section{Le Phénomène de l'Illusion de Compréhension}
\label{sec:illusion_comprehension}

La fluidité des agents conversationnels génératifs constitue leur force pédagogique principale --- mais aussi leur risque le plus insidieux. Lorsqu'un agent présente l'information de manière parfaitement articulée et convaincante, l'apprenant peut confondre la facilité de réception avec la qualité de sa propre compréhension. Ce phénomène, que nous désignons par \textit{illusion de compréhension} (ou \textit{Illusion of Understanding}, IoU), représente le risque métacognitif central examiné par cette thèse. Cette section analyse d'abord les fondements cognitifs de ce phénomène à travers l'Illusion de Profondeur Explicative (\S\ref{subsec:metacognition_IOED}), puis examine le mécanisme de l'heuristique de fluidité (\S\ref{subsec:fluency_heuristic}), avant d'explorer comment l'autorité perçue de l'agent peut compromettre la vigilance épistémique (\S\ref{subsec:autorite_vigilance}).

% ----------------------------------------------------------------------------
% Sous-sections
% ----------------------------------------------------------------------------
% ============================================================================
% Sous-section 2.5.1 : Métacognition et Calibration de la Confiance
% ============================================================================
% Sources : illusion_extraction (§2.2.1), illusion_ai_agent_draft
% Calibrage : ~550 mots
% Type : C (Rédaction originale)
% ============================================================================

\subsection{Métacognition et Calibration de la Confiance}
\label{subsec:metacognition_IOED}

La métacognition désigne la cognition sur la cognition~: la capacité d'un individu à surveiller et réguler ses propres processus cognitifs~\citep{flavell1979metacognition}. Dans le contexte de l'apprentissage, cette capacité se manifeste par deux fonctions distinctes~: le \textit{monitoring}, qui consiste à évaluer sa propre compréhension, et le \textit{control}, qui permet d'ajuster ses stratégies en conséquence. L'auto-évaluation de la compréhension constitue un processus intrinsèquement faillible~\citep{glenberg1982automatic}~: les apprenants croient fréquemment avoir compris un contenu alors même qu'ils échouent à détecter des contradictions évidentes.

Ce déficit métacognitif trouve sa formalisation dans le concept d'\textit{Illusion of Explanatory Depth} (IOED). Ce phénomène désigne la tendance des individus à surestimer la profondeur de leur compréhension des systèmes causaux complexes --- jusqu'au moment où ils sont contraints de produire une explication détaillée~\citep{rozenblit2002misunderstood}. Le protocole expérimental classique procède en trois phases~: une auto-évaluation initiale (T1), la production d'une explication écrite, puis une seconde auto-évaluation (T2). La révélation est systématique~: confrontés à l'exercice d'explication, les participants découvrent que leur compréhension était moins profonde qu'ils ne le croyaient, se traduisant par une chute significative entre T1 et T2.

Ce phénomène s'avère pertinent pour l'étude des interactions avec les agents conversationnels. Certaines modalités de présentation peuvent donner l'impression que le matériel est facile à traiter~\citep{paik2013effects}. Cette facilité apparente conduit les apprenants à sous-estimer la difficulté de la tâche et à développer une métacompréhension excessivement optimiste. Le résultat est une dissociation entre la confiance subjective, qui augmente, et l'apprentissage objectif, qui stagne.

Cette illusion s'inscrit dans un déficit métacognitif plus large. Le phénomène décrit par l'effet Dunning-Kruger met en lumière un \og double fardeau\fg{}~: les individus les moins compétents manquent également des compétences métacognitives nécessaires pour reconnaître leur propre incompétence~\citep{kruger1999unskilled}. Dans le contexte d'une interaction avec un agent conversationnel fluide, ce déficit devient particulièrement problématique~: la facilité apparente de l'échange peut renforcer une confiance injustifiée.

La notion de \textit{calibration} désigne l'écart entre la confiance subjective et la performance objective~\citep{koriat2008subjective}. Les études révèlent une tendance systématique à la \textit{surconfiance}~: les apprenants surestiment leur niveau de compréhension. Cette surconfiance constitue un prédicteur robuste de mauvais apprentissage, car elle conduit à un désengagement prématuré de l'effort cognitif.

Dans le contexte des agents alimentés par IA générative, ce problème prend une dimension nouvelle. Un \og paradoxe métacognitif\fg{} émerge~: bien que l'assistance de l'IA puisse améliorer la performance immédiate, elle dégrade la capacité de l'utilisateur à évaluer cette même performance~\citep{fernandes2026metacognitive}. L'agent, en fournissant des réponses instantanées et fluides, prive l'apprenant des \og difficultés désirables\fg{} --- l'effort de construction et de réorganisation des connaissances --- pourtant essentielles à un apprentissage durable~\citep{bjork2013self}.

% ============================================================================
% Sous-section 2.5.2 : L'Heuristique de Fluidité (Fluency Heuristic)
% ============================================================================
% Sources : illusion_extraction (§2.2.1, §2.2.2), illusion_ai_agent_draft
% Calibrage : ~550 mots
% Type : C (Rédaction originale)
% ============================================================================

\subsection{L'Heuristique de Fluidité}
\label{subsec:fluency_heuristic}

Le mécanisme central de l'illusion de compréhension réside dans ce que la littérature désigne par \textit{processing fluency}~: l'information facile à percevoir et à traiter cognitivement est vécue comme familière~\citep{reber1999effects}. Cette facilité de traitement est alors attribuée par erreur à la propre maîtrise du sujet par l'apprenant plutôt qu'aux qualités de la présentation. L'heuristique de fluidité constitue un raccourci cognitif par lequel nous jugeons plus vraie, plus crédible et mieux comprise l'information qui \og passe bien\fg{}.

La simple facilité de perception d'un énoncé --- police lisible, articulation claire, formulation syntaxiquement simple --- augmente sa crédibilité perçue, indépendamment de sa véracité objective. L'individu interprète la facilité cognitive comme un signal de familiarité, et la familiarité comme un indice de vérité. Dans le contexte des agents conversationnels alimentés par LLM, cette heuristique prend une dimension particulière~: ces modèles possèdent des caractéristiques techniques qui maximisent la fluidité de traitement --- discours instantané, parfaitement structuré et linguistiquement fluide~\citep{shanahan2024talking}. Cette double fluidité --- linguistique et auditive lorsque couplée à une synthèse vocale --- crée les conditions idéales pour que l'heuristique opère.

Une variante pertinente est l'\textit{instructor fluency effect}, qui décrit comment les comportements non-verbaux de l'enseignant --- dynamisme, contact visuel, fluidité verbale --- peuvent biaiser les jugements d'apprentissage~\citep{toftness2018instructor}. Les apprenants confondent la qualité de la délivrance pédagogique avec la qualité de leur propre apprentissage, rapportant une confiance élevée sans gains de performance correspondants. La présence visuelle d'un instructeur peut augmenter la satisfaction et l'apprentissage perçu tout en détournant l'attention du contenu~\citep{wilson2018instructor}. L'apprenant se trouve dans une situation paradoxale où son expérience subjective positive masque un apprentissage objectivement dégradé.

Au-delà de la compréhension conceptuelle, l'heuristique de fluidité peut générer une \og illusion d'acquisition de compétence\fg{}~: l'observation passive d'une démonstration peut conduire l'observateur à confondre la fluidité de traitement visuel avec sa propre capacité à exécuter la tâche~\citep{kardas2018illusion}. Par analogie, un apprenant qui observe un agent expliquer un phénomène avec aisance peut confondre la clarté de la présentation avec sa propre maîtrise du sujet.

La convergence de ces mécanismes crée un risque métacognitif majeur. L'interaction conversationnelle tend à augmenter la crédibilité perçue et à réduire la détection des inexactitudes par rapport au texte statique, car elle active des heuristiques sociales qui diminuent la vigilance critique~\citep{anderl2024conversational}. Même lorsque le contenu n'est pas jugé globalement plus crédible, il est souvent perçu comme plus clair et plus engageant --- une qualité qui peut conduire à une acceptation non critique~\citep{huschens2023unambiguous}. Cette configuration favorise une forme de \og paresse métacognitive\fg{}, où les apprenants délèguent les processus cognitifs coûteux et renoncent à l'effort de construction personnelle des connaissances~\citep{fan2023metacognitive}.

% ============================================================================
% Sous-section 2.5.3 : L'Autorité de l'Agent et la Vigilance Épistémique
% ============================================================================
% Sources : illusion_extraction (§2.2.1, §2.2.2), illusion_ai_agent_draft
% Calibrage : ~550 mots
% Type : C (Rédaction originale)
% ============================================================================

\subsection{L'Autorité de l'Agent et la Vigilance Épistémique}
\label{subsec:autorite_vigilance}

Au-delà de l'illusion de compréhension --- qui constitue une erreur d'auto-évaluation ---, les agents conversationnels génèrent un second risque épistémique~: la surconfiance accordée à la source elle-même. Cette distinction conceptuelle mérite une analyse séparée.

La surconfiance épistémique ne désigne pas une erreur d'auto-évaluation, mais une erreur concernant l'information externe~\citep{kulgemeyer2023epistemic}. Ce biais se manifeste lorsque les apprenants accordent une confiance excessive à des informations factuellement incorrectes parce que la source apparaît autoritaire ou que l'explication est intuitivement séduisante. Les explications fondées sur des conceptions erronées, parce qu'elles s'alignent avec l'expérience quotidienne, sont souvent jugées plus convaincantes que des explications scientifiquement correctes mais contre-intuitives. Cette dynamique prend une dimension critique face aux \og hallucinations\fg{} des LLM --- la génération d'informations incorrectes présentées avec assurance~\citep{zhang2025hallucination}. La fluidité discursive de l'agent crée les conditions d'une acceptation non critique d'informations potentiellement fausses.

L'\textit{illusory truth effect} constitue un mécanisme cognitif qui contribue à cette surconfiance~\citep{fazio2015knowledge}. La simple répétition d'un énoncé, même faux, augmente sa crédibilité perçue en accroissant sa familiarité. Dans le contexte d'interactions répétées avec un agent, ce mécanisme peut consolider des croyances erronées initialement introduites par une hallucination. Il convient toutefois de distinguer cet effet de l'illusion de maîtrise mesurée par le protocole IOED~: l'effet de vérité illusoire constitue une variable explicative de la persistance des fausses croyances plutôt qu'une mesure de l'auto-évaluation. Les deux phénomènes peuvent néanmoins se renforcer~: un apprenant qui croit maîtriser un sujet sera moins enclin à questionner les informations reçues.

L'anthropomorphisme --- notre tendance à attribuer des caractéristiques humaines à des entités non-humaines~\citep{epley2007seeing} --- joue un rôle central dans l'attribution d'autorité aux agents. L'apparence visuelle déclenche ce processus, mais les travaux récents suggèrent que la fluidité conversationnelle pourrait constituer un signal social de premier ordre~\citep{nass1994computers}. Cette dynamique s'avère préoccupante pour les jeunes apprenants, dont la tendance naturelle à l'anthropomorphisme est plus prononcée~\citep{kidd2023anthropomorphism}. Les adolescents, dont les modèles mentaux de la technologie sont en développement, sont plus dépendants des indices de surface pour évaluer la crédibilité~\citep{andries2023trust}.

La convergence de ces mécanismes --- fluidité de traitement, autorité perçue, anthropomorphisme --- tend à désactiver la vigilance épistémique de l'apprenant. Les explications trompeuses générées par l'IA peuvent être plus persuasives que les explications honnêtes~\citep{danry2024don}. Le \og halo de confiance\fg{} observé dans les interactions avec les agents --- où les élèves généralisent la compétence perçue à l'ensemble du domaine --- agit comme un \og bouclier de crédibilité\fg{}. L'apprenant, ayant catégorisé l'agent comme une source fiable, cesse d'appliquer les filtres critiques qu'il mobiliserait face à une source dont l'autorité n'est pas établie. Ce phénomène de \og négligence épistémique\fg{} opère de manière inconsciente, échappant à la régulation métacognitive de l'individu.


% Transition vers Section 2.6
L'analyse de l'illusion de compréhension révèle une tension fondamentale~: les mêmes caractéristiques qui rendent les agents conversationnels engageants --- fluidité, réalisme, crédibilité --- peuvent simultanément compromettre la profondeur de l'apprentissage en désactivant les mécanismes de vigilance critique. Cette tension entre engagement et vigilance constitue le fil conducteur de notre problématique de recherche, qu'il convient maintenant de formuler explicitement.


% ============================================================================
% Section 2.6 : Synthèse et Problématique
% ============================================================================
\input{chapters/Ch2_EtatDeLArt/sections/2_6/2_6.tex}
