%%%%%%%%%%%%%%%%%%%%%%%%%%%%%%%%%%%%%%%%%%%%%%%%%%%%%%%%%%%%%%%%%
% CHAPITRE 2 : ÉTAT DE L'ART
% Fichier d'orchestration principal
%%%%%%%%%%%%%%%%%%%%%%%%%%%%%%%%%%%%%%%%%%%%%%%%%%%%%%%%%%%%%%%%%
% Ce fichier orchestre l'inclusion des différentes sections et
% sous-sections du chapitre 2. Chaque sous-section est rédigée
% dans un fichier séparé pour faciliter la maintenance.
%
% Structure prévue (~14500 mots) :
%   2.1. Cadre Théorique de l'Apprenant (~3000 mots)
%   2.2. Le Contexte Disciplinaire : Histoire (~2000 mots)
%   2.3. Cadre Théorique de l'Interaction (~3500 mots)
%   2.4. L'Émergence des IA Génératives en Éducation (~2500 mots)
%   2.5. Le Phénomène de l'Illusion de Compréhension (~2500 mots)
%   2.6. Synthèse et Problématique (~1000 mots)
%%%%%%%%%%%%%%%%%%%%%%%%%%%%%%%%%%%%%%%%%%%%%%%%%%%%%%%%%%%%%%%%%

\chapter{État de l'Art}
\label{ch:etat-art}

\sommairechapitre

%%%%%%%%%%%%%%%%%%%%%%%%%%%%%%%%%%%%%%%%%%%%%%%%%%%%%%%%%%%%%%%%%
% INTRODUCTION DU CHAPITRE (optionnelle, courte)
%%%%%%%%%%%%%%%%%%%%%%%%%%%%%%%%%%%%%%%%%%%%%%%%%%%%%%%%%%%%%%%%%
% Note : Pas de paragraphe introductif selon ResearchStyle.txt
% Le chapitre commence directement par la première section.

%%%%%%%%%%%%%%%%%%%%%%%%%%%%%%%%%%%%%%%%%%%%%%%%%%%%%%%%%%%%%%%%%
% 2.1. CADRE THÉORIQUE DE L'APPRENANT
%%%%%%%%%%%%%%%%%%%%%%%%%%%%%%%%%%%%%%%%%%%%%%%%%%%%%%%%%%%%%%%%%
\section{Cadre Théorique de l'Apprenant : Motivation, Intérêt et Pertinence}
\label{sec:cadre-apprenant}

% 2.1.1. Dynamiques Motivationnelles : SDT et CET
% ============================================================================
% SOUS-SECTION 2.1.1 - Dynamiques Motivationnelles : SDT et CET
% ============================================================================

\subsection{Dynamiques Motivationnelles : SDT et CET}

La compréhension des processus d'apprentissage repose sur l'identification des mécanismes psychologiques qui orientent le comportement des apprenants. La Théorie de l'Autodétermination (SDT) propose un cadre explicatif articulé autour de trois besoins psychologiques fondamentaux dont la satisfaction conditionne le développement d'une motivation de qualité \citep{DeciRyan2000}. Le besoin d'autonomie correspond à la nécessité de se percevoir comme l'agent causal de ses propres actions, plutôt que contrôlé par des forces externes. Le besoin de compétence renvoie au sentiment d'efficacité dans ses interactions avec l'environnement et à la capacité d'atteindre les résultats souhaités. Le besoin d'affiliation, enfin, traduit la recherche de connexions sociales significatives et le sentiment d'appartenance à un groupe \citep{deci1985}. Ces trois besoins ne fonctionnent pas de manière isolée : leur satisfaction conjointe dans un contexte donné détermine le degré d'internalisation des comportements et la qualité de l'engagement dans une activité. Les environnements qui offrent des choix significatifs, des défis adaptés au niveau de l'apprenant, des retours constructifs et des interactions sociales soutenantes tendent à favoriser le développement d'une motivation autodéterminée \citep{DeciRyan2000}.

La Théorie de l'Évaluation Cognitive (CET), intégrée à la SDT, précise les mécanismes par lesquels les événements externes modulent la motivation intrinsèque \citep{DeciRyan1985}. Selon ce cadre théorique, l'effet d'un événement sur la motivation dépend de son interprétation par l'individu selon trois dimensions fonctionnelles. Les aspects informationnels fournissent un retour sur la compétence de l'individu : un feedback positif tend à renforcer le sentiment d'efficacité et, par conséquent, la motivation intrinsèque. Les aspects contrôlants exercent une pression sur le comportement et orientent l'individu vers des résultats spécifiques, ce qui déplace le locus de causalité perçu vers l'extérieur et tend à diminuer la motivation intrinsèque. Les aspects amotivants, enfin, signalent une incompétence et sapent à la fois le sentiment d'efficacité et l'envie de poursuivre l'activité. Un même événement, tel qu'une récompense ou un feedback de l'enseignant, peut être interprété différemment selon le contexte et la manière dont il est présenté. Une récompense perçue comme une reconnaissance de compétence aura un effet différent d'une récompense perçue comme un moyen de contrôle comportemental \citep{DeciRyan1985}.

L'application de ces cadres théoriques aux contextes éducatifs révèle une tendance préoccupante : la motivation intrinsèque tend à décliner tout au long de la scolarité, avec une inflexion particulièrement marquée lors de la transition vers l'enseignement secondaire \citep{GnambsHanfstingl2016}. Cette période se caractérise non seulement par une augmentation de la pression liée à la performance, mais également par des changements biologiques et une élévation des niveaux d'anxiété chez les élèves \citep{GnambsHanfstingl2016}. Les structures scolaires traditionnelles, en limitant les opportunités d'exploration et en privilégiant des formats d'enseignement transmissifs, peuvent restreindre l'expression de la curiosité naturelle des élèves \citep{Engel2009, Engel2011}. Ce déclin motivationnel ne constitue pas une fatalité développementale mais reflète en partie l'inadéquation entre les besoins psychologiques des apprenants et les caractéristiques de leur environnement éducatif. Les disciplines perçues comme éloignées des préoccupations quotidiennes des élèves, telles que l'histoire, peuvent se trouver particulièrement affectées par ce phénomène.

La motivation, comprise comme la direction et l'intensité de l'effort vers un but, se distingue conceptuellement de l'intérêt, qui désigne un état psychologique caractérisé par une attention focalisée, un affect positif et une volonté de réengagement avec un contenu spécifique \citep{Bergin1999}. Alors que la motivation peut être orientée vers des objectifs extrinsèques sans rapport avec le contenu lui-même, l'intérêt implique une relation particulière avec un domaine ou une activité spécifique. Une activité est considérée comme intrinsèquement motivante lorsqu'elle est poursuivie en l'absence de récompense externe apparente \citep{DeciPorac1978}. L'intérêt, en revanche, inclut des émotions positives envers l'objet et une activité autodirigée, non instrumentale \citep{Bergin1999}. Cette distinction possède des implications pratiques : un élève peut être motivé à réussir un examen d'histoire pour des raisons extrinsèques sans pour autant développer un intérêt pour la discipline elle-même.

% ============================================================================
% RÉFÉRENCES UTILISÉES DANS CETTE SOUS-SECTION :
% - Deci & Ryan (1985) : CET, mécanismes de la motivation intrinsèque
% - Deci & Ryan (2000) : SDT, trois besoins psychologiques fondamentaux
% - Gnambs & Hanfstingl (2016) : déclin motivation intrinsèque, transition secondaire
% - Engel (2009, 2011) : structures scolaires et curiosité
% - Bergin (1999) : distinction intérêt/motivation
% - Deci & Porac (1978) : définition motivation intrinsèque
% ============================================================================


% 2.1.2. Architecture de l'Intérêt
% % ============================================================================
% SOUS-SECTION 2.1.2 - Architecture de l'Intérêt
% ============================================================================

\subsection{Architecture de l'Intérêt}

L'intérêt constitue un état psychologique qui combine deux dimensions interdépendantes \citep{RenningerHidi2015}. La dimension affective se manifeste par une expérience émotionnelle positive associée à l'engagement avec un contenu particulier : plaisir, curiosité, sentiment de fascination. La dimension cognitive, quant à elle, correspond au désir de comprendre et d'approfondir ses connaissances dans un domaine spécifique. Cette dualité distingue l'intérêt d'autres construits apparentés : contrairement à l'attention, qui peut être captée par n'importe quel stimulus saillant, l'intérêt implique une valence positive et une orientation vers la compréhension \citep{Hidi2006}. Contrairement au simple plaisir, il comporte une composante épistémique qui pousse à la recherche active d'information. Cette double nature explique pourquoi l'intérêt exerce une influence sur la qualité et la persistance des apprentissages : il soutient à la fois l'investissement émotionnel nécessaire pour maintenir l'effort et le traitement cognitif profond requis pour la construction de connaissances durables \citep{RenningerHidi2015}.

La recherche distingue deux formes principales d'intérêt qui s'inscrivent dans un continuum développemental \citep{Priniski2018}. L'intérêt situationnel émerge en réponse à des caractéristiques de l'environnement : une activité nouvelle, un problème intrigant, une présentation engageante, ou une interaction sociale stimulante \citep{Krapp1992}. Cette forme d'intérêt est transitoire et dépend largement du contexte immédiat. L'intérêt individuel, en revanche, représente une disposition relativement stable envers un domaine de connaissance ou un type d'activité \citep{Schiefele1991}. Il se caractérise par une tendance à se réengager volontairement avec le contenu au fil du temps, indépendamment des sollicitations externes. La transition de l'intérêt situationnel vers l'intérêt individuel constitue un processus d'internalisation : ce qui était initialement soutenu par l'environnement devient progressivement autodéterminé \citep{Priniski2018}. Cette distinction possède des implications directes pour l'enseignement : si l'intérêt individuel préexistant facilite l'apprentissage, seul l'intérêt situationnel se trouve directement sous l'influence des choix pédagogiques de l'enseignant \citep{Bergin1999}.

Le développement de l'intérêt suit une progression en quatre phases distinctes \citep{HidiRenninger2006}. La première phase, l'intérêt situationnel déclenché, correspond à une réponse attentionnelle initiale face à des stimuli environnementaux tels qu'une information inattendue, une présentation inhabituelle ou une activité qui remet en question les conceptions existantes. Cette phase est typiquement brève et peut ne pas se prolonger au-delà de l'exposition immédiate au stimulus. La deuxième phase, l'intérêt situationnel maintenu, se caractérise par une attention soutenue grâce à un engagement significatif avec le contenu. L'apprenant commence à percevoir une connexion personnelle avec le matériel, ce qui prolonge son implication au-delà de la simple nouveauté. La troisième phase marque l'émergence d'un intérêt individuel : l'apprenant développe une prédisposition à se réengager avec le contenu spécifique au fil du temps, même en l'absence de sollicitation externe. La quatrième phase, l'intérêt individuel développé, représente une disposition durable qui se manifeste par une tendance à générer spontanément des questions et à rechercher activement des occasions d'approfondir ses connaissances dans le domaine \citep{HidiRenninger2006}. Ce modèle séquentiel n'implique pas une progression automatique : le passage d'une phase à l'autre requiert des conditions de soutien appropriées, et l'intérêt peut régresser ou stagner si ces conditions ne sont pas réunies.

Les facteurs susceptibles de déclencher et de maintenir l'intérêt situationnel ont fait l'objet d'une catégorisation distinguant les éléments individuels des éléments situationnels \citep{Bergin1999}. Parmi les facteurs individuels figurent le sentiment d'appartenance culturelle, l'identification à des modèles, les émotions associées au contenu, la perception de compétence et la pertinence par rapport aux objectifs personnels. Les facteurs situationnels, plus directement manipulables par l'enseignant, incluent les activités pratiques, la nouveauté, l'interaction sociale, la modélisation par des pairs ou des experts, les jeux et puzzles, ainsi que la dimension narrative du contenu \citep{Bergin1999}. Quatre types d'interventions se sont révélés efficaces pour favoriser le développement de l'intérêt : l'adaptation des caractéristiques structurelles de l'environnement d'apprentissage, la personnalisation du contexte en fonction des préférences individuelles, l'implémentation d'approches par problèmes, et la mise en évidence de l'utilité du contenu \citep{Harackiewicz2016}. L'efficacité de ces interventions varie selon la phase de développement de l'intérêt : les stratégies appropriées pour déclencher un intérêt situationnel initial diffèrent de celles requises pour soutenir la transition vers un intérêt individuel \citep{RenningerHidi2015}.

% ============================================================================
% RÉFÉRENCES UTILISÉES DANS CETTE SOUS-SECTION :
% - Renninger & Hidi (2015) : nature duale affective/cognitive, soutien développemental
% - Hidi (2006) : distinction intérêt/attention
% - Priniski et al. (2018) : continuum développemental, internalisation
% - Krapp et al. (1992) : intérêt situationnel, déclencheurs environnementaux
% - Schiefele (1991) : intérêt individuel comme disposition stable
% - Bergin (1999) : facteurs individuels et situationnels, rôle enseignant
% - Hidi & Renninger (2006) : modèle en quatre phases
% - Harackiewicz et al. (2016) : quatre types d'interventions
% ============================================================================


% 2.1.3. La Théorie de la Pertinence et de la Valeur
% % ============================================================================
% SOUS-SECTION 2.1.3 - La Théorie de la Pertinence et de la Valeur
% ============================================================================

\subsection{La Théorie de la Pertinence et de la Valeur}

La décision de s'engager dans une tâche d'apprentissage repose sur deux évaluations distinctes mais interdépendantes : l'attente de succès et la valeur attribuée à la tâche \citep{EcclesWigfield2002}. La théorie Expectancy-Value postule qu'un individu choisit de persister dans une activité exigeante lorsqu'il estime à la fois pouvoir y réussir et que cette activité présente une valeur suffisante pour justifier l'investissement requis. Ces deux composantes interagissent de manière multiplicative : une forte attente de succès ne compense pas l'absence de valeur perçue, et inversement, une tâche hautement valorisée mais perçue comme inaccessible ne suscitera pas d'engagement durable. Les croyances relatives à soi-même, incluant le sentiment de compétence et l'auto-efficacité, déterminent la composante expectation, tandis que les croyances relatives à la tâche déterminent la composante valeur \citep{EcclesWigfield2002}. Ce cadre théorique permet de prédire les choix de cours, la persistence dans une filière et les orientations professionnelles, en particulier dans les domaines scientifiques où les abandons précoces constituent un phénomène documenté.

La valeur subjective attribuée à une tâche se décompose en quatre dimensions distinctes \citep{WigfieldEccles1992}. La valeur intrinsèque correspond au plaisir inhérent à l'exécution de la tâche, indépendamment de ses conséquences. La valeur d'accomplissement renvoie à l'importance de la tâche pour l'identité de l'individu : réussir dans ce domaine confirme une facette valorisée du soi. La valeur d'utilité désigne la perception de l'utilité de la tâche pour atteindre des objectifs futurs pertinents pour la vie de l'individu \citep{Harackiewicz2014}. Le coût, enfin, représente les aspects négatifs associés à l'engagement dans la tâche : le temps requis, l'effort cognitif, l'anxiété de performance ou le renoncement à d'autres activités valorisées. Ces quatre dimensions ne sont pas mutuellement exclusives : un même apprenant peut simultanément apprécier une activité pour elle-même, y voir un enjeu identitaire, la considérer comme utile pour ses projets et en percevoir le coût. La configuration relative de ces dimensions varie selon les individus et les contextes, ce qui explique en partie la diversité des patterns d'engagement observés face à un même contenu d'apprentissage.

Parmi ces quatre dimensions, la valeur d'utilité présente une caractéristique particulière : elle repose sur la perception de connexions entre la tâche immédiate et des activités, objectifs ou contextes futurs, ce qui la rend particulièrement susceptible d'intervention externe \citep{Harackiewicz2014}. Un enseignant ou un parent peut expliciter ces connexions, aider l'apprenant à percevoir la pertinence d'un contenu pour ses projets personnels ou professionnels. Cette malléabilité distingue la valeur d'utilité des autres formes de valeur, plus directement liées aux caractéristiques intrinsèques de la tâche ou à l'histoire personnelle de l'individu. Les interventions visant à promouvoir la perception de valeur d'utilité ont démontré des effets positifs sur l'intérêt et la persistence, en particulier lorsque les apprenants génèrent eux-mêmes les connexions plutôt que de les recevoir passivement \citep{HullemanHarackiewicz2009}. Cette auto-génération des liens entre le contenu et la vie personnelle s'avère particulièrement efficace pour les apprenants initialement moins confiants dans leurs capacités, tandis que la simple présentation d'informations sur l'utilité bénéficie davantage aux apprenants déjà intéressés \citep{Harackiewicz2014}.

La pertinence personnelle constitue le mécanisme psychologique sous-jacent à l'efficacité de ces interventions sur la valeur d'utilité \citep{AlbrechtKarabenick2017}. Rendre un contenu pertinent implique d'établir des connexions significatives entre ce contenu et les expériences, les objectifs ou l'identité de l'apprenant \citep{Priniski2018}. Cette conception rejoint la notion de "psychologisation" du curriculum proposée par Dewey : transformer le contenu disciplinaire en l'ancrant dans l'expérience immédiate et les préoccupations actuelles de l'apprenant, de sorte qu'un obstacle intellectuel présent crée le besoin d'acquérir la connaissance en question \citep{AlbrechtKarabenick2017}. La distinction entre pertinence personnelle, affectivement orientée et liée aux intérêts et à l'identité, et pertinence impersonnelle, cognitivement orientée et liée aux applications pratiques, suggère que les voies vers l'engagement peuvent varier selon les individus \citep{AlbrechtKarabenick2017}. Pour l'enseignement de l'histoire, discipline souvent perçue comme déconnectée du quotidien des élèves, ces considérations soulèvent la question des moyens par lesquels le contenu historique peut être rendu personnellement significatif.

% ============================================================================
% RÉFÉRENCES UTILISÉES DANS CETTE SOUS-SECTION :
% - Eccles & Wigfield (2002) : Expectancy-Value Theory, interaction des composantes
% - Wigfield & Eccles (1992) : quatre types de valeur subjective
% - Harackiewicz et al. (2014) : utility value, interventions, auto-génération
% - Hulleman & Harackiewicz (2009) : self-generated utility value
% - Albrecht & Karabenick (2017) : pertinence personnelle, Dewey, psychologisation
% - Priniski et al. (2018) : connexions significatives, identité
% ============================================================================


% 2.1.4. L'Apprentissage Actif et le Cadre ICAP
% % ============================================================================
% SOUS-SECTION 2.1.4 - L'Apprentissage Actif et le Cadre ICAP
% ============================================================================

\subsection{L'Apprentissage Actif et le Cadre ICAP}

L'apprentissage actif désigne un ensemble d'approches pédagogiques dans lesquelles les apprenants participent à leur propre construction de connaissances par l'exploration, l'expérimentation et la collaboration \citep{Mayer2014}. Cette conception dépasse la simple activité physique pour englober l'engagement cognitif, défini comme le fait de penser activement à ce que l'on fait \citep{Yannier2021}. Les manifestations de cet engagement incluent la résolution de problèmes, les discussions de groupe, la création de projets et l'interaction avec des simulations. Les méta-analyses comparant l'apprentissage actif aux méthodes transmissives révèlent des gains significatifs en termes de performance et de réduction des taux d'échec, particulièrement dans les disciplines scientifiques \citep{Freeman2014}. Toutefois, les apprenants perçoivent initialement l'apprentissage actif comme moins efficace que les méthodes passives \citep{Deslauriers2019}. Cette perception résulte d'une interprétation erronée : l'effort cognitif accru requis par les méthodes actives est confondu avec un signe de difficulté, alors qu'il indique un traitement plus profond de l'information.

Le cadre ICAP propose une taxonomie des modes d'engagement cognitif fondée sur les comportements observables de l'apprenant \citep{Chi2014}. Cette taxonomie distingue quatre niveaux ordonnés selon leur efficacité pour l'apprentissage. Le mode passif correspond à la réception d'information sans engagement manifeste : écouter un cours, regarder une vidéo sans prendre de notes. Le mode actif implique une manipulation du matériel d'apprentissage : souligner un texte, recopier des passages, répéter mentalement des informations. Le mode constructif se caractérise par la génération d'éléments nouveaux qui dépassent l'information présentée : formuler des hypothèses, établir des liens avec ses connaissances antérieures, créer des schémas explicatifs originaux. Le mode interactif ajoute à la dimension constructive un échange substantiel avec un partenaire, où les contributions de chacun enrichissent mutuellement la compréhension \citep{Chi2014}.

L'hypothèse centrale du cadre ICAP postule que les résultats d'apprentissage suivent une hiérarchie prévisible : Interactif > Constructif > Actif > Passif \citep{Chi2014}. Cette hiérarchie repose sur des mécanismes cognitifs distincts. Le mode passif permet uniquement le stockage d'information dans la mémoire de travail. Le mode actif active les connaissances préexistantes sans nécessairement les modifier. Le mode constructif génère de nouvelles inférences et intègre l'information nouvelle aux schémas existants. Le mode interactif amplifie ces processus constructifs par la confrontation des représentations et la co-construction de sens \citep{Chi2014}. La validation empirique de cette hiérarchie provient de méta-analyses couvrant des dizaines d'études expérimentales, bien que les frontières entre modes demeurent parfois difficiles à établir dans la pratique \citep{Chi2014}.

L'application du cadre ICAP à l'analyse des pratiques pédagogiques permet d'évaluer le potentiel cognitif des différentes activités proposées aux apprenants. Un cours magistral sans interaction maintient l'apprenant en mode passif. La prise de notes verbatim relève du mode actif, tandis que la reformulation personnelle du contenu caractérise le mode constructif. Les discussions entre pairs, lorsqu'elles impliquent un échange substantiel d'idées et non une simple alternance de monologues, exemplifient le mode interactif \citep{Chi2014}. Cette grille d'analyse révèle que de nombreuses activités présentées comme actives ne dépassent pas le niveau de manipulation superficielle du matériel. Pour l'enseignement de l'histoire, discipline souvent critiquée pour son recours excessif à la transmission magistrale \citep{AudigierFink2010}, le cadre ICAP suggère que les approches favorisant la discussion, l'interprétation des sources et la confrontation des perspectives constituent des voies vers un engagement cognitif plus profond.

% ============================================================================
% RÉFÉRENCES UTILISÉES DANS CETTE SOUS-SECTION :
% - Mayer (2014) : définition apprentissage actif, engagement cognitif
% - Yannier et al. (2021) : engagement cognitif au-delà de l'activité physique
% - Freeman et al. (2014) : méta-analyse apprentissage actif STEM
% - Deslauriers et al. (2019) : perception erronée de l'efficacité
% - Chi & Wylie (2014) : cadre ICAP, quatre modes, hiérarchie, mécanismes
% - Audigier & Fink (2010) : critiques enseignement transmissif histoire
% ============================================================================


% 2.1.5. La Personnalisation de l'Apprentissage
% \input{chapters/Ch2_EtatDeLArt/sections/2_1_5_Personnalisation.tex}

%%%%%%%%%%%%%%%%%%%%%%%%%%%%%%%%%%%%%%%%%%%%%%%%%%%%%%%%%%%%%%%%%
% 2.2. LE CONTEXTE DISCIPLINAIRE : ENSEIGNER ET APPRENDRE L'HISTOIRE
%%%%%%%%%%%%%%%%%%%%%%%%%%%%%%%%%%%%%%%%%%%%%%%%%%%%%%%%%%%%%%%%%
% \section{Le Contexte Disciplinaire : Enseigner et Apprendre l'Histoire}
% \label{sec:contexte-histoire}

% 2.2.1. Épistémologie et Comparaison Interdisciplinaire
% \input{chapters/Ch2_EtatDeLArt/sections/2_2_1_Epistemologie.tex}

% 2.2.2. La Perception des Élèves : Le Paradoxe "Ennuyeux mais Important"
% \input{chapters/Ch2_EtatDeLArt/sections/2_2_2_Perception_Eleves.tex}

% 2.2.3. Pratiques Pédagogiques : Du Transmissif à l'Actif
% \input{chapters/Ch2_EtatDeLArt/sections/2_2_3_Pratiques_Pedagogiques.tex}

%%%%%%%%%%%%%%%%%%%%%%%%%%%%%%%%%%%%%%%%%%%%%%%%%%%%%%%%%%%%%%%%%
% 2.3. CADRE THÉORIQUE DE L'INTERACTION
%%%%%%%%%%%%%%%%%%%%%%%%%%%%%%%%%%%%%%%%%%%%%%%%%%%%%%%%%%%%%%%%%
% \section{Cadre Théorique de l'Interaction : Cognition, Multimédia et Présence Sociale}
% \label{sec:cadre-interaction}

% 2.3.1. Architecture Cognitive et Apprentissage Multimédia (CTML & TCC)
% \input{chapters/Ch2_EtatDeLArt/sections/2_3_1_CTML_TCC.tex}

% 2.3.2. La Présence Sociale et le Paradigme CASA
% \input{chapters/Ch2_EtatDeLArt/sections/2_3_2_Presence_Sociale_CASA.tex}

% 2.3.3. De la Présence à l'Apprentissage : Incarnation et Agence Sociale
% \input{chapters/Ch2_EtatDeLArt/sections/2_3_3_Incarnation_Agence.tex}

% 2.3.4. Les Ruptures de Présence : La Vallée de l'Étrange
% \input{chapters/Ch2_EtatDeLArt/sections/2_3_4_Uncanny_Valley.tex}

%%%%%%%%%%%%%%%%%%%%%%%%%%%%%%%%%%%%%%%%%%%%%%%%%%%%%%%%%%%%%%%%%
% 2.4. L'ÉMERGENCE DES IA GÉNÉRATIVES EN ÉDUCATION
%%%%%%%%%%%%%%%%%%%%%%%%%%%%%%%%%%%%%%%%%%%%%%%%%%%%%%%%%%%%%%%%%
% \section{L'Émergence des IA Génératives en Éducation}
% \label{sec:ia-generatives}

% 2.4.1. Des Agents Scriptés aux Agents Génératifs : Le Saut Qualitatif
% \input{chapters/Ch2_EtatDeLArt/sections/2_4_1_Agents_Generatifs.tex}

% 2.4.2. La Personnalisation de l'Apprentissage par l'IA
% \input{chapters/Ch2_EtatDeLArt/sections/2_4_2_Personnalisation_IA.tex}

% 2.4.3. Fiabilité et Hallucinations : Le Défi Épistémique
% \input{chapters/Ch2_EtatDeLArt/sections/2_4_3_Hallucinations.tex}

%%%%%%%%%%%%%%%%%%%%%%%%%%%%%%%%%%%%%%%%%%%%%%%%%%%%%%%%%%%%%%%%%
% 2.5. LE PHÉNOMÈNE DE L'ILLUSION DE COMPRÉHENSION
%%%%%%%%%%%%%%%%%%%%%%%%%%%%%%%%%%%%%%%%%%%%%%%%%%%%%%%%%%%%%%%%%
% \section{Le Phénomène de l'Illusion de Compréhension}
% \label{sec:illusion-comprehension}

% 2.5.1. Métacognition et Calibration de la Confiance
% \input{chapters/Ch2_EtatDeLArt/sections/2_5_1_Metacognition.tex}

% 2.5.2. L'Heuristique de Fluidité
% \input{chapters/Ch2_EtatDeLArt/sections/2_5_2_Fluency_Heuristic.tex}

% 2.5.3. L'Autorité de l'Agent et la Vigilance Épistémique
% \input{chapters/Ch2_EtatDeLArt/sections/2_5_3_Autorite_Vigilance.tex}

%%%%%%%%%%%%%%%%%%%%%%%%%%%%%%%%%%%%%%%%%%%%%%%%%%%%%%%%%%%%%%%%%
% 2.6. SYNTHÈSE ET PROBLÉMATIQUE
%%%%%%%%%%%%%%%%%%%%%%%%%%%%%%%%%%%%%%%%%%%%%%%%%%%%%%%%%%%%%%%%%
% \section{Synthèse et Problématique}
% \label{sec:synthese-problematique}

% 2.6.1. Convergence et Tensions Théoriques
% \input{chapters/Ch2_EtatDeLArt/sections/2_6_1_Convergences_Tensions.tex}

% 2.6.2. Les Lacunes de la Littérature Actuelle
% \input{chapters/Ch2_EtatDeLArt/sections/2_6_2_Lacunes.tex}

% 2.6.3. Questions et Hypothèses de Recherche
% \input{chapters/Ch2_EtatDeLArt/sections/2_6_3_Questions_Hypotheses.tex}

%%%%%%%%%%%%%%%%%%%%%%%%%%%%%%%%%%%%%%%%%%%%%%%%%%%%%%%%%%%%%%%%%
% FIN DU CHAPITRE 2
%%%%%%%%%%%%%%%%%%%%%%%%%%%%%%%%%%%%%%%%%%%%%%%%%%%%%%%%%%%%%%%%%
